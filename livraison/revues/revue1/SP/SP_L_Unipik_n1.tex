% version 1.01, date 07/03/16, auteur Matthieu Martins-Baltar
\documentclass[compress,xcolor=dvipsnames]{beamer}

%Pour les schémas d'architecture
\usepackage{etex}
\usepackage{tikz}
\usetikzlibrary{shapes,arrows,chains,backgrounds,fit}
%FIN Pour les shémas d'architecture

\usepackage[french]{babel}
\selectlanguage{french}
\usepackage[utf8]{inputenc}
\usepackage[T1]{fontenc}
\usepackage{tikz}
\usepackage{wrapfig}
\usepackage{multirow}
%\usepackage{pgf-pie}
\usepackage{pgfplots}
\usepackage{pdfpages}
\usepackage{vocabulaireUnipikPresentation}
\usepackage{commun/vocabulaireCommun}
\usepackage{hyperref}
\usepackage{movie15}
\usepackage{xcolor}
\usepackage{lscape}
\usepackage{booktabs}
\usepackage{multirow}
\usepackage{colortbl}

\newcommand{\tabincell}[2]{\begin{tabular}{@{}#1@{}}#2\end{tabular}}

\usetheme{eastpic}%{Berlin}%{Madrid}

 
 
%Information to be included in the title page:
\title{Revue de PIC}
\date{\today}
\author{Unipik}
\institute{\insa}

\setbeameroption{show notes}

 
\begin{document}

\speaker{\Sergi} 

\begin{frame}[plain]
	\titlepage
\end{frame}

\begin{frame}{Sommaire}
	\tableofcontents[hideallsubsections]
\end{frame}
 

\speaker{\Sergi}
\section[Présentation]{Présentation}
%version 1.00,	date 19/10/2016	auteur(s) Pierre Porche
\subsection{} % PAs besoin de titre

\speaker{\Pierre}

\begin{frame}

\frametitle{Présentation de l'équipe}
	\begin{figure}[!h]
		\begin{center}
			\includegraphics[scale=0.2]{images/organigrammeFonctionnel.pdf}
			\caption{La nouvelle équipe}
		\end{center}
	\end{figure}

\end{frame}

\begin{frame}
\frametitle{Présentation}
	\frametitle{Présentation du client}
	\begin{center}
		UNICEF Seine-Maritime
	\end{center}

	\begin{block}{Ses actions principales}
		\begin{itemize}
			\item La sensibilisation aux droits de l'enfant
			\item Les événements 
			\item Les ventes sur stands
		\end{itemize}
	\end{block}

	\begin{block}{Les représentants de l'UNICEF Seine-Maritime présents}
		\begin{itemize}
			\item Véronique Davreux
			\item Naoual Guery
		\end{itemize}
	\end{block}
\end{frame}



\section[Cahier des charges]{Cahier des Charges}
\subsection{} % PAs besoin de titre

\begin{frame}
\frametitle{UNICEF 76}
\framesubtitle{Cahier des charges}
	\begin{itemize}
		\item Plaidoyers dans les \'ecoles
		\item Actions frimousses
		\item Villes Amies des Enfants
		\item Ventes ponctuelles
	\end{itemize}
\end{frame}

\speaker{\Sergi}
\section[Avancement]{Avancement du Projet}
% version 1.00, date 07/03/16, auteur Matthieu Martins-Baltar
%\subsection{} % Pas besoin de titre


\subsection{Introduction}
%version 1.00,	date 19/10/2016	auteur(s) François Decq
\speaker{\Francois}

\begin{frame}
\frametitle{Rappels}
\begin{block}{Technologies}
	\begin{itemize}
		\item Symfony3 : framework PHP
		\item Base de données : PostgreSql
		\item ORM : Doctrine
		\item Moteur de templates : TWIG
		\vspace{.5cm}
		\item Architecture MVC
	\end{itemize}
\end{block}
\end{frame}

\begin{frame}
\frametitle{Rappels}
\begin{block}{Fin du semestre dernier : lot 1}
	\begin{itemize}
		\item L'architecture du projet est finalisée
		\begin{figure}[!h]
			\begin{center}
				\includegraphics[scale=0.25]{images/mvcGeneral.pdf}
				\caption{Architecture MVC}
			\end{center}
		\end{figure}
		\item L'architecture de la base de données est stable
	\end{itemize}
\end{block}
\end{frame}

\subsection{Présentation des fonctionnalités}
%version 1.00,	date 19/10/2016	auteur(s) François Decq
\speaker{\Francois}

\begin{frame}
\frametitle{Présentation des fonctionnalités}
\begin{block}{Fonctionnalités lot 2}
	\begin{itemize}
		\item F1, F2 : Ajout/Modification/Suppression Bénévole/Établissement
		\item F3 : Mailing (en cours)
		\item F4 : Formulaire de demande
		\item F5 : Affectation intervention à bénévole
		\item F6 : Mise à jour intervention par bénévole
		\item F7 : Mails de rappel à établissement et bénévole
	\end{itemize}
\end{block}
\end{frame}


\subsection{FRONT-END}
\input{sources/avancementProjet/front-end.tex}

\subsection{Serveur}
% version 1.00, date 10/05/16, auteur Matthieu Martins-Baltar


\speaker{\Matthieu}

\begin{frame}
	\frametitle{Serveur}
    Problématique du serveur :
      \begin{itemize}
        \item Projet nécessitant un serveur
        \item Client avec des ressources limitées
        \item Client non technique
      \end{itemize}
\end{frame}

\begin{frame}
	\frametitle{Serveur}
	Solution explorée jusque là :
	\begin{itemize}	
    \item Hébergement par l'INSA
    \end{itemize}
    
	Accord proposé :
	\begin{itemize}
		\item Mise en place d'un programme UNICEF Campus
		\item Création de Projets d’Ouverture et d’Approfondissement %nouvelles fonctionnalité et/ou admin sys
	\end{itemize}
\end{frame}

\begin{frame}
	 Rencontre avec de nombreux interlocuteurs :
	\begin{itemize}
    	\item Présentation de la problématique pour obtenir leur soutien:
		\begin{itemize}
			\item Anne Caldin (service culturel)
			\item Maxime Reynet (service com)
			\item Direction Générale (via Gilles Gasso)
		\end{itemize}
		\item Présentation du dossier avec les demandes techniques précises:
		\begin{itemize}
			\item Baptiste Blondel-Angot (service juridique)
			\item Laurent Vasseur et Sébastien Bonnegent (DSI)
		\end{itemize}    
	\end{itemize}    
\end{frame}

\begin{frame}
	Demandes techniques :
	\begin{itemize}
		\item serveur web Apache
		\item PHP 5.5.9 avec options (PHP-XML, PDO, etc.) 
		\item PostgreSQL + PostGIS (faible volumétrie)
		\item Paramètres divers (JSON, ctype, libxml, etc.)
	\end{itemize}

    Autres solutions possibles :
      \begin{itemize}
        \item Faire une demande de mécénat auprès d'un hébergeur privé
        \item Faire une demande auprès d'UNICEF France
        \item Envisager une solution payante, la moins chère possible
      \end{itemize}
\end{frame}



%\subsection{Réalisation de vues}
%%version 1.00,	date 12/05/2016	auteur(s) Pierre Porche
\speaker{\Juliana}

\begin{frame}
\frametitle{Développement frontend}
\begin{block}{Technologies utilisées}
	\begin{itemize}
		\item TWIG
		\item CSS
		\item Bootstrap
		\item jQuery
	\end{itemize}
\end{block}
\end{frame}

\begin{frame}
\frametitle{Développement frontend}
\begin{block}{TWIG: Le moteur de templates }

\begin{minipage}[c]{.45\linewidth}
      \begin{figure}[r]
		\includegraphics[scale=0.3]{images/moteursTemp.png}
		%\caption{Moteurs de templates}
	  \end{figure}
\end{minipage} \hfill
\begin{minipage}[c]{.42\linewidth}
	
		\begin{itemize}
			\item Intégré directement dans le framework Symfony3
			\item Héritage de templates
			\item Séparation du code HTML du code PHP
		\end{itemize}
	
\end{minipage} \hfill
\end{block}
\end{frame}

\begin{frame}
\frametitle{Développement frontend}
\begin{block}{ CSS et Bootstrap }
	\begin{itemize}
		\item Génération de vues responsives
		\item Grande collection de composants
		\item Communauté qui propose des centaines d'autres composants
	\end{itemize}
\end{block}
\end{frame}

\begin{frame}
\frametitle{Développement frontend}
\begin{block}{jQuery }
	\begin{itemize}
		\item Modification de la structure HTML et aussi des styles
		\item Dynamisme côté client
		\item Documentation riche et complète
	\end{itemize}
\end{block}
\end{frame}

\begin{frame}
\frametitle{Responsive design}
	\begin{multicols}{2}
		\begin{figure}[!h]
			\begin{center}
				\includegraphics[scale=0.19]{images/screenshot1.png}

				\caption{Capture d'écran: Listes grands écrans }
			\end{center}
		\end{figure}
		\begin{figure}[!h]
			\begin{center}
				\includegraphics[scale=0.16]{images/screenshot2.png}
				\caption{Capture d'écran: Listes petits écrans }
			\end{center}
		\end{figure}
	\end{multicols}

\end{frame}



%\subsection{Avancement Serveur}
%% version 1.00, date 10/05/16, auteur Matthieu Martins-Baltar


\speaker{\Matthieu}

\begin{frame}
	\frametitle{Serveur}
    Problématique du serveur :
      \begin{itemize}
        \item Projet nécessitant un serveur
        \item Client avec des ressources limitées
        \item Client non technique
      \end{itemize}
\end{frame}

\begin{frame}
	\frametitle{Serveur}
	Solution explorée jusque là :
	\begin{itemize}	
    \item Hébergement par l'INSA
    \end{itemize}
    
	Accord proposé :
	\begin{itemize}
		\item Mise en place d'un programme UNICEF Campus
		\item Création de Projets d’Ouverture et d’Approfondissement %nouvelles fonctionnalité et/ou admin sys
	\end{itemize}
\end{frame}

\begin{frame}
	 Rencontre avec de nombreux interlocuteurs :
	\begin{itemize}
    	\item Présentation de la problématique pour obtenir leur soutien:
		\begin{itemize}
			\item Anne Caldin (service culturel)
			\item Maxime Reynet (service com)
			\item Direction Générale (via Gilles Gasso)
		\end{itemize}
		\item Présentation du dossier avec les demandes techniques précises:
		\begin{itemize}
			\item Baptiste Blondel-Angot (service juridique)
			\item Laurent Vasseur et Sébastien Bonnegent (DSI)
		\end{itemize}    
	\end{itemize}    
\end{frame}

\begin{frame}
	Demandes techniques :
	\begin{itemize}
		\item serveur web Apache
		\item PHP 5.5.9 avec options (PHP-XML, PDO, etc.) 
		\item PostgreSQL + PostGIS (faible volumétrie)
		\item Paramètres divers (JSON, ctype, libxml, etc.)
	\end{itemize}

    Autres solutions possibles :
      \begin{itemize}
        \item Faire une demande de mécénat auprès d'un hébergeur privé
        \item Faire une demande auprès d'UNICEF France
        \item Envisager une solution payante, la moins chère possible
      \end{itemize}
\end{frame}




\speaker{\Sergi}
\section[Gestion projet]{Gestion de Projet}
\speaker{\Sergi{}}
\subsection{} % PAs besoin de titre

\begin{frame}
\frametitle{La conduite du projet : une continuité}
\end{frame}

\begin{frame}
\frametitle{La conduite de projet : les nouveautés}
\begin{itemize}
	\item Séparation de l'équipe en deux
	\item Attribution du poste responsable sécurité
\end{itemize}

\end{frame}

\subsection{}
\begin{frame}
\frametitle{L'inspection technique}
\end{frame}

\subsection{}
\begin{frame}
\frametitle{La passation}
\end{frame}






\speaker{\Sergi}
\section[Qualité]{Démarche Qualité}
\speaker{\Pierre}

\subsection{} % PAs besoin de titre

\begin{frame}
\frametitle{Satisfaction client}
Actions mises en place :
	\begin{itemize}
		\item Réunions régulières
		\item Prise en compte des remarques et réclamations
		\item Contacts fréquents par emails
		\item Mise en place d'un indicateur
	\end{itemize}
\end{frame}

\begin{frame}
\frametitle{Amélioration continue}
\framesubtitle{Traitement des faits techniques}
\begin{center}
\begin{figure}
\includegraphics[scale=0.21]{images/cycleCorrectionFT.pdf}
\caption{Cycle de correction d'un fait technique}
\end{figure}
\end{center}
\end{frame}


\begin{frame}
\frametitle{Traçabilité}
Actions mises en place :
	\begin{itemize}
		\item Établissement d'un plan de gestion des configurations
		\item Sauvegardes régulières
		\item Versionnage
		\item Signature
		\item Rédaction de comptes-rendus
	\end{itemize}
\end{frame}

\speaker{\Kafui}

\begin{frame}
\frametitle{Risques et opportunités}
But : \textbf{Améliorer la qualité du PIC}
	\begin{itemize}
		\item Réunion hebdomadaire
		\item Réévaluation des risques
	\end{itemize}
\end{frame}


\begin{frame}
\frametitle{Risques et opportunités}
	\begin{center}
	\begin{figure}
	\includegraphics[scale=0.30]		{images/risque.png}
	\caption{Suivi des risques et opportunités}
	\end{figure}
	\end{center}
\end{frame}






\speaker{\Sergi}
\section[Conclusion]{Conclusion}
\begin{frame}
\frametitle{Conclusion}
\begin{itemize}
 \item Cohésion d'équipe et motivation
 \item De l'architecture vers les fonctionnalités
 \item Gestion de projet efficace pour répondre aux attentes du client
 \item Démarche qualité orientée vers la satisfaction client
 \item De la Seine Maritime vers la Normandie puis vers la France
\end{itemize}
\end{frame}

 
\end{document}

