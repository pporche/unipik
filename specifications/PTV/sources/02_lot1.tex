% version 1.00, date 20/02/16, auteur Michel Cressant
% version 1.01, date 02/03/16, auteur Michel Cressant
	Ce chapitre a pour but de valider les points évoqués dans le \DSE{} et d'expliquer notre démarche pour s'assurer du bon fonctionnement des notre lot.
	
\section{Description du lot}
	Ce paragraphe rappelle les exigences demandées pour le lot 1 :
	\begin{itemize}
		\item Une architecture matérielle et logicielle sera mise pour le fonctionnement de l'application dans sa globalité ;
		\item La base de données doit donc être opérationnelle et peut petre manipulée via des scripts ;
		\item Cette structure sera locale ;
		\item La livraison du lot se fera à T0 + 7 semaines (T0 étant la date du lancement du projet).
	\end{itemize}
	
\section{Conception du lot}
	Ce paragraphe explique les méthodes suivies pour la conception de ce lot :
	\begin{itemize}
		\item Le code devra respecter les bonnes conventions de codage (indentation, nom de variable explicite,... ),
		\item Le code devra être documenté.
	\end{itemize}
	
\section{Déroulement de la recette}
	Ce paragraphe explique comment doit se dérouler la recette : 
	\begin{itemize}
		\item La recette sera effectuée à l'\INSA{} ;
		\item La recette sera effectuée sur une machine de l'équipe \nomEquipe{} ;
		\item La recette sera effectuée sur le lot 1.
	\end{itemize}

\section{Description générale des tests}
	Ce paragraphe présente la description générale des tests effectués sur le lot 1 :	
	\begin{itemize}
		\item Les tests seront effectués sur une application web très simplifiée, développée en Modèle-Vue-Controleur en architecture trois tiers. Ils permettront de démontrer la communication entre les différents éléments du MVC.
	\end{itemize}

\section{Validation}	
	\begin{itemize}
		\item Le lot 1 ne pourra partir en validation que si aucun test unitaire ne produit d'erreur ;
		\item Le lot 1 ne pourra partir en validation que si aucun test d’intégration ne produit d'erreur.
	\end{itemize}
	
