% version 1.00	Auteur Pierre Porche date 29/03/2016

\documentclass [a4paper] {article}
\usepackage[utf8]{inputenc}
\usepackage[francais]{babel}
\usepackage[top=2cm, bottom=4cm, left=2cm, right=2cm]{geometry} 
\usepackage{fancyhdr}
\usepackage{graphicx}
\usepackage{color, colortbl}
\usepackage{longtable}
\usepackage{vocabulaireUnipik}
\usepackage{hyperref}
\pagestyle{fancy}
\definecolor{Gray}{gray}{0.8}



%--- En-t�te et pied de page ---%

\renewcommand{\footrulewidth}{0,01cm}
\rhead{}
\chead{\huge{Compte-rendu de réunion de Tutorat Pédagogique}}					%titre
\begin{document}

25/03/2016			 				%Date 
\hfill   
\hfill 	 9:00 - 9:40 				%Heure de d�but, heure de fin.


\lfoot{Version : 1.00} 			% version
%--- Fin en-t�te et pied de page ---%
\section*{Historique des révisions}
\begin{center}
			\begin{tabular}{| p{2.5cm} | p{3cm} | p{3cm} | p{3cm} | p{3.5cm} |}
				\hline
				\rowcolor{Gray}
				Version & Date & Auteur(s) & Modification(s) & Partie(s) modifiée(s)		 \\
				\hline
				1.00 & 29/03/2016 & \Pierre & Création & Toutes \\
		\hline		
			\end{tabular}
		\end{center}

\section*{Signatures}

		\begin{center}
			\begin{tabular}{| p{2.5cm} | p{4cm} | p{3cm} | p{3cm} | p{2.5cm} |}
				\hline
				\rowcolor{Gray}
				Rôle & Fonction & Nom & Date & Visa		 \\
				\hline
				Vérificateur & \RQA & \Kafui & 31/03/2016 & pgpic \\[30pt]
				\hline
				Validateur & \CP & \Sergi & 31/03/2016 & pgpic \\[30pt]	
				\hline
			\end{tabular}
		\end{center}

%--- Réunion --%

\section{Résumé de la semaine passée}
\paragraph*{}
Nous avons effectué la formation Symfony pour tous les développeurs.

\paragraph*{}
Le client est venu mercredi pour la livraison du lot 1. Nous lui avons déroulé le \CDR{} point par point en lui expliquant l'utilité d'un tel lot et du \CDRCourt{}. \nomTuteurPedago{} nous explique que nous aurions dû montrer, conformément à la fonctionnalité 4, un envoi d'email et gérer la sécurité. Pour cela, il faut ajouter une fiche de risque et nommer un responsable sécurité. Concernant les emails, il faut se renseigner sur les solutions qui s'offrent à nous pour les serveurs SMTP.

\paragraph*{}
Le dossier pour l’hébergement de notre service par l'INSA a encore avancé : \Sergi{} est allé voir M. Gasso pour lui en parler et ce dernier en a profité pour en parler à la réunion des directeurs des départements de l'INSA. À présent, il faut rédiger un dossier technique à mettre entre les mains de la DSI et du service juridique de l'INSA. Dans ce dossier, il faut mettre en avant l'apport pour les étudiants : par exemple inclure le département ASI en prévoyant des PAO à ce sujet. Pour le partenariat entre l'UNICEF et l'INSA, il serait intéressant de contacter des écoles qui ont déjà un partenariat pour se renseigner.

\paragraph*{}
Nous avons vu le client au tournoi sportif UNICEF qui a eu lieu ce jeudi 24 mars 2016. Celui-ci nous a dit qu'une réunion inter-comité avait eu lieu durant lequel notre projet a été évoqué. Il nous a été demandé si notre service pourrait s'intégrer à un autre service qui n'existe pas encore et qui est développé par une autre équipe qui n'est pas encore mise en place. Nous avons répondu qu'en théorie, la réponse serait "oui" mais nous ne mettrons aucune action en place sans nouvelles plus détaillées du projet.

%--- fin de réunion ---%
\newpage



\end{document}





















