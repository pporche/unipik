% version 1.00	Auteur Pierre Porche		Date 29/03/2016

\documentclass [a4paper] {article}
\usepackage[utf8]{inputenc}
\usepackage[francais]{babel}
\usepackage[top=2cm, bottom=4cm, left=2cm, right=2cm]{geometry} 
\usepackage{fancyhdr}
\usepackage{graphicx}
\usepackage{color, colortbl}
\usepackage{longtable}
\usepackage{vocabulaireUnipik}
\pagestyle{fancy}
\definecolor{Gray}{gray}{0.8}
\definecolor{White}{gray}{1.0}


%--- En-t�te et pied de page ---%

\renewcommand{\footrulewidth}{0,01cm}
\rhead{}
\chead{\huge{Compte-rendu de réunion interne}}					%titre
\begin{document}

19/04/2016			 				%Date 
\hfill   
\hfill 	 8:41 - 9:04 				%Heure de d�but, heure de fin.


\lfoot{Version : 1.00} 			% Version
%--- Fin en-t�te et pied de page ---%

\section*{Historique des révisions}
\begin{center}
			\begin{tabular}{| c | c | c | c | p{4cm} |}
				\hline
				\rowcolor{Gray}
				Version & Date & Auteur(s) & Modification(s) & Partie(s) modifiée(s)		 \\
				\hline
				1.00 & 19/04/2016 & \Pierre & Création & Toutes \\
		\hline		
			\end{tabular}
		\end{center}

\section*{Signatures}

		\begin{center}
			\begin{tabular}{| c | c | c | c | p{4cm} |}
				\hline
				\rowcolor{Gray}
				Rôle & Fonction & Nom & Date & Visa		 \\
				\hline
				Vérificateur & \RQA & \Kafui &  & pgpic \\[30pt]
				\hline
				Validateur & \CP & \Sergi &  & pgpic \\[30pt]	
				\hline
			\end{tabular}
		\end{center}



\section{\FS}
Sergi : FS? -> faisez la



\section{Planning}
il reste 4 semaines et demi de temps sachant qu'il y a la semaine de livraison donc 3 semaines et demi pour dev et 4 et demi pour la revue.
Ca va etre le rush tavu.

Reste a faire :
- Fin de phase de concetptio,n
- Front : les diagrammes de navigation, on a  pas la réponse du client. Peut on commencer? -> oui d'apreès les FE.
- Back : diagramme de bundle avec entités et méthodes. C'est check.

pour finir la phase de conception, il faut relire les deux diagrammes.


Kafui a fait le dump, il faudra le dire a MM


\section{Planning}
Après la conception, il faudra passer a la dev. il faut se concentrer sur 3 axes : plaidoyers bénévoles et etablissements
Afin de ne pas partir dans tous les sens, on va procéder par fonctionnalités.

On va faire un "chef" d'équipe par FE/BE. pour le front, c'est Julie. POUR LE BACK C'EST LE ROUQUIN. ca n'a pas d'impact sur le rôle de Michou en RD


\section{Revue de risques et opportunités}

FDR 006 : proba 2 -> 3 car PPORCHE cancer redoublement   //   action prév : voir un chef en cas de coup dur. 

Fiche de Risque 013 : proba 1 -> 3 car si pas cde serveur, samarchpa


Fiche de Risque 015 : proba 2 -> 3 car FLEERCIHE a du cancer

Fiche d’Opportunités 001 proba 2 -> 1 

Fiche d’Opportunités 004 proba 4 -> 3

 



%--- fin de réunion ---%


\end{document}







