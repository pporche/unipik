Dans le but d'identifier clairement et de manière unique les éléments de configuration du \picCourt{}, il est nécessaire de spécifier des règles de nommage. Chaque élément inclus dans un référentiel doit respecter la règle de nommage qui lui convient.\\

L'ensemble des fichiers suit le modèle suivant :
\[
  <Type>\_<Référentiel>\_Unipik\_<Suffixe>
\]
On utilisera des abrévations pour chaque élément.\\
La première version du Plan de Gestion des Configurations par exemple aura pour nom de fichier :
\[
PGC\_Q\_Unipik\_v1.00
\]
Et se trouvera dans le dossier :
\[
Unipik/qualite/GP/PGC/
\]

\section{Nom du \picCourt{}}
Ce champ indique l'appartenance du document au \PICCourt{}, qu'importe sa provenance
 (\PICCourt{}, client, département \ASI{}, autres \dots{}). Il a pour valeur \textbf{\nomEquipe}.

\section{Référentiels}

Tous les éléments réalisés au cours du PIC appartiennent à l'un des 5 référentiels suivants : Développement, Livraison, Qualité, Ressources et Spécification.
\begin{table}[H]
\centering
	\begin{tabularx}{11cm}{|X|c|X|}
	\hline
	\rowcolor[gray]{0.85} Référentiel & Abréviation & Nom du dossier sous \git{} \\
	\hline
	Spécifications & S & specifications\\ 
	\hline
	Qualité & Q & qualite\\
	\hline
	Développement & D & developpement\\
	\hline	
	Livraison & L & livraison\\
	\hline 
	Ressources & R & ressources\\
	\hline
	\end{tabularx}
\caption{Abréviations associées à chaque type}
\label{Référentiel}
\end{table}

\section{Types}

\begin{table}[H]
	\centering
	\begin{tabularx}{18cm}{|X|c|c|}
    \hline
    \rowcolor[gray]{0.85} Documents et Enregistrements & Type & Suffixe\\
    \hline	
    \multicolumn{3}{|c|}{\textbf{\bsc{Référentiel Qualité}}}\\
    \hline
    Dossier de Suivi de la Qualité & DSQ & -\\
    \hline
       Fiche de Fait Technique & FFT & n\\
    \hline
       Fiche de Compétences & FC & p-v\\
    \hline
       Fiche de Formation & FF & c\\
    \hline
       Fiche de Rôle & FR & r-v\\
    \hline
       Organigramme Fonctionnel & OF & v\\
    \hline
       Plan de Formation & PF & v\\
    \hline
       Questionnaire de Satisfaction Client & QSC & d\\
    \hline
       Rapport d'Audit Interne & RAI & d\\
    \hline
       Tableau de Bord & TB & s\\
    \hline
       Fiche d'Indicateurs & FI & n\\
    \hline
    Gestion de Projet & GP & -\\
    \hline
       Compte-Rendu & CR & -\\
    \hline
          Compte-Rendu de réunion Client & CRC & d\\
    \hline
          Compte-Rendu de réunion Exceptionnelle & CRE & d\\
    \hline
          Compte-Rendu de réunion Interne & CRI & d\\
    \hline
          Compte-Rendu de réunion Inter-PIC & CRIP & d\\
    \hline
          Compte-Rendu de réunion Tuteur Pédagogique & CRTP & d\\
    \hline
          Compte-Rendu de réunion Tuteur Qualité & CRTQ & d\\
    \hline
       E-mails & Mails & -\\
    \hline
          Mail du Client & MC & d\\
    \hline
          Mail du Directeur Qualité & MDQ & d\\
    \hline
          Mail de Livraison & ML & d\\
    \hline
          Mail de l'Unité P3 & MP3 & d\\
    \hline
          Mail du Tuteur Pédagogique & MTP & d\\
    \hline
          Mail du tuteur Qualité & MTQ & d\\
    \hline
       Procès-Verbal & PV & PV\\
    \hline
          Revue Formelle de Démarrage & RFD & -\\
    \hline
             Procès-Verval de Démarrage & PVD & d\\
    \hline
          Revue Formelle de Fin de Phase de Conception Préliminaire & RFFPCP & -\\
    \hline
             Procès-Verbal de Fin de Phase de Conception Préliminaire & PVFPCP & d\\
    \hline
          Revue Formelle de Fin de Phase d'Intégration & RFFPI & -\\
    \hline
             Procès-Verbal de Fin de Phase d'Intégration & PVFPI & d\\
    \hline
          Revue Formelle de Fin de Phase de Spécification & RFFPS & -\\
    \hline
             Procès-Verbal de Dossier de Spécifications Externes & PVDSE & d\\
    \hline
             Procès-Verbal de Dossier de Spécifications Internes & PVDSI & d\\
    \hline
             Procès-Verbal de Fin de Phase de Spécifications & PVFPS & d\\
    \hline
             Procès-Verbal de Plan de Tests de Validation & PVPTV & d\\
    \hline
          Revue Formelle de Recettes & RFR & -\\
    \hline
             Procès-Verbal de Recettes & PVR & d\\
    \hline
    Gestion des Configurations & GC & -\\
    \hline
       Etat de Configuration & EC & d-c\\
    \hline
       Fiche de Remise de Documents & FRD & d-c\\
    \hline
       Fiche de Remise de Matériels & FRM & d-c\\
    \hline
       Fiche Récapitulative de Référentiel & FRR & s\\
    \hline
       Plan de Gestion des Configurations & PGC & v\\
    \hline
    Portefeuille des Risques & PR & -\\
    \hline
       Fiche De Risque & FDR & n\\
    \hline
       Fiche D'Opportunité & FDO & n\\
    \hline
    Plan Qualité & PQ & v\\
 \multicolumn{3}{|c|}{\textbf{\bsc{Référentiel Spécification}}}\\
    \hline
    LotX & LotX & -\\
    \hline
       Cahier De Recettes & CDR & n-v\\
    \hline
       Carnet de Produit & CP & n-v\\
    \hline
       Dossier de Spécifications Externes & DSE & n-v\\
    \hline
       Cahier De Spécifications Internes & DSI & n-v\\
    \hline
       Plan de Tests de Validation & PTV & n-v\\
 \multicolumn{3}{|c|}{\textbf{\bsc{Référentiel Développement}}}\\
    \hline
    Dossier d'Audit de Code & DAC & d\\
    \hline
    Implémentation & I & -\\
    \hline
    LotX & LotX & -\\
    \hline
       Dossier de Conception & DC & v\\
    \hline
       Dossier de Tests & DT & -\\
    \hline
          Dossier de Tests d'Intégration & DTI & n-v \\
    \hline
          Dossier de Tests Unitaires & DTU & n-v \\
    \hline
          Plan de Tests d'Intégration & PTI & n-v \\
    \hline
          Plan de Tests d'Intégration & PTU & n-v \\
    \hline
          Rapport Jenkins & RJ & n-v \\
    \hline
       Guide Utilisateur & GU & v\\
    \hline
       Procédure d'Installation & PI & v\\
 \multicolumn{3}{|c|}{\textbf{\bsc{Référentiel Livraison}}}\\
    \hline
    Livraisons Complémentaires & LC & d\\
    \hline
    Lots & Lots & -\\
    \hline
       Lot numéro X & LotX & d\\
    \hline
    Revue & Revues & -\\
    \hline
       Revue numéro X & RevuesX & -\\
    \hline
          Support Présentation & SP & n\\
  \end{tabularx}
  \caption{Formalisme des différents Types}
  \label{Formalisme Types}  
\end{table}
\smallskip

\section{Suffixes}

Lors du nommage nous utilisons des suffixes adaptés en fonction du type de document. Un document peut avoir plusieurs suffixes, dans ce cas ils sont séparés par un "-". Les suffixes possibles sont les suivants : 

	\begin{table}[H]
		\centering
		\begin{tabularx}{10cm}{|X|c|c|}
		\hline
		\rowcolor[gray]{0.85} Suffixe & Abréviation & Format\\
		\hline
		date & d & dAA-MM-JJ\\
		\hline
		version & v & vX.YY\\
		\hline
		numéro & n & nXXX\\
		\hline
		semaine & s & sXX\\
		\hline
		rôle & r & rRole\\
		\hline
		personne & p & pNom\\
		\hline
		commentaire & c & cCommentaire\\
		\hline
		\end{tabularx}
	\caption{Abréviations associées à chaque suffixe}
	\label{Suffixes}
	\end{table}
	


\subsection{Suffixe Date}
\label{suffixe_date}

Le suffixe de date suit le format \textbf{AA-MM-JJ}. AA pour l'année, MM pour le mois et JJ pour le jour. Ce format permet de classer les documents par ordre chronologique.\\

Exemple : $d16-01-27$ 

\subsection{Suffixe Version}
\label{suffixe_version}

Le suffixe version suit le format \textbf{vX.YY}. X correspond au numéro de version, YY au numéro de révision. Le numéro de version représente le numéro de semestre, en considérant que le semestre 4.2 est le numéro de semestre 1 du PIC et le semestre 5.1 est le numéro de semestre 2 du PIC. Le numéro de révision démarre à 00 et est modifié de la façon suivante : le chiffre des dizaines est incrémenté à chaque modification majeure et le chiffre des unités est incrémenté à chaque révision. Quand le numéro de version est incrémenté, celui des révision repasse à 00.\\

Exemple : $v1.00$

\subsection{Suffixe Numéro}
\label{suffixe_numero}

Le suffixe numéro suit le format \textbf{nXXX}. XXX est un nombre entier compris entre 001 et 999. Il commence à la valeur 001 et est incrémenté à chaque nouveau document.\\

Exemple : $n042$

\subsection{Suffixe Semaine}
\label{suffixe_semaine}

Le suffixe semaine suit le format \textbf{sXX}. XX correspond au numéro de la semaine en cours, la première semaine du PIC étant 01. Seules les semaines comportant au moins un jour ouvré sont numérotées.\\

Exemple : $s09$

\subsection{Suffixe Rôle}
\label{suffixe_role}

Le suffixe rôle suit le format \textbf{rRole} où Role correspond au rôle concerné par le document. Les valeurs possibles pour Role sont : 
\begin{itemize}
\item CP : Chef Projet;
\item CPA : Chef Projet Adjoint;
\item RGC : Responsable Gestion des Configurations;
\item RQ : Responsable Qualité;
\item RQA : Responsable Qualité Adjoint;
\item RR : Responsable Réseau;
\item RD : Responsable Développement;
\item D : Développement;
\item PR : Pilote de Risque.\\
\end{itemize}

Exemple : $rCP$

\subsection{Suffixe Personne}
\label{suffixe_personne}

Le suffixe personne suit le format \textbf{pNom} où Nom correspond au nom de famille de la personne concernée par le document. Dans le cas où plusieurs personnes sont concernées le suffixe aura le format \textbf{pNom1Nom2}.\\

Exemple : $pBaron$

\subsection{Suffixe Commentaire}
\label{suffixe_commentaire}

Le suffixe commentaire suit le format \textbf{cX} où X est le commentaire correspondant au fichier. Par exemple le sujet de la formation pour une Fiche de Formation.\\

Exemple : $cLaTeX$

