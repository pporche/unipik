\documentclass [a4paper] {article}
\usepackage[utf8]{inputenc}
\usepackage[francais]{babel}
\usepackage[top=2cm, bottom=4cm, left=2cm, right=2cm]{geometry} 
\usepackage{fancyhdr}
\usepackage{graphicx}
\usepackage{color, colortbl}
\usepackage{longtable}
\usepackage{../../../../ressources/Unipik/vocabulaire/vocabulaireUnipik}
\pagestyle{fancy}
\definecolor{Gray}{gray}{0.8}




%--- En-t�te et pied de page ---%
\renewcommand{\footrulewidth}{0,01cm}

\begin{document}
\rhead{}
\chead{\huge{Compte-rendu de réunion de Tutorat Qualité}}					%titre
\hfill   
\hfill 	9:03-9:56 				% Heure de d�but, heure de fin.


\lfoot{Version : 1.00} 			% version

%--- Fin en-t�te et pied de page ---%
\section*{Historique des révisions}
\begin{center}
			\begin{tabular}{| c | c | c | c | p{4cm} |}
				\hline
				\rowcolor{Gray}
				Version & Date & Auteur(s) & Modification(s) & Partie(s) modifiée(s)		 \\
				\hline
				1.00 & 26/01/2016 & \Pierre & Création & Toutes \\
		\hline		
			\end{tabular}
		\end{center}

\section*{Signatures}

		\begin{center}
			\begin{tabular}{| c | c | c | c | p{4cm} |}
				\hline
				\rowcolor{Gray}
				Rôle & Fonction & Nom & Date & Visa		 \\
				\hline
				Vérificateur & \RQA & \Kafui & 26/01/2016 & pgpic \\[30pt]
				\hline
				Validateur & \CP & \Sergi & 29/01/2016 & pgpic \\[30pt]	
				\hline
			\end{tabular}
		\end{center}

%--- Réunion --%

\section{Réunion}
Le \PQ{} n'est pas à faire approuver par le client seulement 3 parties : le validateur, le vérificateur et Pascal Meslier.
\\
La DGQ1 concerne la contractualisation, la DGQ2 concerne la qualité au sein des pics et la DGQ3 décrit le processus "Manager la qualité".
\\
Concernant les questionnaires de satisfaction, il y en aura 4 pour les clients (aux revues) qui aideront à mettre à jours les indicateurs. Il y en aura aussi 2 pour les élèves (à r2 et r4), 1 pour les services support (à r4) et 1 pour les tuteurs (à r4).
\\
Les service support sont : 
\begin{itemize}
\item DRV pour les engagements de confidentialité
\item DSI
\item Le Secrétariat du département ASI
\end{itemize}

Il faudra écrire un PV de lancement qui décrira les moyens mis en oeuvre pour répondre à la problématique ainsi que la fréquence des réunions. Il s'agira d'un engagement de l'équipe sur la réalisation du projet et sur la mise en place de la qualité.
Il contient le nom d'un contact et de préférence d'un autre au cas où le premier serait indisponible. Il définit aussi les moyens de communication et la fréquence des contacts.
Le \CP{} s'engage sur ce document à réaliser le sujet en respectant la politique qualité.
\\
\nomTuteurQualite{} nous conseille d'initier les documents de spécification et de mettre en place un planning de principe.
\\
Les réunions tutorat qualité se feront à raison d'une heure toutes les semaines au départ et si nous avançons bien, écarter les réunions. De plus, si l'équipe est en période de stress intense dû à la surcharge de travail, il est possible d'annuler une réunion de manière ponctuelle. Ces réunions se tiendront le mardi à 9h30, il n'y aura pas d'ordre du jour défini à l'avance, le tuteur présentera une notion pour la première partie de la réunion et répondra à nos questions pour la deuxième.
\\
Il ne faut pas suivre les risques sur PGPIC car ils sont définis avec une certaine matrice de risques mais il faut utiliser une matrice plus simple. De plus, ce logiciel n'est pas à jour pour la révision 2015 de la norme ISO 9001 et ne permet donc pas la gestion des opportunités.
\\
La qualité n'est pas un domaine figé. Il faut bien réfléchir pour créer le portefeuille de risque mais en gardant à l'esprit que certain risques disparaîtront car ils ne seront pas pertinents, d'autres apparaîtront et d'autres encore prendront de l'importance. Toute l'équipe devrait réfléchir au portefeuille de risques et opportunités et de préférence, chaque personne devrait être pilote d'au moins un risque. On parle de \textbf{détection} d'opportunité.
\\
La méthode AGILE est à adapter : si on suit les principes de cette méthode, on contredit la norme ISO 9001. Concernant la révision 2015 de cette dernière, elle impose une gestion des remarques et réclamation. Il y a peu de modifications à apporter à nos méthodes de travail car l'unité P3 dépassait déjà les exigences de la révision 2008.
\\
L'équipe n'aura pas d'achats à effectuer, \nomTuteurQualite{} nous dit que c'est une bonne nouvelle car la procédure est complexe.

~

CYCLE CORRECTIF : \\
Afin de gérer la traçabilité des évolutions, les non-conformités ou la détection de problème, il faudra rédiger des documents (FFT, FOC,..). Ces documents seront traités lors de réunions appelées \CTFT{} (\CTFTCourt) durant lesquelles, pour chaque point (évolution, NC, problème), l'équipe fera une analyse des causes (n pourquoi conseillé). Il est conseillé de tenir ces réunions avec un maximum de membres du groupe afin de trouver plus facilement les causes racines. Suite à ces constats, il faut mettre en place des actions correctives (élimination des causes racines) ou curatives (élimination du soucis de surface). Les actions correctives sont à privilégier car si on élimine la cause racine, le problème ne réapparaît plus. Pour mesurer cela, il est possible de créer un indicateur sur les réapparitions. Les actions préventives, quant à elles, ne concernent que les risques.

~

Il faut organiser du Team Building afin, notamment, d'éviter le risque "membre démissionnaire"

~

Les documents suivants sont des documents internes :
\begin{itemize}
\item Plan de formation ,
\item DTU, 
\item DTI, 
\item DCD, 
\item DCP.
\end{itemize}

~

Le sujet de la prochaine réunion sera : "Les Indicateurs".



%--- fin de réunion ---%
\end{document}







