\section*{Informations générales}
 
\begin{table}[h]
\centering
	\begin{tabularx}{16.8cm}{|X|X|}
	\hline
	Numéro de l'opportunité & Type d'opportunité \\
	\hline
	004 & Bonne communication client \\
	\hline
	\end{tabularx}
\end{table}

\begin{table}[h]
\centering
	\begin{tabularx}{16.8cm}{|X|X|X|}
	\hline
	Date & Visa du \RQ & Visa du \CP \\
	\hline
	 27/01/16 & pgpic & pgpic \\
	\hline
	\end{tabularx}
\end{table}

\begin{table}[h]
\centering
	\begin{tabularx}{16.8cm}{|X|X|X|X|}
	\hline
	Pilote & Activité WBS & Compte WBS & Phase d'apparition \\
	\hline
	 \Florian & Suivre les Risques et Opportunités & 1.2.3.2 & Tout au long du projet.\\
	\hline
	\end{tabularx}
\end{table}

\section*{Description du risque}

\subsection*{Résumé}

	La bonne planification concerne principalement l’appréciation des délais des tâches
et peut être expliquée par une bonne implication du \CP. Elle permet de ne pas prendre de retard de
livraison des lots.
	
\subsection*{Analyse des causes}
	voir figure.

\subsection*{Criticité}

\begin{table}[h]
\centering
	\begin{tabularx}{12.8cm}{|>{
	%\columncolor{gray!40}
	}X|X|}
	\hline
	Bénéfice & 4\\
	\hline
	Probabilité & 3\\
	\hline
	Criticité & Important\\
	\hline
	\end{tabularx}
\end{table}
\newpage

\section*{Actions}
\subsection*{Actions proactives}

%\begin{table}[H]
\centering
	\begin{longtable}{|p{7cm}|p{7cm}|}
	\hline
	Cause & Actions proactives \\
	\hline
	 Choix des outils de plannification pertinent & \begin{itemize}
	 	\item 
	 \end{itemize} \\
	\hline
        Bonne prise en main des outils & \begin{itemize}
	 	\item Former les membres de l'équipe \PICCourt{}.
	 \end{itemize} \\
	\hline
        Bonne vision globale & \begin{itemize}
	 	\item Avoir un \CP{} impliqué et compétent.
	 \end{itemize} \\
	\hline
	\end{longtable}
%\end{table}

\section*{Décision de clôture}
Par le \CP{} et le pilote du risque.
\begin{table}[h]
\centering
	\begin{tabularx}{16.8cm}{|X|X|}
	\hline
	Date de clôture & Raison de la clôture \\
	\hline
	  & \\
	\hline
	\end{tabularx}
\end{table}

\section*{Historique des modifications}
\begin{table}[h]
\centering
	\begin{tabularx}{16.8cm}{|X|X|}
	\hline
	Date & Modification \\%\rowcolor{gray!40} 
	\hline
	  & \\
	\hline
	\end{tabularx}
\end{table}
\newpage


\begin{figure}
	\centering
	\includegraphics[scale=0.5, angle=90]{images/AnalyseOpportunite_nPourquoi_FDO005}
\end{figure}
