\documentclass [a4paper] {article}
\usepackage[utf8]{inputenc}
\usepackage[francais]{babel}
\usepackage[top=2cm, bottom=4cm, left=2cm, right=2cm]{geometry} 
\usepackage{fancyhdr}
\usepackage{graphicx}
\usepackage{color, colortbl}
\usepackage{longtable}
\usepackage{vocabulaireUnipik}
\pagestyle{fancy}
\definecolor{Gray}{gray}{0.8}



%--- En-t�te et pied de page ---%
\renewcommand{\footrulewidth}{0,01cm}

\begin{document}
\rhead{}
\chead{\huge{Compte-rendu de réunion de Tutorat Qualité}}					%titre
\hfill   
\hfill 	9:03-9:56 				% Heure de d�but, heure de fin.


\lfoot{Version : 1.00} 			% version

%--- Fin en-t�te et pied de page ---%
\section*{Historique des révisions}
\begin{center}
			\begin{tabular}{| c | c | c | c | p{4cm} |}
				\hline
				\rowcolor{Gray}
				Version & Date & Auteur(s) & Modification(s) & Partie(s) modifiée(s)		 \\
				\hline
				1.00 & 02/02/2016 & \Pierre & Création & Toutes \\
		\hline		
			\end{tabular}
		\end{center}

\section*{Signatures}

		\begin{center}
			\begin{tabular}{| c | c | c | c | p{4cm} |}
				\hline
				\rowcolor{Gray}
				Rôle & Fonction & Nom & Date & Visa		 \\
				\hline
				Vérificateur & \RQA & \Kafui & 04/02/2016 & pgpic \\[30pt]
				\hline
				Validateur & \CP & \Sergi & 04/02/2016 & pgpic \\[30pt]	
				\hline
			\end{tabular}
		\end{center}

%--- Réunion --%

\section{Informations diverses}
la diffusion c'est l'envoie aux destinataires. il ne peut etre diffusé qu'après approbation. une fois seulement que le doc est approuvé, on peut remplir les champs diffusion
DSE DSI PTV PQ PGD CDR <- On peut globaliser les noms genre "unipik" au lieu de "machin machin et machin"

Diffusion contrôlée / non contrôlée  :
Si le client ne veut qu'une version pour voir, c'est non controlée, si on met en place une contrôlée, il faut envoyer le document à chaque version. DSE DSI PQ : forcément mettre l'équipe.

Attention aux versions des documents! (Audit) -> ne pas conserver les anciennes versions en tant que doc de référence.



FFT/Cycle correctif : quand on rencontre un problème, stocker la démarche
Identification : FFT, décrire la MaJ ou problème
Analyse des causes : déterminer des actions
Ordre de correction : Synthétise les actions (curatif/correctif) -> correctif= a la source + tracer la vérif. des correction
Cloture de l'OC puis du FT
Analyse à froid : vérif. que tout FT corrigé par du correctif ne réapparaît pas.
S'il réapparait, l'OC inclut une action différente. !
CTFT : minimum 3 pers (CP, RQ, RGC)

kanban : pas mal mais attention avec les post it : Trelo ca peut etre pas mal


CNIL : se renseigner auprès du juriste pour savoir si Sergi a le droit de faire la demande.  ->  lire la iso 27005 

\end{document}















