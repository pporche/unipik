% version 1.00	Auteur Kafui Atanley date 26/09/2016

\documentclass [a4paper] {article}
\usepackage[utf8]{inputenc}
\usepackage[francais]{babel}
\usepackage[top=2cm, bottom=4cm, left=2cm, right=2cm]{geometry} 
\usepackage{fancyhdr}
\usepackage{graphicx}
\usepackage{color, colortbl}
\usepackage{longtable}
\usepackage{vocabulaireUnipik}
\usepackage{hyperref}
\DeclareUnicodeCharacter{00A0}{ }
\pagestyle{fancy}
\definecolor{Gray}{gray}{0.8}



%--- En-t�te et pied de page ---%

\renewcommand{\footrulewidth}{0,01cm}
\rhead{}
\chead{\huge{Compte-rendu de réunion de Tutorat Pédagogique}}					%titre
\begin{document}

03/10/2016			 				%Date 
\hfill   
\hfill 	 13:35 - 14:30				%Heure de d�but, heure de fin.


\lfoot{Version : 1.00} 			% version
%--- Fin en-t�te et pied de page ---%
\section*{Historique des révisions}
\begin{center}
			\begin{tabular}{| p{2.5cm} | p{3cm} | p{3cm} | p{3cm} | p{3.5cm} |}
				\hline
				\rowcolor{Gray}
				Version & Date & Auteur(s) & Modification(s) & Partie(s) modifiée(s)		 \\
				\hline
				1.00 & 04/10/2016 & \Kafui & Création & Toutes \\
				\hline			
			\end{tabular}
		\end{center}

\section*{Signatures}

		\begin{center}
			\begin{tabular}{| p{2.5cm} | p{4cm} | p{3cm} | p{3cm} | p{2.5cm} |}
				\hline
				\rowcolor{Gray}
				Rôle & Fonction & Nom & Date & Visa		 \\
				\hline
				Vérificateur & \RGC & \Melissa & --- & --- \\[30pt]
				\hline
				Validateur & \CP & \Pierre &  --- & --- \\[30pt]	
				\hline
			\end{tabular}
		\end{center}

%--- Réunion --%
\section{Résumé de la semaine passée}
La semaine précédente s'est orientée autour de 9 axes :  
\begin{itemize}
\item Base de données : Les corrections suggérées lors de la réunion du 19/09/2016 ont été apportées .
\item Test unitaires : ceci sont en cours de développement . Ils couvrent actuellement douze pourcents du code.
\item SonarQube (plate-forme open-source destinée à la revue de code) a été mis en place. Les indicateurs sont globalement bons. 
\item Le formulaire de demande d'intervention est en cours d'amélioration du point de vue Backend/Frontend.
\item L'autocomplétion de formulaire est opérationnelle pour ce qui est des adresses.
\item Une réunion avec le client aura lieu ce jeudi 6 octobre à 10h dans leurs locaux. ...,... et .. seront présents.
\item Nous avons eu des retours de Quantic Télécom au vu de l'hébergement du serveur. Les aspects techniques de notre application ont été fournis à Quantic Télécom. \ Le \CP{} s'occupera dans un second temps de discuter de l'aspect financier. 
\item Le \PQ{} et le \PGC{} ont été approuvées.
\item Une \CTFT{} a eu lieu afin de résoudre les \FT{} présents.
\end{itemize} 

\section{Réunion cliente}
Il nous est recommandé de bien préparer cette réunion (ordre du jour et horaires à fixer précisémment).Il faut également que nous envoyons des captures d'écran au client afin que celui-ci puisse avoir une idée du 'look and feel' de l'actuel application. Ceci nous permettra d'avoir un premier feedback nécessaire à la validation opértationnelle de l'application.

\section{Interventions}
\nomTuteurPedago{} nous recommande de donner un maximum de pouvoir d'expression à l'utilisateur. Ceci devrait se répercuter à haut niveau en laissant place à des filtres sur chaque caractéristique de l'intervention. 

\section{Semaine suivante}
Le cahier de recette sera rédigée. L'attribution de bénévole à une intervention sera implémentée. Le design du Frontend sera travaillée. Les tests unitaires/d'intégration vont continuer à être implémentées.


%--- fin de réunion ---%
\newpage



\end{document}





















