% version 1.00	Auteur Kafui Atanley date 10/10/2016

\documentclass [a4paper] {article}
\usepackage[utf8]{inputenc}
\usepackage[francais]{babel}
\usepackage[top=2cm, bottom=4cm, left=2cm, right=2cm]{geometry} 
\usepackage{fancyhdr}
\usepackage{graphicx}
\usepackage{color, colortbl}
\usepackage{longtable}
\usepackage{vocabulaireUnipik}
\usepackage{hyperref}
\DeclareUnicodeCharacter{00A0}{ }
\pagestyle{fancy}
\definecolor{Gray}{gray}{0.8}



%--- En-t�te et pied de page ---%

\renewcommand{\footrulewidth}{0,01cm}
\rhead{}
\chead{\huge{Compte-rendu de réunion de Tutorat Pédagogique}}					%titre
\begin{document}

03/10/2016			 				%Date 
\hfill   
\hfill 	 13:29 - 14:30				%Heure de d�but, heure de fin.


\lfoot{Version : 1.00} 			% version
%--- Fin en-t�te et pied de page ---%
\section*{Historique des révisions}
\begin{center}
			\begin{tabular}{| p{2.5cm} | p{3cm} | p{3cm} | p{3cm} | p{3.5cm} |}
				\hline
				\rowcolor{Gray}
				Version & Date & Auteur(s) & Modification(s) & Partie(s) modifiée(s)		 \\
				\hline
				1.00 & 17/10/2016 & \Kafui & Création & Toutes \\
				\hline			
			\end{tabular}
		\end{center}

\section*{Signatures}

		\begin{center}
			\begin{tabular}{| p{2.5cm} | p{4cm} | p{3cm} | p{3cm} | p{2.5cm} |}
				\hline
				\rowcolor{Gray}
				Rôle & Fonction & Nom & Date & Visa		 \\
				\hline
				Vérificateur & \RGC & \Melissa & 17/10/2016 & email \\[30pt]
				\hline
				Validateur & \CP & \Pierre &  17/10/2016 & email \\[30pt]	
				\hline
			\end{tabular}
		\end{center}

%--- Réunion --%
\section{Résumé de la semaine passée}
La semaine précédente s'est orientée autour de 5 axes :  
\begin{itemize}
	\item Un agenda a été mis sur la page d'accueil du site pour résumer les prochaines interventions du membre lorsqu'il se connecte;
	\item Les redirections dans les menus ont toutes été jointes avec leurs pages correspondantes;
	\item L'attribution d'intervention est opérationnelle et s'effectue selon un code couleur;
	\item La modification d'un établissement est opérationnelle;
	\item Les test unitaires ont été avancés (20 \% de couverture de code).
\end{itemize} 

\section{Résumé de la réunion avec le client}
Une réunion a eu lieu le jeudi 6 octobre avec le client. Durant cette réunion le client a pu avoir un aperçu du lot 2 fini. Les clientes ont demandées 4 modifications : 
\begin{itemize}
	\item	La liste des interventions doit pouvoir être imprimée;
	\item	Lorsque l'on affiche la liste des interventions, des filtres sur le niveau scolaire (la classe) et sur le thème doivent être présents;
	\item	Sur l'agenda d'un bénévole, lorsqu'une intervention est affichée, le nom de l'école doit également apparaître;
	\item	Lorsque l'on consulte une intervention, un lien doit permettre de remonter jusqu'à la demande correspondante.
\end{itemize} 


\section{Hébergement}
Nous n'avons pas de retours de Thibaud Dauce(Quantic Telecom) pour l'hébergement, nous utiliserons le serveur mis à disposition par le département en attendant. Le \CP{} va relancer Quantic Telecom sous peu.

\section{Séance de recette provisoire}
La rédaction du cahier de recette est bientôt terminée. \nomTuteurPedago{} nous recommande d'utiliser un logiciel de simulation de clics pour la partie Frontend afin d'éviter de perdre du temps durant la séance. Nous devons ensuite montrer que ce qui a été fait en couche haute correspond avec ce qui est persisté en couche basse. Le but est de montrer que nous sommes capable de générer une multitute de cas d'utilisation.

\section{Envoi d' email}
\nomTuteurPedago{} nous dit qu'il faudrait mieux cibler les établissements afin de leurs envoyer un email ciblé et un formulaire pré-rempli.
%--- fin de réunion ---%
\newpage



\end{document}





















