\section{Exceptions aux règles de nommage}\label{Regle specifique}

\subsection{Référentiels Ressources}

Le référentiel \textbf{Ressources} ne sera pas soumis aux règles de nommage de \nomEquipe.

\subsection{Référentiel Livraison}

Les répertoires lotN (N le numéro de lot) ne sont \textbf{en aucun cas} des répertoires de travail.
Les fichiers contenus dans ces répertoires sont des fichiers compilés (notamment des \verb+pdf+).
Exemple, pour le lot 1, le répertoire lot1/DateLivraison (format DAA-MM-JJ) pourrait par exemple contenir:
	\begin{itemize}
		\item le \DSE{}: \nomEquipe\_S\_DSE\_V1.00.pdf;
		\item le \DSI{}: \nomEquipe\_S\_DSI\_V1.00.pdf;
	\end{itemize}
Remarque: Les documents contenus dans les répertoires de livraison  lotN respecteront
obligatoirement la convention de nommage associée à leur \textbf{référentiel d'origine}.

\section{Répertoires pdf, images, et sources}

\subsection{pdf}

\`{A} la racine du répertoire de chaque document se trouve un répertoire \textbf{pdf}. C'est
le répertoire dans lequel est placé le \verb+pdf+ généré par la compilation \LaTeX{}. Le \verb+pdf+ n'est pas stocké sur \git{}.

\subsection{images}

\`{A} la racine du répertoire des documents nécessitant l'inclusion d'images, de diagrammes ou
d'autres figures se trouve un répertoire \textbf{images}. Il est ajouté au \git{} par le
rédacteur du document si besoin est. Dans ce répertoire sont stockés les fichiers projet
(\verb+.dia+, \verb+.xcf+, \verb+.jpg+, etc\dots). Les \verb+.pdf+ de ces fichiers sont générés
à la compilation par le \verb+makefile+.
Les fichiers du répertoire \verb+images+ ne sont pas concernés par les règles de nommage.

\subsection{sources}

\`{A} la racine du répertoire des documents nécessitant l'inclusion de sous-fichiers \verb+.tex+
(correspondant à des chapitres, des sections, etc\dots) se trouve un répertoire \textbf{sources} pour
les documents longs nécessitant d'être subdivisés. 
Ces sous-fichiers \verb+.tex+ seront placés dans ce répertoire et ils seront inclus via
le \verb+.tex+ général. Ces sous-fichiers ne sont pas concernés par les règles de nommage.

\section{Comptes-rendus} \label{comptes-rendus}

Les Comptes-Rendus seront écrits en \LaTeX{} et archivés au format pdf dans le serveur \textbf{SFTP}. Ils sont approuvés par courriel ou imprimés et signés.

