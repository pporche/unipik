% version 1.00, date 05/11/16, auteur Kafui Atanley
%L’outil devra faciliter la géolocalisation des établissements, bénévoles et interventions.
%Une carte devra pouvoir être accessible quand l’une des entité possède une adresse et il
%devra être possible de filtrer ces entités en fonction de la distance à une ville ou un point
%géographique particulier à l’aide de liste.
\subsection{Fonctionnalité 9}

La fonctionnalité 9 concerne la géolocalisation des entités possédant une adresse.
Ces entités sont : 
\begin{itemize}
\item les établissements,
\item les interventions quelque soit leur type,
\item les ventes.
\end{itemize}
Une carte devra être affichée dans les fiches détaillées de chaque entité.
La maquette \ref{fonctionnalite9Geolocalisation} présente la fiche descriptive d'une entité. Les fiches descriptives
des entités interventions et ventes devront comporter une carte de manière similaire.
Il devra être possible de filtrer ces entités selon une ville et selon la distance par rapport à une ville.
La maquette \ref{fonctionnalite9Filtre} présente un modèle des filtres attendus sur les entités concernées.

\begin{figure}[H]
	\centering
	\includegraphics[scale=0.4]{images/maquettes/fonctionnalite9Geolocalisation.png}
	 \caption{Maquette~: Géolocalisation dans les fiches descriptives}
	 \label{fonctionnalite9Geolocalisation}
\end{figure}

\begin{figure}[H]
	\centering
	\includegraphics[scale=0.4]{images/maquettes/fonctionnalite9Filtre.png}
	 \caption{Maquette~: Modèle de filtres attendus}
	 \label{fonctionnalite9Filtre}
\end{figure}
