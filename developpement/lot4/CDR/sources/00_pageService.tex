% version 1.00, date 07/12/16, auteur Julie Pain, Pierre Porche
\begin{pagesService}
	\begin{historique}
		% nouvelles versions à rajouter AU-DESSUS en recopiant les lignes suivantes et en les modifiant :
		\unHistorique{1.00}{07/12/2016}{\Pierre \newline \Julie}{Création}{Toutes}

	\end{historique}

%%Signataires
        \begin{signatures}
	   \uneSignature{Vérificateur}{\RGC}{\Kafui}{08/12/2016}{email}
       \uneSignature{Validateur}{\CP{}}{\Pierre}{08/12/2016}{email}
	   \uneSignature{Approbateur}{Le client}{\nomClient}{14/12/2016}{email}
        \end{signatures}
	
	
\section*{Documents de Références}
%\begin{documentsReference}
		\begin{listeDeReferences}
			\uneReference{NF EN ISO 9001}{Octobre 2015}
			\uneReference{\MQ{}}{ASI-MQ-MQASI}
			\uneReference{\DGQ{} du Processus "\DGQDEUX{}"}{ASI-DGQ-DGQ2}
			\uneReference{NF EN ISO 9000}{Octobre 2015}
		\end{listeDeReferences}
%	\end{documentsReference}	
	
\begin{terminologie}
		La terminologie (définitions et abréviations) utilisée dans le présent document est centralisée dans le \MQ{} \ASICourt{} (\emph{cf. ASI-MQ-MQASI}) de l'Unité P3.

		\begin{listeDAbreviations}
			\uneAbreviation{\JTUCourt{}}{\JTU}
			\uneAbreviation{\PICCourt}{\PIC}
			\uneAbreviation{\PTUCourt{}}{\PTU}
			
		\end{listeDAbreviations}
	\end{terminologie}
	
	
	
\end{pagesService}
