% version 1.00	Date 01/03/2016	Auteur Pierre Porche

\subsection*{Introduction}

Le \RQ{} est le garant de l’application de la politique qualité au sein du \PICCourt. Il peut être épaulé dans cette tâche par d’autres membres du \PICCourt.

\subsection*{Tâches liées à sa fonction}

Une passation devra être mise en place entre les \RQs{} du premier et second semestre. Cette passation devra si possible être formalisée sous la forme d’une formation.\\
Tout au long du projet, le \RQ{} devra veiller à la bonne adéquation entre les tâches liées à la réalisation des livrables et le référentiel qualité. Pour assurer le bon déroulement de cette veille qualité, il devra réaliser les tâches suivantes :

\subsubsection*{Tâches liées au \PQCourt}
\begin{itemize}

	\item Rédiger et organiser le suivi du \PQ (\PQCourt) en respectant les exigences du Référentiel Qualité et en particulier de la \DGQDEUXCourt;
	\item Assurer la bonne diffusion (c’est à dire, l’envoi après approbation) du \PQCourt{} aux membres de l’équipe \PICCourt;
	\item Vérifier le \PQCourt{} après l’exécution d’actions correctives (cette tâche peut être déléguée à un autre membre du \PICCourt par dérogation personnelle ou de la part du \CP);


	\item Valider l’ensemble des procédures qualité rédigées au sein du \PICCourt;
	\item Fournir un accompagnement aux équipes de développement dans la démarche qualité;
	\item Réaliser des activités régulières de contrôle de l’ensemble du système qualité;
	\item Sensibiliser les membres de l’équipe \PICCourt{} à la norme \ISOCourt 9001:2015.
\end{itemize}

\subsubsection*{Tâches liées au \PGCCourt}

Cette partie peut être déléguée dès le début du \PICCourt{} à un autre membre du \PICCourt{}, possédant la compétence exigée, qui prendra alors la responsabilité de la gestion des configurations.

\begin{itemize}
	\item Rédiger et organiser le suivi du \PGCCourt{} en respectant les exigences du Référentiel Qualité et en particulier de la \DGQDEUXCourt;


	\item Vérifier le \PGCCourt{} après l’exécution d’actions correctives (cette tâche peut être déléguée à un autre membre du \PICCourt{} par dérogation personnelle ou de la part du \CP);

	\item Garantir l’application du \PGCCourt;
	\item S’assurer du bon déroulement de la gestion des modifications des différents documents.
\end{itemize}

\subsubsection*{Tâches liées à la gestion du référentiel}

\begin{itemize}

	\item Chaque semestre, les \RQs{} doivent se réunir et fournir au pilote de processus 2, une liste de cinq questions par responsable visant à évaluer la maîtrise de ce référentiel ;

	\item Chaque semestre, les \RQs{} doivent se réunir pour se répartir et corriger les demandes d’amélioration présentes sur l’outil de suivi des référentiels ;
	\item Chaque semestre, à chaque incohérence et problème soulevé, les \RQs{} doivent ajouter des demandes d’amélioration à l’aide de l’outil de suivi des référentiels.
\end{itemize}
