% version 1.01 Date 14/03/2016	Auteur Mathieu Medici

Ce chapitre présente comment les documents, versionnés ou non, seront archivés dans le cadre de notre projet. L’archivage se divise en deux parties : l’archivage informatique et l’archivage papier.

\section{Archivage informatique}
L'archivage informatique se divise en trois parties : le stockage du projet courant sur le serveur \git{}, l'archivage des pdf sur PGPic et l'archivage des documents approuvés.

\subsection{Stockage du projet sur \git{}}
Tous les documents courants du \PICCourt sont stockés de façon numérique sur le dépôt du serveur \git{} mis à disposition par le département \ASI{}. On y trouve à la fois les documents relatifs à la démarche qualité, mais aussi le code, les mails, ou tout autre document créé pour le projet. Puisqu'il s'agit de l'espace de travail, on ne doit pas utiliser le serveur \git{} pour accéder aux documents de références.

\subsection{Archivage sur le serveur PGPic}

Chaque semaine, le \RQ{} procède à un archivage des documents au format \verb+pdf+. Cet archivage est stocké sur notre compte du serveur PGPic et donc les documents sont accessibles directement depuis \verb+pgpic.insa-rouen.fr+ dans la section \verb+Documents+.

\subsection{Archivage des documents approuvés}

Les documents approuvés (PQ, PGC, DSE, PTV, ou autres) sont archivés sur le site \verb+monprojet.insa-rouen.fr+, dans l'onglet \verb+Documents+, dans le dossier \verb+archivage documents approuvés+. Toutes les versions y seront conservées. Les documents doivent être identiques à ceux disponibles sur \verb+pgpic.insa-rouen.fr+ dans la section \verb+Documents+.

\section{Archivage papier}
Certains documents nécessitent un archivage papier spécifique car ils comportent pour la plupart des signatures manuscrites. Ils sont obligatoirement rangés dans le classeur correspondant se trouvant dans l’armoire de la salle du projet. Ces documents sont :
\begin{itemize}
\item Engagements de confidentialité;
\item Fiches de compétence;
\item Fiches de formations;
\item Rapports d’audits;
\item Procès verbaux.
\end{itemize}
Ces documents ne doivent pas sortir de la salle.