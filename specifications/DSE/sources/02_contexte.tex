	Les principaux objectifs de ce document seront explicités dans cette partie. Nous aborderons également les différentes hypothèses liées au contexte sur lesquelles se basent la réalisation du projet.


\section{Objectif du document}
	L'objectif de ce document est de rendre compte de l'ensemble des fonctionnalités attendues par le client ainsi que les contraintes de réalisation. \\
	
	Le \PICCourt \nomClient{} a pour but de créer un logiciel facilitant la gestion des interventions externes telles que les plaidoyers, les frimousses ou encore les VAE.
	
	
\section{Hypothèse}
	Les diffèrentes hypothèses liées au contexte sont :
	\begin{itemize}
		\item Le groupe \nomEquipe{} est constitué de 9 personnes.
		\item La réalisation du projet suivra une méthode de type spirale adaptée à la gestion de la qualité.
		\item Le logiciel devra répondre aux fonctionnalités indiquées dans le cahier des charges fourni par le client.
		\item Le logiciel utilisera au maximum des logiciels et des technologies issus du monde libre notamment pour le développement.
		\item Le système de gestion de base de données utilisé sera PostgreSQL.
		\item La résolution de la géolocalisation des écoles se fera via OpenStreetMap.
		\item Une documentation sera fournie avec le projet.
	\end{itemize}
	