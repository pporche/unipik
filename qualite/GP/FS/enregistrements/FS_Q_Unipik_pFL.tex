%version 1.00,	date 12/02/2016	auteur(s) Pierre Porche
\documentclass [a4paper] {article}
\usepackage[utf8]{inputenc}
\usepackage[francais]{babel}
\usepackage[top=2cm, bottom=4cm, left=2cm, right=2cm]{geometry} 
\usepackage{fancyhdr}
\usepackage{graphicx}
\usepackage{color, colortbl}
\usepackage{longtable}
\usepackage{vocabulaireUnipik}
\usepackage{tabularx}
\usepackage{xcolor}
\definecolor{Gray}{gray}{0.8}

\pagestyle{fancy}

%--- En-t�te et pied de page ---%

\renewcommand{\footrulewidth}{0,01cm}
\rhead{}
\chead{\huge{Fiche de Suivi}}

\begin{document}

\section*{\Florian}

\centering
	\begin{longtable}{|>{\columncolor{gray!40}}p{2cm}|p{12cm}|}
	\hline
	Semaine 5 & \begin{itemize}
        \item Évaluation ArgoUML.
        \item Installation, configuration de Symfony et serveur Apache.
        \item Mise en place de l'architecture du projet.
        \end{itemize}\\  
	\hline
        Semaine 6 & \begin{itemize}
	\item Préparation de la revue.
	\item Début de réalisation des controleurs pour le lot 1.
        \item Travail sur le DCP.  
	\end{itemize} \\
	\hline
        Semaine 7 & \begin{itemize}
	\item Réalisation des controlleurs pour le lot 1.
        \item Création et mise en place de la base de données pour le lot 1.
	\item Réalisation de tests unitaires pour le lot 1.
	\end{itemize} \\
	\hline
        Semaine 8 & \begin{itemize}
	\item Réalisation de la fiche de formation Symfony.
        \item Formation Symfony.
        \item Livraison client et démonstration du lot 1.  
	\end{itemize} \\
	\hline
 	Semaine 9 & \begin{itemize}
	\item Fiche de risque pour la sécurité.
        \item Recherche sur les serveurs smtp disponibles pour l'envoi de mail.
        \item Gestion de l'envoi de mail via Symfony et ticket à la DSI pour accès au port 587.
	\end{itemize} \\
	\hline
        Semaine 10 & \begin{itemize}
	\item Création de l'architecture : Bundles, controleurs. 
        \item Diagramme de bundles.
        \item Intégration du FOSUserBundle.
        \item Formulaires d'inscription et connexion fonctionnels (persistance en BD + mots de passe cryptés)
        \item Configuration pour l'envoie de mail via l'application.  
	\end{itemize} \\
	\hline
        Semaine 11 & \begin{itemize}
	\item Surcharge des controlleurs de FOSUserBundle. 
        \item Changement de mot de passe fonctionnel.
        \item Controlleur pour le profil.
        \item Envoi de mail avec lien d'activation lors d'une inscription.  
	\end{itemize} \\
	\hline
	Semaine 12 & \begin{itemize}
	\item Préparation de la revue.
	\end{itemize} \\
	\hline
	Semaine 37 & \begin{itemize}
	\item Back-end : Créer bénévole.
	\item Back-end : Modifier bénévole.
	\item Back-end : Supprimer bénévole.
	\item Back-end : Créer établissement.
	\item Back-end : Redirection après register.
	\item Refactoring.
	\end{itemize} \\
	\hline
	Semaine 38 & \begin{itemize}
	\item Modification controller
	\item Suppression multiple avec ajax sur toutes les listes
	\item Réparation
	\end{itemize} \\
	\hline
	Semaine 39 & \begin{itemize}
	\item Service toolbox database
	\item Modification bénévole
	\item Réparation
	\item Autocomplétion Ville/CodePostal dans les formulaires 
	\end{itemize} \\
	\hline
	Semaine 40 & \begin{itemize}
	\item Affecter bénévole intervention
	\item Modifier établissement
	\item Gestion bénévoles
	\end{itemize} \\
	\hline
	Semaine 41 & \begin{itemize}
	\item Formulaire hiérarchique
	\item csrf token
	\item mailing smtp
	\end{itemize} \\
	\hline
	Semaine 42 & \begin{itemize}
	\item password form
	\item preparation revue
	\item crontab
	\end{itemize} \\
	\hline
\end{longtable}

\end{document}
