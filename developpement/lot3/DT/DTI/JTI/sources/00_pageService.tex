% version 1.00, date 09/03/16, auteur Michel Cressant
\begin{pagesService}
	\begin{historique}
		% nouvelles versions à rajouter AU-DESSUS en recopiant les lignes suivantes et en les modifiant :
		\unHistorique{1.00}{02/12/2016}{\Julie}{Création}{Toutes}

	\end{historique}

%        \begin{suiviDiffusions}
%
%            % On place ici les diffusions
%        	\unSuivi{1.00}{}{\nomEquipe{}}
%          
%          
%        \end{suiviDiffusions}

%%Signataires
        \begin{signatures}
	   \uneSignature{Vérificateur}{\RQ{}}{\Kafui{}}{05/12/2016}{email}
       \uneSignature{Validateur}{\CP{}}{\Sergi}{05/12/2016}{email}
        \end{signatures}
	
	
	\section*{Documents de Références}
%\begin{documentsReference}
		\begin{listeDeReferences}
			\uneReference{NF EN ISO 9001}{Octobre 2015}
			\uneReference{\MQ{}}{ASI-MQ-MQASI}
			\uneReference{\DGQ{} du Processus "\DGQDEUX{}"}{ASI-DGQ-DGQ2}
			\uneReference{\PTI }{\PTICourt\_ D\_\nomEquipe\_l3 }
			\uneReference{NF EN ISO 9000}{Octobre 2015}
		\end{listeDeReferences}
%	\end{documentsReference}	
	
\begin{terminologie}
		La terminologie (définitions et abréviations) utilisée dans le présent document est centralisée dans le \MQ{} \ASICourt{} (\emph{cf. ASI-MQ-MQASI}) de l'Unité P3.

		\begin{listeDAbreviations}
			\uneAbreviation{\asiCourt}{\asi}
			\uneAbreviation{\DGQDEUXCourt}{\DGQDEUX}
			\uneAbreviation{\DSECourt}{\DSE}
			\uneAbreviation{\INSACourt}{\INSA}
			\uneAbreviation{\JTICourt{}}{\JTI}
			\uneAbreviation{\MQCourt}{\MQ}
			\uneAbreviation{\PICCourt}{\PIC}
			\uneAbreviation{\PTICourt{}}{\PTI}
			
		\end{listeDAbreviations}
	\end{terminologie}
	

	
	
\end{pagesService}
