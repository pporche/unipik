\documentclass[asi, sansVersion]{picINSA}

\usepackage{vocabulaireUnipik}

\begin{document}

\title{Description modèle Entité Association}
\author{\Florian, \Mathieu, \Julie}
\date{02/02/2016} 

\maketitle

\tableofcontents

\chapter{Modèle Entité-Association}

%% Inclure le modele E-A 
\begin{landscape}
\begin{figure}
	\centering
	\includegraphics[scale=1]{images/modeleEA}
	\caption{\label{modele}}
\end{figure}
\end{landscape}

\chapter{Description du modèle E-A}

Le modèle Entité association réalisé est composé de plusieurs entités et de plusieurs associations. Nous allons les décrire dans cette partie. \\ 

\section{Les entités}

\subsection*{Personne}

L'entité PERSONNE est l'entité mère des entités UTILISATEUR et CONTACT. \\
Cette entité possède plusieurs attributs : 
\begin{itemize}
\item un identifiant unique;
\item un nom; %Cdc - Non NULL
\item un prénom; %Cdc - Non NULL
\item un numéro de téléphone fixe; %Entier - Peut être NULL
\item un numéro de téléphone portable. %Entier - Peut être NULL
\end{itemize}

\subsection*{Utilisateur}

L'entité UTILISATEUR est une entité fille de l'entité PERSONNE. Cette entité décrit les bénévoles de l'UNICEF qui gèreront le site ainsi que ceux qui effectueront des actions dans des structures. \\
Cette entité a plusieurs attributs : 
\begin{itemize}
\item un identifiant unique pouvant être une adresse électronique ou un pseudo;  
\item un type qui permet de déterminer si l'utilisateur est un administarteur global, un administrateur local, ou un bénévole;
\item une adresse postale;
\item une ou plusieurs activités potentielles (action ponctuelle, frimousse, plaidoyer); % attribut multivalué
\item un ou plusieurs encadrement de projet (VAE, action ponctuelle, projet d'élève). % attribut multivalué
\end{itemize}

\subsection*{Contact}

L'entité CONTACT est une entité fille de l'entité PERSONNE. Cette entité décrit les personnes travaillant dans un établissement. \\
Cette entité a plusieurs attributs : 
\begin{itemize}
\item un identifiant unique;
\item un type de contact qui permet de déterminer si un contact est un enseignant, un animateur ou a une autre fonction; 
\item un type 
\end{itemize} 

\subsection*{Etablissement}

L'entité ETABLISSEMENT est l'entité mère des entités ENSEIGNEMENT et CENTRE\_DE\_LOISIRS. \\
Cette entité a plusieurs attributs : 
\begin{itemize}
\item un identifiant unique;
\item la ville de l'établissement;
\item une adresse sans le code postal;
\item un code postal;
\item un numéro de téléphone fixe;
\item une adress e-mail; % attribut multivalué
\item une géolocalisation WGS84; % Peut être NULL
\item un responsable de la structure qui représente un CONTACT; 
\item un contact pour les plaidoyers qui représente un CONTACT;
\item un contact pour les frimousses qui représente un CONTACT;
\item un contact pour les activités ponctuelles qui représente un CONTACT.
\end{itemize}

\subsection*{Enseignement}
L'entité ENSEIGNEMENT est une entité fille de l'entité ETABLISSEMENT. Cette entité décrit les établissements ayant un rapport avec l'éductation c'est à dire les écoles primaires, les collèges, les lycées, etc. \\
Cette entité a plusieurs attributs : 
\begin{itemize}
\item un UAI (Unité Administrative Immatriculée) qui est une combinaison de chiffre et de lettres unique pour chaque enseignement;
\item un type d'enseignement qui permet de déterminer si l'enseignement est une école maternelle, une école élémentaire, un collège, un lycée, ou un établissement supérieur. 
\end{itemize} 


\subsection*{Centre de loisirs}
L'entité CENTRE\_DE\_LOISIRS est une entité fille de l'entité l'entité ETABLISSEMENT. Cette entité décrit les établissement ayant un rapport avec les loisirs. \\
Cette entité a un attribut : 
\begin{itemize}
\item un type de centre de loisirs qui permet de déterminer si un centre de loisirs prend en charge des enfants d'école maternelle,d'école élémentaire ou des adolescents.
\end{itemize}  

\subsection*{Intervention}
L'entité INTERVENTION est l'entité mère des entités PLAIDOYER et FRIMOUSSE. Cette entité décrit une intervention que peut demander un établissement. \\
Cette entité a plusieurs attributs :
\begin{itemize}
\item un identifiant unique;
\item une date précise déterminée par les bénévoles lors de l'attribution des bénévoles aux interventions;
\item le nombre de personnes concernées par cette intervention;
\end{itemize}

\section{Les associations}

\chapter{Contraites d'intégrité}


\end{document}