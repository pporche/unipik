\documentclass [a4paper] {article}
\usepackage[utf8]{inputenc}
\usepackage[francais]{babel}
\usepackage[top=2cm, bottom=4cm, left=2cm, right=2cm]{geometry} 
\usepackage{fancyhdr}
\usepackage{graphicx}
\usepackage{color, colortbl}
\usepackage{longtable}
\usepackage{vocabulaireUnipik}
\pagestyle{fancy}
\definecolor{Gray}{gray}{0.8}



%--- En-t�te et pied de page ---%
\renewcommand{\footrulewidth}{0,01cm}

\begin{document}
\rhead{}
\chead{\huge{Compte-rendu de réunion de Tutorat Qualité}}					%titre
\hfill   
\hfill 	9:03-9:56 				% Heure de d�but, heure de fin.


\lfoot{Version : 1.00} 			% version

%--- Fin en-t�te et pied de page ---%
\section*{Historique des révisions}
\begin{center}
			\begin{tabular}{| c | c | c | c | p{4cm} |}
				\hline
				\rowcolor{Gray}
				Version & Date & Auteur(s) & Modification(s) & Partie(s) modifiée(s)		 \\
				\hline
				1.00 & 02/02/2016 & \Pierre & Création & Toutes \\
		\hline		
			\end{tabular}
		\end{center}

\section*{Signatures}

		\begin{center}
			\begin{tabular}{| c | c | c | c | p{4cm} |}
				\hline
				\rowcolor{Gray}
				Rôle & Fonction & Nom & Date & Visa		 \\
				\hline
				Vérificateur & \RQA & \Kafui & 04/02/2016 & pgpic \\[30pt]
				\hline
				Validateur & \CP & \Sergi & 04/02/2016 & pgpic \\[30pt]	
				\hline
			\end{tabular}
		\end{center}

%--- Réunion --%

\section{Informations diverses}
M. \nomTuteurQualite{} nous explique qu'il est souhaitable qu'un \OC{} ne soit lié qu'à un seul \FT{}. Nous lui demandons ensuite dans quel cas il nous faudrait faire une émission de \FFT{} pour modification du \PQ{}. Il nous répond qu'il faut le faire à chaque fois qu'il y a une demande de modification à partir de la première approbation. Par ailleurs, les remarques que nous avons reçues sur le \PQCourt{} sont d'une gravité relativement réduite.
~
Avant le début du second semestre, il est important de bien former les personnes qui arrivent et donc de remplir le plan formation.

\section{Indicateurs}
Un des indicateurs les plus importants est l'indicateur de satisfaction client. Sur les questionnaires de satisfaction, le client ne donne pas qu'une note mais également un coefficient.
\\
Un autre indicateur important est l'indicateur d'avancement, utile pour savoir si nous sommes en avance ou en retard. Il faut toutefois faire attention au calcul de cet indicateur afin qu'il soit le plus significatif possible.
\\
Concernant l'indicateur de temps de présence, il n'est pas conseillé de mettre en place un seuil de surcharge mais il faut suivre ce problème avec attention.
\\
Un indicateur quasi-obligatoire est l'indicateur sur la répétition des \FTCourt{}. Un même \FT{} ne peut se répéter qu'au maximum une fois et au maximum cinq \FTCourt{}. Il faut toutefois penser aux exclusions qui concernent tout \FTCourt{} ne dépendant pas de l'équipe PIC (ex. : retards de signature par le client,..) ainsi que celles traitées par action curative. Dans PGPIC, il faut toujours remplir la colonne "analyse à froid", s'il n'y a pas eu d'action curative, y écrire "a été vue mais pas d'action curative" afin de minimiser les problèmes lors des audits.
\\
Un indicateur doit être pertinent pour les clients. Il existe des logiciels spécialisés permettant d'avoir des indicateurs sur le code (Sonar,..), notamment un indicateur sur le code dupliqué. Ne pas se fier à l'indicateur sur le pourcentage de commentaire. Un indicateur doit être "SMART", comporter une valeur cible, une valeur seuil et être suivi régulièrement.

\section{Recette}

Dans le \PTV{} est défini le déroulement de la recette. Sont décrits :
\begin{itemize}
\item Qui assiste à la recette ;
\item Où a lieu la recette ;
\item Comment se déroulera la recette ;
\item Que contiendra la livraison.
\end{itemize}
~
Au début, un cahier de recette vierge est fourni au client afin qu'il valide les tests. Il y a un cahier de recette par lot et il faut à chaque fois effectuer des tests de non-régression.
\\
Le client approuve ensuite le cahier de recette vierge et déclenche à ce moment la livraison. Lors de cette phase, l'équipe présente un cahier de recette provisoire au client.

\section{Possibilités de recette}
\subsection{approbation sans remarque du cahier de recette provisoire}
Si celui-ci est approuvé sans remarque du client, il devient le cahier de recette définitif. Il est par ailleurs conseillé de faire figurer deux cases sur la première page des cahier de recettes : "recette provisoire" et "recette définitive" qui seront toutes deux cochées dans ce cas.

\subsection{approbation avec remarques du cahier de recette provisoire}
Si le client a des remarques, nous devons les relever, lever des \FFT{}, les joindre au cahier de recette et nous passons en période probatoire. durant cette période, il nous faut corriger les remarques reçues lors de la présentation du cahier de recette ainsi que celles qui pourront être faite durant la période probatoire. Enfin, nous présentons un cahier de recette définitive au client.
\subsubsection{Approbation sans réserve du cahier de recette définitive}
Si le cahier de recette est approuvé sans réserves, le lot est terminé.
\subsubsection{Approbation avec réserves du cahier de recette définitive}
Si le cahier de recette est approuvé avec des réserves, la livraison se poursuit jusqu'à la levée de la dernière réserve qui vaut pour approbation. Ces réserves sont généralement des remarques de faible importance.
\subsubsection{Approbation avec report de fonctionnalités}
Si le lot est accepté avec report de fonctionnalités sur le prochain lot, il faut implémenter les fonctionnalités manquantes lors du prochain lot.
\subsubsection{Refus total}
Le lot peut également être refusé totalement par le client, le cycle recommence.

~

A la fin de l'approbation des documents de spécifications, il faut un \PV{} du \CP{}. Concernant ces documents, si le client exprime son désire de les regrouper, c'est possible sous réserve d'acceptation par le tuteur 

\section{\PTV{}}
Ce document définit avec le client les conditions de réalisations de la livraison, les possibilités de validation (voir plus haut), les supports de livraison et la déscription des tests que nous ferons (sans entrer dans le détail du \CDR{}). Tout ce qui sera défini grossièrement dans le \PTV{} aidera à la rédaction du cahier de recette.

\section{Autres}
Il faut impérativement faire attention à ce que l'archivage soit utile et régulier.

\end{document}















