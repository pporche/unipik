% version 1.00	Auteur Pierre Porche

\documentclass [a4paper] {article}
\usepackage[utf8]{inputenc}
\usepackage[francais]{babel}
\usepackage[top=2cm, bottom=4cm, left=2cm, right=2cm]{geometry} 
\usepackage{fancyhdr}
\usepackage{graphicx}
\usepackage{color, colortbl}
\usepackage{longtable}
\usepackage{vocabulaireUnipik}
\pagestyle{fancy}
\definecolor{Gray}{gray}{0.8}
\definecolor{White}{gray}{1.0}


%--- En-t�te et pied de page ---%

\renewcommand{\footrulewidth}{0,01cm}
\rhead{}
\chead{\huge{Compte-rendu de réunion interne}}					%titre
\begin{document}

23/02/2016			 				%Date 
\hfill   
\hfill 	 08:38 - 09:15 				%Heure de d�but, heure de fin.


\lfoot{Version : 1.00} 			% Rédacteur
%--- Fin en-t�te et pied de page ---%

\section*{Historique des révisions}
\begin{center}
			\begin{tabular}{| c | c | c | c | p{4cm} |}
				\hline
				\rowcolor{Gray}
				Version & Date & Auteur(s) & Modification(s) & Partie(s) modifiée(s)		 \\
				\hline
				1.00 & 23/02/2016 & \Pierre & Création & Toutes \\
		\hline		
			\end{tabular}
		\end{center}

\section*{Signatures}

		\begin{center}
			\begin{tabular}{| c | c | c | c | p{4cm} |}
				\hline
				\rowcolor{Gray}
				Rôle & Fonction & Nom & Date & Visa		 \\
				\hline
				Vérificateur & \RQA & \Kafui & 24/02/2016 & pgpic \\[30pt]
				\hline
				Validateur & \CP & \Sergi & 25/02/2016 & pgpic \\[30pt]	
				\hline
			\end{tabular}
		\end{center}
		
\newpage		



\section{Choix du framework}
L'équipe s'est réunie afin de décider du framework à utiliser. Une recherche a été faite en amont afin d'établir une liste des possibilités. Cinq frameworks différents ont été retenus :
\begin{itemize}
\item Pour le langage Java :
	\begin{itemize}
	\item Struts ;
	\item Spring ;
	\item Play.
	\end{itemize}
\item Pour le langage PHP :
	\begin{itemize}
	\item Symfony ;
	\item Laravel.
	\end{itemize}	
\end{itemize}

Après cela, nous avons dû éliminer le framework Struts car il est très peu utilisé, presque exotique et cela ne faciliterait pas la reprise et la maintenance de notre application par des tiers.
\\
Nous avons ensuite dressé deux tableaux : "Spring contre Play" et "Symfony contre Laravel" qui sont exposés ci-après :

\begin{tabular}{cc}

\begin{tabular}{| p{3cm} | p{3cm} |}
	\hline
	\rowcolor{Gray}
	Spring & Play		 \\
	\hline
	\rowcolor{White}
	+ Facilité de maintenance & + MVC natif \\
	- MVC non-natif & + contrôle des utilisateurs \\	
	 & + Restfull \\[30pt]	
	\hline
\end{tabular}

\begin{tabular}{| p{3cm} | p{3cm} |}
	\hline
	\rowcolor{Gray}
	Symfony & Laravel		 \\
	\hline
	\rowcolor{White}
	+ Facilité de maintenance &  \\
	+ Utilisation de package &  \\	
	++ Bundle &  \\[30pt]	
	\hline
\end{tabular}

\end{tabular}

\subsubsection*{}
Cette analyse nous a permis de "choisir" les frameworks Play pour Java et Symfony pour PHP. Nous avons donc ensuite fait un tableau comparatif entre Java avec Play et PHP avec Symfony. Ce tableau est donnée ci-après :

\begin{center}
\begin{tabular}{| p{4cm} | p{4cm} |}
	\hline
	\rowcolor{Gray}
	Java & PHP		 \\
	\hline
	\rowcolor{White}
	+ Statique & - Dynamique \\
	+ Typage fort & + Plus utilisé dans le développement web \\	
	+ Scalable & - Typage faible \\		
	+ Collections & + Bugs moins bloquants \\		
	+ Bibliothèques & + Apprentissage aisé \\		
	 & + Prix faible \\		
	 & + Scalable \\	[20pt]
	\hline
\end{tabular}
\end{center}

La décision s'est donc finalement portée sur le framework Symfony.

\section{Choix de l'outil UML}
Nous avons décidé de faire une recherche d'un outil pour l'UML qui satisferait plusieurs critères. Parmi ceux-ci, une compatibilité avec le framework choisi (voir plus haut), une simplicité d'utilisation, des exports de code fiables et une grande popularité. Une tâche de recherche et de décision sur l'outil a été confiée à \Julie.


%--- fin de réunion ---%


\end{document}







