% version 1.00, date 12/11/16, auteur Kafui Atanley
Ce chapitre présente les maquettes pour chaque fonctionnalité pour chaque fonctionnalité additionnelle du lot 3 par rapport au lot 2.


\section{Fonctionnalité 8}
Ce paragraphe décrit les maquettes concernant la fonctionnalité 8 soit la géolocalisation. \\

La figure suivante \ref{maquette8-1} montre la maquette d'une fiche descriptive d'une entité possédant une adresse. Un espace devra être réservé pour afficher la location de l'entité comme visible sur la fmaquette.
\begin{figure}[H]
	\centering
	\includegraphics[scale=0.40]{images/maquettes/fonctionnalite9Geolocalisation.png}
	\caption{Maquette~: Fiche descriptive d'une entité possédant une adresse}
	\label{maquette8-1}
\end{figure}
La figure suivante \ref{maquette8-2} présente la maquette des filtres de tri attendus sur la géolocalisation dans les pages permettant de lister les entités.
\begin{figure}[H]
	\centering
	\includegraphics[scale=0.40]{images/maquettes/fonctionnalite9Filtre.png}
	\caption{Maquette~: Modèle de filtre attendues dans les liste de tri}
	\label{maquette8-2}
\end{figure}

\section{Fonctionnalité 9}
Ce paragraphe décrit les maquettes concernant la fonctionnalité 9 soit l'attribution de frimousse. \\

La figure suivante montre l'email type de prise en charge d'une intervention de type frimousse.

La figure suivante montre la maquette de l'interface pour la prise en charge d'une intervention de type frimousse. L'utilisateur pourra s'attribuer une intervention de type frimousse via la liste d'intervention ou via la fiche descriptive de cette intervention.


\section{Fonctionnalité 10}
Ce paragraphe décrit les maquettes concernant la fonctionnalité 10 soit la gestion des ventes. \\

La figure suivante montre la maquette de la page permettant d'ajouter une vente .

La figure suivante montre la maquette de la page permettant de modifier une vente .

La figure suivante \ref{maquette10-3} montre la maquette de la page de la fiche descriptive d'une vente.
\begin{figure}[H]
	\centering
	\includegraphics[scale=0.40]{images/maquettes/fonctionnnalite10ConsulterVente.png}
	\caption{Maquette~: Consulter la fiche descriptive d'une vente}
	\label{maquette10-3}
\end{figure}

La figure suivante \ref{maquette10-4} Listing des ventes montre la maquette de la page permettant d'effectuer le listing des ventes.
\begin{figure}[H]
	\centering
	\includegraphics[scale=0.40]{images/maquettes/fonctionnalite10Ventes.png}
	\caption{Maquette~: Listing des ventes}
	\label{maquette10-4}
\end{figure}
Il est possible de supprimer une vente à partir de la fiche descriptive d'une vente 


