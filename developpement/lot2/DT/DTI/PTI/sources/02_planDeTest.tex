% version 1.00, date 30/03/16, auteur Michel Cressant
  
  Ce chapitre a pour but de présenter les éléments de test, ???????? les fonctionnalités qui devront être testées ??????????, les documents et éléments nécessaires à la réalisation des tests d'intégration, la procédure à suivre une fois un test exécuté, l'environnement de travail nécessaire à la réalisation d'un test et la validation d'un test. 
 
 
  \section{Elements de test}
 	Les éléments qui seront testés permettront de vérifier l'intégration des modules de code développés par l'équipe \nomEquipe.
 	
 \section{Fonctionnalités à tester}
 	Ce paragraphe décrit l'organisation de notre application ainsi que les différents tests d'intégrations qui devront être réalisés. \\
 	
 	La figure suivante (figure \ref{diagrammeDeBundles}) représente le diagramme de package (ici ce sont des bundles) de notre application.
 	\begin{figure}[H]
 		\centering
 		\includegraphics[scale=0.49]{images/diagrammeDeBundles.png}
 		\caption{Diagramme de package}
 		\label{diagrammeDeBundles}
 	\end{figure}
 	
 	Le but des tests d'intégration est de vérifier la bonne communication entre les bundles. Les tests seront donc les suivants :
 	
 	\subsection*{Tests d'Intégration entre le package FOSUserBundle et le package InterventionBundle}
 	
 	\begin{center}
    		\begin{tabular}[h]{|p{0.45\textwidth}|p{0.45\textwidth}|}
		\hline
			ID & Test d'intégration \\\hline
			TI001 & FOSUserBundle <-> InterventionBundle \\\hline
    	 	\end{tabular}
  	\end{center}
  	
  	
  	\subsection*{Tests d'Intégration entre le package FOSUserBundle et le package SwiftMailerBundle}
 	
 	\begin{center}
    		\begin{tabular}[h]{|p{0.45\textwidth}|p{0.45\textwidth}|}
		\hline
			ID & Test d'intégration \\\hline
			TI002 & FOSUserBundle <-> SwiftMailerBundle \\\hline
    	 	\end{tabular}
  	\end{center}
  	
  	\subsection*{Tests d'Intégration entre le package InterventionBundle et le package SwiftMailerBundle}
 	
 	\begin{center}
    		\begin{tabular}[h]{|p{0.45\textwidth}|p{0.45\textwidth}|}
		\hline
			ID & Test d'intégration \\\hline
			TI003 & InterventionBundle <-> SwiftMailerBundle \\\hline
    	 	\end{tabular}
  	\end{center}
  	
  	
  	
 	
  \section{Eléments nécessaires pour les Test d'Intégration}
  	Ce paragraphe décrit les éléments livrés nécessaires à la réalisation des tests d'intégration. \\
  	
  	Les éléments suivant doivent être livrés avant que les tests d'intégrations puissent commencer :
  	\begin{itemize}
  		\item Le \PTU du lot 2,
  		\item Les chapitres 1, 2 et 3 de ce document.
  	\end{itemize}
  	
  	Les éléments suivant doivent être livrés avant qu'un test d'intégration spécifique ne commence :
  	\begin{itemize}
  		\item Le \JTU du lot 2,
  		\item Les composants impliqués,
  		\item Les données tests spécifiques à ce test d'intégration.
  	\end{itemize}
  
  \section{Rapport de Test d'Intégration}  
  	Ce paragraphe décrit la procédure à exécuter une fois un test exécuté.  \\
  	
  	Les éléments suivants sont à réaliser lorsqu'un test d'intégration spécifique est fini : 
  	\begin{itemize}
  		\item Remplir le rapport du test dans le \JTI du lot 2, qui listera tous les tests d'intégration,
  		\item Rapporter les problèmes rencontrés (si nécessaire).
  	\end{itemize}
  	
  	Les éléments suivant sont à réaliser lorsque la session de test sur tous les modules de code sont finis : 
  	\begin{itemize}
  		\item Remplir tous les rapports de test dans le \JTI du lot 2.
  	\end{itemize}
  
  \section{Environnement de Test}
	  Ce paragraphe décrit l'environnement nécessaire à la réalisation des tests d'intégration. \\
	  
	  Afin de réaliser les tests d'intégration il suffit d'un ordinateur ayant une connection Internet.
  
  \section{Validation d'un Test}
	Ce paragraphe décrit les conditions à remplir pour qu'un test d'intégration soit validé.\\
  
  	Un test d'intégration est validé s'il ne provoque aucune erreur, et retourne les résultats escomptés. 
  	
  	