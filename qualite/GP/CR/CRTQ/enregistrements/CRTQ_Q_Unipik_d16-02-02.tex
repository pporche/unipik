\documentclass [a4paper] {article}
\usepackage[utf8]{inputenc}
\usepackage[francais]{babel}
\usepackage[top=2cm, bottom=4cm, left=2cm, right=2cm]{geometry} 
\usepackage{fancyhdr}
\usepackage{graphicx}
\usepackage{color, colortbl}
\usepackage{longtable}
\usepackage{../../../../ressources/Unipik/vocabulaire/vocabulaireUnipik}
\pagestyle{fancy}
\definecolor{Gray}{gray}{0.8}

\begin{document}
Un FT -> 1 OC <- c'est mieux
FFT pour le PQ? à chaque damnde de corrections dès approbation

2eme semestre : bien former les personnes qui arrivent <- formations pour le Portefeuille de formation. 
Rq sur le RQ pas très graves.

Indicateurs :
L'indicateur sur la satisfaction client <- !!
Le client ne donne pas qu'une note, il donne aussi un coeff sur les questionnaires.
indicateur d'avancement <- pour savoir si on est en avance ou en retard. bien faire attention au calcul de l'indicateur (tache facile-difficile..)
temps de présence : pas de valeur de surcharge mais faut faire gaffe.
indicateur sur la répétition des FT <- pas le droit d'avoir + d'1 répétition d'UN FT sur la totalité du PIC et pas + de (5?) FT qui subissent une répet. Penser aux exclusions : tout FT qui ne dépend pas de nous (retards signature,..) ou traité par action curative.
dans PGPIC : tableau récap des FT, colonne "analyse à froid" <- la remplir! "a été vue mais pas d'action curative" -> colonne au vert, pas de pbs durant les audits.
Indicateurs pertinents pour le client et indicateurs au niveau du code genre avec des logiciels spécialisés (Sonar,..) indic sur le pourcentage de commentaire -> nul. sur le code dupliqué -> bien.
Indicateur : SMART, valeur cible, valeur seuil, suivi régulièrement

Recette :
dans le PTV on défini comment se déroule la recette (qui, où, comment) livraison : document, app fournie,.. 
cahier de recette vierge pour le client pour valider les tests,..    -> cahier de recette par lot. tests de non-régression <- c important tavu.
approbation du vierge -> déclenchement de la livraison -> cahier de recette provisoire (deux cases "recette provisoire", "recette définitive") dans le PTV, si c'est approuvé sans remarque du client, ca passe en définitif (deux cases cochées) -> si le client a des remarques, on le marque, on met les FFT dans le cahier de recette, -> periode probatoire : correction des remarques + nouvelles remarques? -> on balance le cahier de recette définitive -> approbation sans reserve : approbation avec reserve (levée de la dernière reserve vaut pour approbation ("le bandeau il est bleu, je le veux vert"); acceptation du lot mais report des fonctionnalités sur le lot prochain (le lot, on le prend mais les fctnalités pas bien, on les décale) ; refus total (periode probatoire, correction définitive).

a la fin de l'approbation des docs de specs, il faut un PV du CP 
sur les docs de specs -> si le client trouve relou tous ces documents, on peut regrouper.
simplification de DCD, DCP -> sous accord du tuteur, on peut fusionner les deux.

PTV : comment le remplir? -> ca définit avec le client les conditions de réalisations de la livraison, possibilité de validation (voir plus haut), comment on livre (support), en décrivant les tests qu'on va faire (sans le détail du CDR) -> les trucs définis grossièrement dans le PTV vont aider à faire le CDR

archivage : faire gaffe à ce que ce soit utile !


\end{document}















