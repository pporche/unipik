% version 1.00, date 07/10/16, auteur Julie Pain

\section{Fonctionnalité F1}

	\subsection*{Critère 1.1 : Ajouter un bénévole}
	
		\begin{center}
    	 		\begin{tabular}[h]{|p{0.95\textwidth}|}
			\hline
				Lorsque l'utilisateur est connecté en tant qu'administrateur, il a la possibilité d'ajouter un bénévole. \\
				Nous allons voir le liste des bénévoles : elle est vide. \\
				Nous ajoutons quatre bénévoles. \\
				Reconsultons la liste : les quatre bénévoles sont présents. \\
				
				$\square$ Oui \hfill \hfill $\square$ Non \\\hline Remarques : \\ ~\\
			 \\\hline
     		\end{tabular}
  		\end{center}	
  		
  		
	\subsection*{Critère 1.2 : Mail de confirmation}
	
		\begin{center}
    	 		\begin{tabular}[h]{|p{0.95\textwidth}|}
			\hline
				Nous nous connectons sur l'adresse mail du premier bénévole ajouté. \\
				Le bénévole a bien reçu un mail de confirmation, il doit cliquer sur un lien permettant l'activation du compte. \\ 
				
				$\square$ Oui  \hfill \hfill $\square$ Non \\\hline Remarques : \\ ~\\
			 \\\hline
     		\end{tabular}
  		\end{center}	
  		
  	\subsection*{Critère 1.3 : Connexion réussie}
	
		\begin{center}
    	 		\begin{tabular}[h]{|p{0.95\textwidth}|}
			\hline
				Lorsque le bénévole clique sur le lien du mail, il est redirigé vers la page d'activation du compte et est connecté. \\
				Un message "Félicitations \textit{login}, votre compte est maintenant activé". \\
				En consultant le profil du bénévole, les informations entrées par l'administrateur sont bien enregistrées. \\
				
				$\square$ Oui  \hfill \hfill $\square$ Non \\\hline Remarques : \\ ~\\
			 \\\hline
     		\end{tabular}
  		\end{center}	
  		
  	\subsection*{Critère 1.4 : Modification d'un bénévole}
	
		\begin{center}
    	 		\begin{tabular}[h]{|p{0.95\textwidth}|}
			\hline
				Le bénévole peut modifier les informations le concernant, comme par exemple l'adresse ou le numéro de téléphone. \\
				Un message "Félicitations! Le profil a été mis à jour". \\
				Sur la page de profil du bénévole, les informations ont bien été modifiées. \\
				
				$\square$ Oui  \hfill \hfill $\square$ Non \\\hline Remarques : \\ ~\\
			 \\\hline
     		\end{tabular}
  		\end{center}	
  		
  	\subsection*{Critère 1.5 : Suppression d'un bénévole}
	
		\begin{center}
    	 		\begin{tabular}[h]{|p{0.95\textwidth}|}
			\hline
				Nous nous connectons en tant qu'administrateur car seul un administrateur peut supprimer un bénévole. \\
				Nous supprimons le bénévole. \\
				Sur la liste des bénévoles, le bénévole en question n'apparaît plus. \\
				
				$\square$ Oui  \hfill \hfill $\square$ Non \\\hline Remarques : \\ ~\\
			 \\\hline
     		\end{tabular}
  		\end{center}	
  		
  	\subsection*{Critère 1.6 : Connexion échouée}
	
		\begin{center}
    	 		\begin{tabular}[h]{|p{0.95\textwidth}|}
			\hline
				Si le bénévole qui vient d'être supprimé essaie de se reconnecter, la connexion échoue et un message "Erreur! Identifiants invalides" apparaît. \\
				
				$\square$ Oui  \hfill \hfill $\square$ Non \\\hline Remarques : \\ ~\\
			 \\\hline
     		\end{tabular}
  		\end{center}	
  		
  		
\section{Fonctionnalité F2}
  		
  	\subsection*{Critère 2.1 : Ajouter un établissement}
  		\begin{center}
    	 		\begin{tabular}[h]{|p{0.95\textwidth}|}
			\hline
				Seul un administrateur peut ajouter un établissement, il faut donc être connecté en tant qu'administrateur.\\
				Nous ajoutons un établissement. \\
				Lorsque nous consultons la liste des établissements, celui-ci est bien présent. \\
						
				
				$\square$ Oui  \hfill \hfill $\square$ Non \\\hline Remarques : \\ ~\\
			 \\\hline
     		\end{tabular}
  		\end{center}	
  		
  	\subsection*{Critère 2.2 : Modification d'un établissement}
  		\begin{center}
    	 		\begin{tabular}[h]{|p{0.95\textwidth}|}
			\hline
				Nous modifions les informations de l'établissement qui vient d'être ajouté.\\
				Lorsque l'on consulte l'établissement, les modifications apportées ont bien été enregistrées. \\			
				
				$\square$ Oui  \hfill \hfill $\square$ Non \\\hline Remarques : \\ ~\\
			 \\\hline
     		\end{tabular}
  		\end{center}	
  		
  	\subsection*{Critère 2.3 : Supression d'un établissement}
  		\begin{center}
    	 		\begin{tabular}[h]{|p{0.95\textwidth}|}
			\hline
				Nous supprimons un établissement. \\
				Sur la liste des établissements, celui-ci n'apparaît plus. \\
				
				$\square$ Oui  \hfill \hfill $\square$ Non \\\hline Remarques : \\ ~\\
			 \\\hline
     		\end{tabular}
  		\end{center}	
  		
  		
\section{Fonctionnalité F3}


\section{Fonctionnalité F4}
	\subsection*{Critère 4.1 : Formulaire de demande }
  		\begin{center}
    	 		\begin{tabular}[h]{|p{0.95\textwidth}|}
			\hline
				Le formulaire de demande permet de rentrer les informations de l'établissement, ses préférences concernant les dates, autant d'interventions que l'établissement veut, et les informations sur le contact qui fait la demande.\\
				Nous entrons trois interventions. \\
				Lorsque nous consultons la liste des interventions, celles-ci sont bien présentes. \\
				
				$\square$ Oui  \hfill \hfill $\square$ Non \\\hline Remarques : \\ ~\\
			 \\\hline
     		\end{tabular}
  		\end{center}	
  		
  	\subsection*{Critère 4.2 : Mail de confirmation de prise en compte}
  		\begin{center}
    	 		\begin{tabular}[h]{|p{0.95\textwidth}|}
			\hline
				Lorsque nous consultons la boîte mail de l'établissement, celui-ci a bien reçu un mail de confirmation de prise en compte.\\
				
				$\square$ Oui  \hfill \hfill $\square$ Non \\\hline Remarques : \\ ~\\
			 \\\hline
     		\end{tabular}
  		\end{center}	
  		
  		
  	\subsection*{Critère 4.3 : Suppression d'une intervention}
  		\begin{center}
    	 		\begin{tabular}[h]{|p{0.95\textwidth}|}
			\hline
				Nous supprimons une intervention. \\
				Sur la liste des interventions, celle-ci n'apparaît plus. \\
				
				$\square$ Oui  \hfill \hfill $\square$ Non \\\hline Remarques : \\ ~\\
			 \\\hline
     		\end{tabular}
  		\end{center}	

\section{Fonctionnalité F5}
	\subsection*{Critère 5.1 : Aide à l'attribution des bénévoles aux interventions }
  		\begin{center}
    	 		\begin{tabular}[h]{|p{0.95\textwidth}|}
			\hline
				La page de la liste des interventions permet d'effectuer des filtres sur les interventions pour aider l'utilisateur à choisir les interventions qui lui conviennent.\\
				Il est possible de trier selon les filtres suivants: 
				\begin{itemize}
				\item le type d'intervention : plaidoyers, frimousses, autres;
				\item le statut de l'intervention : attribuées, non attribuées, réalisées;
				\item le niveau scolaire de l'intervention : maternelle, primaire, collège, lycée, autre;
				\item le thème de l'intervention;
				\item le lieu de l'intervention; 
				\end{itemize}
				
				$\square$ Oui  \hfill \hfill $\square$ Non \\\hline Remarques : \\ ~\\
			 \\\hline
     		\end{tabular}
  		\end{center}	
  		
  	\subsection*{Critère 5.2 : Attribution d'une intervention à un bénévole par l'administrateur}
  		\begin{center}
    	 		\begin{tabular}[h]{|p{0.95\textwidth}|}
			\hline
				Si l'utilisateur est connecté en tant qu'administrateur, il peut attribuer l'intervention au bénévole qu'il souhaite. \\
				Nous attribuons une intervention à un bénévole. \\
				En consultant l'intervention en question, on voit que l'intervention est bien attribuée. \\
				
				$\square$ Oui  \hfill \hfill $\square$ Non \\\hline Remarques : \\ ~\\
			 \\\hline
     		\end{tabular}
  		\end{center}	
  		
  	\subsection*{Critère 5.3 : Attribution d'une intervention à un bénévole par lui même}
  		\begin{center}
    	 		\begin{tabular}[h]{|p{0.95\textwidth}|}
			\hline
				Si l'utilisateur est connecté en tant que simple bénévole, il ne peut que s'attribuer l'intervention. \\
				Nous nous attribuons une intervention. \\
				En consultant l'intervention en question, on voit que l'intervention est bien attribuée. \\
				Si nous allons voir la page "Mes interventions", celle-ci est bien présente. \\
				
				$\square$ Oui  \hfill \hfill $\square$ Non \\\hline Remarques : \\ ~\\
			 \\\hline
     		\end{tabular}
  		\end{center}	
  		
  	\subsection*{Critère 5.4 : Mail de confirmation de prise en charge}
  		\begin{center}
    	 		\begin{tabular}[h]{|p{0.95\textwidth}|}
			\hline
				Lorsque nous consultons la boîte mail de l'établissement, celui-ci a bien reçu un mail de confirmation de prise en charge de l'intervention par le bénévole concerné.\\
				
				$\square$ Oui  \hfill \hfill $\square$ Non \\\hline Remarques : \\ ~\\
			 \\\hline
     		\end{tabular}
  		\end{center}	
  		
  		
  	\subsection*{Critère 5.5 : Désattribution d'une intervention}
  		\begin{center}
    	 		\begin{tabular}[h]{|p{0.95\textwidth}|}
			\hline
				De la même manière que l'attribution, le bénévole peut se désattribuer une intervention alors que l'administrateur peut désattribuer des interventions à tous les bénévoles. \\
				
				$\square$ Oui  \hfill \hfill $\square$ Non \\\hline Remarques : \\ ~\\
			 \\\hline
     		\end{tabular}
  		\end{center}	
 
\section{Fonctionnalité F6}
	\subsection*{Critère 6.1 : Modification des informations d'une intervention }
  		\begin{center}
    	 		\begin{tabular}[h]{|p{0.95\textwidth}|}
			\hline
				Un bénévole peut modifier une intervention si celle-ci lui est attribuée (l'administrateur peut modifier toutes les interventions), notamment pour spécifier la date de l'intervention ou encore modifier d'autres informations. \\
				Modifions une intervention. \\
				Lorsque nous consultons la page de cette intervention, celle-ci a bien été modifiée. \\
				
				$\square$ Oui  \hfill \hfill $\square$ Non \\\hline Remarques : \\ ~\\
			 \\\hline
     		\end{tabular}
  		\end{center}	
  		
  	\subsection*{Critère 6.3 : Consultation du planning}
  		\begin{center}
    	 		\begin{tabular}[h]{|p{0.95\textwidth}|}
			\hline
				Il est possible de consulter le planning de l'utilisateur et nous pouvons voir que l'intervention est présente. \\
				
				$\square$ Oui  \hfill \hfill $\square$ Non \\\hline Remarques : \\ ~\\
			 \\\hline
     		\end{tabular}
  		\end{center}	