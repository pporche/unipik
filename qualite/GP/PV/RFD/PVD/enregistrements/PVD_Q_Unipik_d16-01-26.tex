% Version 1.00	auteur Sergi Colomies

\documentclass[asi, sansVersion]{picInsa}

\usepackage{vocabulaireUnipik}

\referenceVersion{\PVDCourt\_Q\_\nomEquipe\_d16-01-26}

\begin{document}
 
 \begin{center}
  \LARGE{}
    Procès Verbal de Réunion Formelle de Démarrage\\
 \end{center}
 
 \normalsize{}
 
L'équipe \nomEquipe{} atteste en ce jour, le 27/01/2016, avoir démarré le \PIC{} le 18/01/2016.
Lors de la réunion formelle de démarrage (ou de lancement) avec le client du 26 Janvier 2016, nous avons défini le sujet et clarifié le cahier des charges.


\paragraph{}
\textbf{Informations sur le \PICCourt{} :}

\begin{description}
  \item[Sujet~:]"Création d'un outil de gestion des interventions externes"
  \item[Intitulé~:]      \PICCourt{} \nomPIC
  \item[Nom de l'équipe~:]  \nomEquipe
\end{description}

\paragraph{}
\textbf{Informations sur le client :}

\begin{description}
\item[Nom de l'organisme~:] \nomPIC
\item[Nom de son représentant~:] \representantClient
\item[Adresse de l'organisme~:] 26 rue Saint Nicolas, 76000, Rouen
\item[Téléphone~:] 02 35 88 98 88
\item[E-mail~:] unicef76@unicef.fr
\item[Nom de son suppléant~:] Véronique \textsc{Davreux}
\end{description}

\paragraph{}
\textbf{Engagement de l'équipe :}\\


L'équipe \nomEquipe{} s'engage à réaliser le sujet grâce aux moyens mis à disposition et dans le temps qui lui est imparti.


L'équipe \nomEquipe{} s'engage également à la mise en place d'une politique qualité de type ISO 9001 : 2015 notamment en réalisant un \PQ{} et un \PGC{}.

Enfin, le \CP{}, \Sergi{}, s'engage à mettre en place une gestion de projet de type agile et à créer un planning de principe et suivre l'avancement du projet. \\


À la vue de tous ces éléments, la phase de spécifications a pu débuter.

\begin{center}
Décision : Validée [ \checkmark{} ] - Ajournée [ ]
\end{center}

\subsection*{Signature}
\begin{figure}[H]
		\centering
		\begin{tabularx}{17cm}{|p{4cm}|X|X|X|X|}
		\hline
		\rowcolor[gray]{0.85} Fonction & Nom & Date & Visa \\
		\hline
		\CP{} & \Sergi{} & 27/01/16 & courriel\\
		\hline
		\end{tabularx}
\end{figure}

\end{document}