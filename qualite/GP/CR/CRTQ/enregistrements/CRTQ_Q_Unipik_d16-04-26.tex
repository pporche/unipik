% version 1.00	date 26/04/2016	Auteur Pierre Porche

\documentclass [a4paper] {article}
\usepackage[utf8]{inputenc}
\usepackage[francais]{babel}
\usepackage[top=2cm, bottom=4cm, left=2cm, right=2cm]{geometry} 
\usepackage{fancyhdr}
\usepackage{graphicx}
\usepackage{color, colortbl}
\usepackage{longtable}
\usepackage{vocabulaireUnipik}
\pagestyle{fancy}
\definecolor{Gray}{gray}{0.8}



%--- En-t�te et pied de page ---%
\renewcommand{\footrulewidth}{0,01cm}

\begin{document}
\rhead{}
\chead{\huge{Compte-rendu de réunion de Tutorat Qualité}}					%titre

26/04/2016
\hfill   
\hfill 	9:33 - 10:16 				% Heure de d�but, heure de fin.


\lfoot{Version : 1.00} 			% version

%--- Fin en-t�te et pied de page ---%
\section*{Historique des révisions}
\begin{center}
			\begin{tabular}{| c | c | c | c | p{4cm} |}
				\hline
				\rowcolor{Gray}
				Version & Date & Auteur(s) & Modification(s) & Partie(s) modifiée(s)		 \\
				\hline
				1.00 & 26/04/2016 & \Pierre & Création & Toutes \\
		\hline		
			\end{tabular}
		\end{center}

\section*{Signatures}

		\begin{center}
			\begin{tabular}{| c | c | c | c | p{4cm} |}
				\hline
				\rowcolor{Gray}
				Rôle & Fonction & Nom & Date & Visa		 \\
				\hline
				Vérificateur & \RQA & \Kafui &  & pgpic \\[30pt]
				\hline
				Validateur & \CP & \Sergi &  & pgpic \\[30pt]	
				\hline
			\end{tabular}
		\end{center}

%--- Réunion --%

\section{Résumé de la semaine}
\paragraph*{}
\nomTuteurQualite{} nous explique que l'utilisation du script de coloration des documents pour les nouvelles versions des documents devrait être utilisé à chaque fois. Changer d'outil est peu recommandé car celui ci est qualifié alors qu'un nouvel outil ne serait pas sûr.

\paragraph*{}
L'évaluation par les pairs est à faire aussi tôt que possible. \nomTuteurQualite{} nous explique que l'outil est peu maintenable et qu'il serait certainement remplacé dès l'année prochaine par l'outil de la mission qualité.

\paragraph*{}
Notre BD n'étant pas définie manuellement, il est important d'effectuer des tests en "worst case scenario" afin d'être certains de prendre en compte toutes les éventualités.

\paragraph*{}
\Pierre{} rend compte à \nomTuteurQualite{} de la réunion des \RQ{} lors de laquelle le sujet de la "passation qualité" a été évoqué. Selon les \RQCourt{}, il serait souhaitable pour les étudiants suivants d'avoir le point de vue des \RQ{} et \CP{} de cette année sur la gestion des PICs. \nomTuteurQualite{} propose l'idée de faire passer les étudiants dans un cours de MGPI.


\end{document}















