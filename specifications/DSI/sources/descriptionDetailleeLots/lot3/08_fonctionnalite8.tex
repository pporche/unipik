% version 1.00, date 05/11/16, auteur Kafui Atanley
% version 2.00, date 28/11/16, auteur François Decq

%La fonctionnalité 8 permet l’attribution d’intervention de type frimousse. La fonctionnalité
%8 consistera à considérer les interventions de type frimousse en sus des interventions de
%type action éducative. Le principe reste le même, un chargé d’action éducative s’attribue
%une intervention. Lors de chaque affectation, un e-mail d’information de prise en charge de
%la demande par l’équipe de chargés de frimousse sera envoyé à l’établissement demandeur
%ainsi qu’au responsable de l’activité demandée. Une intervention de type frimousse ne peut
%être attribuée qu’à un utilisateur.
\subsection{Fonctionnalité 8}

La fonctionnalité 8 consistera à planifier les interventions de type frimousse c'est-à-dire qu'un intervenant s'attribue une intervention de type frimousse.
Lors de chaque affectation, un e-mail d'information de prise en charge de la demande par l'équipe d'intervenants sera envoyé à l'établissement demandeur ainsi qu'au responsable de l'activité frimousse.
La figure \ref{maquetteCourrielConfirmation} présente la maquette de cet email d'information de prise en charge. 
Une intervention de type frimousse ne peut être attribuée qu'à un utilisateur.
  La figure \ref{visualiserESAttribuer} présente le diagramme de cas d'utilisation sur la visualisation et l'attribution des interventions.\\
