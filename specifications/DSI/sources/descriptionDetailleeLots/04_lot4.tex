% version 1.00, date 30/11/16, auteur Kafui Atanley

%La fonctionnalité 11 concerne le module de statistiques concernant les données de l’Uni-
%cef. Ce module de statistiques contribuera à effectuer le bilan annuel. Il devra permettre de
%comptabiliser le nombre d’établissement partenaires ainsi que leur type. Il devra aussi per-
%mettre de comptabiliser le nombre de personnes ayant été sensibilisé. Il devra permettre de
%comptabiliser le nombre d’intervention effectué, les thèmes les plus plebiscités en fonction
%de critère tel que le type d’établissement, le niveau scolaire des participants. Il devra enfin
%permettre de comptabiliser les ventes effectuées à l’année en fonction de la zone géogra-
%phique et du chiffre d’affaire. Il devra être possible de générer un rapport au format PDF ou
%au format CSV figurant chacune de ces données pour l’année actuelle. L’équipe Projet INSA
%Certifié devra être force de proposition compte à la représentation graphique de celle-ci de
%celle-ci.
\subsection{Fonctionnalité 11}

La fonctionnalité 11 consistera à mettre en place un module de statistique. Ce module de statistique étant utilisé pour établir une bilan de fin d'année, il devra être le plus précis possible.
Ce module de statistique devra porter sur la gestion des interventions et des ventes associées.
Une attention particulière sera porté sur la gestion des devises.
Il devra être possible de :
\begin{itemize}
\item comptabiliser les interventions de type action éducative par thème, type d'établissement, nombre de participant;
\item comptabiliser les interventions de type frimousse par type d'établissement, nombre de participant, chiffre d'affaire engrangé;
\item comptabiliser les ventes par chiffre d'affaire.
\end{itemize}
Ces données devront pouvoir être générées par année. La représentation graphique de ces données devra être faite à l'aide de diagramme circulaire et/ou de diagramme en barre. La maquette \ref{fonctionnalite11Stat} présente comment sera figurée la liste des ventes. La repésentation graphique pourra également être figurée de manière alternative selon le temps restant en PIC.
Il devra être possible d'avoir des tableaux représentant ces données. Ces tableaux devront pouvoir être exportés au format CSV et PDF.

\begin{figure}[H]
	\centering
	\includegraphics[scale=0.4]{images/maquettes/fonctionnalite11Stat.png}
	 \caption{Maquette~: Maquette du module de statistiques}
	 \label{fonctionnalite11Stat}
\end{figure}
