  % version 1.00	date 29/01/2016  	Auteur Sergi Colomies

\section*{Informations générales}
 
\begin{table}[H]
\centering
	\begin{tabularx}{16.8cm}{|X|X|}
	\hline
	\rowcolor{gray!40} Numéro du risque & Type du risque \\
	\hline
	011 & Retard de remise de données de la part du client \\
	\hline
	\end{tabularx}
\end{table}

\begin{table}[H]
\centering
	\begin{tabularx}{16.8cm}{|X|X|X|}
	\hline
	\rowcolor{gray!40} Date & Visa du \RQ & Visa du \CP \\
	\hline
	 29/01/2016 & pgpic & pgpic \\
	\hline
	\end{tabularx}
\end{table}

\begin{table}[H]
\centering
	\begin{tabularx}{16.8cm}{|X|X|X|X|}
	\hline
	\rowcolor{gray!40} Pilote & Activité WBS & Compte WBS & Phase d'apparition \\
	\hline
	 \Sergi & Suivre les Risques et Opportunités & 1.2.3.2 & À tout moment\\
	\hline
	\end{tabularx}
\end{table}

\section*{Description du risque}

\subsection*{Résumé}
	Le risque lié au retard des données peut impacter sur la durée critique du projet. Celui-ci pourrait intervenir sur n'importe quel phase étant donnée que le terme donnée est ici générique.
	
\subsection*{Analyse des causes}
	voir figure \ref{risque retard data}.

\subsection*{Criticité}

\begin{table}[H]
\centering
	\begin{tabularx}{16.8cm}{|>{\columncolor{gray!40}}X|X|}
	\hline
	Gravité & 2\\
	\hline
	Probabilité & 2\\
	\hline
	Criticité & A surveiller \\
	\hline
	\end{tabularx}
\end{table}
\newpage

\section*{Actions}
\subsection*{Actions préventives}

%\begin{table}[H]
\centering
	\begin{longtable}{|p{7cm}|p{7cm}|}
	\hline
	\rowcolor{gray!40} Numéro de cause & Actions préventives \\
	\hline
	 1 & \begin{itemize}
	 	\item Demander un envoi de données de deux manières différentes
	 \end{itemize} \\
	\hline
	2 & \begin{itemize}
		\item Voir les actions préventives au risque associé 
	\end{itemize} \\
	\hline
	3 & \begin{itemize}
		\item Voir actions préventives pour le risque associé
	\end{itemize} \\
	\hline
	\end{longtable}
%\end{table}

\flushleft
\subsection*{Plan de contournement}

\begin{enumerate}
	\item Utiliser des données test en attendant un retour du client
\end{enumerate}

\section*{Décision de clôture}
Par le \CP{} et le pilote du risque.
\begin{table}[H]
\centering
	\begin{tabularx}{16.8cm}{|X|X|}
	\hline
	\rowcolor{gray!40} Date de clôture & Raison de la clôture \\
	\hline
	 14/03/2016 & Données reçues\\
	\hline
	\end{tabularx}
\end{table}

\section*{Historique des modifications}
\begin{table}[H]
\centering
	\begin{tabularx}{16.8cm}{|X|X|}
	\hline
	\rowcolor{gray!40} Date & Modification \\
	\hline
	 14/03/2016 & clôture \\
	\hline
	\end{tabularx}
\end{table}
\newpage

%\begin{landscape}
\begin{figure}
	\centering
	\includegraphics[scale=0.30]{images/nPourquoiFDR011}
	\caption{\label{risque retard data} Retard de remise des données - méthode des n pourquoi}
\end{figure}
%\end{landscape}
