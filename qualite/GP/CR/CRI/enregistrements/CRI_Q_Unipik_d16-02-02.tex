\documentclass [a4paper] {article}
\usepackage[utf8]{inputenc}
\usepackage[francais]{babel}
\usepackage[top=2cm, bottom=4cm, left=2cm, right=2cm]{geometry} 
\usepackage{fancyhdr}
\usepackage{graphicx}
\usepackage{color, colortbl}
\usepackage{longtable}
\usepackage{../../../../ressources/Unipik/vocabulaire/vocabulaireUnipik}
\pagestyle{fancy}
\definecolor{Gray}{gray}{0.8}



%--- En-t�te et pied de page ---%

\renewcommand{\footrulewidth}{0,01cm}
\rhead{}
\chead{\huge{Compte-rendu de réunion interne}}					%titre
\begin{document}

03/02/2016			 				%Date 
\hfill   
\hfill 	 13:59 - 15:30 				%Heure de d�but, heure de fin.


\lfoot{Version : 1.00} 			% Rédacteur
%--- Fin en-t�te et pied de page ---%

\section*{Historique des révisions}
\begin{center}
			\begin{tabular}{| c | c | c | c | p{4cm} |}
				\hline
				\rowcolor{Gray}
				Version & Date & Auteur(s) & Modification(s) & Partie(s) modifiée(s)		 \\
				\hline
				1.00 & 03/02/2016 & \Kafui & Création & Toutes \\
		\hline		
			\end{tabular}
		\end{center}

\section*{Signatures}

		\begin{center}
			\begin{tabular}{| c | c | c | c | p{4cm} |}
				\hline
				\rowcolor{Gray}
				Rôle & Fonction & Nom & Date & Visa		 \\
				\hline
				Vérificateur & \RQA & \Pierre &  & courriel \\[30pt]
				\hline
				Validateur & \CP & \Sergi & & courriel \\[30pt]	
				\hline
			\end{tabular}
		\end{center}
		
\newpage		



%--- Réunion --%

%\section{Tour de table}
%Les fonctions sont réparties comme suit :
%\begin{itemize}
%	\item Michel Cressant : \RD (\RDCourt),
%	\item Matthieu Martins-Baltar : Responsable Serveur et Réseau (RSR),
%	\item Mathieu Medici : \RGC (\RGCCourt),
%	\item Kafui Atanley : \RQCourt Développeur,
%	\item Mélissa Bignoux : Développeuse,
%	\item Julie Pain : Développeuse,
%	\item Florian Leriche : Développeur.
%\end{itemize}

\section{But de la réunion}
L'équipe s'est réunie afin de se concerter autour du modèle Entité-Association réalisé par Julie Pain, Matthieu Medici et Florian Leriche. Ce modèle a permis de dégager plusieurs questions qui seront posés au client lors de la réunion prévu pour le jour suivant.

\section{Explication du modèle Entité/Association}

\subsection{Cas de l'entité utilisateur}
Il a été décidé de séparer l'entité utilisateur et l'entité contact car un contact ne fait pas forcément parti de l'association contrairement à un utilisateur.
Le type de l'entité utilisateur désigne son statut vis-à-vis de l'application développé là ou ce lui de l'entité contact désigne le métier du contact de l'établissement (enseignant de l'établissement, animateur).
Le groupe pense que l'attribut clef id\_ utilisateur de l'entité Utilisateur et l'attribut clef id\_ contact de l'entité Contact est redondant. Ceux-ci seront donc enlevés du modèle final au profit de la l'attribut id\_ personne de l'entité Personne dont hérite les entités Utilisateur et Contact.

\subsection{Cas de l'entité Etablissement et de ses enfants}
Les entités Enseignement et Centre\_ loisirs ont été différentiés car les centres de loisirs ne dispose pas d'Unité administrative immatriculée d'ou la nécessité de créer un attribut UAI pour l'entité enseignement uniquement.
\\
L'équipe pense qu'il serait bien de faire de figurer le nom de chaque chef de l'établissement. Un attribut respo\_ structure sera donc ajouté à l'entité Etablissement.
Une réflexion est menée autour des attributs contact\_ activité\_ ponctuelle, contact\_ frimousse et contact\_ plaidoyer. Ces attributs pourrait s'avérer vide assez souvent. La solution proposé serait de faire un attribut contact pouvant prendre comme valeur de 3 à Null, ces attributs contiendrait l'id chaque contact.

\subsection{Cas de l'entité Actions Ponctuelle}
L'entité action ponctuelle ne ressemble pas à l'entité Intervention d'où la différenciation. Les détails associés à cette entité ne sont pas assez clairs pour l'équipe. L'équipe décide donc qu'il sera redemandé au client d'explicité ce qui est attendu.
\\
L'équipe ayant travaillée sur le modèle a noté plusieurs questions qui seront également posées au client. 


%--- fin de réunion ---%


\end{document}