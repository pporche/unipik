\documentclass [a4paper] {article}
\usepackage[utf8]{inputenc}
\usepackage[francais]{babel}
\usepackage[top=2cm, bottom=4cm, left=2cm, right=2cm]{geometry} 
\usepackage{fancyhdr}
\usepackage{graphicx}
\usepackage{color, colortbl}
\usepackage{longtable}
\usepackage{../../../../ressources/Unipik/vocabulaire/vocabulaireUnipik}
\pagestyle{fancy}
\definecolor{Gray}{gray}{0.8}



%--- En-t�te et pied de page ---%

\renewcommand{\footrulewidth}{0,01cm}
\rhead{}
\chead{\huge{Compte-rendu de réunion de Tutorat Pédagogique}}					%titre
\begin{document}

29/01/2015			 				%Date 
\hfill   
\hfill 	 9:02 - 10:03 				%Heure de d�but, heure de fin.


\lfoot{Version : 1.00} 			% version
%--- Fin en-t�te et pied de page ---%
\section*{Historique des révisions}
\begin{center}
			\begin{tabular}{| c | c | c | c | p{4cm} |}
				\hline
				\rowcolor{Gray}
				Version & Date & Auteur(s) & Modification(s) & Partie(s) modifiée(s)		 \\
				\hline
				1.00 & 29/01/2016 & \Pierre & Création & Toutes \\
		\hline		
			\end{tabular}
		\end{center}

\section*{Signatures}

		\begin{center}
			\begin{tabular}{| c | c | c | c | p{4cm} |}
				\hline
				\rowcolor{Gray}
				Rôle & Fonction & Nom & Date & Visa		 \\
				\hline
				Vérificateur & \RQA & \Kafui & 29/01/2016 & pgpic \\[30pt]
				\hline
				Validateur & \CP & \Sergi & 29/01/2016 & pgpic \\[30pt]	
				\hline
			\end{tabular}
		\end{center}

%--- Réunion --%


\section{Résumé de la semaine}
Nous avons fini le \PQ{} et le \PGC{} et nous les avons envoyés à M. Pascal Meslier pour approbation.
Le mardi 26 Février, nous avons vu le client. Ce qui est ressorti de cette réunion, c'est que l'\nomClient{} ne se préoccupait pas de la méthode de réalisation du projet mais seulement du livrable. Concernant celui ci, le client nous a exprimé son vif désir d'avoir un livrable fonctionnel et utilisable à la fin du premier semestre de PIC.
\\
Nous avons revu avec le client le cahier des charges afin de lui soumettre nos questions et cela nous a éclairé sur les attentes du client.

~

Nous avons ensuite entamé la phase de spécification en nous séparant en trois groupes : 
\begin{itemize}
\item un groupe sur le \DSE{}, le \DSI{} et le \PTV{} ; 
\item un groupe sur le modèle Entité/Association ;
\item un groupe sur la réalisation d'une maquette.
\end{itemize}

Concernant le DSE, M. \nomTuteurPedago{} nous explique que celui ci sera assez rapide à faire puisque nous disposons d'un cahier des charges complet et précis. Il nous conseille donc de se concentrer sur le \PTV{}.
\\
Concernant la réalisation de la maquette, \nomTuteurPedago{} nous explique qu'elle n'a de sens qu'une fois les fonctionnalités de notre application définies. Il faut donc réaliser le \DSICourt{} et définir l'architecture avant. Cela permettra de réaliser les maquettes en ayant hiérarchisé nos fonctionnalités en terme de dépendances.

~

Nous avons également fait une analyse des risques et opportunités (13 risques et 7 opportunités) que nous avons succinctement présenté à \nomTuteurPedago{}.

~

Le dimensionnement total du système n'a pas été fait. Nous aurions dû demander au client un ordre de grandeur du nombre de tuples pour chaque niveau (local, régional, national). Cela permettra de définir les technologies à utiliser et d'avancer sur le \DSICourt{}. Une fois fait, cela permet de proposer une architecture qui a du sens en raisonnant sur les structures physiques de nos données. Il faudra mettre l'architecture en couches et de ce fait, pouvoir faire des sous-groupes selon le même découpage. Il est important de monter en puissance sur le dimensionnement et l'architecture des systèmes d'information.

~

Concernant les fiches de compétences, il faut enlever les compétences qui ne sont pas pertinentes concernant le projet. Il faut se former sur l'IHM en terme de design et les bases de données. Pour cela, le \RQCourt{} doit remplir les fiches de formation avec au moins deux des trois thèmes suivants :
\begin{itemize}
\item dimensionnement et architecture des systèmes d'information;
\item Bases de Données;
\item IHM design.
\end{itemize}
Il faudra ensuite planifier ces formations au plus vite et les mettre en place afin d'éviter de se retrouver au pied du mur.


\section{Revue du GIT}

\nomTuteurPedago{} a fait une "inspection technique" sur le dossier qualité.
\\
Il est à noter qu'il y a des problèmes de chemins absolus. Il n'en faut pas et il faut donc les repérer et les remplacer par des chemins relatifs.
\\
Un autre gros problème est la présence de fichiers inutiles (archives des années passées, doublons, CLS inutiles,...). Ceux ci sont à supprimer.
\\
Les makefile ne fonctionnent pas tous et certains ont des numéros de version directement dans les chemins d'accès. Il faut donc les refaire afin qu'à tout moment, ils soient fonctionnels et modifiables. La prise de bonnes habitudes concernant notre GIT est essentielle puisqu'elle nous permettra d'avancer plus sereinement à l'avenir. Ces tâches sont à la charge du \RD{}.

\section{Objectifs semaine prochaine}
Les points à travailler pour la semaine prochaine sont :
\begin{itemize}
\item Traiter toutes les remarques de \nomTuteurPedago{} remontées par la présente réunion ;
\item Avancer, voir terminer, la rédaction du \PTV{} ;
\item Avancer, voir terminer, la rédaction du \DSI{} ;
\item Terminer la rédaction du \DSE{} ;
\item Terminer la conception du modèle E/A, de préférence avant jeudi, afin de l'envoyer à \nomTuteurPedago{}.
\end{itemize}


%--- fin de réunion ---%
\newpage



\end{document}





















