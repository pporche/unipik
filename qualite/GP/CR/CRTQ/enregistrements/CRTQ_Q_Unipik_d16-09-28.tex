% version 1.00	date 28/09/2016	Auteur Kafui Atanley

\documentclass [a4paper] {article}
\usepackage[utf8]{inputenc}
\usepackage[francais]{babel}
\usepackage[top=2cm, bottom=4cm, left=2cm, right=2cm]{geometry} 
\usepackage{fancyhdr}
\usepackage{graphicx}
\usepackage{color, colortbl}
\usepackage{longtable}
\usepackage{vocabulaireUnipik}
\pagestyle{fancy}
\definecolor{Gray}{gray}{0.8}



%--- En-t�te et pied de page ---%
\renewcommand{\footrulewidth}{0,01cm}

\begin{document}
\rhead{}
\chead{\huge{Compte-rendu de réunion de Tutorat Qualité}}					%titre

28/09/2016
\hfill   
\hfill 	10:43 - 11:04 				% Heure de d�but, heure de fin.


\lfoot{Version : 1.00} 			% version

%--- Fin en-t�te et pied de page ---%
\section*{Historique des révisions}
\begin{center}
			\begin{tabular}{| c | c | c | c | p{4cm} |}
				\hline
				\rowcolor{Gray}
				Version & Date & Auteur(s) & Modification(s) & Partie(s) modifiée(s)		 \\
				\hline
				1.00 & 28/09/2016 & \Kafui & Création & Toutes \\
		\hline		
			\end{tabular}
		\end{center}

\section*{Signatures}

		\begin{center}
			\begin{tabular}{| c | c | c | c | p{4cm} |}
				\hline
				\rowcolor{Gray}
				Rôle & Fonction & Nom & Date & Visa		 \\
				\hline
				Vérificateur & \RGC & \Melissa & -- & -- \\[30pt]
				\hline
				Validateur & \CP & \Pierre & -- & -- \\[30pt]	
				\hline
			\end{tabular}
		\end{center}

%--- Réunion --%

\section{Démarche Qualité}
\paragraph*{}
L'utilisation de pgpic pour vérification, approbation de document est à proscrire. Pgpic ne permet pas de déterminer précisemment la personne ayant donnée son approbation. Nous avons donc une non-conformité de type majeure. Il ne faudra pas modifier tous les documents déjà approuvés.
Redmine dispose maintenant de l'étiquette action de correction, il faudra donc juste préciser dans le champs Sujet si l'action est corrective ou currative.
Il faudra penser à récupérer les questionnaires de la deuxième revue.
Il faut commencer à penser à l'audit de certification.

\section{Avancement du projet}
\paragraph*{}
Le Backend de l'application est bientôt fini, le Frontend de l'application avance bien. La livraison aura lieu le 17 octobre. 

\end{document}















