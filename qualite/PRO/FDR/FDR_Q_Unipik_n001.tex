% version 1.01	date 23/03/2016  	Auteur Pierre Porche
% version 1.00	date 29/01/2016  	Auteur Matthieu Martins-Baltar

\section*{Informations générales}
 
\begin{table}[H]
\centering
	\begin{tabularx}{16.8cm}{|X|X|}
	\hline
	\rowcolor{gray!40} Numéro du risque & Type du risque \\
	\hline
	001 & Crash du serveur \\
	\hline
	\end{tabularx}
\end{table}

\begin{table}[H]
\centering
	\begin{tabularx}{16.8cm}{|X|X|X|}
	\hline
	\rowcolor{gray!40} Date & Visa du \RQ & Visa du \CP \\
	\hline
	 07/12/15 & pgpic & pgpic \\
	\hline
	\end{tabularx}
\end{table}

\begin{table}[H]
\centering
	\begin{tabularx}{16.8cm}{|X|X|X|X|}
	\hline
	\rowcolor{gray!40} Pilote & Activité WBS & Compte WBS & Phase d'apparition \\
	\hline
	 \Matthieu & Suivre les Risques et Opportunités & 1.2.3.2 & À partir de l’installation serveur\\
	\hline
	\end{tabularx}
\end{table}

\section*{Description du risque}

\subsection*{Résumé}
	Le risque lié au crash du serveur peut entraîner la perte de toutes les données et l'impossibilité d'utiliser le service mis en place.
	
\subsection*{Analyse des causes}
	voir figure \ref{risque crash serveur}.

\subsection*{Criticité}

\begin{table}[H]
\centering
	\begin{tabularx}{16.8cm}{|>{\columncolor{gray!40}}X|X|}
	\hline
	Gravité & 4\\
	\hline
	Probabilité & 1\\
	\hline
	Criticité & Critique\\
	\hline
	\end{tabularx}
\end{table}
\newpage

\section*{Actions}
\subsection*{Actions préventives}

%\begin{table}[H]
\centering
	\begin{longtable}{|p{7cm}|p{7cm}|}
	\hline
	\rowcolor{gray!40} Numéro de cause & Actions préventives \\
	\hline
	 1 & \begin{itemize}
	 	\item Boîte à idée
	 	\item Réunions fréquentes
	 	\item Laisser la parole à tout le monde
	 	\item Hiérarchie bien définie
	 	\item Choisir les bons outils de communication
	 	\item Faire des compte rendus de réunion à envoyer au client
	 	\item Reformulation des spécifications
	 	\item Avoir un interlocuteur unique entre le client et le \PICCourt
	 	\item Suivi hebdomadaire de l'avancement
	 \end{itemize} \\
	\hline
	3 & \begin{itemize}
		\item Choisir de bonnes valeurs seuils de présence
		\item Présence et assiduité
		\item Créneau horaire respecté
		\item Objectifs personnels
	\end{itemize} \\
	\hline
	4 & \begin{itemize}
		\item La personne qui fait les tests doit être différente de celle qui code
		\item Faire un maximum de tests
		\item Définir clairement les méthodes de travail dès le début du \PICCourt{} et faire vérifier par un référent
	\end{itemize} \\
	\hline
	5 & \begin{itemize}
		\item Effectuer régulièrement des audits sécurité
		\item Faire des tests en boîte noire et boîte blanche
	\end{itemize} \\
	\hline
	6 & \begin{itemize}
		\item Faire les démarches nécessaires
		\item Être économes
	\end{itemize} \\
	\hline
	7 & \begin{itemize}
		\item Faire un planning
		\item Faire un GANTT/PERT
		\item Chacun respecte son rôle
		\item Bien définir les rôles
		\item Faire des réunions fréquentes
		\item Utiliser les bons outils de partage
	\end{itemize} \\
	\hline
	8 & \begin{itemize}
		\item Avoir une méthode de paiement de secours
		\item Se renseigner auprès des prestataires
		\item Demander au département les méthodes de paiement autorisées
	\end{itemize} \\
	\hline
	9 & \begin{itemize}
		\item Former un bénévole
	\end{itemize} \\
	\hline
	10 & \begin{itemize}
		\item Actions de prévention et de formation
		\item Effectuer des sauvegardes régulières
	\end{itemize} \\
	\hline
	\end{longtable}
%\end{table}

\flushleft
\subsection*{Plan de contournement}

\begin{enumerate}
	\item Bloquer le site
	\item Demander un serveur temporaire
	\item Faire les démarches nécessaires pour les réparations
	\item Relancer le site
\end{enumerate}

\section*{Décision de clôture}
Par le \CP{} et le pilote du risque.
\begin{table}[H]
\centering
	\begin{tabularx}{16.8cm}{|X|X|}
	\hline
	\rowcolor{gray!40} Date de clôture & Raison de la clôture \\
	\hline
	  & \\
	\hline
	\end{tabularx}
\end{table}

\section*{Historique des modifications}
\begin{table}[H]
\centering
	\begin{tabularx}{16.8cm}{|X|X|}
	\hline
	Date & Modification \\
	\hline
	 23/03/2016 & modification actions préventives \\
	\hline
	\end{tabularx}
\end{table}
\newpage

\begin{landscape}
\begin{figure}
	\centering
	\includegraphics[scale=0.24]{images/nPourquoiFDR001}
        \caption{\label{risque crash serveur}risque crash serveur - méthode des n pourquoi}
\end{figure}
\end{landscape}
