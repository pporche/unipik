\documentclass [a4paper] {article}
\usepackage[utf8]{inputenc}
\usepackage[francais]{babel}
\usepackage[top=2cm, bottom=4cm, left=2cm, right=2cm]{geometry} 
\usepackage{fancyhdr}
\usepackage{graphicx}
\usepackage{color, colortbl}
\usepackage{longtable}
\usepackage{../../../../ressources/Unipik/vocabulaire/vocabulaireUnipik}
\pagestyle{fancy}
\definecolor{Gray}{gray}{0.8}



%--- En-t�te et pied de page ---%

\renewcommand{\footrulewidth}{0,01cm}
\rhead{}
\chead{\huge{Compte-rendu de réunion avec le client}}					%titre
\begin{document}

26/01/2015			 				%Date 
\hfill   
\hfill 	 10:02 - 11:13 				%Heure de d�but, heure de fin.


\lfoot{Version : 1.00} 			% version
%--- Fin en-t�te et pied de page ---%
\section*{Historique des révisions}
\begin{center}
			\begin{tabular}{| c | c | c | c | p{4cm} |}
				\hline
				\rowcolor{Gray}
				Version & Date & Auteur(s) & Modification(s) & Partie(s) modifiée(s)		 \\
				\hline
				1.00 & 26/01/2016 & \Pierre & Création & Toutes \\
		\hline		
			\end{tabular}
		\end{center}

\section*{Signatures}

		\begin{center}
			\begin{tabular}{| c | c | c | c | p{4cm} |}
				\hline
				\rowcolor{Gray}
				Rôle & Fonction & Nom & Date & Visa		 \\
				\hline
				Vérificateur & \RQA & \Kafui & 29/01/2016 & pgpic \\[30pt]
				\hline
				Validateur & \CP & \Sergi & 29/01/2016 & pgpic \\[30pt]	
				\hline
			\end{tabular}
		\end{center}

%--- Réunion --%

\section{Tour de table}
Les fonctions sont réparties comme suit :
\begin{itemize}
	\item Sergi Colomies : \CP (\CPCourt) ;
	\item Pierre Porche : \RQ (\RQCourt) + \CPA (\CPACourt) ;
	\item Mathieu Medici : \RGC (\RGCCourt) ;
	\item Michel Cressant : \RD (\RDCourt) ;
	\item Matthieu Martins-Baltar : Responsable Serveur et Réseau (RSR) ;
	\item Mélissa Bignoux : Développeuse ;
	\item Julie Pain : Développeuse ;
	\item Kafui Atanley : Développeur ;
	\item Florian Leriche : Développeur.


\end{itemize}


\section{Relecture du cahier des charges}
Les réponses aux questions concernant le cahier des charges sont les suivantes : 
\begin{itemize}
\item 1.1 : L'UNICEF prend note du fait que l'hébergeur ne propose pas de technologie ;
\item 1.2 : Le terme "léger" signifiait : pas d'installation, adapté à l'utilisation de tous. Ce terme peut être remplacé par "accessible" ;
\item 1.2 : La liste des navigateurs n'est pas exhaustive mais on peut imaginer un certain "seuil" de version minimale des navigateurs ;
\item 1.2 : Il n'y aura pas de test par le client en cours de développement mais seulement au moment de la livraison ;
\item 1.2 : Concernant le mécénat, l'équipe pourrait demander à l'INSA ;
\item 1.2 : Aucun problème pour l'utilisation de MySQL ;
\item 1.3 : l'administrateur global sera le président d'UNICEF et on ne doit développer qu'une structure locale qui sera "copiée" ultérieurement ;
\item 1.4 : Les informations externes correspondent par exemple aux informations concernant un établissement ;
\item 1.5 : Les interlocuteurs ont été mis au courant de l'existence du projet ;
\item 2.1 : La priorité est de rester en accord avec la loi informatique et libertés ;
\item 2.1 : Les VAE sont les Villes Amies des Enfants, elles ont un partenariat avec l'UNICEF ;
\item 2.1 : Si un bénévole fait deux activités et qu'il n'a pas d'adresse mail, un mail est envoyé à tous ses responsables d'activité ;
\item 2.2 : La ville de rattachement est la ville dans laquelle l’établissement est situé ;
\item 2.3.1 : La distance renseignée correspond à celle entre l'établissement et une des villes citées ;
\item 2.3.2 : Il faut que les champs soient pré-remplis ;
\item 2.3.4 : Les rappels par SMS ne sont pas essentiels ;
\item 2.4 : F9 : le lieu de vente est défini par l'établissement qui fait la demande ;
\item 2.5 : L’agrégation correspond à un total des données ;
\item 3.1 : T0 est le 18 janvier 2016 ;
\item 3.1 : le 1er lot correspond à un remplissage de la base de données par des scripts livrés. La date est modifiable mais la livraison du deuxième lot est fixe.
\end{itemize}


\section{Divers}
Il est important de respecter la charte graphique de l'UNICEF car il s'agit d'une marque déposée.


%--- fin de réunion ---%
\newpage



\end{document}





















