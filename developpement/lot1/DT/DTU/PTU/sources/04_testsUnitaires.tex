Ici sont décris les différents tests unitaires qui seront effectués sur le lot 1 :

\section{Test Unitaire TU001}
	Ce test a pour but de se connecter à la base de données via un terminal. Ce test vérifie donc l'existante de la base de données. Afin de se connecter, entrez la commande suivante dans un terminal :\\
	
	 psql -U USER -d UNIBASE -h 127.0.0.1 -W suivi du mot de passe nécessaire. \\
	
	Résultat attendu : La connexion à la base de données UNIBASE. Le terminal devra afficher l'élément suivant : UNIBASE =>

\section{Test Unitaire TU002}
	Ce test a pour but de vérifier si la base de données est remplie via un terminal. Pour cela connecter vous à la base de données UNIBASE et effectuez les commandes suivantes : \\
	
	\begin{itemize}
		\item SELECT * FROM BENEVOLE;
		\item SELECT * FROM ENSEIGNEMENT;
		\item SELECT * FROM CONTACT; 
		\item SELECT * FROM PROJET;
		\item SELECT * FROM CENTRELOISIRS
		\item SELECT * FROM FRIMOUSSE;
		\item SELECT * FROM PLAIDOYER;
		\item SELECT * FROM THEME;
	\end{itemize} 
	
	Résultat attendu : Les tables affichées devront être non vide.	
	
\section{Test Unitaire TU003}
	Ce test a pour but de vérifier qu'il est possible d'ajouter et de supprimer un établissement dans la table ETABLISSEMENT. Pour cela connecter vous à la base UNIBASE et effectuez les commandes suivantes : \\
	
	\begin{itemize}
		\item INSERT INTO ETABLISSEMENT VALUES ('valeur 1', 'valeur 2', ...)
		\item SELECT * FROM ETABLISSEMENT WHERE nom = valeur1 AND telfixe = valeur2 etc;
	\end{itemize}
	
	Résultat attendu : L'établissement ajouté devra apparaître à l'écran. \\
	
	Supprimer ensuite cette établissement : \\
	
	\begin{itemize}
		\item DELETE FROM ETABLISSIMENT WHERE idEtablissement=celui trouvé juste avant
		\item SELECT * FROM ETABLISSEMENT WHERE id= celui trouvé juste avant
	\end{itemize}	 
	
	Résultat attendu : Aucun n'est établissment affiché à l'écran.
	
\section{Test Unitaire TU004}
	Ce test a pour but de vérifier qu'il est possible d'ajouter et de supprimer un bénévole dans la table BENEVOLE. Pour cela connecter vous à la base UNIBASE et effectuez les commandes suivantes : \\
	
	\begin{itemize}
		\item INSERT INTO BENEVOLE VALUES ('valeur 1', 'valeur 2', ...)
		\item SELECT * FROM BENEVOLE WHERE email=valeur1;
	\end{itemize}
	
	Résultat attendu : Le bénévole ajouté devra apparaître à l'écran. \\
	
	Supprimer ensuite ce bénévole : \\
	
	\begin{itemize}
		\item DELETE FROM BENEVOLE WHERE email=valeur1
		\item SELECT * FROM BENEVOLE WHERE email=valeur1
	\end{itemize}	 
	
	Résultat attendu : Aucun bénévole n'est affiché à l'écran.
	
	
\section{Test Unitaire TU005}
	Ce test a pour but de vérifier qu'il est possible d'ajouter et de supprimer un intervention dans la table INTERVENTION. Pour cela connecter vous à la base UNIBASE et effectuez les commandes suivantes : \\
	
	\begin{itemize}
		\item INSERT INTO INTERVENTION VALUES ('valeur 1', 'valeur 2', ...)
		\item SELECT * FROM INTERVENTION WHERE ...;
	\end{itemize}
	
	Résultat attendu : L'intervention ajoutée devra apparaître à l'écran. \\
	
	Supprimer ensuite cette intervention : \\
	
	\begin{itemize}
		\item DELETE FROM INTERVENTION WHERE ...
		\item SELECT * FROM INTERVENTION WHERE ...
	\end{itemize}	 
	
	Résultat attendu : Aucune intervention n'est affichée à l'écran.
	
	
	
\section{Test Unitaire TU006}
	Ce test a pour but de vérifier qu'il est possible d'ajouter et de supprimer un contact dans la table CONTACT. Pour cela connecter vous à la base UNIBASE et effectuez les commandes suivantes : \\
	
	\begin{itemize}
		\item INSERT INTO CONTACT VALUES ('valeur 1', 'valeur 2', ...)
		\item SELECT * FROM CONTACT WHERE ...;
	\end{itemize}
	
	Résultat attendu : Le contact ajouté devra apparaître à l'écran. \\
	
	Supprimer ensuite ce contact : \\
	
	\begin{itemize}
		\item DELETE FROM CONTACT WHERE ...
		\item SELECT * FROM CONTACT WHERE ...
	\end{itemize}	 
	
	Résultat attendu : Aucune contact n'est affiché à l'écran.
	
	
\section{Test Unitaire TU007}
	Ce test a pour but de vérifier qu'il est possible d'ajouter et de supprimer un centre de loisirs dans la table CENTRELOISIRS. Pour cela connecter vous à la base UNIBASE et effectuez les commandes suivantes : \\
	
	\begin{itemize}
		\item INSERT INTO CENTRELOISIRS VALUES ('valeur 1', 'valeur 2', ...)
		\item SELECT * FROM CENTRELOISIRS WHERE ...;
	\end{itemize}
	
	Résultat attendu : Le centre de loisirs ajouté devra apparaître à l'écran. \\
	
	Supprimer ensuite ce centre de loisirs : \\
	
	\begin{itemize}
		\item DELETE FROM CENTRELOISIRS WHERE ...
		\item SELECT * FROM CENTRELOISIRS WHERE ...
	\end{itemize}	 
	
	Résultat attendu : Aucun centre de loisirs n'est affiché à l'écran.
	
	
\section{Test Unitaire TU008}
	Faire des tests du genre interjection et tout et tout