% version 1.00	Date 01/03/2016	Auteur Pierre Porche

\documentclass[11pt]{article}
\usepackage{draftcopy}
\usepackage[francais]{babel}
\usepackage[utf8]{inputenc}
\usepackage{tabularx}
\usepackage{graphicx}
\usepackage[table]{xcolor}
\usepackage{fancyhdr}
\usepackage{vocabulaireUnipik}

\begin{document}
\chead{\huge Fiche de rôle \CPA}

\section*{Introduction}

Tout au long du projet, le \RD{} doit s'occuper de la gestion de toutes les parties concernant le développement. Il aura donc sous sa charge un certain nombre de développeurs et va devoir s'assurer que les résultats issus du développement sont conformes au \DGQDEUXCourt.

\section*{Tâches liées à sa fonction}

Le \RD{} devra remplir les missions suivantes :
\begin{itemize}
	\item Définir avec le \CP{} les différentes phases de conception et de développement;
	\item Établir les phases de vérification et de validation du développement;

	\item Participer à la réalisation du \DSECourt{}, du \DSICourt{} et du \PTVCourt;

	\item S'assurer de la réussite des tests unitaires et d'intégration;
	\item S'assurer que les codes fournis fonctionneront également chez le client.
\end{itemize}

\end{document}