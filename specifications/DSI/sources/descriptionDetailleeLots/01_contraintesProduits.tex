
% version 1.00, date 23/02/16, auteur Matthieu Martins-Baltar
\subsection{Contraintes Produits}
Dans cette sous-partie, nous allons décrire les contraintes produits, c'est-à-dire des contraintes ayant trait au système à proprement parler.\\

L'interface de l'application se présentera sous la forme d'une page web qui sera visualisée à partir d'un navigateur. Les navigateurs principaux, Chrome, Firefox, Safari doivent être compatibles et capables d'afficher correctement la page. L'équipe \PIC{} devra spécifier au client les versions de navigateur compatible avec l'application.  Le code de cette page devra répondre aux normes de la W3C. De plus, l'interface devra être \emph{responsive} et utilisable depuis un smartphone. Enfin, le design de l'interface doit permettre une utilisation simple à un utilisateur inexpérimenté.\\

La partie serveur de l'application devra être légère et pouvoir fonctionner sur un serveur de faible dimension.\\
