% version 1.00	Auteur Pierre Porche		Date 26/04/2016

\documentclass [a4paper] {article}
\usepackage[utf8]{inputenc}
\usepackage[francais]{babel}
\usepackage[top=2cm, bottom=4cm, left=2cm, right=2cm]{geometry} 
\usepackage{fancyhdr}
\usepackage{graphicx}
\usepackage{color, colortbl}
\usepackage{longtable}
\usepackage{vocabulaireUnipik}
\pagestyle{fancy}
\definecolor{Gray}{gray}{0.8}
\definecolor{White}{gray}{1.0}


%--- En-t�te et pied de page ---%

\renewcommand{\footrulewidth}{0,01cm}
\rhead{}
\chead{\huge{Compte-rendu de réunion interne}}					%titre
\begin{document}

26/04/2016			 				%Date 
\hfill   
\hfill 	 8:36 - 9:25				%Heure de d�but, heure de fin.


\lfoot{Version : 1.00} 			% Version
%--- Fin en-t�te et pied de page ---%

\section*{Historique des révisions}
\begin{center}
			\begin{tabular}{| c | c | c | c | p{4cm} |}
				\hline
				\rowcolor{Gray}
				Version & Date & Auteur(s) & Modification(s) & Partie(s) modifiée(s)		 \\
				\hline
				1.00 & 26/04/2016 & \Pierre & Création & Toutes \\
		\hline		
			\end{tabular}
		\end{center}

\section*{Signatures}

		\begin{center}
			\begin{tabular}{| c | c | c | c | p{4cm} |}
				\hline
				\rowcolor{Gray}
				Rôle & Fonction & Nom & Date & Visa		 \\
				\hline
				Vérificateur & \RQA & \Kafui &  & pgpic \\[30pt]
				\hline
				Validateur & \CP & \Sergi &  & pgpic \\[30pt]	
				\hline
			\end{tabular}
		\end{center}


\section{Planning}
Il reste trois semaines et demi d'ici à la revue, donc deux semaines et demi de développement. \Sergi{} va dire au client que nous allons avancer au maximum jusqu'aux vacances quitte à terminer le travail à la rentrée puisque le client n'a besoin de l'outil qu'à partir du mois d'octobre. De plus, il faudrait décaler la fonction de géolocalisation car elle n'est pas essentielle et qu'elle semble compliquée pour le moment. D'ici à son développement, il serait envisageable de faire un lien vers un web-service type OpenStreetMap avec une adresse en entrée.

\section{\CP{} de secours}
Dans l'éventualité où \Pierre{} redoublerait, il faudrait un \CP{} de secours. Il est donc demandé à \Florian, \Melissa, \Matthieu{} et \Julie{} de réfléchir à ce problème.


\section{Top down / Bottom up}
Concernant le bottom-up, le modèle de la base de données est fini à part les triggers mais même avec des tables les plus simples possible, l'ORM ne génère rien malgré les recherches intensives de \Michel. \\
\Kafui a réussi à faire marcher l'ORM en utilisant du MySQL avec des tables très simples mais, l'ORM génère du code de mauvaise qualité avec des tables plus complexes. \\
La solution est d'aller voir \nomTuteurPedago{} et de lui expliquer notre manque de solution autre que de repartir en top-down.


\section{Vues}
Plutôt que développer des vues à la chaîne, l'équipe FrontEnd fait des vues "types" pour pouvoir les intégrer plus tard . 


\section{Revue de risques et opportunités}

La \FDR{} 001 (Crash du serveur) passe d'une probabilité de 1 à 2 car la méthode de programmation choisie peut augmenter la probabilité d'apparition de ce problème. \\

La \FDR{} 002 (Mauvaise ambiance interne) passe d'une probabilité de 1 à 2 à cause des tensions provoquées par le manque de temps et d'une gravité de 3 à 4 car une mauvaise ambiance induirait encore plus de prise de retard.\\

La \FDR{} 004 (Mauvaise communication client) passe d'une probabilité de 3 à 4 car le client ne répond plus aux emails de \Sergi{} depuis deux semaines.\\

Concernant la \FDR{} 006, aucune modification n'est faite mais il est à noter que des réorganisations des équipes par l'unité P3 sont possibles.\\

La \FDR{} 008 (Indisponibilité du client pour la remise des recettes) passe d'une probabilité de 2 à 3 car le client semble avoir perdu de sa disponibilité.\\

La \FDR{} 009 (Serveur extérieur non trouvé) passe d'une probabilité de 3 à 4 car l'échéance se rapproche sans progression dans la recherche.\\

La \FDR{} 015 (Un PC d’un membre de l’équipe tombe en panne) passe d'une probabilité de 4 à 3 car les problèmes d'instabilité sont résolus.\\

La \FDO{} 005 (Bonne planification) passe d'un bénéfice de 3 à 2 car même avec une excellente planification, le timing sera difficile à tenir.

La \FDO{} 006 (Bonne ambiance interne) passe d'une probabilité de 4 à 2 car l'ensemble des membres de l'équipe est fatiguée par cette fin de semestre.



\section{Organisation de la semaine}
\Matthieu, \Julie{} et \Mathieu{} font des vues.\\
\Melissa{} cherche un autre ORM.\\
\Florian{} et \Kafui{} fait du contrôleur.\\
\Michel cherche des solutions au problème de bottom-up.\\

En ce qui concerne la documentation, nous utiliserons PHPdoc et nous ajouterons des commentaires dans le code si celui-ci n'est pas assez clair.



%--- fin de réunion ---%


\end{document}







