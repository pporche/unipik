\section{Exception aux règles de nommage}

Le référentiel \textbf{Ressources} ne sera pas soumis aux règles de nommage de \nomEquipe mais leur nom doit toujours être évocateur du contenu.

\section{Dossiers spécifiques}

\subsection{Dossiers images}

\`{A} la racine du répertoire des documents nécessitant l'inclusion d'images, de diagrammes ou
d'autres figures se trouve un répertoire \textbf{images}. Il est ajouté au \git{} par le
rédacteur du document si besoin est. Dans ce répertoire sont stockés les fichiers projet
(\verb+.dia+, \verb+.xcf+, \verb+.jpg+, etc\dots). Les \verb+.pdf+ de ces fichiers sont générés
à la compilation par le \verb+makefile+.
Les fichiers du répertoire \verb+images+ ne sont pas concernés par les règles de nommage.

\subsection{Dossiers sources}

\`{A} la racine du répertoire des documents nécessitant l'inclusion de sous-fichiers \verb+.tex+
(correspondant à des chapitres, des sections, etc\dots) se trouve un répertoire \textbf{sources} pour
les documents longs nécessitant d'être subdivisés. 
Ces sous-fichiers \verb+.tex+ seront placés dans ce répertoire et ils seront inclus via
le \verb+.tex+ général. Ces sous-fichiers ne sont pas concernés par les règles de nommage.

\subsection{Dossiers annexes}

\`{A} la racine du répertoire des documents nécessitant l'inclusion d'annexes se trouve un répertoire \textbf{annexes}.
Ces documents seront placés dans ce répertoire et ils seront inclus via
le \verb+.tex+ général. Ces sous-fichiers ne sont pas concernés par les règles de nommage.

\subsection{Dossiers pdf}

\`{A} la racine du répertoire de chaque document se trouve un répertoire \textbf{pdf}. C'est
le répertoire dans lequel est placé le \verb+pdf+ généré par la compilation \LaTeX{}. Le \verb+pdf+ n'est pas stocké sur \git{}.

\subsection{Dossiers approbations}

Lorsqu'un document est approuvé par une signature manuscrite, la copie papier de l'enregistrement est archivée dans la salle PIC de  \nomEquipe$\ $et une numérisation de la page de signatures est effectuée et stockée dans le dossier \textbf{approbations}. Les numérisations seront nommées de la manière suivante:
\[
ApprobSignature\_<Nom\ du\ document>
\]

Exemple : $ApprobSignature\_PGC\_Q\_Unipik\_v1.00$

\section{Règles de rédaction des dates}

Les dates figurant dans les documents sont toutes au format \textbf{JJ/MM/AAAA}. Attention à ne pas confondre avec le format de date des  suffixes dans la convention de nommage (dAA-MM-JJ).

