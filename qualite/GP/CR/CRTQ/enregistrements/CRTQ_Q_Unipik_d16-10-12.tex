% version 1.00	date 17/10/2016	Auteur Kafui Atanley

\documentclass [a4paper] {article}
\usepackage[utf8]{inputenc}
\usepackage[francais]{babel}
\usepackage[top=2cm, bottom=4cm, left=2cm, right=2cm]{geometry} 
\usepackage{fancyhdr}
\usepackage{graphicx}
\usepackage{color, colortbl}
\usepackage{longtable}
\usepackage{vocabulaireUnipik}
\pagestyle{fancy}
\definecolor{Gray}{gray}{0.8}



%--- En-t�te et pied de page ---%
\renewcommand{\footrulewidth}{0,01cm}

\begin{document}
\rhead{}
\chead{\huge{Compte-rendu de réunion de Tutorat Qualité}}					%titre

12/10/2016
\hfill   
\hfill 	14:30 - 14:41 				% Heure de d�but, heure de fin.


\lfoot{Version : 1.00} 			% version

%--- Fin en-t�te et pied de page ---%
\section*{Historique des révisions}
\begin{center}
			\begin{tabular}{| c | c | c | c | p{4cm} |}
				\hline
				\rowcolor{Gray}
				Version & Date & Auteur(s) & Modification(s) & Partie(s) modifiée(s)		 \\
				\hline
				1.00 & 17/10/2016 & \Kafui & Création & Toutes \\
		\hline		
			\end{tabular}
		\end{center}

\section*{Signatures}

		\begin{center}
			\begin{tabular}{| c | c | c | c | p{4cm} |}
				\hline
				\rowcolor{Gray}
				Rôle & Fonction & Nom & Date & Visa		 \\
				\hline
				Vérificateur & \RGC & \Melissa & -- & -- \\[30pt]
				\hline
				Validateur & \CP & \Pierre & -- & -- \\[30pt]	
				\hline
			\end{tabular}
		\end{center}

%--- Réunion --%

\section{Avancement du projet}
\paragraph*{}
La livraison aura lieu le 17 octobre, la démonstration à \nomTuteurPedago{} aura lieu le 13 octobre. Nous avons rattrapé notre reatard dans les tâches. 

\section{Audit de certification}
\paragraph*{}
\nomTuteurQualite{} nous indique que celle-ci aura lieu le 2 janvier 2017. Une première réunion aura lieu pour présenter les \RQ s et les \CP s. L'audit durera autour de 2h30 par PIC.
Chacun sera confronté a l'auditeur, il est donc nécessaire d'avoir à minima lu la \DGQ 2, il faudra aussi possiblement lire la norme 9001 afin de pouvoir comprendre l'auditeur.

\end{document}















