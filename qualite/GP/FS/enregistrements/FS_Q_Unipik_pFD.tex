%version 1.00,	date 12/02/2016	auteur(s) Pierre Porche
\documentclass [a4paper] {article}
\usepackage[utf8]{inputenc}
\usepackage[francais]{babel}
\usepackage[top=2cm, bottom=4cm, left=2cm, right=2cm]{geometry} 
\usepackage{fancyhdr}
\usepackage{graphicx}
\usepackage{color, colortbl}
\usepackage{longtable}
\usepackage{vocabulaireUnipik}
\usepackage{tabularx}
\usepackage{xcolor}
\definecolor{Gray}{gray}{0.8}

\pagestyle{fancy}

%--- En-t�te et pied de page ---%

\renewcommand{\footrulewidth}{0,01cm}
\rhead{}
\chead{\huge{Fiche de Suivi}}

\begin{document}

\section*{\Francois}

\centering
	\begin{longtable}{|>{\columncolor{gray!40}}p{2cm}|p{12cm}|}
	\hline
	Semaine 14 & \begin{itemize}
	\item Formation Symfony.
	\item Début formation Bootstrap.
	\item Avancement sur le front-end du formulaire de création d'un bénévole.
	\end{itemize}	 \\
	\hline
	
	Semaine 15 & \begin{itemize}
	\item Suite de formation Bootstrap en contexte, travail sur le front-end.
	\item Installation de PostGis et test en base de données.
	\item Filtre des dates dans la liste des bénévoles.
	\item Début de pagination sur listes.
\end{itemize}	 \\
	\hline
	
	Semaine 16 & \begin{itemize}
	\item Fin de pagination sur liste des interventions sauf avec le tri.
	\item Création page de redirection si Javascript non actif.
	\item Derniers réglages sur les activités potentielles et les responsabilités (ajout d'un bénévole).
\end{itemize}	 \\
	\hline
	
	Semaine 17 & \begin{itemize}
	\item Pagination fonctionne avec tri.
	\item Front-end du formulaire de demande : clean de la page + début de filtre sur les niveaux.
	\item Réunion client du 06/10/2016.
\end{itemize}	 \\
	\hline
	
	Semaine 18 & \begin{itemize}
	\item Travail sur la modification d'une intervention.
	\item Résolution bugs sur l'ajout d'un bénévole.
	\item Résolution bugs sur tâches mineures diverses.
\end{itemize}	 \\
	\hline
	
	Semaine 19 & \begin{itemize}
	\item Préparation de la revue 3.
	\item Résolution demandes client.
\end{itemize}	 \\
	\hline
	
	Semaine 20 & \begin{itemize}
	\item Préparation de la revue 3.
	\item Revue 3 du 4 novembre.
\end{itemize}	 \\
	\hline
	
\end{longtable}

\end{document}