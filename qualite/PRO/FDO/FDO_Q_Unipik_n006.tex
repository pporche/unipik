% version 1.01, date 14/03/16, auteur Michel Cressant Pierre Porche
\section*{Informations générales}
 
\begin{table}[h]
\centering
	\begin{tabularx}{16.8cm}{|X|X|}
	\hline
	\rowcolor{gray!40} Numéro de l'opportunité & Type d'opportunité \\
	\hline
	006 & Bonne ambiance \\
	\hline
	\end{tabularx}
\end{table}

\begin{table}[h]
\centering
	\begin{tabularx}{16.8cm}{|X|X|X|}
	\hline
	\rowcolor{gray!40} Date & Visa du \RQ & Visa du \CP \\
	\hline
	 29/01/2016 & pgpic & pgpic \\
	\hline
	\end{tabularx}
\end{table}

\begin{table}[h]
\centering
	\begin{tabularx}{16.8cm}{|X|X|X|X|}
	\hline
	\rowcolor{gray!40} Pilote & Activité WBS & Compte WBS & Phase d'apparition \\
	\hline
	 \Michel & Suivre les Risques et Opportunités & 1.2.3.2 & Tout au long du projet.\\
	\hline
	\end{tabularx}
\end{table}

\section*{Description de l'opportunité}

\subsection*{Résumé}
	Une bonne ambiance au sein de l'équipe PIC permettrait d'augmenter considérablement la productivité des membres. \\
	
\subsection*{Analyse des causes}
	voir figure \ref{opportunite bonne ambiance}.

\subsection*{Criticité}

\begin{table}[h]
\centering
	\begin{tabularx}{16.8cm}{|>{\columncolor{gray!40}}X|X|}
	\hline
	Bénéfice & 3\\
	\hline
	Probabilité & 3\\
	\hline
	Criticité & Important \\
	\hline
	\end{tabularx}
\end{table}
\newpage

\section*{Actions}
\subsection*{Actions proactives}

%\begin{table}[H]
{\centering
	\begin{longtable}{|p{7cm}|p{7cm}|}
	\hline
 	\rowcolor{gray!40} Cause & Actions proactives \\
	\hline
	 Bonne communication & \begin{itemize}
	 	\item Maintenir un rythme régulier de réunions collectives.
	 	\item Mettre en place des entretiens individuels.
	 	\item Mettre en place du team-building.
	 \end{itemize} \\
	\hline
	Equipe impliquée dans le projet & \begin{itemize}
		\item Mettre en place du team-building.
	\end{itemize} \\
	\hline
	Bonne planification & \begin{itemize}
		\item Mettre les indicateurs à jour.
		\item Effectuer un suivi régulier du travail des membres.
	\end{itemize} \\
	\hline
	\end{longtable}}
%\end{table}

\section*{Décision de clôture}
Par le \CP{} et le pilote du risque.
\begin{table}[h]
\centering
	\begin{tabularx}{16.8cm}{|X|X|}
	\hline
	\rowcolor{gray!40} Date de clôture & Raison de la clôture \\
	\hline
	  & \\
	\hline
	\end{tabularx}
\end{table}

\section*{Historique des modifications}
\begin{table}[h]
\centering
	\begin{tabularx}{16.8cm}{|X|X|}
	\hline
	\rowcolor{gray!40} Date & Modification \\%\rowcolor{gray!40} 
	\hline
	11/05/2016  & probabilité augmentée car le stress et la fatigue sont retombés \\
	\hline
	27/04/2016  & probabilité baissée car stress et fatigue \\
	\hline
	14/03/2016  & criticité \\
	\hline
	\end{tabularx}
\end{table}
\newpage


\begin{figure}
	\centering
	\includegraphics[scale=0.35]{images/nPourquoiFDO006}
	\caption{\label{opportunite bonne ambiance}Opportunité bonne ambiance - méthode des n pourquoi}
\end{figure}