% version 1.00	Date 01/03/2016	Auteur Florian Leriche

\documentclass[11pt]{article}
\usepackage{draftcopy}
\usepackage[francais]{babel}
\usepackage[utf8]{inputenc}
\usepackage{tabularx}
\usepackage{graphicx}
\usepackage[table]{xcolor}
\usepackage{fancyhdr}
\usepackage{vocabulaireUnipik}

\begin{document}


\chead{\huge Fiche de rôle \CPA}

\section*{Introduction}

Au début du projet, le \RRS{} doit s'occuper de la mise en place du réseau dans la salle \PICCourt afin que chaque membre ait accès à internet et puisse travailler dans de bonnes conditions. Il sera chargé de maintenir ce réseau fonctionnel tout au long du projet.

\section*{Tâches liées à sa fonction}

Le \RRS{} devra remplir les missions suivantes :
\begin{itemize}
	\item Mettre en place le réseau en salle \PICCourt;
	\item S'assurer du bon fonctionnement du réseau tout au long du \PICCourt;
	\item S'assurer du bon état de fonctionnement du serveur;
	\item Faire évoluer l'architecture du réseau en fonction des besoins utilisateurs;
	\item Participer à la gestion technique des équipements. Cela consiste à réceptionner les matériels informatiques et de télécommunications, les tester, les adapter, les insérer dans le réseau en fonctionnement et effectuer le suivi du parc de matériels.
\end{itemize}

\end{document}
