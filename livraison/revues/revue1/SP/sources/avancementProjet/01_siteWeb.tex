\begin{frame}
  \frametitle{Pourquoi une application web ?}
  \begin{itemize}
    \item Parc informatique limité et diversifié.
      \begin{enumerate}
        \item Ne nécessite qu'un navigateur web.
        \item Prise en compte des OS évoqués (Windows, Mac OS, Linux).
        \item Pas de restriction au niveau du matériel informatique.
      \end{enumerate}
    \item Une solution non intrusive.
      \begin{enumerate}
        \item Ne nécessite qu'un navigateur web.
        \item Aucune installation annexe.
        \item Aucun matériel ou téléchargement supplémentaire requis.
      \end{enumerate}
    \item Niveau de connaissance informatique des acteurs très hétérogène.
      \begin{enumerate}
        \item Pas d'inquiétude concernant une quelconque installation ou configuration.
        \item Pas besoin de se préoccuper des mises à jours.  
      \end{enumerate}
    \item Une solution accessible proche de la gratuité.
    \item Utilisation de technologies issues du monde libre.
  \end{itemize}
\end{frame}

\begin{frame}
  \frametitle{Le choix du langage}
  \begin{center}
    \begin{tabular}[h]{|p{0.45\textwidth}|p{0.45\textwidth}|}
	\hline
	\cellcolor{gray!40}PHP & \cellcolor{gray!40}Java JEE \\\hline
        Serveur de type web & Serveur d'application ou conteneur de servlet \\\hline
        Moins lourd, consomme moins de ressources sur des applications peu complexes & Efficace sur des applications complexes \\\hline
        Evolutions fréquentes sans rétrocompatibilité & Moins d'évolutions et rétrocompatibilité avec Java \\\hline
        Restreint à l'univers web & Formation et utilisation du code une seule fois pour tout \\\hline
        Plus difficile de gérer la sécurité & Controle et validation des données inclue au coeur du langage \\\hline
        Intégration et apprentissage rapide des développeurs & Requiert un bon niveau d'abstraction \\\hline
    \end{tabular}
  \end{center}
\end{frame}

\begin{frame}
  \frametitle{Le choix du framework : Symfony}
  \begin{itemize}
    \item Open source.
    \item Le système de Bundle : Gain de temps et optimisation.
    \item Communauté très active et excellente réputation : bon support.    
    \item Développer en MVC avec une bonne séparation des trois couches.
    \item Tests unitaires et fonctionnels intégrés.
    \item Utilisation de fichiers de configuration simplifiée.
    \item URL rewritting simplifié
    \item Un code modifiable facilement et bien structuré : meilleure maintenance et évolutivité.
  \end{itemize}
\end{frame}
