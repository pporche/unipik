% version 1.00	Auteur Pierre Porche		Date 29/03/2016

\documentclass [a4paper] {article}
\usepackage[utf8]{inputenc}
\usepackage[francais]{babel}
\usepackage[top=2cm, bottom=4cm, left=2cm, right=2cm]{geometry} 
\usepackage{fancyhdr}
\usepackage{graphicx}
\usepackage{color, colortbl}
\usepackage{longtable}
\usepackage{vocabulaireUnipik}
\pagestyle{fancy}
\definecolor{Gray}{gray}{0.8}
\definecolor{White}{gray}{1.0}


%--- En-t�te et pied de page ---%

\renewcommand{\footrulewidth}{0,01cm}
\rhead{}
\chead{\huge{Compte-rendu de réunion interne}}					%titre
\begin{document}

19/04/2016			 				%Date 
\hfill   
\hfill 	 8:41 - 9:04 				%Heure de d�but, heure de fin.


\lfoot{Version : 1.00} 			% Version
%--- Fin en-t�te et pied de page ---%

\section*{Historique des révisions}
\begin{center}
			\begin{tabular}{| c | c | c | c | p{4cm} |}
				\hline
				\rowcolor{Gray}
				Version & Date & Auteur(s) & Modification(s) & Partie(s) modifiée(s)		 \\
				\hline
				1.00 & 19/04/2016 & \Pierre & Création & Toutes \\
		\hline		
			\end{tabular}
		\end{center}

\section*{Signatures}

		\begin{center}
			\begin{tabular}{| c | c | c | c | p{4cm} |}
				\hline
				\rowcolor{Gray}
				Rôle & Fonction & Nom & Date & Visa		 \\
				\hline
				Vérificateur & \RQA & \Kafui & 22/04/2016 & pgpic \\[30pt]
				\hline
				Validateur & \CP & \Sergi & 25/04/2016 & pgpic \\[30pt]	
				\hline
			\end{tabular}
		\end{center}



\section{\FS}
\Sergi{} demande à l'équipe si tout le monde a rempli sa \FS{} et invite les retardataires à le faire au plus vite.


\section{Planning}
Il reste 4 semaines et demi jusqu'à la revue sachant qu'il y a une semaine prévue pour la livraison. Cela donne 3 semaines et demi pour le développement, ce qui implique pour l'équipe d'entrer en période de développement intensif. \\
Il reste à finir la phase de conception en réalisant les diagrammes de navigation dès maintenant et sans attendre la réponse du client. Pour finir cette phase, nous devons également relire les diagrammes de bundles. \\
\Kafui{} a fait le dump de Symfony, il faudra en parler à \nomTuteurPedago. \\

\indent
Après la conception, il faudra passer au développement. il faut se concentrer sur 3 axes : 
\begin{itemize}
\item les plaidoyers ;
\item les bénévoles ;
\item les établissements ;
\end{itemize}
Afin de ne pas partir dans tous les sens, on va procéder par fonctionnalités. \\

\indent
Un chef d'équipe sera choisi dans l'équipe Back End comme dans l'équipe Front End afin d'assurer la bonne avancée de l'équipe. Ces chefs seront respectivement \Florian{} et \Julie{} et cette organisation n'impacte pas le rôle de \Michel{} en tant que \RD{}.



\section{Revue de risques et opportunités}

la \FDR{} 004 (Mauvaise communication avec le client) passe d'une probabilité de 2 à 3 à cause d'un départ en vacances du client.



la \FDR{} 006 (Mauvaise mise en route du second semestre) passe d'une probabilité de 2 à 3 à cause des rattrapages de \Pierre. Une action préventive est de prévoir un autre \CP{} en cas de redoublement.

la \FDR{} 013 (Le livrable ne fonctionne pas chez le client) passe d'une probabilité de 1 à 3 car si nous n'avons pas de serveur, la livraison ne fonctionnera pas.

la \FDR{} 015 (Un PC d’un membre de l’équipe tombe en panne) passe d'une probabilité de 2 à 3 car \Florian{} a quelques problèmes avec son PC.

la \FDO{} 001 (Serveur gratuit) passe d'une probabilité de 2 à 1 car le dossier n'est pas encore envoyé aux personnes concernées.

la \FDO{} 004 (Bonne communication client) passe d'une probabilité de 4 à 3 car les vacances du client semblent se prolonger.
 



%--- fin de réunion ---%


\end{document}







