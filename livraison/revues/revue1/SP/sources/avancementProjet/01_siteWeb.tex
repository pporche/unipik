\begin{frame}
  \frametitle{Choix d'une application web 1/2}
  \begin{itemize}
    \item Parc informatique limité et diversifié
      \begin{itemize}
        \item[$\rightarrow $] Ne nécessite qu'un navigateur web
        \item[$\rightarrow $] Prise en compte des OS évoqués (Windows, Mac OS, Linux)
        \item[$\rightarrow $] Pas de restriction au niveau du matériel informatique
      \end{itemize}
    \item Une solution non intrusive
      \begin{itemize}
        \item[$\rightarrow $] Ne nécessite qu'un navigateur web
        \item[$\rightarrow $] Aucune installation annexe
        \item[$\rightarrow $] Aucun matériel ou téléchargement supplémentaire requis
      \end{itemize}
\end{itemize}
\end{frame}

\begin{frame}
  \frametitle{Choix d'une application web 2/2}
  \begin{itemize}
    \item Niveau de connaissance informatique des acteurs très hétérogène
      \begin{itemize}
        \item[$\rightarrow $] Pas d'installation ou de configuration
        \item[$\rightarrow $] Pas de mises à jour à installer
      \end{itemize}
    \item Une solution accessible proche de la gratuité
    \item Des technologies issues du monde libre
  \end{itemize}
\end{frame}

\begin{frame}
  \frametitle{Choix du langage}
  \begin{center}
    \begin{tabular}[h]{|p{0.45\textwidth}|p{0.45\textwidth}|}
	\hline
	\cellcolor{blue!15}PHP & \cellcolor{blue!15}Java J2EE \\\hline
        Serveur de type web & Serveur d'application ou conteneur de servlet \\\hline
        Moins lourd, consomme moins de ressources sur des applications peu complexes & Efficace sur des applications complexes \\\hline
        Evolutions fréquentes sans rétrocompatibilité & Moins d'évolutions et rétrocompatibilité avec Java \\\hline
        Langage adapté à l'univers web & Formation et utilisation du code une fois pour toute \\\hline
        Plus difficile de gérer la sécurité & Contrôle et validation des données inclue au coeur du langage \\\hline
        Intégration et apprentissage rapide des développeurs & Nécessité d'un bon niveau d'abstraction \\\hline
    \end{tabular}
  \end{center}
\end{frame}

\begin{frame}
  \frametitle{Choix du framework : Symfony}
  \begin{itemize}
    \item Un framework open source
    \item Un système de Bundle : gain de temps et optimisation
    \item Une communauté très active et excellente réputation : bon support  
    \item Un développement en MVC avec une bonne séparation des trois couches
    \item Des tests unitaires et fonctionnels intégrés
    \item Des fichiers de configuration simplifiés
    \item Un routing organisé et simplifié
    \item Un code modifiable facilement et bien structuré : meilleure maintenance et évolutivité
  \end{itemize}
\end{frame}
