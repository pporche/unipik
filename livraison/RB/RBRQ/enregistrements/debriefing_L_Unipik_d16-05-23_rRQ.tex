% version 1.00	Auteur Sergi Colomies

\documentclass[asi]{picInsa}
\DeclareGraphicsRule{*}{pdf}{*}{}
\usepackage{pdfpages}


\usepackage{vocabulaireUnipik}

\setcounter{secnumdepth}{4}
\setcounter{tocdepth}{4}
\newcommand{\ligneMaj}[3] {
	\rowcolor[gray]{0.55} \textbf{\textit{#1}} & #2  &  #3\\
	\hline
}
\newcommand{\ligneSup}[3] {
	\rowcolor[gray]{0.65} |\textunderscore \textbf{\textit{#1}} & #2  &  #3\\
	\hline
}
\newcommand{\ligneMed}[3] {
	\rowcolor[gray]{0.75} \hspace{0.25cm} |\textunderscore #1  & #2 & #3 \\
	\hline
}
\newcommand{\ligneSub}[3] {
	\rowcolor[gray]{0.85}  \hspace{0.5cm} |\textunderscore #1 & #2 & #3\\
	\hline
}
\newcommand{\ligneSubSub}[3] {
	\rowcolor[gray]{0.95}  \hspace{0.75cm} |\textunderscore #1 & #2 & #3\\
	\hline
}
\newcommand{\ligneTache}[3] {
	\hspace{1.00cm} |\textunderscore #1 & #2 & #3\\
	\hline
}


\titreGeneral{Débriefing \RQ{}}
\sousTitreGeneral{\nomEquipe}
\titreAcronyme{\Huge{Débriefing}}
\version{v1.00}
\titreDetaille{\large{debriefing\_L\_Unipik\_d16-05-23\_rRQ}}
\referenceVersion{debriefing\_L\_Unipik\_d16-05-23\_rRQ}
\auteurs{\Pierre{}}
\destinataires{Unité P3}
\resume{Le présent document contient la présentation du débriefing \RQ{} \nomEquipe.}
\motsCles{débriefing, \RQ, gestion de projet}
\natureDerniereModification{Création}
\modeDiffusionControle{}

\begin{document}

\couverture{}

\informationsGenerales{}
%% version 1.00, date 29/02/16, auteur Michel Cressant
\begin{pagesService}
	\begin{historique}
		% nouvelles versions à rajouter AU-DESSUS en recopiant les lignes suivantes et en les modifiant :
		\unHistorique{1.00}{02/02/2016}{\Sergi \newline \Pierre \newline \Michel}{Création}{Toutes}

	\end{historique}

        \begin{suiviDiffusions}

            % On place ici les diffusions
        	\unSuivi{1.00}{}{\nomEquipe{}, \nomPIC{}}
          
          
        \end{suiviDiffusions}

%%Signataires
        \begin{signatures}
	   \uneSignature{Vérificateur}{\RQ{},\newline \CPA}{\Pierre{}}{26/01/2016}{PGPic}
       \uneSignature{Validateur}{\CP{}}{\Sergi}{26/01/2016}{PGPic}
	   \uneSignature{Approbateur}{Le client}{\nomClient}{}{}
        \end{signatures}
	
	
	\section*{Documents de Références}
%\begin{documentsReference}
		\begin{listeDeReferences}
			\uneReference{NF EN ISO 9001}{Octobre 2015}
			\uneReference{\MQ{}}{ASI-MQ-MQASI}
			\uneReference{\DGQ{} du Processus "\DGQDEUX{}"}{ASI-DGQ-DGQ2}
			\uneReference{\PTU }{\PTUCourt\_ S\_\nomEquipe\_V1.00 }
			\uneReference{NF EN ISO 9000}{Octobre 2015}
		\end{listeDeReferences}
%	\end{documentsReference}	
	
\begin{terminologie}
		La terminologie (définitions et abréviations) utilisée dans le présent document est centralisée dans le \MQ{} \ASICourt{} (\emph{cf. ASI-MQ-MQASI}) de l'Unité P3.

		\begin{listeDAbreviations}
			\uneAbreviation{\asiCourt}{\asi}
			\uneAbreviation{\DSECourt}{\DSE}
			\uneAbreviation{\DGQDEUXCourt}{\DGQDEUX}
			\uneAbreviation{\INSACourt}{\INSA}
			\uneAbreviation{\MQCourt}{\MQ}
			\uneAbreviation{\PICCourt}{\PIC}
			\uneAbreviation{\PTVCourt}{\PTV}
			\uneAbreviation{MVC}{Modéle-Vue-Controleur}
			
		\end{listeDAbreviations}
	\end{terminologie}
	

	
	
\end{pagesService}

\begin{pagesService}
	\begin{historique}
		% nouvelles versions à rajouter AU-DESSUS en recopiant les lignes suivantes et en les modifiant :
		\unHistorique{1.00}{25/04/2016}{\Pierre{}}{Création}{Toutes}

	\end{historique}


%%Signataires
        \begin{signatures}
	   \uneSignature{Vérificateur}{\RQA}{\Kafui}{}{pgpic}
       \uneSignature{Validateur}{\CP}{\Sergi}{}{pgpic}
        \end{signatures}
\end{pagesService}

\tableofcontents

\setcounter{chapter}{0}

\chapter{Introduction}
\label{Introduction}

Les \PIC{} sont organisés par l'unité P3 du département \ASI{} de l'INSA de Rouen à partir du deuxième semestre de la quatrième année. Un produit conforme à un cahier des charges fourni par un client devra être réalisé par une équipe de neuf étudiants. Afin de structurer le travail au sein de l'équipe, différents rôles ainsi qu'une hiérarchie ont été mis en place, parmis lesquels ceux de \CP{} et \RQ{}. Ce document traite de ce dernier, que j'ai occupé ce semestre.
\paragraph*{}
L'unité P3 prétendant à la certification ISO 9001 : 2015, la gestion de la Qualité est un des enjeux majeurs des \PICCourt{}. Le rôle du \RQ{} est de veiller au respect de cette norme via notamment le respect de la DGQ2 et l'établissement de trois principales mission du \RQCourt{} :
\begin{itemize}
\item la mise en place d'un \SMQ{} ; 
\item l'amélioration continue ;
\item le suivi de la Qualité.
\end{itemize}
\paragraph*{}
Ce document va d’abord présenter le projet du PIC \nomEquipe{} pour lequel j’ai exercé le rôle de \RQ{}, puis s’étendra sur les différentes tâches qui ont été attribuées à mon rôle, pour conclure sur le résultat de l’audit qualité et son explication.


\chapter{Le projet PIC UNICEF}
\label{le_projet}

\section{Présentation du projet}
\subsection*{Le client}
L'\nomClient{} Seine Maritime est un comité de l'\nomClient{} qui est une agence de l'Organisation des Nations Unies (ONU) consacrée à l'amélioration et à la promotion de la condition des enfants. Elle a activement participé à la rédaction, la conception et la promotion de la Convention relative aux droits de l'enfant (CIDE), adoptée lors du sommet de New York le 20 novembre 1989. L'\nomClient{} a reçu le prix Nobel de la paix le 12 janvier 1965. \\
Les principales actions de l'\nomClient{} sont : \begin{itemize}
\item l'éducation des filles,
\item la vaccination et la lutte contre le SIDA et le VIH,
\item la protection de l'enfance,
\item la santé des nouveau-nés,
\item l'égalité hommes-femmes.
\end{itemize}
Les autres priorités traitent de la place de l'enfant dans la famille, de la pratique sportive.\\
En particulier, l'\nomClient{} Seine Maritime met en place des actions de plaidoyer afin de sensibiliser les enfants des écoles du département aux problèmes ci-dessus. Ce comité organise également des actions de ventes afin de lever des fonds pour l'\nomClient{} ainsi que des actions ponctuelles pour sensibiliser un public le plus large possible.

\subsection*{Ses besoins}
Dans le cadre des activités citées précédemment, l'\nomClient{} Seine Maritime et, à plus long terme, l'\nomClient{} France a besoin d'un outil afin d'en informatiser la gestion. Les utilisateurs de l'outil ayant un niveau de connaissance en informatique et un parc informatique très hétérogène, notre outil se devra d'être le plus portable et le plus intuitif possible. Les fonds de l'\nomClient{} ont vocation à remplir les missions qui lui sont attribuées, ainsi, la solution apportée devra être le plus proche possible de la gratuité avec, notamment, l'utilisation de technologies issues du monde libre. \newpage

\section{Présentation de l'équipe}
A l’issue de la réunion de répartition des sujets PIC, notre équipe était composée de
neuf personnes. Le choix du \CP{} s’est fait au bénéfice de \Sergi{}, qui était la personne la plus apte à remplir ce poste.
Le poste de \RQ{} m'a attiré pour son côté gestion du projet ainsi que pour la partie organisation et formalisation.
L’organisation de l’équipe pour ce premier semestre était donc la suivante : Figure \ref{organigramme}.

\begin{figure}[H]
	\includegraphics[scale=0.35]{images/organigrammeFonctionnel.pdf}
	\caption{Organigramme fonctionnel}
	\label{organigramme}
\end{figure}
~\\


\section{Résultats attendus}
Après concertation avec le client, il a été décidé que pour des raisons de portabilité et d'accessibilité, une application web serait développée. Celle ci sera concue en PHP avec utilisation du Framework Symfony et reposera sur une architecture 3 tiers et un patron de conception MVC. Le lotissement s’effectuera de la manière suivante :
\begin{itemize}
\item Le lot 1 couvre l'architecture matérielle et logicielle du projet ;
\item Le lot 2 couvre la gestion des utilisateurs et des plaidoyers ;
\item Le lot 3 couvre la gestion des frimousses ;
\item Le lot 4 couvre la gestion des actions ponctuelles.
\end{itemize}

La majeure partie du projet en terme de fonctions sera donc effectué au premier semestre et le second semestre se concentrera plus sur la partie IHM design et finalisation du projet, ce qui est cohérent avec la légère baisse d’effectif entre semestres.


\chapter{Descriptif du poste de \RQ}
\label{Descriptif}

\section{Objectifs qualité}
Une des priorités du PIC étant la satisfaction du client, nous y avons porté une attention particulière dès la mise en place du \SMQ{} en début de semestre 1. Afin d’identifier les besoins de notre client, qui n'est pas un client technique, et d'y répondre, nous avons mis en place de nombreux outils de gestion de la Qualité. Cette attention particulière se reflète dans les objectifs qualités que nous avons définis dans notre Plan Qualité dès la première version (PQ\_Q\_Unipik\_v1.00).


\section{Gestion de projet : attribution des rôles}
Les rôle à préciser obligatoirement lors de la création des équipes PIC sont le poste de \CP{} qui est revenu à \Sergi{}, ceux de \CPA{} et de \RQ{} qui sont revenus à \Pierre{} et celui de \RQA{} qui est revenu à \Kafui{}. Cela dit, un bon nombre de rôle essentiels ont du être définis par le duo \CP{} et \RQ{} après le début du projet. Le rôle de \RGC{} est revenu à \Mathieu{}, celui de \RD{} à \Michel, celui de \RRS{} à \Matthieu{} et celui de \RS{} à \Florian{}.

\section{Indicateurs}
Les indicateurs permettent, par le biais de valeurs cibles et seuils, de lever des alertes lorsque certains objectifs suivants ne sont pas remplis :
\begin{itemize}
\item Avancement du projet ;
\item Bonne communication ;
\item Respect des délais ;
\item Efficacité du traitement des Faits Techniques ;
\item Conformité des produits.
\end{itemize}
La valeur cible représente la meilleure valeur que peut prendre cet indicateur, celle qui est obtenue si tout se passe au mieux. La valeur seuil représente la valeur limite que peut prendre l’indicateur : si elle est dépassée, un objectif risque de ne pas être rempli et un fait technique doit être levé afin d’éliminer la cause de cette anomalie.

Les indicateurs fixés pour le projet \nomEquipe{} sont donc liés à ces objectifs et ils sont détaillés dans la figure \ref{indicateurs}.

\begin{table}

\begin{tabular}[H]{|p{0.35\textwidth}|p{0.35\textwidth}|p{0.2\textwidth}|}
	\hline
		\rowcolor[gray]{0.45} Intitulé & Valeur cible & Valeur seuil \\\hline
	\rowcolor[gray]{1}
	\multicolumn{3}{|c|}{\cellcolor[gray]{0.65} Indicateurs hebdomadaires} \\\hline

	\multicolumn{3}{|c|}{Processus-Objectif} \\\hline
	\multicolumn{3}{|c|}{\cellcolor[gray]{0.85} Gestion de Projet} \\
	\multicolumn{3}{|c|}{\cellcolor[gray]{0.85} Assurer l'avancement du projet et le respect des délais} \\\hline
	Volume horaire insuffisant & 27h (CP \& RQ) & 22h (CP \& RQ)  \\
	 & 22h (autres) & 17h (autres)  \\\hline
	Retard de tâches & 0\% & 30\% \\\hline
	\multicolumn{3}{|c|}{\cellcolor[gray]{0.85} Gestion de Projet} \\
	\multicolumn{3}{|c|}{\cellcolor[gray]{0.85} Assurer une bonne communication} \\\hline
	Délai entre la tenue de la réunion et l'émission du CR & 4 jours & 7 jours \\\hline
	\multicolumn{3}{|c|}{\cellcolor[gray]{0.85} Management de la qualité} \\
	\multicolumn{3}{|c|}{\cellcolor[gray]{0.85} Assurer un bon système de gestion de la qualité} \\\hline
	Délai entre la date de correction de FT annoncé et la date réelle & 0 jours & 7 jours d'écart \\\hline
	
	\hline
	\rowcolor[gray]{0.85}
	\multicolumn{3}{|c|}{\cellcolor[gray]{0.65} Indicateurs ponctuels} \\\hline 
	\multicolumn{3}{|c|}{Processus-Objectif} \\\hline
	\multicolumn{3}{|c|}{\cellcolor[gray]{0.85} Gestion de Projet} \\
	\multicolumn{3}{|c|}{\cellcolor[gray]{0.85} Assurer l'avancement du projet et le respect des délais} \\\hline
	Écart entre la date prévue de livraison et la date effective de livraison & 0 jour ouvré & 7 jours ouvrés  \\\hline
	Durée de la période probatoire & 2 jours ouvrés & 7 jours ouvrés \\\hline
	\multicolumn{3}{|c|}{\cellcolor[gray]{0.85} Gestion de Projet} \\
	\multicolumn{3}{|c|}{\cellcolor[gray]{0.85} Assurer une bonne communication} \\\hline
	Note de satisfaction client suite au questionnaire de satisfaction de l'unité P3 & 17 (note sur 20) & 14 (note sur 20) \\\hline
	\multicolumn{3}{|c|}{\cellcolor[gray]{0.85} Management de la qualité} \\
	\multicolumn{3}{|c|}{\cellcolor[gray]{0.85} Assurer un bon système de gestion de la qualité} \\\hline
	Nombre de non-conformités (documents de spécification, audits de code, SMQ, Gestion des Configurations & 0 & 0 \\\hline
	Nombre de remarques (documents de spécification, audits de code, SMQ, Gestion des Configurations) & 0 & 15 \\\hline
	
\end{tabular}

\caption{Indicateurs du projet \nomEquipe{}} \label{indicateurs}
\end{table}



\section{Suivi des risques et opportunités}
Lors de l'accomplissement d'un projet, la gestion des risques permet de minimiser les probabilité d’apparition d'une difficulté et d'être prêts à mettre en place les actions permettant de passer outre. La gestion des opportunités, quand à elle, permet d'augmenter les probabilité d’apparition d'un fait bénéfique pour le projet et de profiter au maximum de son apparition. \\
Les risques et opportunités sont ensuite consigné dans un \PRO{}. A la fin du semestre, ce dernier comportait les risques suivants :
\begin{itemize}
\item Crash du serveur ;
\item Mauvaise ambiance interne ;
\item Absence collective ;
\item Mauvaise communication client ;
\item Mauvaise planification ;
\item Mauvaise mise en route du second semestre ;
\item Perte de documents ;
\item Indisponibilité du client pour la remise des recettes ;
\item Serveur extérieur non trouvé ;
\item Problème avec la CNIL ;
\item Retard de remise des données (clôturé le 14/03/2016 après remise des données par le client) ;
\item Retard de remise du livrable ;
\item Le livrable ne fonctionne pas chez le client ;
\item Le serveur n’implémente pas PostgreSQL ni PostGis ;
\item Un PC d’un membre de l’équipe tombe en panne ;
\item Attaque extérieure.
\end{itemize}
~\\
Le \PRO{} contenait également les opportunités suivantes : 
\begin{itemize}
\item Serveur gratuit ;
\item Livrable terminé en avance ;
\item Bonne passation inter-semestre ;
\item Bonne communication client ;
\item Bonne planification ;
\item Bonne ambiance interne ;
\item Remise des données en avance (clôturé le 14/03/2016 après remise des données par le client).
\end{itemize}
~\\
\indent Pour chacun de ces risques et opportunités, une \FDR{} ou \FDO{} a été rédigée contenant la probabilité d'apparition, le niveau d'impact, la criticité ainsi qu'un arbre des causes. \\
\indent Chaque semaine, lors de la réunion hebdomadaire, chaque pilote de risque ou opportunité est interrogé sur l'évolution de son risque ou opportunité par rapport à la semaine précédente et les changements sont reportés dans le \PRO{} via les \FDR{} et \FDO{}.

\section{Formations et compétences}
Une autre direction pour laquelle le rôle de \RQ{} comprend de la gestion de projet est la planification des formations en fonction des compétences de chacun et des nécessités imposées par le projet. Pour cela, avons utilisé la démarche suivante :
\begin{itemize}
\item Un ou deux membres de l’équipe se sont formés sur une technologie via des cours, des documentations ou des tutoriels ;
\item Ces membres ont formé les autres personnes en leur confectionnant un support synthétique ;
\item Les personnes formées ont été évaluées par un QCM, lui aussi réalisé par les membres formateurs.
\end{itemize}
Une fois les formations passées et réussies, les \FC{} ont été modifiées en conséquence.


\chapter{Résultats de l'audit}
\label{Resultats}

\section{Remarques soulevées}
L’audit Qualité du PIC a été réalisé le 15/03/2016 par \nomApprobateur{} et \nomTuteurQualite. Durant cet audit, cinq remarques ont été soulevées :
\begin{itemize}
\item Une incohérence entre les noms de certains documents et leurs entêtes a été remarquée ;
\item Les formations complémentaires à acquérir ne sont pas renseignées dans les \FC{} ;
\item Les documents de référence ne sont pas stockés de manière claire ;
\item La planification ne comprend pas les tâches de formation ;
\item Certains documents laissent place à l'interprétation ou aux doutes à cause de champs non complétés.
\end{itemize}

\section{Analyse}
\subsection{Première remarque}
\paragraph*{Causes} Cette erreur est due à une inattention de la part du rédacteur du document et de son vérificateur. Le champ contenant l'en-tête n'a pas été modifié, et cette erreur n’a pas été détectée durant la phase de vérification du document.

\paragraph*{Conséquences} Une incohérence dans les en-têtes peut provoquer une mauvaise lecture du document et une confusion concernant la version de référence de celui ci. Dans le cas présent, les conséquences de cette erreur sont limitées voire nulles. Il est cependant important de la corriger et de sensibiliser l’équipe à ce genre de problèmes afin d’éviter de potentielles futures incohérences.

\paragraph*{Actions mises en place}
\begin{itemize}
\item Action curative : modification de l'en tête sur les documents incriminés ;
\item Action corrective : sensibilisation de l'équipe à la rédaction des documents et à la vérification.
\end{itemize}


\subsection{Deuxième remarque}
\paragraph*{Causes} Cette erreur est due à un manque de rigueur de la part de chaque membre de l'équipe et du \RQ{} qui n'a pas relevé cette erreur.

\paragraph*{Conséquences} Un manque de renseignement sur les formations nécessaires peut amener à un déficit de formation qui peut mener à une mauvaise réalisation des lots.

\paragraph*{Actions mises en place} 
\begin{itemize}
\item Action curative : renseignement des formations complémentaires à acquérir ;
\end{itemize}


\subsection{Troisième remarque}
\paragraph*{Causes} Cette erreur est due à la présence de différents espaces de stockages et absence de précision concernant la place des documents de référence.

\paragraph*{Conséquences} Un manque de clarté sur le stockage des documents de référence peut mener à la modification et à la diffusion des mauvais documents.

\paragraph*{Actions mises en place}
\begin{itemize}
\item Action corrective : modification du \PGC{} en conséquence afin de clarifier le stockage ;
\end{itemize}

\subsection{Quatrième remarque}
\paragraph*{Causes} Cette erreur est due à un manque de rigueur de la part du \CP{} et du \CPA{} lors du remplissage du planning.

\paragraph*{Conséquences} Une absence de renseignement des tâches de formations dans le planning peut mener à une mauvaise planification et à des retards.

\paragraph*{Actions mises en place}
\begin{itemize}
\item Action curative : modification de la planification en conséquence ;
\item Action corrective : changement des habitudes de planification de la part du \CP{}.
\end{itemize}


\subsection{Cinquième remarque}
\paragraph*{Causes} Cette erreur est due à une inattention de la part des rédacteurs des documents et de leurs vérificateurs. Les champs non complétés n'ont pas été supprimés ou remplis avec "N/A", et cette erreur n’a pas été détectée durant la phase de vérification du document.

\paragraph*{Conséquences} Une interprétation personnelle d'un document peut mener à une erreur de compréhension de la part du lecteur.

\paragraph*{Actions mises en place}
\begin{itemize}
\item Action curative : modification des documents incriminés ;
\item Action corrective : sensibilisation de l'équipe à la rédaction des documents et à la vérification.
\end{itemize}

\chapter{Conclusion et avis personnel}
\paragraph*{} Comme nous avons pu le voir, la gestion de la Qualité est un des enjeux principaux des \PIC{} car elle permet de garantir des méthodes de travail, de communication et de livraison au client afin de maximiser la satisfaction de ce dernier. De plus, la mise en place de la norme ISO 9001 : 2015 permet d'assurer une traçabilité complète des décisions, documents, remarques et réclamations client.

\paragraph*{} L'attribution du poste de \RQ{} a été une expérience très enrichissante. En effet, ce poste correspond tout à fait à mon profil et m'a permis de mettre en application les connaissances acquises au cours de ma formation dans le département \ASI{}. De plus, l'équipe dans laquelle j'ai eu la chance de travailler était soudée et pleine de bonne humeur malgré les difficultés ponctuelles. Nous avons formé un bon binôme avec \Sergi{}, le \CP{}, en mettant en commun les différentes tâches de gestion de projet et de gestion de la Qualité. Enfin, tenir ce rôle ainsi que celui de \CPA{} m'a permis de monter en puissance au niveau de la gestion de projet et de travailler dans une équipe avec une hiérarchie établie pour la première fois dans mon cursus.

\paragraph*{} Cependant, la mise en place du \SMQ{} au sein du projet n'a pas été chose facile. En effet, du fait du contexte de formation des \PIC{}, le complément de formation sur la gestion de la Qualité qu'il nous manquait après les cours traitant de ce sujet n'a pu être acquise que "sur le tas". Toutefois, l'accompagnement tutoral étant bien encadré et suffisant, cette montée en puissance s'est faite rapidement.

\paragraph*{} Je tire de ce semestre en tant que \RQ{} une expérience très positive et une réelle envie de développer mes compétence dans ce domaine afin d'y exercer éventuellement un emploi. La répartition des rôles et des projets a donc été très bénéfique pour mon équipe et moi même.

\pageQuatriemeCouverture

\end{document}
