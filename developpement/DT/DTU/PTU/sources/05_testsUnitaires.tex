% version 1.00, date 15/03/16, auteur Michel Cressant

Ici sont décris les différents tests unitaires qui seront effectués sur le lot 1 :

\section{Test Unitaire TU001}
	
		Ce test vérifie les bonnes fonctionnalités du package Unipik/ArchitectureBundle/Controller \\
		
		
  		\begin{center}
    	 		\begin{tabular}[h]{|p{0.25\textwidth}|p{0.65\textwidth}|}
			\hline
				Eléments testés & Toutes les classes situées dans le package \\ 																& Unipik/ArchitectureBundle/Controller \\\hline
    				A faire & Ecrire dans un terminal : \\ 
    						& phpunit tests/Unipik/ArchitectureBundle/Controller/\\\hline
    				Résultat attendu & OK (n tests, n assertions) \\\hline
     		\end{tabular}
  		\end{center}	
	
\section{Test Unitaire TU002}

		Ce test vérifie les bonnes fonctionnalités du package Unipik/ArchitectureBundle/Entity \\
		
		
  		\begin{center}
    	 		\begin{tabular}[h]{|p{0.25\textwidth}|p{0.65\textwidth}|}
			\hline
				Eléments testés & Toutes les classes situées dans le package \\ 																& Unipik/ArchitectureBundle/Entity \\\hline
    				A faire & Ecrire dans un terminal : \\ 
    						& phpunit tests/Unipik/ArchitectureBundle/Entity/\\\hline
    				Résultat attendu & OK (n tests, n assertions) \\\hline
     		\end{tabular}
  		\end{center}	
  		
  		
  		
\section{Test Unitaire TU003}

		Ce test vérifie les bonnes fonctionnalités du package Unipik/ArchitectureBundle/Form \\
		
		
  		\begin{center}
    	 		\begin{tabular}[h]{|p{0.25\textwidth}|p{0.65\textwidth}|}
			\hline
				Eléments testés & Toutes les classes situées dans le package \\ 																& Unipik/ArchitectureBundle/Form \\\hline
    				A faire & Ecrire dans un terminal : \\ 
    						& phpunit tests/Unipik/ArchitectureBundle/Form/\\\hline
    				Résultat attendu & OK (n tests, n assertions) \\\hline
     		\end{tabular}
  		\end{center}	
  	
  		
\section{Test Unitaire TU004}

		Ce test vérifie les bonnes fonctionnalités du package Unipik/ArchitectureBundle/Repository \\
		
		
  		\begin{center}
    	 		\begin{tabular}[h]{|p{0.25\textwidth}|p{0.65\textwidth}|}
			\hline
				Eléments testés & Toutes les classes situées dans le package \\ 																& Unipik/ArchitectureBundle/Repository \\\hline
    				A faire & Ecrire dans un terminal : \\ 
    						& phpunit tests/Unipik/ArchitectureBundle/Repository/\\\hline
    				Résultat attendu & OK (n tests, n assertions) \\\hline
     		\end{tabular}
  		\end{center}	
  	


\section{Test Unitaire TU005}

		Ce test vérifie les bonnes fonctionnalités du package Unipik/ArchitectureBundle/Resources/view \\
		
		
  		\begin{center}
    	 		\begin{tabular}[h]{|p{0.25\textwidth}|p{0.65\textwidth}|}
			\hline
				Eléments testés & Toutes les classes situées dans le package \\ 																& Unipik/ArchitectureBundle/Resources/view \\\hline
    				A faire & Ecrire dans un terminal : \\ 
    						& phpunit tests/Unipik/ArchitectureBundle/Resources/view/\\\hline
    				Résultat attendu & OK (n tests, n assertions) \\\hline
     		\end{tabular}
  		\end{center}	
  	