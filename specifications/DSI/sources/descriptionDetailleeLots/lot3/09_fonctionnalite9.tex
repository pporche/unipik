% version 1.00, date 05/11/16, auteur Kafui Atanley
\subsection{Fonctionnalité 9}

