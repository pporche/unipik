% version 1.00, date 08/12/16, auteur Pierre Porche

Ici sont décris les différents tests unitaires qui seront effectués sur le lot 4 :

\section{Test Unitaire TU001}
	
	Ce test vérifie les bonnes fonctionnalités du bundle Unipik/ArchitectureBundle/ \\
				
  	\begin{center}
  		\begin{tabular}[h]{|p{0.25\textwidth}|p{0.65\textwidth}|}
		\hline
			Eléments testés & Toutes les classes situées dans le bundle \\ 																& Unipik/ArchitectureBundle/ \\\hline
    			A faire & Ecrire dans un terminal : \\ 
    					& phpunit tests/Unipik/Unit/ArchitectureBundle/\\\hline
    			Résultat attendu & OK (n tests, n assertions) \\\hline
     	\end{tabular}
  	\end{center}	
  		
\section{Test Unitaire TU002}
	
	Ce test vérifie les bonnes fonctionnalités du bundle Unipik/InterventionBundle/ \\
				
  	\begin{center}
  		\begin{tabular}[h]{|p{0.25\textwidth}|p{0.65\textwidth}|}
		\hline
			Eléments testés & Toutes les classes situées dans le bundle \\ 																& Unipik/InterventionBundle/ \\\hline
    			A faire & Ecrire dans un terminal : \\ 
    					& phpunit tests/Unipik/Unit/InterventionBundle/\\\hline
    			Résultat attendu & OK (n tests, n assertions) \\\hline
     	\end{tabular}
  	\end{center}	   		
   		
\section{Test Unitaire TU003}
	
	Ce test vérifie les bonnes fonctionnalités du bundle Unipik/UserBundle/ \\
				
  	\begin{center}
  		\begin{tabular}[h]{|p{0.25\textwidth}|p{0.65\textwidth}|}
		\hline
			Eléments testés & Toutes les classes situées dans le bundle \\ 																& Unipik/UserBundle/ \\\hline
    			A faire & Ecrire dans un terminal : \\ 
    					& phpunit tests/Unipik/Unit/UserBundle/\\\hline
    			Résultat attendu & OK (n tests, n assertions) \\\hline
     	\end{tabular}
  	\end{center}	 	
