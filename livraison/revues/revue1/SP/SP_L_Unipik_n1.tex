% version 1.01, date 07/03/16, auteur Matthieu Martins-Baltar
\documentclass[compress,xcolor=dvipsnames]{beamer}

%Pour les schémas d'architecture
\usepackage{etex}
\usepackage{tikz}
\usetikzlibrary{shapes,arrows,chains,backgrounds,fit}
%FIN Pour les shémas d'architecture

\usepackage[french]{babel}
\selectlanguage{french}
\usepackage[utf8]{inputenc}
\usepackage[T1]{fontenc}
\usepackage{tikz}
\usepackage{wrapfig}
\usepackage{multirow}
%\usepackage{pgf-pie}
\usepackage{pgfplots}
\usepackage{pdfpages}
\usepackage{vocabulaireUnipikPresentation}
\usepackage{commun/vocabulaireCommun}
\usepackage{hyperref}
\usepackage{movie15}
\usepackage{xcolor}
\usepackage{lscape}
\usepackage{booktabs}
\usepackage{multirow}
\usepackage{colortbl}

\newcommand{\tabincell}[2]{\begin{tabular}{@{}#1@{}}#2\end{tabular}}

\usetheme{eastpic}%{Berlin}%{Madrid}

 
 
%Information to be included in the title page:
\title{Revue de PIC}
\date{\today}
\author{Unipik}
\institute{\insa}

\setbeameroption{show notes}

 
\begin{document}

\speaker{\Sergi} 

\begin{frame}[plain]
	\titlepage
\end{frame}

\begin{frame}{Sommaire}
	\tableofcontents[hideallsubsections]
\end{frame}
 

\speaker{\Sergi}
\section[Présentation]{Présentation}
\subsection{} % PAs besoin de titre

\begin{frame}
\frametitle{Présentation de l'équipe}
	\begin{figure}
		\includegraphics[scale=0.25]{images/organigrammeFonctionnel.png}
		\caption{Organigramme fonctionnel}
		\label{OF}
	\end{figure}
\end{frame}

\speaker{Mélissa Bignoux}
\begin{frame}
\frametitle{Présentation du client}
	\begin{center}
		UNICEF
	\end{center}
	\begin{itemize}
		\item Un des principaux organismes d’aide humanitaire et de développement
		\item Acteur partout dans le monde en faveur des droits de chaque enfant
	\end{itemize}
	
	\begin{center}
		UNICEF Seine-Maritime
	\end{center}
	Ses actions principales sont : 
	\begin{itemize}
		\item La sensibilisation aux droits de l'enfant
		\item Les événements 
		\item Les ventes sur stands
	\end{itemize}
	
\end{frame}
\speaker{Mélissa Bignoux}
\begin{frame}
\frametitle{Présentation du sujet}
Outil de gestion informatique des interventions externes suivantes :

	\begin{itemize}
		\item Les plaidoyers dans les \'ecoles
		\item Les actions frimousses
		\item Les projets de lycéens et d'étudiants
		\item Les actions ponctuelles
	\end{itemize}
	
\end{frame}


\section[Cahier des charges]{Cahier des Charges}
\subsection{} % PAs besoin de titre

\begin{frame}
\frametitle{UNICEF 76}
\framesubtitle{Les actions}
	\begin{itemize}
		\item Plaidoyers dans les \'ecoles
		\item Actions frimousses
		\item Villes Amies des Enfants
		\item Ventes ponctuelles
	\end{itemize}
\end{frame}

\speaker{\Sergi}
\section[Avancement]{Avancement du Projet}
\subsection{} % PAs besoin de titre

\begin{frame}
\frametitle{UNICEF 76}
\framesubtitle{Les actions}
	\begin{itemize}
		\item Plaidoyers dans les \'ecoles
		\item Actions frimousses
		\item Villes Amies des Enfants
		\item Ventes ponctuelles
	\end{itemize}
\end{frame}

\speaker{\Sergi}
\section[Gestion projet]{Gestion de Projet}
\subsection{} % PAs besoin de titre

\begin{frame}
\frametitle{UNICEF 76}
\framesubtitle{Les actions}
	\begin{itemize}
		\item Plaidoyers dans les \'ecoles
		\item Actions frimousses
		\item Villes Amies des Enfants
		\item Ventes ponctuelles
	\end{itemize}
\end{frame}


\speaker{\Sergi}
\section[Qualité]{Démarche Qualité}
\speaker{\Pierre}

\subsection{} % PAs besoin de titre

\begin{frame}
\frametitle{Satisfaction client}
Actions mises en place :
	\begin{itemize}
		\item Réunions régulières
		\item Prise en compte des remarques et réclamations
		\item Contacts fréquents par emails
		\item Mise en place d'un indicateur
	\end{itemize}
\end{frame}

\begin{frame}
\frametitle{Amélioration continue}
\framesubtitle{Traitement des faits techniques}
\begin{center}
\begin{figure}
\includegraphics[scale=0.21]{images/cycleCorrectionFT.pdf}
\caption{Cycle de correction d'un fait technique}
\end{figure}
\end{center}
\end{frame}


\begin{frame}
\frametitle{Traçabilité}
Actions mises en place :
	\begin{itemize}
		\item Établissement d'un plan de gestion des configurations
		\item Sauvegardes régulières
		\item Versionnage
		\item Rédaction de comptes-rendus
	\end{itemize}
\end{frame}

\speaker{\Kafui}

\begin{frame}
\frametitle{Risques et opportunités}
But : \textbf{Améliorer la qualité du PIC}
	\begin{itemize}
		\item Réunion hebdomadaire
		\item Réévaluation des risques
	\end{itemize}
\end{frame}


\begin{frame}
\frametitle{Risques et opportunités}
	\begin{center}
	\begin{figure}
	\includegraphics[scale=0.30]		{images/risque.pdf}
	\caption{Suivi des risques et opportunités}
	\end{figure}
	\end{center}
\end{frame}






\speaker{\Sergi}
\section[Conclusion]{Conclusion}
\begin{frame}
\frametitle{Conclusion}
\begin{itemize}
 \item Cohésion d'équipe et motivation
 \item De l'architecture vers les fonctionnalités
 \item Gestion de projet efficace pour répondre aux attentes du client
 \item Démarche qualité orientée vers la satisfaction client
 \item De la Seine Maritime vers la Normandie puis vers la France
\end{itemize}
\end{frame}

 
\end{document}

