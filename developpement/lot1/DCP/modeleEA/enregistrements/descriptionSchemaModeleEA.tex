%version 2.00, date 23/02/16, auteur Mathieu MEDICI, Julie PAIN

\documentclass[asi, sansVersion]{picInsa}

\usepackage{vocabulaireUnipik}

\begin{document}

\title{Description modèle Entité Association}
\author{\Florian, \Mathieu, \Julie}
\date{04/02/2016} 

\maketitle

\tableofcontents

\chapter{Modèle Entité-Association}

%% Inclure le modele E-A 
\begin{landscape}
\begin{figure}
	\centering
	\includegraphics[scale=1]{images/modeleEAv3}
	\caption{\label{modele}modèle Entité/Association}
\end{figure}
\end{landscape}

\chapter{Description du modèle E-A}

Le schéma du modèle Entité association réalisé est composé de plusieurs entités et de plusieurs associations. Nous allons les décrire dans cette partie. \\ 

\section{Les entités}

\subsection*{Personne}

L'entité PERSONNE est l'entité mère des entités UTILISATEUR, CONTACTPROJET et CONTACTETABLISSEMENT. \\
Cette entité possède plusieurs attributs : 
\begin{itemize}
\item un nom; %Cdc - Non NULL
\item un prénom; %Cdc - Non NULL
\item un numéro de téléphone fixe; %Entier - Peut être NULL
\item un numéro de téléphone portable; %Entier - Peut être NULL
\item un email. %Ne peut pas être NULL (Clé)
\end{itemize}

\subsection*{Utilisateur}

L'entité UTILISATEUR est une entité fille de l'entité PERSONNE. Cette entité décrit les bénévoles de l'UNICEF qui gèreront le site ainsi que ceux qui effectueront des actions dans des structures. \\
Cette entité a plusieurs attributs : 
\begin{itemize} 
\item un type qui permet de déterminer si l'utilisateur est un administarteur global (plusieurs administrateurs globaux possibles), un administrateur local ou un simple bénévole;
\item une adresse correspondant à la rue et au numéro;
\item un code postal;
\item une ville;
\item aucune, une ou plusieurs activités potentielles (action ponctuelle, frimousse, plaidoyer); % attribut multivalué
\item aucune, une ou plusieurs responsabilités d'activité (action ponctuelle, frimousse, plaidoyer);
\item un mot de passe. 
\end{itemize}

\subsection*{ContactEtablissement}

L'entité CONTACTETABLISSEMENT est une entité fille de l'entité PERSONNE. Cette entité décrit les personnes travaillant dans un établissement ainsi que les élèves participant à un projet. \\
Cette entité a un attribut : 
\begin{itemize}
\item un type de contact qui permet de déterminer si un contact est un enseignant, un animateur, un contact frimousse, un contact plaidoyer, un contact activité ponctuelle ou un responsable d'établissement; 
\end{itemize} 

\subsection*{ContactProjet}

L'entité CONTACTPROJET est une entité fille de l'entité PERSONNE. Cette entité décrit les élèves participant à un projet.


\subsection*{Etablissement}

L'entité ETABLISSEMENT est l'entité mère des entités ENSEIGNEMENT et CENTRELOISIRS. \\
Cette entité a plusieurs attributs : 
\begin{itemize}
\item un identifiant unique;
\item un nom;
\item la ville de l'établissement;
\item une adresse sans le code postal;
\item un code postal;
\item un numéro de téléphone fixe;
\item une ou plusieurs adresses e-mail; % attribut multivalué
\item une géolocalisation WGS84; % Peut être NULL
\end{itemize}

\subsection*{Enseignement}
L'entité ENSEIGNEMENT est une entité fille de l'entité ETABLISSEMENT. Cette entité décrit les établissements qui relève de l'éducation nationale c'est à dire les écoles primaires, les collèges, les lycées, etc. \\
Cette entité a plusieurs attributs : 
\begin{itemize}
\item un UAI (Unité Administrative Immatriculée) qui est une combinaison de chiffres et de lettres unique pour chaque enseignement;
\item un type d'enseignement qui permet de déterminer si l'enseignement est une école maternelle, une école élémentaire, un collège, un lycée ou un établissement supérieur. 
\end{itemize} 


\subsection*{Centre de loisirs}
L'entité CENTRELOISIRS est une entité fille de l'entité ETABLISSEMENT. Cette entité décrit les établissements ayant un rapport avec les loisirs. \\
Cette entité a un attribut : 
\begin{itemize}
\item un type de centre de loisirs qui permet de déterminer si un centre de loisirs prend en charge des enfants d'école maternelle, d'école élémentaire ou des adolescents.
\end{itemize}  

\subsection*{Intervention}
L'entité INTERVENTION est l'entité mère des entités PLAIDOYER et FRIMOUSSE. Il s'agit d'une entité faible. Cette entité décrit une intervention que peut demander un établissement ou un contact. \\
Cette entité a plusieurs attributs :
\begin{itemize}
\item un identifiant unique;
\item une date précise déterminée par les bénévoles lors de l'attribution des bénévoles aux interventions;
\item le nombre de personnes concernées par cette intervention (exemple : le nombre d'élève d'une classe);
\item le matériel disponible pour l'intervention (présence ou non d'un rétroprojecteur, d'un tableau intéractif ou d'enceintes);
\item un lieu dans l'établissement, c'est à dire la salle où aura lieu l'intervention;
\item un champ remarque permettant à l'utilisateur de saisir des remarques spécifiques à cette intervention. 
\end{itemize}

\subsection*{Plaidoyer}
L'entité PLAIDOYER est une entité fille de l'entité INTERVENTION. C'est une entité faible. \\
Cette entité a plusieurs attributs : 
\begin{itemize}
\item un thème qui dépend du type de l'établissement faisant une demande d'intervention;
\item un niveau qui correspond à la classe dans laquelle va se dérouler l'intervention, par exemple, pour une école primaire, la classe peut être CE1 mais aussi CE1-CE2.
\end{itemize}

\subsection*{Frimousse}
L'entité FRIMOUSSE est une entité fille de l'entité INTERVENTION. C'est une entité faible. Une activité frimousse est réalisée avec des enfants étant en école élémentaire. \\
Cette entité a plusieurs attributs :
\begin{itemize}
\item les matériaux nécessaires à la réalisation de l'activité (patron, bourre, décoration); % Attribut multivalué
\item un niveau qui correspond à la classe dans laquelle va se dérouler l'intervention (CP, CE1, etc).
\end{itemize}

\subsection*{Projet}
L'entité PROJET représente les projets réalisées par des étudiants ou des lycéens. Il s'agit d'une entité faible.\\
Cette entité possède plusieurs attributs : 
\begin{itemize}
\item un identifiant unique;
\item  le type de projet permet de savoir si le projet est réalisé par des étudiants, des lycéens, des élèves d'école primaire ou de collège; 
\item la description du projet est un champ permettant à l'utilisateur de saisir toutes les informations importantes sur le projet (le nom du projet, s'il y a un tuteur);
\item les profits réalisés par ce projet.
\end{itemize}

\subsection*{Vente}
L'entité VENTE représente les actions ponctuelles réalisées par les établissements. Il s'agit d'une entité faible.\\ 
Cette entité a plusieurs attributs : 
\begin{itemize}
\item un identifiant unique; 
\item les profits réalisés par cette vente;
\item la date de la vente;
\item un champ remarques permettant à l'utilisateur d'ajouter des informations importantes. 
\end{itemize}
\section{Les associations}

\subsection*{estResponsable}
L'association ESTRESPONSABLE relie un UTILISATEUR et un PROJET. Un utilisateur est responsable d'aucun, un ou plusieurs projets et un projet est sous la responsabilité d'un ou plusieurs utilisateurs.

\subsection*{participe}
L'association PARTICIPE relie un CONTACTPROJET et un PROJET. Un contact participe à aucun, un ou plusieurs projets et un projet a pour participant un ou plusieurs contacts.

\subsection*{réalise}

L'association REALISE est une association ternaire qui relie une VENTE, un ETABLISSEMENT et un PLAIDOYER. Un établissement réalise aucune, une ou plusieurs ventes. Une vente est réalisée par un et un seul établissement. Un plaidoyer réalise aucune ou une intervention dans un établissement générant par la suite une vente. Une vente est réalisé grâce à l'intervention de un et un seul plaidoyer.


\subsection*{demande}

L'association DEMANDE est une association ternaire qui relie un ETABLISSEMENT, un CONTACTETABLISSEMENT et une INTERVENTION. Un établissement demande aucune, une ou plusieurs interventions. Une intervention est demandée par un et un seul établissement ou un et un seul contactEtablissement. Un contactEtablissement demande aucune, une ou plusieurs interventions. Cette association possède trois attributs : les moments souhaités, les moments à éviter et une plage de date. Les moments souhaités correspondent aux préférences de l'établissement en ce qui concerne le jour de la semaine et le moment (matin, après-midi) réservés à l'intervention, les moments à éviter correspondent aux jours et moments (matin, après-midi) qui n'arrange pas l'établissement et la plage de date correspond à la période de temps pour laquelle l'établissement souhaiterait avoir cette intervention. 


\subsection*{effectue} 

L'association EFFECTUE relie un UTILISATEUR et une INTERVENTION. Un utilisateur effectue aucune, une ou plusieurs interventions. Une intervention est effectuée par un et un seul utilisateur. 

\subsection*{travaille}

L'association TRAVAILLE relie un CONTACT et un ETABLISSEMENT. Un contact travaille dans aucun, un ou plusieurs établissements. Un établissement a un ou plusieurs contacts.   

\chapter{Contraites d'intégrité}


\end{document}