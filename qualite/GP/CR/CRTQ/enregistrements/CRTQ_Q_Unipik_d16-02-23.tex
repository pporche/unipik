% version 1.00	Auteur Pierre Porche

\documentclass [a4paper] {article}
\usepackage[utf8]{inputenc}
\usepackage[francais]{babel}
\usepackage[top=2cm, bottom=4cm, left=2cm, right=2cm]{geometry} 
\usepackage{fancyhdr}
\usepackage{graphicx}
\usepackage{color, colortbl}
\usepackage{longtable}
\usepackage{vocabulaireUnipik}
\pagestyle{fancy}
\definecolor{Gray}{gray}{0.8}



%--- En-t�te et pied de page ---%
\renewcommand{\footrulewidth}{0,01cm}

\begin{document}
\rhead{}
\chead{\huge{Compte-rendu de réunion de Tutorat Qualité}}					%titre

23/02/2016
\hfill   
\hfill 	9:03 - 9:56 				% Heure de d�but, heure de fin.


\lfoot{Version : 1.00} 			% version

%--- Fin en-t�te et pied de page ---%
\section*{Historique des révisions}
\begin{center}
			\begin{tabular}{| c | c | c | c | p{4cm} |}
				\hline
				\rowcolor{Gray}
				Version & Date & Auteur(s) & Modification(s) & Partie(s) modifiée(s)		 \\
				\hline
				1.00 & 24/02/2016 & \Pierre & Création & Toutes \\
		\hline		
			\end{tabular}
		\end{center}

\section*{Signatures}

		\begin{center}
			\begin{tabular}{| c | c | c | c | p{4cm} |}
				\hline
				\rowcolor{Gray}
				Rôle & Fonction & Nom & Date & Visa		 \\
				\hline
				Vérificateur & \RQA & \Kafui & 24/02/2016 & pgpic \\[30pt]
				\hline
				Validateur & \CP & \Sergi & 25/02/2016 & pgpic \\[30pt]	
				\hline
			\end{tabular}
		\end{center}

%--- Réunion --%

\section{Diffusion}
La diffusion est l'envoi aux destinataires d'un document. Celui-ci ne peut être diffusé qu'après approbation, donc seulement après que le document soit approuvé, l'équipe peut remplir les champs "diffusion". \\
Dans les \DSE{}, \DSI{}, \PTV{}, \PQ{}, \PGC{} et \CDR{}, il est possible de globaliser les noms (ex.: "Unipik" au lieu de citer chaque membre de l'équipe).

\paragraph*{Diffusion contrôlée / non contrôlée}
Si le client (ou autre personne) ne veut disposer que d'une version d'un document, c'est une diffusion non contrôlée. Si on met en place une diffusion contrôlée, il faut envoyer le document à tous les destinataires à chaque nouvelle version. Pour les \DSE{}, \DSI{} et \PQ{}, il faut forcément mettre l'équipe comme destinataire en diffusion contrôlée.

\paragraph*{}
Lors des audits, il faut faire attention aux versions des documents. En effet, il ne faut pas conserver les anciennes versions en tant que document de référence.


\section{\FT}
Lorsque l'équipe rencontre un problème, elle doit stocker la démarche.  Celle-ci se déroule comme suit :

\begin{itemize}
\item Identification : émission de \FFT{}, description de la Mise à Jour ou du problème ;
\item Analyse des causes : méthode des n-pourquoi, apparition d'actions possibles ;
\item \OC : Synthèse des actions curatives et correctives. Pour mémoire, les actions correctives éliminent la source du problème. Il faut aussi tracer la vérification des corrections ;
\item Clôture de l'\OC{} puis du \FT{} ;
\item Analyse à froid : vérification que tout \FT{} corrigé par des actions correctives ne réapparaisse pas. S'il réapparaît, le nouvel \OC{} inclut une action corrective différente.
\end{itemize}

Pour une \CTFT{}, il faut au minimum trois personnes (\CP{}, \RQ{} et \RGC{}) mais le plus possible serait le mieux.


\section{Divers}

Utiliser un kanban n'est pas une mauvaise idée mais il faut faire attention avec les post-it. Pour résoudre ce problème, le logiciel Trello peut aider.

\paragraph*{}
Concernant la demande auprès de la CNIL, il faut se renseigner auprès du juriste de l'INSA pour savoir si \Sergi{} a le droit de faire la demande. \nomTuteurQualite{} se charge de le contacter et nous conseille de lire la norme ISO 27005.

\end{document}















