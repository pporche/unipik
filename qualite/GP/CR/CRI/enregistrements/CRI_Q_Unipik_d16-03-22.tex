% version 1.00	Auteur Pierre Porche		Date 22/03/2016

\documentclass [a4paper] {article}
\usepackage[utf8]{inputenc}
\usepackage[francais]{babel}
\usepackage[top=2cm, bottom=4cm, left=2cm, right=2cm]{geometry} 
\usepackage{fancyhdr}
\usepackage{graphicx}
\usepackage{color, colortbl}
\usepackage{longtable}
\usepackage{vocabulaireUnipik}
\pagestyle{fancy}
\definecolor{Gray}{gray}{0.8}
\definecolor{White}{gray}{1.0}


%--- En-t�te et pied de page ---%

\renewcommand{\footrulewidth}{0,01cm}
\rhead{}
\chead{\huge{Compte-rendu de réunion interne}}					%titre
\begin{document}

22/03/2016			 				%Date 
\hfill   
\hfill 	 9:31 - 10:14 				%Heure de d�but, heure de fin.


\lfoot{Version : 1.00} 			% Version
%--- Fin en-t�te et pied de page ---%

\section*{Historique des révisions}
\begin{center}
			\begin{tabular}{| c | c | c | c | p{4cm} |}
				\hline
				\rowcolor{Gray}
				Version & Date & Auteur(s) & Modification(s) & Partie(s) modifiée(s)		 \\
				\hline
				1.00 & 22/03/2016 & \Pierre & Création & Toutes \\
		\hline		
			\end{tabular}
		\end{center}

\section*{Signatures}

		\begin{center}
			\begin{tabular}{| c | c | c | c | p{4cm} |}
				\hline
				\rowcolor{Gray}
				Rôle & Fonction & Nom & Date & Visa		 \\
				\hline
				Vérificateur & \RQA & \Kafui &  & pgpic \\[30pt]
				\hline
				Validateur & \CP & \Sergi &  & pgpic \\[30pt]	
				\hline
			\end{tabular}
		\end{center}
		
\newpage		



\section{Planning des prochaines semaines}
Durant cette semaine, l'équipe se formera sur Symfony grâce à la \FF{} que \Florian{} fera dès ce matin. Il serait souhaitable que la formation soit terminée vendredi soir. \\
Ce mercredi à 13:30, le client vient pour la livraison et le remplissage du \CDR{} puis nous recommençons notre modèle en spirale.


\section{Gestion des configurations}
Il faut à présent déclarer la variable UNIPIKGENPATH afin de pouvoir faire un make dans le git. De plus, en ce qui concerne les changements de version, il faut changer la variable dans le fichier "gestion des configurations" à la source du git, dans le nom du .tex ainsi que dans le fichier latex lui même, à la ligne "\textbackslash version\{vX.YY\}".
Pour les changements de version dans le code, il faut le renseigner dans la documentation.


\section{Remarques de \nomTuteurPedago{} concernant notre lot}
Il faut maintenant passer au relationnel. Pour cela, deux options s'offrent à nous : 
\begin{itemize}
	\item le relationnel classique pour lequel un attribut multivalué entraîne plusieurs relations pour n'avoir que des attributs atomiques ;
	\item le relationnel étendu qui inclut le concept d'objet et de classes.
\end{itemize}
Il est à noter que Symfony est fait pour fonctionner avec du relationnel classique. \\
En ce qui concerne la modélisation du schéma relationnel, nous décidons de coder la BD sur Symfony, d'exporter le schéma puis de le retoucher si besoin est. \\
Il faut dès maintenant mettre en place des conventions de nommages et autres outils facilitant la phase de codage.


\section{Technologie pour le Front End}
Afin de choisir la technologie que nous allons utiliser pour la réalisation du Front-End, \Matthieu{} a fait un état de l'art et le framework qui est sorti du lot est Bootstrap. Les raisons sont les suivantes : 
\begin{itemize}
	\item De nombreux thèmes ;
	\item Un bon support ;
	\item Une mise en avant de la productivité par rapport à la customisation, sans toutefois empêcher celle ci ;
	\item Prise en charge par Symfony native ;
	\item Un "feel \& look" caractéristique.
\end{itemize}



\section{Revu des risques et opportunités}
Nous passons en revue tous les risques et opportunités afin de suivre leurs évolutions. 
\begin{itemize}
	\item Le risque 003 passe de la probabilité 1 à la probabilité 3 car il y a une vague d'épidémie grippale sur la région et plus particulièrement sur l'INSA ;
	\item le risque 004 passe de la probabilité 1 à la probabilité 2 car le client part en vacances pour un long moment ;
	\item le risque 010 passe de la probabilité 2 à la probabilité 1 car la CNIL a déjà accepté notre deuxième demande de traitement ;
	\item le risque 012 passe de la probabilité 2 à la probabilité 4 car le lot 1 sera livré en retard ;
	\item le risque 013 passe de la probabilité 2 à la probabilité 1 car le lot 1 ne sera pas à faire tourner chez le client ;
	\item l'opportunité 006 a maintenant pour pilote \Michel{} et passe d'un bénéfice de 2 à un bénéfice de 3 car la cohésion de l'équipe est d'autant plus importante que nous avons fait deux équipes (Front-End et Back-End) ;
	\item un nouveau risque apparait : "panne d'un PC dans l'équipe" suite au problème avec le PC de Michel. Sergi sera pilote de ce risque.
\end{itemize}





%--- fin de réunion ---%


\end{document}







