\documentclass [a4paper] {article}
\usepackage[utf8]{inputenc}
\usepackage[francais]{babel}
\usepackage[top=2cm, bottom=4cm, left=2cm, right=2cm]{geometry} 
\usepackage{fancyhdr}
\usepackage{graphicx}
\usepackage{color, colortbl}
\usepackage{longtable}
\usepackage{../../../../ressources/Unipik/vocabulaire/vocabulaireUnipik}
\pagestyle{fancy}
\definecolor{Gray}{gray}{0.8}



%--- En-t�te et pied de page ---%

\renewcommand{\footrulewidth}{0,01cm}
\rhead{}
\chead{\huge{Compte-rendu de réunion avec le client}}					%titre
\begin{document}

26/01/2016			 				%Date 
\hfill   
\hfill 	 10:04 - 12:18 				%Heure de d�but, heure de fin.


\lfoot{Version : 1.00} 			% version
%--- Fin en-t�te et pied de page ---%
\section*{Historique des révisions}
\begin{center}
			\begin{tabular}{| c | c | c | c | p{4cm} |}
				\hline
				\rowcolor{Gray}
				Version & Date & Auteur(s) & Modification(s) & Partie(s) modifiée(s)		 \\
				\hline
				1.00 & 04/02/2016 & \Pierre & Création & Toutes \\
		\hline		
			\end{tabular}
		\end{center}

\section*{Signatures}

		\begin{center}
			\begin{tabular}{| c | c | c | c | p{4cm} |}
				\hline
				\rowcolor{Gray}
				Rôle & Fonction & Nom & Date & Visa		 \\
				\hline
				Vérificateur & \RQA & \Kafui & 05/02/2016 & pgpic \\[30pt]
				\hline
				Validateur & \CP & \Sergi & 05/02/2016 & pgpic \\[30pt]	
				\hline
			\end{tabular}
		\end{center}

%--- Réunion --%

\section{Résumé de la semaine}
Cette semaine, nous avons fait la phase de spécification. Notamment le modèle entité/association qui sera expliqué lors de cette réunion.


\section{Explication du modèle Entité/Association (E/A)}
\Julie présente le schéma du modèle E/A au client tout en posant des questions sur les points d'incompréhension. Les informations recueillies seront notées ci après et le modèle E/A présenté sera joint.
\\

Au mois de juin, l'UNICEF envoie un e-mail aux écoles afin de rappeler qu'au titre du partenariat UNICEF/eduNat, les plaideurs sont disponibles pour intervention.
\\

Le double niveau grande-section/CP existe dans certains cas. Il pourra être traité en marquant un niveau CP dans le logiciel livrée et en précisant en remarque que c'est un double niveau.
\\

Les frimousses ne peuvent avoir lieu en maternelle.
\\

Tous les utilisateurs de l'application auront un e-mail.
\\

Il est important de suivre les encadrements de projets (spectacle de danse, course,..) pour les statistiques.
\\

Les ventes organisées en interne par l'UNICEF ne sont pas à prendre en compte. Les ventes dans les établissements sont reliées à un établissement et découlent d'une intervention. La gestion des stocks ne sera à faire qu'au second semestre si l'avancement du projet est suffisant.
\\

Les utilisateurs de l'application seront les bénévoles plaideurs ou encadrant des projets.
\\

Chaque plaideur rentre lui-même dans l'application les intervention qu'il fera.
\\

Il arrive que des établissements contactent directement les plaideurs, dans ce cas, le plaideur devra demander à l'établissement de faire sa demande via l'application en lui renvoyant le mail de demande.
\\

Pour plus de souplesse, les dates d'intervention seront modifiable simplement et à tout moment.
\\

Les modifications d'informations seront conservées un an.
\\

Il faut inclure les thèmes "droit/discrimination" et "droit/harcèlement" dans les demandes des primaires et des collège.
\\

Pour les lycées et établissements, aucun thème n'est à proposer.
\\

Les classes de type CLIS ou SEGPA ne doivent pas être considérées à part mais cette information sera écrite dans le champ "remarques".
\\

Le nombre de bénévoles pour UNICEF 76 est de l'ordre de 10 plaideurs, sur l'ensemble de la Normandie, de l'ordre de 100 et sur la France, de l'ordre de 500.
\\

Les demandes d'intervention se font généralement en deux semaines.
\\

Pour toute question sur la charte graphique, contacter UNICEF France.


\section{Divers}




%--- fin de réunion ---%
\newpage



\end{document}





















