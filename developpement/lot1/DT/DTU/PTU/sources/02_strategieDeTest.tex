Les tests unitaires se divisent en deux types :

\section{Les tests "boîte blanche"}
	Ce sont les tests structurels. Ils permettent de tester la structure du composant, ils sont donc la base de la fiabilité du produit. Le composant est considéré comme testé de manière structurée lorsque les résultats attendus et les résultats obtenus concordent. \\
	
	Tous les défauts et défaillances détectés sont corrigés.
	
\section{Les tests "boîte noire"}
	Ce sont les tests fonctionnels. Ils sont utilisés pour l’analyse partitionnelle, le graphe de cause à effet et les tests aux limites. Il est recommendé de tester tous les appels à des fonctionnalités externes au composant. \\
	
	La préparation des jeux d’essais, scénarios et résultats attendus doit être réalisée avant de les mettre en œuvre. \\
	
	Tous les défauts et défaillances détectés sont corrigés.
	