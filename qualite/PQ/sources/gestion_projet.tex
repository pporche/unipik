Le processus de \textit{gestion de projet} se décompose en trois sous processus : 
\begin{itemize}
\item Planifier le projet
\item Réaliser le suivi
\item Gérer les risques et opportunités
\end{itemize}

\section{Planification du projet}

Le processus \textit{Planifier le projet} peut se découper en trois sous-processus.

\begin{itemize}
\item Analyser le projet : consiste à établir la liste de l'ensemble des activités du projet.
\item Modéliser le projet : consiste en la création des tâches.
\item \textit{Ordonnancer les activités} : a pour objectif d'établir un planning du projet, sous la forme d'un diagramme GANTT.
\end{itemize}

Nous étudierons tout d'abord ces trois sous-processus, puis nous présenterons le planning de principe et enfin nous verrons les outils  disposition pour la gestion de projet. 

\subsection{Analyse du projet}
Le but de l'analyse est de préparer et de faciliter la modélisation. Nous utiliserons pour cela la structure la structure WBS qui permet de décomposer chaque activité en allant du plus général au particulier. Chaque feuille de la WBS représente une activité. Les WBS proviennent du besoin du client exprimé dans le Cahier des charges et au début de chaque sprint. \\

Dans le cadre de la plannification, nous pourrons également utiliser d’autres structures hiérarchiques comme la \FBSCourt (\FBS), l’ OBS (Object \BS) ou encore la \RBSCourt (\RBS). L’analyse du projet va nous permettre de bien préparer l’étape suivante, c’est-à-dire la modélisation.


\subsection{Modélisation du projet}

Au cours de cette modélisation, nous allons reprendre le WBS réalisé lors de l’analyse du projet et nous allons décomposer chaque activité en une ou plusieurs tâches de manière à établir une liste des différentes tâches. Ensuite nous allons définir chaque tâche en précisant sa durée, sa charge,les effectifs en ressources nécessaires et les prédécesseurs de cette tâche. Cette étape va nous aider à préparer l’ordonnancement des tâches.\\

Chaque attribution de tâche se fera via PGPic et/ou sur le kanbanboard utilisé par les membres de l'équipe, notamment pour les tâches de réalisation d’un Sprint. Les tâches seront attribuées aux différents membres de l’équipe PIC en prenant en compte des compétences et des disponibilités de chacun.

\subsection{Ordonnancement}

L'objectif de cette phase est d'ordonnancer les activités afin d'établir un planning du projet. Ce planning sera sous la forme d'un GANTT. Ce type de diagramme permet d’avoir une représentation visuelle dans le temps.\\

Pour répertorier les compétences de chacun, des fiches de compétences seront créées au début du projet et mises à jour au fur et à mesure des formations. L’ensemble des formations ainsi que les analyses effectuées sont renseignées dans la Fiche de Suivi des Formations.\\ 

Ce planning sera mis à jour par le \CP ou par son adjoint chaque semaine lors de la réunion d’avancement hebdomadaire. Il sera par la suite disponible sur PGPic.


\subsection{Planning de principe}

Comme dit ci-dessus, le planning du \PICCourt est fait par le \CP et le \CPA. Le responsable développement pourra aussi être sollicité selon le type de tâche.

%\indent Le planning de principe est généré par Ganttify. En ce qui concerne la planification du projet pour le premier semestre, c’est-à-dire de janvier à juin, le PIC comportera 12 semaines de travail. Pour le second semestre, de septembre à janvier, nous disposerons de 14 semaines de travail. La réalisation du PIC au sein de l'INSA est à temps partiel. \\
%
%\indent La planification du projet suit une planification de type Agile (SCRUM). Lors du premier semestre, le planning était divisé en "sprints" de durée constante, fixée par défaut à 4 semaines. Cette durée pouvait éventuellement être modifiée en accord avec le client en cas
%de retard. Le Sprint 0 étant réservé à la mise en place du PIC et à la formation de l’équipe.
%Pour le second semestre, aucune durée de sprint par défaut n'a été fixée, bien qu'une phase de mise en place et formation soit nécessaire.

\subsection*{Semestre 1 : Janvier 2015 - Mai 2015}

Durant le premier semestre, nous disposons de 12 semaines de \PICCourt. Le planning global sera donc le suivant : 
\begin{itemize}
\item Installation, initialisation des procédures de qualité, prise en main des technologies internes (1 semaine);
\item Premier sprint: réalisation de 1er lot (4 semaines);
\item Deuxième sprint (4 semaines);
\item Troisième sprint: Finalisation du 2ème lot (3 semaines);
\end{itemize}

\subsection*{Semestre 2 : Septembre 2015 - Janvier 2016}

Durant le second semestre, nous disposons de 14 semaines de \PICCourt. Le planning global sera donc le suivant : 
\begin{itemize}
\item Premier sprint: réalisation de 1er lot (4 semaines);
\item Deuxième sprint (4 semaines);
\item Troisième sprint: Finalisation du 2ème lot (3 semaines);
\end{itemize}

\subsection{Outils pour la gestion du projet}


\subsubsection*{PGPic}

\subsubsection*{}

\subsubsection*{}

\section{Suivi du projet} 


\subsection{Suivi d'avancement}

\subsection{Communication}
\subsubsection*{La communication avec le \tuteurPedagogique{}}
\subsubsection*{La communication avec le \tuteurQualite{}}
\subsubsection*{La communication avec le client}
\subsubsection*{Les communications internes}
\subsubsection*{Les communications inter\picCourt{}}  

\section{Gestion des risques et opportunités} 

\subsection{Identification des risques et opportunités}
\subsection{Suivi des risques et opportunités}
\subsection{Réduction des risques et opportunités}

%\subsection{Outils pour la gestion du projet}
%
%\indent Nous allons vous présenter dans cette partie les logiciels et méthodes utilisés pour la gestion du PIC.
%
%\subsubsection*{PGPic}
%
%\indent PGPic est un logiciel développé initialement par l'équipe Geotopic, une équipe PIC de l'INSA. Il est utilisé pour une grande partie de la gestion de projet.
%
%\indent Cet outil intranet permet entre autre :
%
%\begin{itemize}
%\item la création et le suivi des tâches;
%\item le suivi des temps de travail par personne et par tâche;
%\item l'organisation des tâches;
%\item la création de diagramme de Gantt;
%\item la gestion des riques;
%\item la gestion des faits techniques;
%\item le suivi des documents et de leur validation;
%\item la planification de réunions.
%\end{itemize}
%
%\subsubsection*{Kanbanboards} \label{Kanbanboards}
%
%\indent Les Kanbanboards sont des tableaux de suivi des tâches. Celles-ci sont triées dans quatre catégories : "à faire", "en cours", "à vérifier" et "terminées". Nous utiliserons cet outil pour superviser l'avancement technique du projet.
%\section{Suivi du projet} 
%\label{suivi}
%
%\indent Le processus \textit{Réaliser le suivi} est décomposé en quatre sous-processus.
%
%\begin{itemize}
% \item \textit{Suivre les tâches ou suivi statique} ;
%\item \textit{Réajuster le planning ou suivi dynamique} ;
%\item \textit{Communiquer en interne et en externe} ;
%\item \textit{Évaluer la conduite du projet ou suivi prévisionnel}.
%\end{itemize}
%
%\subsection{Suivi d'avancement}
%
%\indent Nous mettons en place un suivi hebdomadaire ayant pour objectif de mesurer l'avancement du PIC. Il se traduit entre autre par des réunions régulières qui permettront d'établir et de discuter du planning à venir et de déclarer et analyser les risques potentiels. Avant chacun de ces réunions, chaque membre du PIC doit décrire l'avancement des tâches qui lui sont attribuées.\\
%À la suite de ces réunions, le \CP{} distribue des éventuelles fiches de tâches modifiées et établit un diagramme de GANTT des tâches futures pour l'ensemble de l'équipe.
%
%\subsection{Communication}
%\label{Communication}
%Afin d’assurer une bonne communication, plusieurs échanges vont avoir lieu au sein et
%à l'extérieur du \picCourt{}. Il y a cinq types d’échanges :
%\begin{itemize}
% \item les échanges internes au \PICCourt{};
% \item les échanges inter\PICCourt{};
% \item les échanges avec le tuteur Qualité;
% \item les échanges avec le tuteur Pédagogique;
% \item les échanges avec le client.
%\end{itemize}
%Toutes les réunions ne se déroulent pas de la même façon. Cependant, l’archivage d’un compte-rendu se fait toujours en version numérique et parfois en version papier (lorsqu’il est signé).
%
%\subsubsection*{La communication avec le \tuteurPedagogique{}:}
%
%La communication se base sur des réunions hebdomadaires et en cas de besoin, l’envoi de courriels. Des réunions en dehors de ce créneau pourront être organisées exceptionnellement. Ces dernières se déroulent de la façon suivante :
%\begin{enumerate}
% \item Le \CP{} définit avant chaque réunion un ordre du jour qui recense toutes les questions et points à aborder pendant la réunion. Chaque membre du \PICCourt{} a la possibilité de consulter et de 
% rajouter des éléments à cet ordre du jour. Cela peut se faire par oral ou bien via le bloc-notes de \lintranet.
% \item Un secrétaire est désigné à chaque début de réunion pour prendre des notes sur les points abordés et rédiger le compte-rendu au plus tard dans les 7 jours qui suivent.
% \item Une fois rédigé, le compte-rendu est soumis à vérification puis validation. Le CR doit être validé avant la prochaine réunion au plus tard.
% \item Le CR est soumis à l'approbation du tuteur Pédagogique en début de réunion, la semaine suivante.
% \item Si le CR n’est pas approuvé, il est réédité, vérifié, validé et soumis à approbation à la prochaine réunion.
% \item Si le CR est approuvé, il est archivé.
%
%\end{enumerate}
%
%\subsubsection*{La communication avec le \tuteurQualite{}:}
%
%La communication se base sur des réunions et en cas de besoin, l’envoi de courriels. Des réunions en dehors de ce créneau pourront être organisées exceptionnellement. Ces dernières se déroulent de la façon suivante :
%\begin{enumerate}
% \item Le \CP{} et le \RQ{} définissent avant chaque réunion un ordre du jour qui recense toutes les questions et points à aborder pendant la réunion. Chaque membre du \PICCourt{} a la possibilité de consulter et de rajouter des éléments cet ordre du jour. Cela peut se faire par oral ou bien via le bloc-notes de \lintranet.
% \item Un secrétaire est désigné à chaque début de réunion pour prendre des notes sur les points abordés et rédiger le compte-rendu au plus tard dans les 7 jours qui suivent.
% \item Une fois rédigé, le compte-rendu est soumis à vérification puis validation. Le CR doit être validé avant la prochaine réunion au plus tard.
% \item Le CR est soumis à l'approbation du tuteur Qualité en début de réunion, la semaine suivante.
% \item Si le CR n’est pas approuvé, il est réédité, vérifié, validé et soumis à approbation à la prochaine réunion.
% \item Si le CR est approuvé, il est archivé.
%\end{enumerate}
%
%\subsubsection*{La communication avec le client}
%La communication avec le client est basée sur les éléments suivants :
%\begin{itemize}
% \item échanges de courriels entre le \CP{} et le client ;
% \item appels téléphoniques entre le \CP{} et le client ;
% \item visio-conférence entre le \CP{} et le client ;
% \item visio-conférence entre l’équipe et le client.
%\end{itemize}
%
%Le \CP{} est l’interlocuteur privilégié du client. En cas d’absence et de nécessité, le \CP{} Adjoint peut devenir cet interlocuteur. Les échanges se font sur une base hebdomadaire. 
%Ces derniers regroupent les appels téléphoniques ou en visio-conférence et la communication par courriel. En cas de nécessité, une prise de contact téléphonique peut être initiée par le client ou le 
%\CP{}. L’échange de courriels est mené depuis l’adresse du \CP{} puis par l'adresse mise en place dans le cadre des \PICCourt{}, associée au \CP{}. Ces courriels sont archivés sur 
%l’ordinateur du \CP{}  et par conséquent, sur le serveur du \PICCourt. Les réunions avec le client sont prédéfinies à l’avance entre l’équipe \PICCourt{} et ce dernier.
%La réunion avec le client se déroule de la façon suivante :
%
%\begin{enumerate}
% \item Le \CP{} et le \CPA{} définissent avant chaque réunion un ordre du jour qui recense toutes les questions et points à aborder pendant la réunion. Chaque membre du \PICCourt{} a la 
% possibilité de consulter et de rajouter des éléments à cet ordre du jour. Cela peut se faire par oral ou bien via le bloc-notes de \lintranet.
% \item Un secrétaire est désigné à chaque début de réunion pour prendre des notes sur les points abordés et rédiger le compte-rendu au plus tard dans les 7 jours qui suivent.
% \item Une fois rédigé, le compte-rendu est envoyé pour vérification puis validation.
% \item Le compte-rendu est envoyé par courriel au client pour approbation. Si le client ne fait aucune remarque sur le compte-rendu dans les 7 jours suivants l’envoi du mail, le compte-rendu est 
% considéré comme approuvé.
% \item Si le compte-rendu n’est pas approuvé, il est réédité, vérifié et validé et soumis à approbation une nouvelle fois.
% \item Si le compte-rendu est approuvé, il est archivé.
%\end{enumerate}
%
%Pour ce compte-rendu, le \CP{} est forcément rédacteur ou validateur.
%
%\subsubsection*{Les communications internes} 
%Comme précisé dans la partie précédente, une réunion d’avancement hebdomadaire aura lieu toutes les semaines. Au cours de cette réunion seront discutés entre autres l’avancement du projet et la gestion des 
%risques. Le \RQ{} devra aussi avertir l’équipe \PICCourt{} de la disponibilité du nouveau tableau de bord (s’il est disponible au moment de la réunion). \\
%Il n'y aura pas de compte rendu de ces réunions sauf si l'équipe le juge vraiment nécessaire.
%
%\subsubsection*{Les communications inter\picCourt{}} 
%Les réunions inter\PICCourt{} vont être effectuées soit entre les différents chefs \PICCourt{}, soit entre les différents Responsables Qualité. Ces réunions se feront à minima toutes les 3 semaines et au cours de ces réunions 
%seront discutés l'ensemble des problèmes rencontrés par les participants.
%
%Il n'y aura pas de compte rendu de ces réunions mais des OJ seront envoyés par mail avant la réunion.
%
%\section{Gestion des risques} 
%
%\indent Le processus \textit{Gérer les riques} se décompose en trois sous-processus.
%
%\begin{itemize}
%\item \textit{Identifier les risques};
%\item \textit{Suivre les risques};
%\item \textit{Réduire les risques}.
%\end{itemize}
%
%Ce processus permet au chef PIC et à son équipe de porter une attention particulière à certaines parties du projet qui pourraient être sensibles.
%
%\subsection{Identification des risques} \label{identificationRisques}
%
%Tous les membres de l'équipe participent à l'identification des risques. Les risques sont identifiés en début de semestre et aboutissent à l'élaboration du Portefeuille de Risques (PR). 
%	L'identification des risques se poursuit tout au long du semestre.\\
%	\newline
%	 L'identification des risques a un double objectif : 
%	\begin{itemize}
%		\item prévenir l'apparition d'écart par rapport à un objectif ;
%		\item prévenir l'apparition d'une non-conformité.
%	\end{itemize}
%	 Si un risque est découvert et validé par l'équipe \PICCourt{}, une Fiche de Risques sera rédigée et ajoutée au Portefeuille de Risques.\\
%	\newline
%	Tous les risques n'ont pas le même impact. Celui-ci peut-être :
%	\begin{itemize}
%		\item mineur : le risque peut entraîner une difficulté à achever une tâche mais n'entraîne pas de retard ;
%		\item moyen : le risque peut entraîner un retard sur le projet ;
%		\item important : le risque entraîne forcement un retard sur le projet ;
%		\item maximum : le risque bloque le projet.\\
%	\end{itemize}
%
%Chaque risque a également une priorité. Cette priorité est définie en fonction de l'impact du risque mais aussi de sa probabilité d'apparition.
%
%Nous devons également définir sa probabilité :
%\begin{itemize}
% \item peu probable : (P < 20\%) ;
% \item possible : (20\% < P < 40\%) ;
% \item probable : (40\% < P < 60\%) ;
% \item très probable (P > 60\%).\\
%
%\end{itemize}
%
%La priorité du risque est exprimée en fonction de l'impact et de la probabilité d'apparition.
%
%\begin{table}[h]
%\centering
%\begin{tabular}{c|
%>{\columncolor[HTML]{009901}}c |c|c|
%>{\columncolor[HTML]{FE0000}}c |}
%\cline{2-5}
% & \cellcolor[HTML]{BBDAFF}Peu probable & \cellcolor[HTML]{BBDAFF}Possible & \cellcolor[HTML]{BBDAFF}Probable & \cellcolor[HTML]{BBDAFF}Très probable \\ \hline
%\multicolumn{1}{|c|}{\cellcolor[HTML]{BBDAFF}Mineur} & {\color[HTML]{000000} \textbf{Acceptable}} & \cellcolor[HTML]{009901}{\color[HTML]{000000} \textbf{Acceptable}} & \cellcolor[HTML]{FFC702}{\color[HTML]{000000} \textbf{À Surveiller}} & {\color[HTML]{000000} \textbf{Critique}} \\ \hline
%\multicolumn{1}{|c|}{\cellcolor[HTML]{BBDAFF}Moyen} & {\color[HTML]{000000} \textbf{Acceptable}} & \cellcolor[HTML]{FFC702}{\color[HTML]{000000} \textbf{À Surveiller}} & \cellcolor[HTML]{FFC702}{\color[HTML]{000000} \textbf{À Surveiller}} & {\color[HTML]{000000} \textbf{Critique}} \\ \hline
%\multicolumn{1}{|c|}{\cellcolor[HTML]{BBDAFF}Important} & {\color[HTML]{000000} \textbf{Acceptable}} & \cellcolor[HTML]{FFC702}{\color[HTML]{000000} \textbf{À Surveiller}} & \cellcolor[HTML]{FE0000}{\color[HTML]{000000} \textbf{Critique}} & {\color[HTML]{000000} \textbf{Critique}} \\ \hline
%\multicolumn{1}{|c|}{\cellcolor[HTML]{BBDAFF}Maximum} & \cellcolor[HTML]{FFC702}{\color[HTML]{000000} \textbf{À Surveiller}} & \cellcolor[HTML]{FE0000}{\color[HTML]{000000} \textbf{Critique}} & \cellcolor[HTML]{FE0000}{\color[HTML]{000000} \textbf{Critique}} & {\color[HTML]{000000} \textbf{Critique}} \\ \hline
%\end{tabular}
%\caption{Les priorités des risques} \label{table:prioriteRisque}
%\end{table}
%
%Les risques peuvent être de differente nature :
%\begin{itemize}
% \item juridique (lié au contrat) ;
% \item technique ;
% \item projet (équipe, organisationnel) ;
% \item externe (conjoncturel, par exemple : grève, épidémie, catastrophe naturelle, fermeture
%des locaux) ;
% \item infrastructure (matériel et locaux à disposition de l’équipe PIC) ;
% \item humain (maladies, problemes de differentes types) .
%\end{itemize}
%
%La phase d'apparition du risque est également indiquée sur la fiche de risque. Elle correspond à la phase du projet dans laquelle le risque peut survenir.
%
%\subsection{Suivi des risques}  \label{suivi_des_risques}
%
%Cette phase permettra de garder à jour le \PR{} . Les évolutions de gestion des risques seront soumises, validées et ré-actualisées collectivement lors des réunions hebdomadaires. Les Fiches de Risques pourront donc être mises à jour au fur et à mesure de l'évolution des risques. Chaque membre de l'équipe \PICCourt{} peut proposer l'ajout d'un nouveau risque au \PR{}.\\
%
%Si on estime que le risque ne présente plus aucun problème pour le projet, ce dernier est clôturé. \\
%
%Toute modification opérée sur le risque va entrainer un changement sur sa Fiche de Risque. Le suivi des risques permet de maîtriser et minimiser un maximum de risques.
%
%\subsection{Réduction des risques}  \label{reductionRisques}
%
%L'équipe PIC doit essayer de réduire au maximum les risques qui sont identifiés. \\
%
%L'émission d'une nouvelle Fiche de Risque donnera lieu à une analyse de risque afin de définir des actions préventives et de proposer un plan de contournement. Le résultat de cette analyse sera présent dans la Fiche de Risque.
