Le but de cette partie est de décrire les spécifications opérationnelles du projet. Une première sous-partie permettra de décrire les demandes de performances du projet avec des exigences chiffrés précises, une seconde sous-partie permettra de décrire les besoins en sécurité, intégrité et sûreté tel que le chiffrage, la gestion des droit et les logs, enfin, une dernière sous-partie détaillera les besoins en terme de base de donnée tels que les contraintes d'intégrités ou les types d'accès.

\subsection{Performances}

Cette sous-partie a pour objectif de décrire les besoins en performances du projet avec des valeurs chiffrées précises.




\subsection{Sécurité, intégrité et sûreté}

L'objectif de cette sous-partie est de détailler les besoins en sécurité, en intégrité et en sûreté du logiciel final.



%Les mots de passe des utilisateurs seront hachés et salés avant d'être stocké dans la base de donnée. La récupération du mot de passe devra se faire de manière sécurisé, soit par envoie par email d'un lien pour le réinitialiser, soit par le biais d'une question secrète (uniquement si l'utilisateur n'a pas d'email car moins sécurisé). 

%Un fichier log sera conservé avec les connexion de chacun des bénévoles et leurs actions principales.
%Aucun fichier de log n'est nécessaire…

\subsection{Base de données}

Cette sous-partie permet de détailler les besoins concernant la base de données, tels que les logiques de comportement de chacune des données écrites, de la fréquence d'utilisation, des types d'accès ou encore des contraintes d'intégrité.