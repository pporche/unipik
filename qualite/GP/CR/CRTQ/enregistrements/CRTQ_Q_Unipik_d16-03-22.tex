% version 1.00	date 22/03/2016	Auteur Pierre Porche

\documentclass [a4paper] {article}
\usepackage[utf8]{inputenc}
\usepackage[francais]{babel}
\usepackage[top=2cm, bottom=4cm, left=2cm, right=2cm]{geometry} 
\usepackage{fancyhdr}
\usepackage{graphicx}
\usepackage{color, colortbl}
\usepackage{longtable}
\usepackage{vocabulaireUnipik}
\pagestyle{fancy}
\definecolor{Gray}{gray}{0.8}



%--- En-t�te et pied de page ---%
\renewcommand{\footrulewidth}{0,01cm}

\begin{document}
\rhead{}
\chead{\huge{Compte-rendu de réunion de Tutorat Qualité}}					%titre

22/03/2016
\hfill   
\hfill 	9:30 - 10:06 				% Heure de d�but, heure de fin.


\lfoot{Version : 1.00} 			% version

%--- Fin en-t�te et pied de page ---%
\section*{Historique des révisions}
\begin{center}
			\begin{tabular}{| c | c | c | c | p{4cm} |}
				\hline
				\rowcolor{Gray}
				Version & Date & Auteur(s) & Modification(s) & Partie(s) modifiée(s)		 \\
				\hline
				1.00 & 22/03/2016 & \Pierre & Création & Toutes \\
		\hline		
			\end{tabular}
		\end{center}

\section*{Signatures}

		\begin{center}
			\begin{tabular}{| c | c | c | c | p{4cm} |}
				\hline
				\rowcolor{Gray}
				Rôle & Fonction & Nom & Date & Visa		 \\
				\hline
				Vérificateur & \RQA & \Kafui & 23/03/2016 & pgpic \\[30pt]
				\hline
				Validateur & \CP & \Sergi &  & pgpic \\[30pt]	
				\hline
			\end{tabular}
		\end{center}

%--- Réunion --%

\section{Versionnage}
Suite aux revues, \nomTuteurQualite{} nous informe que la première version d'un document est la version 0.00 puis 0.01, 0.02,.. et que dès l'approbation, on passe à la version 1.00 et c'est celle ci que l'on diffuse. \\
Ensuite, à la première modification, on envoie une version 1.01 et l'on incrémente. Pour les documents sans approbation, on commence à la v1.00. \\
Il est important de communiquer avec le client à ce sujet. \\ ~ \\
Sur le tableau de suivi des évolutions, on ne renseigne que les versions (X.00) mais avec \emph{toutes} les modifications depuis la dernière version approuvée.\\
Toutes les révisions sont à conserver.


\section{Audit}
En ce qui concerne les en-têtes et les dates de diffusion notamment, nous manquons de rigueur. Chaque remarque déclenche un \FFT{} individuelle qui sera associée à un \OC{} individuel. \\
Il faudra remplir le rapport d'Audit, le scanner et l'envoyer à \nomTuteurQualite{} en format pdf qui nous la renverra avec un récapitulatif.


\end{document}















