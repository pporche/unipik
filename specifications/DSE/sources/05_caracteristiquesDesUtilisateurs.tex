% version 1.00, date 20/02/16, auteur Michel Cressant
Dans cette partie, nous décrivons les utilisateurs de l'application, leurs compétences et leurs expériences.\\ 

Les utilisateurs principaux de l'application seront des bénévoles de \nomClient{}, pour la plupart retraités. Ils ont une expérience très limitée des outils informatiques. Nous pouvons considérer que les bénévoles sont capables (ou recevront une formation) de recevoir, de lire et d'envoyer des emails mais aussi de se servir d'un navigateur web, de parcourir un site internet, de s'enregistrer, de s'authentifier et enfin de remplir des formulaires web.\\

L'application devra être très simple d'utilisation afin de s'adapter au mieux aux compétences des utilisateurs.\\

Les autres utilisateurs de l'application seront les contacts de \nomClient{}, qui sont les différents établissements souhaitant effectuer des demandes d'interventions. On peut considérer qu'ils possèdent un niveau égal ou supérieur à celui des bénévoles quant à l'utilisation des outils informatiques.
