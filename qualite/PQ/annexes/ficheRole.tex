\section{\CP}
\subsection*{Introduction}

Le \CP doit garantir le bon déroulement du \PICCourt. Il possède des missions d’organisation et de validation du travail effectué par les membres de l’équipe. Il est également l’interlocuteur privilégié du Tuteur Pédagogique, du Tuteur Qualité et du client.

\subsection*{Tâches liées à sa fonction}

Une passation devra être mise en place entre les \CPs du premier et second semestre. Cette passation devra si possible être formalisée sous la forme d’une formation.

\subsection*{Tâches effectuées au démarrage du \PICCourt}

Le \CP devra se conformer aux exigences de la période de démarrage du \PICCourt :
\begin{itemize}
	\item Organiser l’équipe \PICCourt au premier semestre et de la fin du premier au second semestre en tenant compte des éventuels départs à l’étranger, redoublements, réorientations ou retours de mobilité académique. Ainsi qu’organiser la formation en début du second semestre du \PICCourt
	\item Créer le dépôt \git sur https://monprojet.insa-rouen.fr et des espaces publics et privés des membres de l’équipe \PICCourt
	\item Rédiger ou faire rédiger, puis valider les \FC de son équipe et éventuellement attester certaines compétences d’ordre personnel.
	\item Établir l’organigramme des fonctions de son \PICCourt et les descriptifs de ces fonctions.
	\item Associer les ressources aux fonctions de son \PICCourt.
\end{itemize}

\subsection*{Tâches effectuées au cours du \PICCourt}

Le \CP devra remplir au cours du \PICCourt les missions suivantes :

\begin{itemize}
	\item Mettre à jour les \FC des membres du \PICCourt.
	\item Archiver de manière hebdomadaire l’ensemble des espaces privés des membres et les conserver sur clé USB jusqu’à la semaine suivante.
	\item Garantir les journaux de tests.
	\item Participer à la \CTFT (CTFT).
	\item Gérer le budget de fonctionnement du \PICCourt.
	\item Représenter les ressources globales du \PICCourt et le solde des ressources consommées par un \WBSCourt minimal, un OBS, un \RBSCourt ou un \FBSCourt.
	\item Collecter les risques de tâches mis en lumière par les membres de l’équipe.
	\item Participer à la réunion de la conduite de projet ou suivi prévisionnel tenue au minimum une fois par semestre.
	\item Effectuer le débriefing à la revue de \PICCourt.
	\item Accorder les dérogations de possibilité de diffusion des non-conformités en accord avec le client.
\end{itemize}

Le \CP devra également remplir ou déléguer ces missions au \CPA :
\begin{itemize}
	\item Animer la réunion d’avancement hebdomadaire.
	\item Établir les GANTT pour le suivi des tâches passées, en cours et futures.
	\item Reprendre le planning de la semaine passée et le mettre à jour en fonction des retards estimés, des réunions exceptionnelles et des corrections possibles.
	\item Vérifier les fiches de suivi hebdomadaire des membres de l’équipe.
\end{itemize}

\subsection*{Tâches effectuées en fin de période du \PICCourt}

Le \CP devra en fin de période remplir les missions suivantes :
\begin{itemize}
	\item Livrer au secrétariat de la Direction du Département \ASICourt l’archivage de l’espace public des membres de l’équipe en fin de semestre.
	\item Garantir avec la Direction Qualité dans un \PVCourt la fin de la phase d’intégration.
\end{itemize}
\newpage

\section{\CPA}
\subsection*{Introduction}

Durant le premier semestre, le \CPA doit seconder le \CP et effectuer les tâches qu’il lui aura déléguées. À la fin de ce semestre, dans la majorité des cas, il devra se préparer à assurer le rôle de \CP.

\subsection*{Tâches liées à sa fonction}

Le \CPA devra remplir les missions suivantes :
\begin{itemize}
	\item Établir avec le \CP le planning détaillé des activités à réaliser ;
	\item Réaliser avec le \CP le suivi du projet ;
	\item Remplacer le \CP si celui-ci est indisponible ;
	\item Préparer le semestre suivant, en collaboration avec le \CP, à la fin du premier semestre, s’il prend la fonction de \CP au second semestre.
\end{itemize}

\newpage
\section{\RQ}
\subsection*{Introduction}

Le \RQ est le garant de l’application de la politique qualité au sein du \PICCourt. Il peut être épaulé dans cette tâche par d’autres membres du \PICCourt.

\subsection*{Tâches liées à sa fonction}

Une passation devra être mise en place entre les \RQs du premier et second semestre. Cette passation devra si possible être formalisée sous la forme d’une formation.\\
Tout au long du projet, le \RQ devra veiller à la bonne adéquation entre les tâches liées à la réalisation des livrables et le référentiel qualité. Pour assurer le bon déroulement de cette veille qualité, il devra réaliser les tâches suivantes :

\subsubsection*{Tâches liées au \PQCourt}
\begin{itemize}
	\item Rédiger et organiser le suivi du \PQ (\PQCourt) en respectant les exigences du Référentiel Qualité et en particulier de la \DGQDEUXCourt.
	\item Assurer la bonne diffusion (c’est à dire, l’envoi après approbation) du \PQCourt aux membres de l’équipe \PICCourt.
	\item Vérifier le \PQCourt après l’exécution d’actions correctives (cette tâche peut être déléguée à un autre membre du \PICCourt par dérogation personnelle ou de la part du \CP).
	\item Valider l’ensemble des procédures qualité rédigées au sein du \PICCourt.
	\item Fournir un accompagnement aux équipes de développement dans la démarche qualité.
	\item Réaliser des activités régulières de contrôle de l’ensemble du système qualité.
	\item Sensibiliser les membres de l’équipe \PICCourt à la norme \ISOCourt 9001:2015.
\end{itemize}

\subsubsection*{Tâches liées au \PGCCourt}

Cette partie peut être déléguée dès le début du \PICCourt à un autre membre du \PICCourt, possédant la compétence exigée, qui prendra alors la responsabilité de la gestion des configurations.

\begin{itemize}
	\item Rédiger et organiser le suivi du \PGCCourt en respectant les exigences du Référentiel Qualité et en particulier de la \DGQDEUXCourt.
	\item Vérifier le \PGCCourt après l’éxécution d’actions correctives (cette tâche peut être déléguée à un autre membre du \PICCourt par dérogation personnelle ou de la part du \CP).
	\item Garantir l’application du \PGCCourt.
	\item S’assurer du bon déroulement de la gestion des modifications des différents documents.
\end{itemize}

\subsubsection*{Tâches liées à la gestion du référentiel}

\begin{itemize}
	\item Chaque semestre, les \RQs doivent se réunir et fournir au pilote de processus 2, une liste de 5 questions par responsable visant à évaluer la maîtrise de ce référentiel ;
	\item Chaque semestre, les \RQs doivent se réunir pour se répartir et corriger les demandes d’amélioration présentes sur l’outil de suivi des référentiels ;
	\item Chaque semestre, à chaque incohérence et problème soulevé, les \RQs doivent ajouter des demandes d’amélioration à l’aide de l’outil de suivi des référentiels.
\end{itemize}

\newpage
\section{\RQA}
\subsection*{Introduction}

Durant le premier semestre, le \RQA{} doit seconder le \RQ{} et effectuer les tâches qu’il lui aura déléguées. À la fin de ce semestre, dans la majorité des cas, il devra se préparer à assurer le rôle de \RQ .

\subsection*{Tâches liées à sa fonction}

Le \RQA devra remplir les missions suivantes :
\begin{itemize}
	\item Tenir à jour les différents indicateurs mis en place pour le \PICCourt.
	\item Remplacer le \RQA{} si celui-ci est indisponible ;
	\item Seconder le \RQ{} dans la réalisation du \PQ, du \PGC{} ainsi que des autres documents liés au projet.
        \item Participer au développement des fonctionnalités nécessaires au projet. 
\end{itemize}

\newpage
\section{\RGC}
\subsection*{Introduction}

Au début du projet, le \RGC{} doit s'occuper de la rédaction du \PGCCourt{} qui sert à contrôler l’activité de gestion des configurations pendant toute la durée du \PICCourt. L’élaboration de ce document permet donc de fixer toutes les règles de la gestion des configurations. Ce dernier sera amené à évoluer tout au long du projet afin que cette gestion soit toujours adaptée au \PICCourt.

\subsection*{Tâches liées à sa fonction}

Le \RGC{} devra remplir les missions suivantes :
\begin{itemize}
	\item Mettre en place le gestionnaire de sources du \PICCourt.
	\item Rédiger le \PGC.
	\item Fixer les règles de la gestion des configurations.
        \item Clôturer les ordres de corrections après avoir vérifié que la procédure de correction/vérification avait bien été respectée.
        \item Se charger de l'effacement du dépôt des sources.
        \item S'assurer que les membres de l'équipe \PICCourt{} respectent le \PGCCourt.
\end{itemize}

\newpage
\section{\RRS}
\subsection*{Introduction}

Au début du projet, le \RRS doit s'occuper de la mise en place du réseau dans la salle \PICCourt afin que chaque membre ait accès à internet et puisse travailler dans de bonnes conditions. Il sera chargé de maintenir ce réseau fonctionnel tout au long du projet.

\subsection*{Tâches liées à sa fonction}

Le \RRS devra remplir les missions suivantes :
\begin{itemize}
	\item Mettre en place le réseau en salle \PICCourt.
	\item S'assurer du bon fonctionnement du réseau tout au long du \PICCourt.
	\item S'assurer du bon état de fonctionnement du serveur.
	\item Faire évoluer l'architecture du réseau en fonction des besoins utilisateurs.
	\item Participer à la gestion technique des équipements. Cela consiste à réceptionner les matériels informatiques et de télécommunications, les tester, les adapter, les insérer dans le réseau en fonctionnement et effectuer le suivi du parc de matériels.
\end{itemize}

\newpage
\section{\RD}
\subsection*{Introduction}

Tout au long du projet, le \RD doit s'occuper de la gestion de toutes les parties concernant le développement. Il aura donc sous sa charge un certain nombre de développeurs et va devoir s'assurer que les résultats issus du développement sont conformes au \DGQDEUXCourt.

\subsection*{Tâches liées à sa fonction}

Le \RD devra remplir les missions suivantes :
\begin{itemize}
	\item Définir avec le \CP les différentes phases de conception et de développement.
	\item Établir les phases de vérification et de validation du développement.
	\item Participer à la réalisation du \DSECourt, du \DSICourt et du \PTVCourt.
	\item S'assurer de la réussite des tests unitaires et d'intégration.
	\item S'assurer que les codes fournis fonctionneront également chez le client.
\end{itemize}