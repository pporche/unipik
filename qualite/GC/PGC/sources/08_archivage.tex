% version 1.00	Auteur Mathieu Médici

Ce chapitre présente comment les documents, versionnés ou non, seront archivés dans le cadre de notre projet. L’archivage se divise en deux parties : l’archivage informatique et l’archivage papier.

\section{Archivage informatique}
Tous les documents du \PICCourt sont archivés de façon numérique sur le serveur \git{} mis à disposition par le département \ASI{}. On y trouve à la fois les documents relatifs à la démarche qualité, mais aussi le code, les mails, ou tout autre document créé pour le projet.

\section{Archivage papier}
Certains documents nécessitent un archivage papier spécifique car ils comportent pour la plupart des signatures manuscrites. Ils sont obligatoirement rangés dans le classeur correspondant se trouvant dans l’armoire de la salle du projet. Ces documents sont :
\begin{itemize}
\item Engagements de confidentialité;
\item Rapports d’audits;
\item Formations;
\item Procès verbaux.
\end{itemize}
Ces documents ne doivent pas sortir de la salle.