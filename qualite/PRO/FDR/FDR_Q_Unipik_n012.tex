\section*{Informations générales}
 
\begin{table}[H]
\centering
	\begin{tabularx}{16.8cm}{|X|X|}
	\hline
	\rowcolor{gray!40} Numéro du risque & Type du risque \\
	\hline
	012 & Retard de remise du livrable au client \\
	\hline
	\end{tabularx}
\end{table}

\begin{table}[H]
\centering
	\begin{tabularx}{16.8cm}{|X|X|X|}
	\hline
	\rowcolor{gray!40} Date & Visa du \RQ & Visa du \CP \\
	\hline
	  &  &  \\
	\hline
	\end{tabularx}
\end{table}

\begin{table}[H]
\centering
	\begin{tabularx}{16.8cm}{|X|X|X|X|}
	\hline
	\rowcolor{gray!40} Pilote & Activité WBS & Compte WBS & Phase d'apparition \\
	\hline
	 \Kafui & Suivre les Risques et Opportunités & 1.2.3.2 & À tout moment\\
	\hline
	\end{tabularx}
\end{table}

\section*{Description du risque}

\subsection*{Résumé}
	Le risque lié au retard de remise du livrable au client pourrait considérablement impacter la notoriété de l'établissement et de l'équipe PIC . Ce risque pourrait aussi éventuellement impliquer sur les activités du client.
	
\subsection*{Analyse des causes}
	voir figure \ref{risque retard de remise du livrable}.

\subsection*{Criticité}

\begin{table}[H]
\centering
	\begin{tabularx}{16.8cm}{|>{\columncolor{gray!40}}X|X|}
	\hline
	Gravité & 4\\
	\hline
	Probabilité & 1\\
	\hline
	Criticité & Critique\\
	\hline
	\end{tabularx}
\end{table}
\newpage

\section*{Actions}
\subsection*{Actions préventives}

%\begin{table}[H]
\centering
	\begin{longtable}{|p{7cm}|p{7cm}|}
	\hline
	\rowcolor{gray!40} Numéro de cause & Actions préventives \\
	\hline
	 1 & \begin{itemize}
	 	\item Communiquer régulièrement avec le client.
	 	\item
	 	Prévoir des réunions à un rythme régulier
	 \end{itemize} \\
	\hline
	2 & \begin{itemize}
		\item Voir les actions préventives au risque associé 
	\end{itemize} \\
	\hline
	3 & \begin{itemize}
		\item Rechercher un support convenant à la fois au client et à l'équipe PIC.
	\end{itemize} \\
	\hline
	4 & \begin{itemize}
		\item Voir les actions préventives au risque associé 
	\end{itemize} \\
	\hline
	5 & \begin{itemize}
		\item S'assurer que chaque tâche soit réparti
à la personne approprié.
	\item
	S'assurer que chaque tâche soit réparti
à la personne approprié.
\item Maintenir les fiches de compétences à jour. 
\item Proposer une formation adéquat dans le cas où un manque se déclare.	
	
	\end{itemize} \\
	\hline
	\end{longtable}
%\end{table}

\flushleft
\subsection*{Plan de contournement}

\begin{enumerate}
	\item Utiliser des données test en attendant un retour du client
\end{enumerate}

\section*{Décision de clôture}
Par le \CP{} et le pilote du risque.
\begin{table}[H]
\centering
	\begin{tabularx}{16.8cm}{|X|X|}
	\hline
	\rowcolor{gray!40} Date de clôture & Raison de la clôture \\
	\hline
	  & \\
	\hline
	\end{tabularx}
\end{table}

\section*{Historique des modifications}
\begin{table}[H]
\centering
	\begin{tabularx}{16.8cm}{|X|X|}
	\hline
	\rowcolor{gray!40} Date & Modification \\
	\hline
	  & \\
	\hline
	\end{tabularx}
\end{table}
\newpage

\begin{figure}
	\centering
	\includegraphics[scale=0.35]{images/AnalyseRisque_nPourquoi_FDR012}
	\caption{\label{risque retard de remise du livrable} Retard de remise du livrable - méthode des n pourquoi}
\end{figure}
