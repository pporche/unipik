% version 1.00, date 08/03/16, auteur Michel Cressant
  
  Ce chapitre à pour but de présenter les éléments de test, les fonctionnalités qui devront être testées, les documents et éléments nécessaires à la réalisation des tests d'intégration, la procédure à suivre une fois un test exécuté, l'environnement de travail nécessaire à la réalisation d'un test et la validation d'un test. 
 
 
 \section{Elements de test}
 	Les éléments qui seront testés permettront de vérifier l'intégration des modules de code développer par l'équipe \nomEquipe.
 	
 \section{Fonctionnalités à tester}
	Ce paragraphe décrit l'organisation d'un modèle MVC ainsi que les différents tests d'intégration qui devront être réalisés. \\ 	
 	
 	La figure \ref{modeleMVC} représente l'organisation d'un modèle MVC communiquant avec une base de données. Les flèches représente comment communique les différentes parties du modèle. Les tests d'intégration devront vérifier que ces parties communiquent correctement.
 	
 	\begin{figure}[H]
 		\centering
 		\includegraphics[scale=0.85]{images/modeleMVCPlusBD.pdf}
 		\caption{Représentation d'un Modèle Vue Contrôleur}
 		\label{modeleMVC}
 	\end{figure}
 	
 Les différents tests seront :\\
 
 \subsection*{Tests d'Intégration Contrôleur Vue}
  \begin{center}
    \begin{tabular}[h]{|p{0.45\textwidth}|p{0.45\textwidth}|}
		\hline
		ID & Test d'intégration \\\hline
        TI001 & Contrôleur -> Vue \\\hline
     \end{tabular}
  \end{center}
  
  \begin{figure}[H]
  	\centering
  	\includegraphics[scale=0.8]{images/testControleurVue.pdf}
  	\caption{Tests d'intégration du Contrôleur et des Vues}
  	\label{testControleurEtVue}
  \end{figure}
  
  
 \subsection*{Tests d'Intégration du Modèle Vue Contrôleur}
  \begin{center}
    \begin{tabular}[h]{|p{0.45\textwidth}|p{0.45\textwidth}|}
	\hline
	ID & Test d'intégration \\\hline
        TI002 & Contrôleur -> Modèle \\ & Modèle -> Contrôleur \\\hline 
     \end{tabular}
  \end{center}
  
  \begin{figure}[H]
  	\centering
  	\includegraphics[scale=0.8]{images/testControleurModele.pdf}
  	\caption{Tests d'intégration sur le Modèle Vue Contrôleur}
  	\label{testModele}
  \end{figure}    
  
  
  \subsection*{Tests d'Intégration du Modèle Vue Contrôleur communiquant avec la Base de Données}
  \begin{center}
    \begin{tabular}[h]{|p{0.45\textwidth}|p{0.45\textwidth}|}
	\hline
		ID & Test d'intégration \\\hline
		TI003 & Modèle -> Base de données \\
         & Base de données -> Modèle \\\hline
     \end{tabular}
  \end{center}
  
  \begin{figure}[H]
  	\centering
  	\includegraphics[scale=1.15]{images/testModeleBD.pdf}
  	\caption{Test d'intégration sur le Modèle Vue Contrôleur avec la Base de Données}
  	\label{testBaseDeDonnees}
  \end{figure}
  
  
  
  
  
  \section{Eléments nécessaires pour les Test d'Intégration}
  	Ce paragraphe décrit les éléments livrés nécessaires à la réalisation des tests d'intégration. \\
  	
  	Les éléments suivant doivent être livrés avant que les tests d'intégrations puissent commencer :
  	\begin{itemize}
  		\item Le \PTU,
  		\item Les chapitres 1, 2 et 3 de ce document.
  	\end{itemize}
  	
  	Les éléments suivant doivent être livrés avant qu'un test d'intégration spécifique ne commence :
  	\begin{itemize}
  		\item Le \JTU,
  		\item Les composants impliqués,
  		\item Les données tests spécifiques à ce test d'intégration.
  	\end{itemize}
  
  \section{Rapport de Test d'Intégration}  
  	Ce paragraphe décrit la procédure à éxecuter une fois un test éxécuté.  \\
  	
  	Les éléments suivant sont à réaliser lorsqu'un test d'intégration spécifique est finis : 
  	\begin{itemize}
  		\item Remplir le rapport du test dans le \JTU, qui listera tous les tests d'intégration,
  		\item Rapporter les problèmes rencontrés (si nécessaire).
  	\end{itemize}
  	
  	Les éléments suivant sont à réaliser quand lorsque la session de test sur tous les modules de code sont finis : 
  	\begin{itemize}
  		\item Remplir tous les rapports de test dans le \JTU.
  	\end{itemize}
  
  \section{Environnement de Test}
	  Ce paragraphe décrit l'environnement nécessaire à la réalisation des tests d'intégration. \\
	  
	  L'installation des éléments suivants sont nécessaires à la réalisations des tests dintégration : 
	  \begin{itemize}
	  	\item php5,
	  	\item apache2 (à configurer),
	  	\item postgresql,
	  	\item Symfony (framework).
	  \end{itemize}
  
  \section{Validation d'un Test}
	Ce paragraphe décrit les conditions à remplir pour qu'un test d'intégration soit validé.\\
  
  	Un test d'intégration est validé s'il ne provoque aucune erreur, et retourne les résultats escomptés. 
  	
  	