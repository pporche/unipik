% version 1.00	Auteur Kafui Atanley date 16/09/2016

\documentclass [a4paper] {article}
\usepackage[utf8]{inputenc}
\usepackage[francais]{babel}
\usepackage[top=2cm, bottom=4cm, left=2cm, right=2cm]{geometry} 
\usepackage{fancyhdr}
\usepackage{graphicx}
\usepackage{color, colortbl}
\usepackage{longtable}
\usepackage{vocabulaireUnipik}
\usepackage{hyperref}
\DeclareUnicodeCharacter{00A0}{ }
\pagestyle{fancy}
\definecolor{Gray}{gray}{0.8}



%--- En-t�te et pied de page ---%

\renewcommand{\footrulewidth}{0,01cm}
\rhead{}
\chead{\huge{Compte-rendu de réunion de Tutorat Pédagogique}}					%titre
\begin{document}

12/09/2016			 				%Date 
\hfill   
\hfill 	 13:30 - 14:06				%Heure de d�but, heure de fin.


\lfoot{Version : 1.00} 			% version
%--- Fin en-t�te et pied de page ---%
\section*{Historique des révisions}
\begin{center}
			\begin{tabular}{| p{2.5cm} | p{3cm} | p{3cm} | p{3cm} | p{3.5cm} |}
				\hline
				\rowcolor{Gray}
				Version & Date & Auteur(s) & Modification(s) & Partie(s) modifiée(s)		 \\
				\hline
				1.00 & 16/09/2016 & \Kafui & Création & Toutes \\
				\hline			
			\end{tabular}
		\end{center}

\section*{Signatures}

		\begin{center}
			\begin{tabular}{| p{2.5cm} | p{4cm} | p{3cm} | p{3cm} | p{2.5cm} |}
				\hline
				\rowcolor{Gray}
				Rôle & Fonction & Nom & Date & Visa		 \\
				\hline
				Vérificateur & \RGC & \Melissa & 20/09/2016 & pgpic \\[30pt]
				\hline
				Validateur & \CP & \Pierre &  --- & --- \\[30pt]	
				\hline
			\end{tabular}
		\end{center}

%--- Réunion --%
\section{Résumé de la semaine passée}
La semaine précédente s'est orientée autour de 8 axes :  
\begin{itemize}
\item Script bash de création de la base de données et de son remplissage opérationnels ;
\item Documentation du wiki mis à jour ;
\item Backend/Frontend de création/consultation/modification/suppression de bénévole terminés ;
\item Backend/Frontend de création/consultation/modification/suppression d'établissement terminés ;
\item Formation de \Juliana{} et \Francois{} terminée ;
\item Backend/Frontend du formulaire de demande avancés ;
\item Modification des documents relatifs la qualité ;
\item Veille des indicateurs et documents propres à la gestion de projet.
\end{itemize} 

\section{Base de données}
\nomTuteurPedago{} nous recommade très fortement d'utiliser PostGis dès que possible. Nous avons une incohérence de méthode : ne pas intégrer PostGis dans ce lot impliquerait de retravailler la conception de la base de données ce qui semble impensable étant donné les délais impartis. 
\nomTuteurPedago{} nous fait remarquer plusieurs incohérences dans la base de données  : 
\begin{itemize}
\item certains domaines comportent des erreurs (exemple : département Val de Marine) ;
\item la cohérence de domaine est à revoir ( les expressions régulières pour les e-mails, les heures et les code postaux sont incorrectes) ;
\item le choix de la longueur des chaînes de caractères semble être aléatoire, un domaine devrait être utilisé pour le fixer (longueur courte, moyenne, longue) ;
\item les noms de certains attributs sont des noms systèmes (ex : date), ceci pourrait poser problème au niveau du mapping avec le framework.
\end{itemize}

\section{Gestion de Projet}
La date de livraison a été fixé entre le \RD{} et le \CP{}. Celle-ci aura lieu le 17 octobre. Cette date variera en fonction de la disponibilité du client.

\section{Semaine suivante}
Il est prévu de terminer le Backend/Frontend sur la création/consultation/modification/suppression d' intervention.
Il est prévu de terminer le Backend/Frontend sur le formulaire de demande d'intervention.
Il est également prévu d'établir des vues avec des filtres de recherche sur les bénévoles et interventions. \\
\Francois{} et \Juliana{} passeront dans l'équipe Frontend. 
\Matthieu{} s'occupera de mettre en place les tests d'intégration et les tests unitaires. 
\Matthieu{} s'occupera également d'établir des fichiers de configuration pour permettre un déploiement automatique de notre application sur un serveur.
\Pierre{} s'occupera de communiquer avec le client pour la date de livraison.


%--- fin de réunion ---%
\newpage



\end{document}





















