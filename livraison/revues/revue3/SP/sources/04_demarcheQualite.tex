%version 1.00,	date 19/10/2016	auteur(s) Pierre Porche

\speaker{\Kafui}

\subsection{} % PAs besoin de titre


\begin{frame}
\frametitle{Continuité dans la démarche Qualité}
\begin{block}{Principe de l'amélioration continue}
\begin{itemize}
\item Passation 
\item Mise à jour des documents de Qualité
\item Suivi de l'évolution des risques et opportunités
\end{itemize}
\end{block}
\end{frame}




\begin{frame}
\frametitle{Suivi des risques et opportunités}
\begin{figure}
\begin{longtable}{|p{1.8cm}||p{3.5cm}|p{3.5cm}|}
\hline
 & \textbf{Risques} & \textbf{Opportunités} \\\hhline{|=||=|=|}
\multirow{1}{*}{\textbf{clôturés}} & \small Mauvaise mise en route du second semestre & \small Bonne passation inter-semestre \\\hline
\multirow{1}{*}{\textbf{à surveiller}} & \small \begin{itemize}	
						\item Retard de remise du livrable 
						\end{itemize}& \\\hline
\multirow{1}{*}{\textbf{déclenchés}} & \small Retard de remise du livrable & \\\hline
\end{longtable}
\caption{Récapitulatif des changements des risques et opportunités}
\end{figure}
\end{frame}



\begin{frame}
\frametitle{Risque retard du livrable}
\begin{block}{Contexte}
\begin{itemize}
\item Lot 2 aurait dû être livré fin septembre après négociation intiale
\item Livraison effectué le lundi 17 octobre
\item Risque déclenché le 1 octobre par le \RQ{} et le \CP{}
\end{itemize}
\end{block}
\end{frame}

\begin{frame}
\frametitle{Risque retard du livrable : Analyse des n-pourquoi}
\begin{itemize}
\item Inclure le graph
\end{itemize}
\end{frame}


\begin{frame}
\frametitle{Risque retard du livrable : actions curratives}
\begin{itemize}
\item Augmentation du nombre d'heure de travail (montrer le nombre d'heure est-il une bonne chose)
\item Décalage de la date de rendu du lot 2
\end{itemize}
\end{frame}


