% version 1.00	date 30/11/2016	Auteur Kafui Atanley

\documentclass [a4paper] {article}
\usepackage[utf8]{inputenc}
\usepackage[francais]{babel}
\usepackage[top=2cm, bottom=4cm, left=2cm, right=2cm]{geometry} 
\usepackage{fancyhdr}
\usepackage{graphicx}
\usepackage{color, colortbl}
\usepackage{longtable}
\usepackage{vocabulaireUnipik}
\pagestyle{fancy}
\definecolor{Gray}{gray}{0.8}



%--- En-t�te et pied de page ---%
\renewcommand{\footrulewidth}{0,01cm}

\begin{document}
\rhead{}
\chead{\huge{Compte-rendu de réunion de Tutorat Qualité}}					%titre

23/11/2016
\hfill   
\hfill 	11:21 - 12:04 				% Heure de d�but, heure de fin.


\lfoot{Version : 1.00} 			% version

%--- Fin en-t�te et pied de page ---%
\section*{Historique des révisions}
\begin{center}
			\begin{tabular}{| c | c | c | c | p{4cm} |}
				\hline
				\rowcolor{Gray}
				Version & Date & Auteur(s) & Modification(s) & Partie(s) modifiée(s)		 \\
				\hline
				1.00 & 30/11/2016 & \Kafui & Création & Toutes \\
		\hline		
			\end{tabular}
		\end{center}

\section*{Signatures}

		\begin{center}
			\begin{tabular}{| c | c | c | c | p{4cm} |}
				\hline
				\rowcolor{Gray}
				Rôle & Fonction & Nom & Date & Visa		 \\
				\hline
				Vérificateur & \RGC & \Melissa & -- & -- \\[30pt]
				\hline
				Validateur & \CP & \Pierre & -- & -- \\[30pt]	
				\hline
			\end{tabular}
		\end{center}

%--- Réunion --%

\section{Etude de la pertinence des indicateurs}
\paragraph*{}
Nous disposons actuellement de deux indicateurs spécifiques à la gestion de projet (temps de travail en salle PIC et retard de tâche). Nous avons également deux indicateurs spécifiques à la qualité du projet (temps de rédaction de compte-rendu et écart de date prévue/réelle de correction de \FT{}).
Avec la mise en place récente de SonarQube, nous avons pensé à mettre en place deux indicateurs supplémentaires afin de veiller à la qualité du code que sont la duplication et le nombre de bug détectés. Ceux-ci sont en cours de référencement dans le \PQ{} mais ils sont déjà utilisés.
Leurs seuils sont fixés respectivement à 15 \% de duplication maximum et à 2 bugs détectés au maximum.\\
\nomTuteurQualite{} nous signifie que nos indicateurs sont plutôt pertinents mais que nous devrions rechercher un indicateur supplémentaire vis-à-vis de la qualité du projet.
\section{Recommendation}
\paragraph*{}
\nomTuteurQualite{} nous conseille d'établir une documentation consistente car nous avons un client non technique. Cette documentation devra couvrir le déploiement et l'utilisation de l'outil.
\nomTuteurQualite{} nous recommande de ne pas négliger cet étape car la partie "service après-vente" des \PIC{} posent souvent problème.

\end{document}















