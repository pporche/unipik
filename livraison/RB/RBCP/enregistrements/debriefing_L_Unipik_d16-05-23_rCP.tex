% version 1.00	Auteur Sergi Colomies

\documentclass[asi]{picInsa}
\DeclareGraphicsRule{*}{pdf}{*}{}
\usepackage{pdfpages}


\usepackage{vocabulaireUnipik}

\setcounter{secnumdepth}{4}
\setcounter{tocdepth}{4}
\newcommand{\ligneMaj}[3] {
	\rowcolor[gray]{0.55} \textbf{\textit{#1}} & #2  &  #3\\
	\hline
}
\newcommand{\ligneSup}[3] {
	\rowcolor[gray]{0.65} |\textunderscore \textbf{\textit{#1}} & #2  &  #3\\
	\hline
}
\newcommand{\ligneMed}[3] {
	\rowcolor[gray]{0.75} \hspace{0.25cm} |\textunderscore #1  & #2 & #3 \\
	\hline
}
\newcommand{\ligneSub}[3] {
	\rowcolor[gray]{0.85}  \hspace{0.5cm} |\textunderscore #1 & #2 & #3\\
	\hline
}
\newcommand{\ligneSubSub}[3] {
	\rowcolor[gray]{0.95}  \hspace{0.75cm} |\textunderscore #1 & #2 & #3\\
	\hline
}
\newcommand{\ligneTache}[3] {
	\hspace{1.00cm} |\textunderscore #1 & #2 & #3\\
	\hline
}
\title{Débriefing \CP{}}
\author{\Sergi}


\titreGeneral{Débriefing \CP{}}
\sousTitreGeneral{\nomEquipe}
\titreAcronyme{\Huge{debriefing}}
\version{v1.00}
\titreDetaille{\large{debriefing\_L\_Unipik\_d16-05-23}}
\referenceVersion{debriefing\_L\_Unipik\_d16-05-23}
\auteurs{\Sergi{}}
\destinataires{Unité P3}
\resume{Le présent document contient la présentation du débriefing \CP{} \nomEquipe.}
\motsCles{débriefing, \CP, gestion de projet}
\natureDerniereModification{Création}
\modeDiffusionControle{}

\begin{document}

\couverture{}

\informationsGenerales{}
%% version 1.00, date 29/02/16, auteur Michel Cressant
\begin{pagesService}
	\begin{historique}
		% nouvelles versions à rajouter AU-DESSUS en recopiant les lignes suivantes et en les modifiant :
		\unHistorique{1.00}{02/02/2016}{\Sergi \newline \Pierre \newline \Michel}{Création}{Toutes}

	\end{historique}

        \begin{suiviDiffusions}

            % On place ici les diffusions
        	\unSuivi{1.00}{}{\nomEquipe{}, \nomPIC{}}
          
          
        \end{suiviDiffusions}

%%Signataires
        \begin{signatures}
	   \uneSignature{Vérificateur}{\RQ{},\newline \CPA}{\Pierre{}}{26/01/2016}{PGPic}
       \uneSignature{Validateur}{\CP{}}{\Sergi}{26/01/2016}{PGPic}
	   \uneSignature{Approbateur}{Le client}{\nomClient}{}{}
        \end{signatures}
	
	
	\section*{Documents de Références}
%\begin{documentsReference}
		\begin{listeDeReferences}
			\uneReference{NF EN ISO 9001}{Octobre 2015}
			\uneReference{\MQ{}}{ASI-MQ-MQASI}
			\uneReference{\DGQ{} du Processus "\DGQDEUX{}"}{ASI-DGQ-DGQ2}
			\uneReference{\PTU }{\PTUCourt\_ S\_\nomEquipe\_V1.00 }
			\uneReference{NF EN ISO 9000}{Octobre 2015}
		\end{listeDeReferences}
%	\end{documentsReference}	
	
\begin{terminologie}
		La terminologie (définitions et abréviations) utilisée dans le présent document est centralisée dans le \MQ{} \ASICourt{} (\emph{cf. ASI-MQ-MQASI}) de l'Unité P3.

		\begin{listeDAbreviations}
			\uneAbreviation{\asiCourt}{\asi}
			\uneAbreviation{\DSECourt}{\DSE}
			\uneAbreviation{\DGQDEUXCourt}{\DGQDEUX}
			\uneAbreviation{\INSACourt}{\INSA}
			\uneAbreviation{\MQCourt}{\MQ}
			\uneAbreviation{\PICCourt}{\PIC}
			\uneAbreviation{\PTVCourt}{\PTV}
			\uneAbreviation{MVC}{Modéle-Vue-Controleur}
			
		\end{listeDAbreviations}
	\end{terminologie}
	

	
	
\end{pagesService}

\begin{pagesService}
	\begin{historique}
		% nouvelles versions à rajouter AU-DESSUS en recopiant les lignes suivantes et en les modifiant :
		\unHistorique{1.00}{25/04/2016}{\Sergi{}}{Création}{Toutes}

	\end{historique}

        \begin{suiviDiffusions}

            % On place ici les diffusions
        	\unSuivi{1.00}{}{}
          
       
        \end{suiviDiffusions}

%%Signataires
        \begin{signatures}
	   \uneSignature{Vérificateur}{}{}{}{pgpic}
       \uneSignature{Validateur}{\CPA{}, \newline \RQ{}}{\Pierre}{}{pgpic}
	   \uneSignature{Approbateur}{}{}{}{}
        \end{signatures}
\end{pagesService}

\tableofcontents

\setcounter{chapter}{0}

\chapter{Introduction}
\label{Introduction}





\chapter{Présentation du PIC}
\label{presentation_PIC}
\section{Le client}
L'\nomClient{} Seine Maritime est un comité de l'\nomClient{},  agence de l'Organisation des Nations unies consacrée à l'amélioration et à la promotion de la condition des enfants.  L'\nomClient{} a activement participé à la rédaction, la conception et la promotion de la Convention relative aux droits de l'enfant (CIDE), adoptée lors du sommet de New York le 20 novembre 1989. L'\nomClient{} a reçu le prix Nobel de la paix le 12 janvier 1965. L'\nomClient{} s'est donné les objectifs prioritaires suivants :
\begin{itemize}
	\item l'éducation des filles,
	\item la vaccination et la lutte contre le SIDA et le VIH,
	\item la protection de l'enfance,
	\item la santé des nouveau-nés,
	\item l'égalité hommes-femmes\vspace{0.5cm}.
\end{itemize}

Pour notre projet nous travaillons avec l'\nomClient{} Seine-Maritime. Les fonctions de ce comité sont : l'organisation de plaidoyers dans les écoles sur des thèmes centraux de l'\nomClient, l'organisation d'opérations frimousses consistant à la fabrication de poupées "frimousses" par des élèves afin d'être revendues par la suite, et des actions ponctuelles telles que des ventes lors d’événements particuliers.



\section{Ses besoins}
Afin de pouvoir gérer ses différentes actions (plaidoyer, frimousse, actions ponctuelles,...), l'\nomClient{} Seine-Maritime a besoin d'un outil de gestion des interventions externes de ses bénévoles. Cette outil devra être capable de gérer l'ensemble des bénévoles du comité et de les relier aux différentes interventions externes de manière intelligente.\\
Le parc informatique étant assez hétérogène, tant au niveau matériel que logique, notre service devra fonctionner dans différents environnements (Windows, Linux, MAC). De plus les connaissances informatiques du client étant limitées dans certains cas, nous devrons produire un service intuitif et facile à prendre en main.



\section{Les livrables}
Le projet à réaliser a été découpé en quatre livrables sur la période des deux semestres.\vspace{0.5cm}\\
Le premier livrable constate de l'architecture matérielle et logicielle de l'outil final. Nous avons décidé de réaliser un service web développé en PHP avec une base de données en PostgreSQL. Le framework Symfony permettra de faire le lien entre les deux. Nous utiliserons une architecture trois tiers avec le patron de conception Modèle-Vue-Contrôleur (MVC).\vspace{0.5cm}\\
Le deuxième livrable couvre toute la partie de gestion des plaidoyers. Le livrable doit fonctionner à la fin du semestre pour pouvoir commencer à être utilisé par l'UNICEF dès la rentrée scolaire.\vspace{0.5cm}\\
Le troisième livrable couvrira la partie gestion des frimousses.\vspace{0.5cm}\\
Enfin le dernier livrable couvrira la partie gestion des actions ponctuelles et devra être rendu pour la fin du second semestre de PIC.



\section{L'équipe}
L'équipe PIC se décompose suivant l'organigramme fonctionnel ci-dessous (figure \ref{organigramme} :
\begin{figure}[!h]
	\begin{center}
	\includegraphics[scale=0.35]{images/organigrammeFonctionnel.pdf}
	\label{organigramme}
	\caption{Organigramme fonctionnel}
	\end{center}
\end{figure} 





\chapter{L'avancement du PIC}
\label{avancement}
\section{Prévisions pour le premier semestre}
En collaboration avec notre client, l'\nomClient{} Seine-Maritime, nous avons divisé le projet en quatre grands livrables : deux livrables par semestre. Notre client nous a bien fait comprendre qu'il souhaitait pouvoir commencer à utiliser le service pour les plaidoyers dès la fin du premier semestre. Il fallait donc que toute la partie concernant les plaidoyers soit traitée au premier semestre. Le client a aussi laissé sous-entendre que les livrables attendus au second semestre n'étaient pas les plus importants à leurs yeux et qu'ils seraient amenés à changer.\vspace{0.5cm}\\
Dans les parties suivantes vont être décrits les lots du premier semestre tels qu'ils ont été déterminés initialement.

\subsection{Lot 1}
Le premier lotissement est décrit de la manière suivante dans le cahier des charges : <<Le lot 1 couvre l'architecture matérielle et logicielle à mettre en place pour le fonctionnement de l'application dans sa globalité>>. Pour ce livrable il est donc attendu une réflexion de l'équipe sur l'architecture du système à mettre en place, non pas au niveau des langages de développement mais au niveau structurel et physique. Cette architecture a également été mise en place sur un exemple de base que l'on a développé. Ce livrable doit être rendu à $T_{0}+5$. 

\subsection{Lot 2}
Le deuxième lotissement va couvrir les fondamentaux et la gestion des plaidoyers. Il implémente sept fonctionnalités et doit être achevé pour la fin du premier semestre. Les fonctionnalités demandées sont les suivantes :
\begin{itemize}
	\item gestion des bénévoles avec la création, la suppression et la modification de ceux-ci dans la base de données.
	\item gestion des établissements avec la création, la suppression et la modification de ceux-ci dans la base de données.
	\item envoi de mails automatique aux établissements pour la démarche de création d'intervention.
	\item création d'un formulaire intelligent permettant aux établissements de s'inscrire dans la liste des interventions facilement et rapidement.
	\item affectation d'interventions aux plaideurs en fonction de leur géolocalisation et par envoi de mails automatiques.
	\item modification des interventions par un bénévole.
	\item envoi de mails de rappels une semaine avant la date prévue de l'intervention.
\end{itemize}



\section{Réalisations du premier semestre}
\subsection{Lot 1 : l'architecture}
Le premier lotissement correspond à l'architecture logicielle et matérielle du service demandé. Nous avons décidé de réaliser une architecture trois tiers composée d'un client, un serveur et une base de données. Le client représente les ordinateurs des bénévoles qui auront accès à l'interface du service et pourront l'utiliser sans se soucier de la logique métier et de la base données. Le serveur hébergera notre service et répondra aux requêtes des clients. Enfin la base de données sera séparée du serveur d'application afin de bien délimiter les différents cadres de développement. Nous avons également décidé de mettre en place le patron de conception Modèle-Vue-Contrôleur (MVC).\vspace{0.5cm}\\
En ce qui concerne la base de données, le client nous a imposé de la développer en PostgreSQL. Elle doit également pouvoir utiliser des objets spatiaux via la technologie PostGIS. Nous avons aussi pris en considération la faible volumétrie de la base : au niveau de la Seine-Maritime ne sont dénombrés qu'une dizaine de plaideurs pour 1500 établissements et un taux de connexion qui ne dépassera pas les dix connexions par seconde.\vspace{0.5cm}\\
En ce qui concerne le service web, nous avons décidé de le développer en PHP. Pour faire ce choix nous avons réalisé un état de l'art puis une comparaison des différents langages possibles. Suite à cette comparaison (figure \ref{comparaison_PHP}), il est apparu que le PHP est le langage de développement qui correspond le mieux à notre projet.
\begin{figure}[!h]
\begin{center}
    \begin{tabular}[h]{|p{0.45\textwidth}|p{0.45\textwidth}|}
    
	\hline
	\cellcolor{blue!15}PHP & \cellcolor{blue!15}Java J2EE \\\hline
        Serveur de type web & Serveur d'application ou conteneur de servlet \\\hline
        Moins lourd, consomme moins de ressources sur des applications peu complexes & Efficace sur des applications complexes \\\hline
        Evolutions fréquentes sans rétrocompatibilité & Moins d'évolutions et rétrocompatibilité avec Java \\\hline
        Langage adapté à l'univers web & Formation et utilisation du code une fois pour toute \\\hline
        Plus difficile de gérer la sécurité & Contrôle et validation des données inclue au coeur du langage \\\hline
        Intégration et apprentissage rapide des développeurs & Nécessité d'un bon niveau d'abstraction \\\hline
    \end{tabular}
    \caption{Comparaison des langages de développement web}
    \label{comparaison_PHP}
\end{center}
\end{figure}

Enfin afin de faire le lien entre notre base de données et notre application web, nous avons pris la décision d'utiliser le framework Symfony. En effet celui-ci possède de nombreux avantages dont celui de générer du code PHP à partir d'une base de données. Ses autres principaux avantages sont : être un logiciel open source, inclure un développement MVC avec une bonne séparation des trois couches et avoir des tests unitaires et fonctionnels intégrés.

\subsection{Lot 2 : les plaidoyers}
Le second livrable couvre toute la partie gestion des plaidoyers. Pour rappel, les plaidoyers sont des interventions menées par des bénévoles de l'\nomClient{} 76 dans des établissements scolaires sur des thèmes précis. Les grandes fonctionnalités à développer sont : la création de bénévoles dans la base de données, la création d'établissements dans la base de données, la création d'interventions dans la base de données et l'association de ces trois grandes entités pour former une intervention par un bénévole dans un établissement. Sont aussi demandées l'envoi de mails automatiques et la géolocalisation des interventions.

\subsection{Éléments non traités}



\section{Prévisions du second semestre}
\subsection{Lot 2 : les plaidoyers}


\subsection{Lot 3 : les frimousses}
Le troisième lot qui s'étend sur la première moitié du second semestre devra couvrir la totalité de la gestion des frimousses. Pour rappel, les opérations frimousses sont des opérations menées en établissements scolaires et consistant en la réalisation de poupées par les élèves afin que celles-ci puissent être vendues par la suite.\\
Les deux grandes fonctionnalités attendues pour ce livrable sont les suivantes :
\begin{itemize}
	\item envoie de mails automatiques
	\item association de bénévoles aux interventions de type frimousse\vspace{0.5cm}
\end{itemize}

Ce livrable est attendu pour la date $T_{1}+6$ semaines avec $T_{1}$ la date de début du second semestre.

\subsection{Lot 4 : les actions ponctuelles}
La dernière fonctionnalité qui s'étend sur la deuxième moitié du second semestre couvrira quant à elle la gestion des actions ponctuelles. Pour rappel les actions ponctuelles sont des ventes sur stands organisées par l'\nomClient{} ou bien des projets étudiants encadrés par l'\nomClient{}.\\
Les grandes fonctionnalités attendues pour ce lotissement sont les suivantes : 
\begin{itemize}
	\item géolocalisation des activités ponctuelles
	\item association entre un bénévole et une action ponctuelle dans la base de données
	\item gestion des statistiques de l'\nomClient{}\vspace{0.5cm}
\end{itemize}
Ce dernier livrable devra être rendu à la fin du temps imparti pour le PIC.




\chapter{La gestion d'équipe}
\label{gestion_equipe}


 
\begin{appendix}
\listoffigures
\addcontentsline{toc}{chapter}{Table des figures}
	 
\listoftables
\addcontentsline{toc}{chapter}{Liste des tableaux}
\end{appendix}
\pageQuatriemeCouverture

\end{document}
