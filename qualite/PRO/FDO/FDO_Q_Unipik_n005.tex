% version 1.02	Auteur Florian Leriche Pierre Porche

\section*{Informations générales}
 
\begin{table}[h]
\centering
	\begin{tabularx}{16.8cm}{|X|X|}
	\hline
	\rowcolor{gray!40} Numéro de l'opportunité & Type d'opportunité \\
	\hline
	005 & Bonne planification \\
	\hline
	\end{tabularx}
\end{table}

\begin{table}[h]
\centering
	\begin{tabularx}{16.8cm}{|X|X|X|}
	\hline
	\rowcolor{gray!40} Date & Visa du \RQ & Visa du \CP \\
	\hline
	 29/01/2016 & pgpic & pgpic \\
	\hline
	\end{tabularx}
\end{table}

\begin{table}[h]
\centering
	\begin{tabularx}{16.8cm}{|X|X|X|X|}
	\hline
	\rowcolor{gray!40} Pilote & Activité WBS & Compte WBS & Phase d'apparition \\
	\hline
	 \Pierre & Suivre les Risques et Opportunités & 1.2.3.2 & Tout au long du projet.\\
	\hline
	\end{tabularx}
\end{table}

\section*{Description de l'opportunité}

\subsection*{Résumé}

	La bonne planification concerne principalement l’appréciation des délais des tâches
et peut être expliquée par une bonne implication du \CP. Elle permet de ne pas prendre de retard de
livraison des lots.
	
\subsection*{Analyse des causes}
	voir figure \ref{opportunite bonne planification}.

\subsection*{Criticité}

\begin{table}[h]
\centering
	\begin{tabularx}{16.8cm}{|>{\columncolor{gray!40}}X|X|}
	\hline
	Bénéfice & 2\\
	\hline
	Probabilité & 1\\
	\hline
	Criticité & Faible \\
	\hline
	\end{tabularx}
\end{table}
\newpage

\section*{Actions}
\subsection*{Actions proactives}

%\begin{table}[H]
{\centering
	\begin{longtable}{|p{7cm}|p{7cm}|}
	\hline
	\rowcolor{gray!40}Cause & Actions proactives \\
	\hline
        Bonne prise en main des outils & \begin{itemize}
	 	\item Former les membres de l'équipe \PICCourt{}.
	 \end{itemize} \\
	\hline
        Bonne vision globale & \begin{itemize}
	 	\item Avoir un \CP{} impliqué et compétent.
	 \end{itemize} \\
	\hline
	\end{longtable}}
%\end{table}

\section*{Décision de clôture}
Par le \CP{} et le pilote du risque.
\begin{table}[h]
\centering
	\begin{tabularx}{16.8cm}{|X|X|}
	\hline
	\rowcolor{gray!40} Date de clôture & Raison de la clôture \\
	\hline
	  & \\
	\hline
	\end{tabularx}
\end{table}

\section*{Historique des modifications}
\begin{table}[h]
\centering
	\begin{tabularx}{16.8cm}{|X|X|}
	\hline
	\rowcolor{gray!40} Date & Modification \\%\rowcolor{gray!40} 
	\hline
	 06/09/2016 & modification pilote car changement d'équipe \\
	\hline
	 11/05/2016 & Probabilité baissée car fin du semestre en approche \\
	\hline
	 27/04/2016 & Bénéfice baissé car moins d'impact \\
	\hline
	 14/03/2016 & criticité \\
	\hline
	\end{tabularx}
\end{table}
\newpage


\begin{figure}
	\centering
	\includegraphics[scale=0.5]{images/nPourquoiFDO005}
	\caption{\label{opportunite bonne planification}Opportunité bonne planification - méthode des n pourquoi}
\end{figure}
