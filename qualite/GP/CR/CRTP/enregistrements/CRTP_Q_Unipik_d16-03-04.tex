% version 1.00	Auteur Pierre Porche date 07/03/2016

\documentclass [a4paper] {article}
\usepackage[utf8]{inputenc}
\usepackage[francais]{babel}
\usepackage[top=2cm, bottom=4cm, left=2cm, right=2cm]{geometry} 
\usepackage{fancyhdr}
\usepackage{graphicx}
\usepackage{color, colortbl}
\usepackage{longtable}
\usepackage{vocabulaireUnipik}
\usepackage{hyperref}
\pagestyle{fancy}
\definecolor{Gray}{gray}{0.8}



%--- En-t�te et pied de page ---%

\renewcommand{\footrulewidth}{0,01cm}
\rhead{}
\chead{\huge{Compte-rendu de réunion de Tutorat Pédagogique}}					%titre
\begin{document}

04/03/2016			 				%Date 
\hfill   
\hfill 	 9:00 - 10:07 				%Heure de d�but, heure de fin.


\lfoot{Version : 1.00} 			% version
%--- Fin en-t�te et pied de page ---%
\section*{Historique des révisions}
\begin{center}
			\begin{tabular}{| p{2.5cm} | p{3cm} | p{3cm} | p{3cm} | p{3.5cm} |}
				\hline
				\rowcolor{Gray}
				Version & Date & Auteur(s) & Modification(s) & Partie(s) modifiée(s)		 \\
				\hline
				1.00 & 07/03/2016 & \Pierre & Création & Toutes \\
		\hline		
			\end{tabular}
		\end{center}

\section*{Signatures}

		\begin{center}
			\begin{tabular}{| p{2.5cm} | p{4cm} | p{3cm} | p{3cm} | p{2.5cm} |}
				\hline
				\rowcolor{Gray}
				Rôle & Fonction & Nom & Date & Visa		 \\
				\hline
				Vérificateur & \RQA & \Kafui & 29/02/2016 & pgpic \\[30pt]
				\hline
				Validateur & \CP & \Sergi & 01/03/2016 & pgpic \\[30pt]	
				\hline
			\end{tabular}
		\end{center}

%--- Réunion --%

\section{Résumé semaine passée}
Vendredi dernier, nous avons fait une formation à ArgoUML dont l'évaluation s'est faite lundi. Tout le monde a eu une note suffisante pour valider la formation. Lundi, le client a accepté de décaler la date de rendu du lot 1 de deux semaines. Mardi, nous avons envoyé à l'\nomClient{} les documents de spécification (\DSE{}, \DSI{} et \PTV{}) et n'avons toujours pas de réponse, \nomTuteurPedago{} nous conseille de le relancer mardi prochain. La demande à la CNIL a été effectuée et nous avons commencé à préparer l'audit du 15 mars.\\
\Sergi{} a envoyé des mails à Anne Caldin et à Maxime Reynet au sujet du serveur prêté par l'INSA et Anne Caldin semble plutôt enthousiaste. \nomTuteurPedago{} nous rappelle qu'il ne fait pas sous estimer les discussions informelles qui peuvent être utiles afin de préparer les réunions plus formelles. Cela s'applique lorsque nous traitons avec des personnes avec un faible pouvoir décisionnel. Gilles Gasso est un soutien supplémentaire qu'il serait intéressant de contacter.\\
Nous avons apporté les corrections nécessaires sur les \DSI{} et \PTV{} et avons lu les remarques sur le diagramme de classe. Nous avons également avancé la rédaction du \DCP{} et avons commencé la rédaction des \DCD{}, \PTI{} et \PTU{}. \\
Dans le \DSI{}, il est écrit que nous utilisons un patron MVC et une architecture trois tiers, il faut donc expliquer ce que c'est et pourquoi nous l'utilisons. Dans le \DCP{}, il faut retrouver la manière dont l'équipe sera organisée. Le \DCD{}, lui, répond aux bonnes pratiques et est plus proche du code. \\
Le \PTV{} doit inclure des tests sur chacun des trois tiers et sur le Modèle, la Vue et le Contrôleur. Symfony intègre un framework de test unitaire. \\
Un test d'intégration possible est de créer deux ou trois pages qui font des appels à la base de donnée et de vérifier si ces appels (ajout, modification et consultation) sont bien répercutés dans la base. \\
Nous avons également commencé à nous pencher sur la revue. \nomTuteurPedago{} nous indique que la première revue sert principalement à rassurer le client sur le fait que nous avons compris ses attentes.


\section{Diagramme de classes}
\Julie{} pose des questions concernant les remarques de \nomTuteurPedago{} sur le diagramme de classes.\\
Sur le modèle du domaine, il faut encore une fois bien comprendre la différence entre constante de type et constante de domaine. Une adresse n'est pas une constante. \\
Les thèmes et les comités ont une association M/N car chaque comité peut traiter des thèmes qu'il souhaite. Page 11, il y a un problème de structuration sur contact/demande/etablissement. Il faut faire attention à la hiérarchisation des comités (départemental, régional, national). Lier un département à un autre n'est pas forcément une bonne approche et il faudrait réfléchir en termes géographiques. Le thème doit être rattaché au niveau.\\
Il faut demander au client si il y a besoin de vidéoprojecteur pour la confection de frimousses et si nous devons traiter les maisons de retraite dans notre projet. Concernant les VAE, pourquoi ne pas considérer leur mairies comme des établissements?


%--- fin de réunion ---%
\newpage



\end{document}





















