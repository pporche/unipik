\subsection{Performances}

Dans un premier temps, l'application ne sera déployée qu'au niveau régional, puis cette dernière pourra se voir étendre au niveau national. Elle doit donc supporter un nombre d'utilisateurs simultanés de l'ordre de la centaine. \\

Ensuite, concernant les informations qui seront stockées dans la base de données, cette dernière devra pouvoir accueillir les données relatives aux différents bénévoles dont le nombre est de l'ordre de la dizaine ainsi qu'à environ 1600 établissements. La base de données devra aussi pouvoir gérer les données concernant les interventions réalisées par les bénévoles dont le nombre s'élève à environ 200 par an.

\subsection{Sécurité, intégrité et sûreté}

Les utilisateurs devront utiliser un compte sécurisé à l'aide d'un login et d'un mot de passe dans le but d'utiliser l'application. Les mots de passe seront hashés et salés avant d'être stocké dans la base de données. En effet, cette méthode permet de renforcer la sécurité des informations en leur rajoutant une donnée supplémentaire afin d'empêcher que deux informations identiques conduisent à la même empreinte. \\

 La récupération du mot de passe devra se faire de manière sécurisée, soit par envoi par email d'un lien pour le réinitialiser, soit par le biais d'une question secrète (uniquement si l'utilisateur n'a pas d'email car cette solution est moins sécurisé).\\

En outre, il est important de conserver un historique de toutes les actions effectuées par un bénévole, y compris si le compte de celui-ci a été supprimé. 

\subsection{Base de données}

Dans un premier temps, concernant les informations relatives aux utilisateurs, ces derniers devront pouvoir posséder au moins une activité potentielle (parmi Plaidoyer, Frimousse et Action ponctuelle), avoir la possibilité d'encadrer un projet élève ou un projet étudiant et pourront posséder le statut de responsable d'une ou plusieurs activités. Ils devront aussi obligatoirement posséder une adresse e-mail. \\

Chacune des activités doit obligatoirement être liée à un ou plusieurs responsables. Ces derniers peuvent être en charge de plusieurs activités. De plus, une autre contrainte d'intégration du système est que toute action ayant été effectuée le soit par un bénévole existant. \\

Ensuite, concernant les informations relatives aux établissements,ils peuvent posséder plusieurs emails et la base de données doit recenser le nom des contacts dans l'école. Il en existe deux types : le contact pour une activité en particulier ainsi que le responsable de l'école. Un établissement est rattaché à une ville unique, est identifié par un UAI unique mais ne possède pas forcément de nom. Lorsque ces derniers souhaitent effectuer une demande d'intervention, ils peuvent proposer une plage de date et préciser pour chaque jour de la semaine le moment à privilégier (matin, après-midi, indifférent) et le moment à éviter. \\

Enfin, pour les données liées aux interventions, il faudra pouvoir stocker les informations concernant le matériel disponible (présence ou non d'un rétroprojecteur, d'un tableau interactif ou d'enceintes). Les plaidoyers sont associés à un thème qui dépend du niveau de la classe concernée par l'intervention :

\begin{tabular}{|p{5cm}|p{5cm}|p{5cm}|}
  \hline
  Maternelle & Primaire & Collège \\
  \hline
  Convention Internationale des Droits de l'Enfant & Convention Internationale des Droits de l'Enfant & Convention Internationale des Droits de l'Enfant \\
  \hline
  L'éducation & L'éducation & L'éducation \\
  \hline
  La santé - Alimentation & La santé - Alimentation & La santé - Alimentation \\
  \hline
  L'eau & L'eau & L'eau \\
  \hline
  Le harcèlement & Le harcèlement & Le harcèlement \\
  \hline
  & La santé (en général) & La santé (en général) \\
  \hline
  & Le travail des enfants & Le travail des enfants \\
  \hline
  & & Les enfants soldats \\
  \hline
  & & Les urgences mondiales \\
  \hline
\end{tabular}
