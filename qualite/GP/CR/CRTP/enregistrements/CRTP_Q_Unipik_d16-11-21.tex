% version 1.00	Auteur Kafui Atanley date 22/11/2016

\documentclass [a4paper] {article}
\usepackage[utf8]{inputenc}
\usepackage[francais]{babel}
\usepackage[top=2cm, bottom=4cm, left=2cm, right=2cm]{geometry} 
\usepackage{fancyhdr}
\usepackage{graphicx}
\usepackage{color, colortbl}
\usepackage{longtable}
\usepackage{vocabulaireUnipik}
\usepackage{hyperref}
\DeclareUnicodeCharacter{00A0}{ }
\pagestyle{fancy}
\definecolor{Gray}{gray}{0.8}



%--- En-t�te et pied de page ---%

\renewcommand{\footrulewidth}{0,01cm}
\rhead{}
\chead{\huge{Compte-rendu de réunion de Tutorat Pédagogique}}					%titre
\begin{document}

21/11/2016			 				%Date 
\hfill   
\hfill 	 13:30 - 13:54				%Heure de d�but, heure de fin.


\lfoot{Version : 1.00} 			% version
%--- Fin en-t�te et pied de page ---%
\section*{Historique des révisions}
\begin{center}
			\begin{tabular}{| p{2.5cm} | p{3cm} | p{3cm} | p{3cm} | p{3.5cm} |}
				\hline
				\rowcolor{Gray}
				Version & Date & Auteur(s) & Modification(s) & Partie(s) modifiée(s)		 \\
				\hline
				1.00 & 22/11/2016 & \Kafui & Création & Toutes \\
				\hline			
			\end{tabular}
		\end{center}

\section*{Signatures}

		\begin{center}
			\begin{tabular}{| p{2.5cm} | p{4cm} | p{3cm} | p{3cm} | p{2.5cm} |}
				\hline
				\rowcolor{Gray}
				Rôle & Fonction & Nom & Date & Visa		 \\
				\hline
				Vérificateur & \RGC & \Melissa & 23/11/2016 & email \\[30pt]
				\hline
				Validateur & \CP & \Pierre &  23/11/2016 & email \\[30pt]	
				\hline
			\end{tabular}
		\end{center}

%--- Réunion --%
\section{Résumé de la semaine passée}
La semaine précédente s'est orientée autour de 5 axes :  
\begin{itemize}
	\item La liste récapitulative des interventions est maitenant imprimable au format PDF;
	\item La demande d'une intervention est maintenant accessible depuis l'intervention;
	\item La base de données a été modifiée afin de rajouter les triggers nécessaires;
	\item Le formulaire de demande anonyme a été avancé (50 \% fait);
	\item Il y a eu contact avec le client afin de fixer le lotissement.
\end{itemize} 

\section{Lotissement}
	Un email a été envoyé au client afin de confirmer le contenu du lot 3 et 4. 
Le lot 3 contiendra les éléments suivants :
\begin{itemize}
	\item Finalisation du formulaire de demande dans sa version anonyme et dans sa version pré-remplie;
	\item Prise en compte des remarques du client vis-à-vis du lot précédent;
	\item Implémentation de la géolocalisation;
	\item Vente de Frimousse;
	\item Finalisation du mailing.
\end{itemize} Il sera livré le 5 ou le 7 décembre en fonction de la disponibilité du client.
Le lot 4 contiendra les éléments suivants et ajoutera un module de statistiques et prendra en compte les remontées du client vis-à-vis du lot 3.
Il sera livré le 14 décembre en fonction de la disponibilité du client.

\section{Contact avec le client}
Le client ne nous a pas fourni les modèles d'email à envoyer dans les cas de confirmation/rappel d'intervention. \nomTuteurPedago{} nous conseille d'anticiper en rédigeant ceux-ci par nous-mêmes et en les présentant au client. Il faut également que nous nous chargions d'aller chercher les plaquettes de prospection au client afin de les faire suivre avec les emails de prospection. 
Une réunion aura lieu lundi prochain chez le client dans ses locaux afin d'initier un bénévole lambda à l'application.
Le client nous a envoyé le jeu de données pour l'année 2016.

\section{Hébergement}
Nous n'avons pas eu de nouvelles de Quantic Télécom. Le \CP{} va se charger de les relancer cette semaine.
%--- fin de réunion ---%
\newpage



\end{document}





















