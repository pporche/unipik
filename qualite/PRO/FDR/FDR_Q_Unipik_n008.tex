\section*{Informations générales}
 
\begin{table}[H]
\centering
	\begin{tabularx}{16.8cm}{|X|X|}
	\hline
	\rowcolor{gray!40} Numéro du risque & Type du risque \\
	\hline
	008 & Indisposition du client pour le passage d'une recette \\
	\hline
	\end{tabularx}
\end{table}

\begin{table}[H]
\centering
	\begin{tabularx}{16.8cm}{|X|X|X|}
	\hline
	\rowcolor{gray!40} Date & Visa du \RQ & Visa du \CP \\
	\hline
	 29/01/2016 & pgpic & pgpic \\
	\hline
	\end{tabularx}
\end{table}

\begin{table}[H]
\centering
	\begin{tabularx}{16.8cm}{|X|X|X|X|}
	\hline
	\rowcolor{gray!40} Pilote & Activité WBS & Compte WBS & Phase d'apparition \\
	\hline
	 \Julie & Suivre les Risques et Opportunités & 1.2.3.2 & À partir du début du projet\\
	\hline
	\end{tabularx}
\end{table}

\section*{Description du risque}

\subsection*{Résumé}
	Une indisponibilité du client lors d'un rendez-vous prévu pour un passage de recette peut retarder la livraison du produit par l'équipe. 
	
\subsection*{Analyse des causes}
	voir figure \ref{risque indisposition client}.

\subsection*{Criticité}

\begin{table}[H]
\centering
	\begin{tabularx}{16.8cm}{|>{\columncolor{gray!40}}X|X|}
	\hline
	Gravité & 3\\
	\hline
	Probabilité & 3\\
	\hline
	Criticité & \`A surveiller\\
	\hline
	\end{tabularx}
\end{table}
\newpage

\section*{Actions}
\subsection*{Actions préventives}

\centering
	\begin{longtable}{|p{7cm}|p{7cm}|}
	\hline
	\rowcolor{gray!40} Numéro de cause & Actions préventives \\
	\hline
	1 & \begin{itemize}
		\item Rappel des rendez-vous quelques jours avant
		\end{itemize} \\
	\hline
	2 & \begin{itemize}
		\item Prévoir un deuxième créneau de rendez-vous
		\end{itemize} \\
	\hline
	3 & \begin{itemize}
		\item Vérification de la plannification \CP{} par le \CPA{}
		\end{itemize} \\
	\hline
	\end{longtable}

\flushleft
\subsection*{Plan de contournement}

\begin{enumerate}
	\item Fixer une nouvelle date de recette avec le client dans les meilleurs délais.
\end{enumerate}

\section*{Décision de clôture}
Par le \CP{} et le pilote du risque.
\begin{table}[H]
\centering
	\begin{tabularx}{16.8cm}{|X|X|}
	\hline
	\rowcolor{gray!40} Date de clôture & Raison de la clôture \\
	\hline
	  & \\
	\hline
	\end{tabularx}
\end{table}

\section*{Historique des modifications}
\begin{table}[H]
\centering
	\begin{tabularx}{16.8cm}{|X|X|}
	\hline
	\rowcolor{gray!40} Date & Modification \\
	\hline
	  & \\
	\hline
	\end{tabularx}
\end{table}
\newpage


\begin{figure}
	\centering
	\includegraphics[scale=1.2]{images/AnalyseRisque_nPourquoi_FDR008}
	\caption{\label{risque indisposition client}risque indisposition du client pour le passage d'une recette - méthode des n pourquoi}
\end{figure}
