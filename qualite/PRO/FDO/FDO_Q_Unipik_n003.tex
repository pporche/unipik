% version 1.00, date 28/01/16, auteur Pierre Porche
\section*{Informations générales}
 
\begin{table}[h]
\centering
	\begin{tabularx}{16.8cm}{|X|X|}
	\hline
	\rowcolor{gray!40} Numéro de l'opportunité & Type d'opportunité \\
	\hline
	003 & Bonne passation \\
	\hline
	\end{tabularx}
\end{table}

\begin{table}[h]
\centering
	\begin{tabularx}{16.8cm}{|X|X|X|}
	\hline
	\rowcolor{gray!40} Date & Visa du \RQ & Visa du \CP \\
	\hline
	 29/01/2016 & pgpic & pgpic \\
	\hline
	\end{tabularx}
\end{table}

\begin{table}[h]
\centering
	\begin{tabularx}{16.8cm}{|X|X|X|X|}
	\hline
	\rowcolor{gray!40} Pilote & Activité WBS & Compte WBS & Phase d'apparition \\
	\hline
	 \Pierre & Suivre les Risques et Opportunités & 1.2.3.2 & A partir de la fin du premier semestre.\\
	\hline
	\end{tabularx}
\end{table}

\section*{Description de l'opportunité}

\subsection*{Résumé}
	Grâce à une bonne passation, le temps de prise en main du projet par la nouvelle équipe pourrait être grandement réduit.
	
\subsection*{Analyse des causes}
	voir figure \ref{opportunite bonne passation}.

\subsection*{Criticité}

\begin{table}[h]
\centering
	\begin{tabularx}{16.8cm}{|>{\columncolor{gray!40}}X|X|}
	\hline
	Bénéfice & 3\\
	\hline
	Probabilité & 3\\
	\hline
	Criticité & Important\\
	\hline
	\end{tabularx}
\end{table}
\newpage

\section*{Actions}
\subsection*{Actions proactives}

%\begin{table}[H]
{\centering
	\begin{longtable}{|p{7cm}|p{7cm}|}
	\hline
	\rowcolor{gray!40} Cause & Actions proactives \\
	\hline
	 Rédaction de documents de passation & \begin{itemize}
	 	\item Attribuer cette tâche
	 	\item Inclure cette tâche dans le planning
	 	\item Commencer cette tâche à temps
	 \end{itemize} \\
	\hline
	Arrivée de la nouvelle équipe avant départ de l'ancienne & \begin{itemize}
		\item Demander à la nouvelle équipe d'arriver au plus tôt
	\end{itemize} \\
	\hline
	Communication préalable avec la nouvelle équipe & \begin{itemize}
		\item Organiser des rendez-vous téléphoniques
		\item Prévenir la nouvelle équipe de ces rendez-vous
	\end{itemize} \\
	\hline
	Passation prévue et organisée & \begin{itemize}
		\item Réfléchir au processus de passation
		\item Inclure la préparation de la passation dans le planning
		\item Nommer un responsable de la passation
	\end{itemize} \\
	\hline
	Mise en place d'outils de communication performants & \begin{itemize}
		\item Rechercher les outils les plus adaptés
		\item Tester différents outils
	\end{itemize} \\
	\hline
	Bon investissement de la nouvelle équipe & 
	 \\
	\hline
	Bonne planification & \begin{itemize}
		\item Former le \CP à la planification
		\item Bien prévoir les temps nécessaires à chaque tâche
	\end{itemize} \\
	\hline
	Bon investissement de l'ancienne équipe & 
	 \\
	\hline

	\end{longtable}}
%\end{table}


\section*{Décision de clôture}
Par le \CP{} et le pilote du risque.
\begin{table}[h]
\centering
	\begin{tabularx}{16.8cm}{|X|X|}
	\hline
	\rowcolor{gray!40} Date de clôture & Raison de la clôture \\
	\hline
	  & \\
	\hline
	\end{tabularx}
\end{table}

\section*{Historique des modifications}
\begin{table}[h]
\centering
	\begin{tabularx}{16.8cm}{|X|X|}
	\hline
	\rowcolor{gray!40} Date & Modification \\%\rowcolor{gray!40} 
	\hline
	  & \\
	\hline
	\end{tabularx}
\end{table}
\newpage


\begin{figure}
	\centering
	\includegraphics[scale=0.5]{images/AnalyseOpportunite_nPourquoi_FDO003}
	\caption{\label{opportunite bonne passation}Opportunité bonne passation - méthode des n pourquoi}
\end{figure}
