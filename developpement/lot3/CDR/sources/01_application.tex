% version 1.00, date 07/10/16, auteur Julie Pain

\section{Fonctionnalité F8}

	\subsection*{Critère 1.1 : Aide à l’attribution des bénévoles aux interventions}
	
		\begin{center}
    	 		\begin{tabular}[h]{|p{0.95\textwidth}|}
			\hline
				La page de la liste des interventions permet d’effectuer des filtres sur les interventions pour aider l’utilisateur à choisir les interventions qui lui conviennent.\\
Il est possible de trier selon les filtres suivants :
		\begin{itemize}
			\item le type d’intervention : plaidoyers, frimousses, autres ;
			\item le statut de l’intervention : attribuées, non attribuées, réalisées ;
			\item le niveau scolaire de l’intervention : maternelle, primaire, collège, lycée, autre ;
			\item le thème de l’intervention ;
			\item le lieu de l’intervention ;
		\end{itemize}
				
				$\square$ Oui \hfill \hfill $\square$ Non \\\hline Remarques : \\ ~\\
			 \\\hline
     		\end{tabular}
  		\end{center}	

	\subsection*{Critère 1.2 : Attribution d’une intervention à un bénévole par l’administrateur}
	
		\begin{center}
    	 		\begin{tabular}[h]{|p{0.95\textwidth}|}
			\hline
				Si l’utilisateur est connecté en tant qu’administrateur, il peut attribuer l’intervention au
bénévole qu’il souhaite.\\
Nous attribuons une intervention à un bénévole.\\
En consultant l’intervention en question, on voit que l’intervention est bien attribuée.\\
				
				$\square$ Oui \hfill \hfill $\square$ Non \\\hline Remarques : \\ ~\\
			 \\\hline
     		\end{tabular}
  		\end{center}	

	\subsection*{Critère 1.3 : Attribution d’une intervention à un bénévole par lui même}
	
		\begin{center}
    	 		\begin{tabular}[h]{|p{0.95\textwidth}|}
			\hline
				Si l’utilisateur est connecté en tant que simple bénévole, il ne peut que s’attribuer l’intervention.
Nous nous attribuons une intervention.
En consultant l’intervention en question, on voit que l’intervention est bien attribuée.
Si nous allons voir la page "Mes interventions", celle-ci est bien présente.
				
				$\square$ Oui \hfill \hfill $\square$ Non \\\hline Remarques : \\ ~\\
			 \\\hline
     		\end{tabular}
  		\end{center}	

	\subsection*{Critère 1.4 : Aide à l’attribution des bénévoles aux interventions}
	
		\begin{center}
    	 		\begin{tabular}[h]{|p{0.95\textwidth}|}
			\hline
				La page de la liste des interventions permet d’effectuer des filtres sur les interventions pour aider l’utilisateur à choisir les interventions qui lui conviennent.\\
Il est possible de trier selon les filtres suivants :
		\begin{itemize}
			\item le type d’intervention : plaidoyers, frimousses, autres ;
			\item le statut de l’intervention : attribuées, non attribuées, réalisées ;
			\item le niveau scolaire de l’intervention : maternelle, primaire, collège, lycée, autre ;
			\item le thème de l’intervention ;
			\item le lieu de l’intervention ;
		\end{itemize}
				
				$\square$ Oui \hfill \hfill $\square$ Non \\\hline Remarques : \\ ~\\
			 \\\hline
     		\end{tabular}
  		\end{center}	

	\subsection*{Critère 1.5 : Aide à l’attribution des bénévoles aux interventions}
	
		\begin{center}
    	 		\begin{tabular}[h]{|p{0.95\textwidth}|}
			\hline
				La page de la liste des interventions permet d’effectuer des filtres sur les interventions pour aider l’utilisateur à choisir les interventions qui lui conviennent.\\
Il est possible de trier selon les filtres suivants :
		\begin{itemize}
			\item le type d’intervention : plaidoyers, frimousses, autres ;
			\item le statut de l’intervention : attribuées, non attribuées, réalisées ;
			\item le niveau scolaire de l’intervention : maternelle, primaire, collège, lycée, autre ;
			\item le thème de l’intervention ;
			\item le lieu de l’intervention ;
		\end{itemize}
				
				$\square$ Oui \hfill \hfill $\square$ Non \\\hline Remarques : \\ ~\\
			 \\\hline
     		\end{tabular}
  		\end{center}	
  		
  		
\section{Fonctionnalité F9}
  		
  	\subsection*{Critère 2.1 : Ajouter un établissement}
  		\begin{center}
    	 		\begin{tabular}[h]{|p{0.95\textwidth}|}
			\hline
				Seul un administrateur peut ajouter un établissement, il faut donc être connecté en tant qu'administrateur.\\
				Nous ajoutons un établissement. \\
				Lorsque nous consultons la liste des établissements, celui-ci est bien présent. \\
						
				
				$\square$ Oui  \hfill \hfill $\square$ Non \\\hline Remarques : \\ ~\\
			 \\\hline
     		\end{tabular}
  		\end{center}	
  		
  		
\section{Fonctionnalité F10}
\subsection*{Critère 3.1 : Mailing }
  		\begin{center}
    	 		\begin{tabular}[h]{|p{0.95\textwidth}|}
			\hline
				La page de mailing permet de sélectionner certains établissements en fonction de leur niveau scolaire puis de leur envoyer un mail contenant un lien qui redirige vers le formulaire de demande. \\
				Nous envoyons un mail de test à deux adresses email que nous relevons et nous suivons le lien vers le formulaire. \\
				Lorsque nous cliquons sur le lien, nous arrivons bien au formulaire de demande pré-rempli.
				
				$\square$ Oui  \hfill \hfill $\square$ Non \\\hline Remarques : \\ ~\\
			 \\\hline
     		\end{tabular}
  		\end{center}	
