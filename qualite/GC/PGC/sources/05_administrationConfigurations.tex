\section{Suivi des configurations}

Afin de permettre le suivi de ces documents ainsi que leur état en temps réel, \nomEquipe{} utilise l'intranet \lintranet. 
Il permet de savoir quelle est la dernière version applicable d'un document et récapitule l'ensemble du contenu des référentiels. 
Cette page est actualisée par la personne concernée lors de la création, la vérification, la validation, l'approbation, la diffusion et l'archivage d'un document. Sur cette page figurent les informations suivantes:
\begin{itemize}
	\item \textbf{référentiel}: nom du référentiel auquel le document appartient;
	\item \textbf{document}: nom du document;
	\item \textbf{rédacteur}: nom du rédacteur qui a initié le document;
	\item \textbf{vérificateur}: nom du vérificateur;
	\item \textbf{validateur}: nom du validateur;
	\item \textbf{approbation}: le document a été approuvé ou non (case verte si le document a été
	approuvé, rouge sinon);
	\item \textbf{diffusion}: le document a été diffusé ou non (case verte si le document a été
	diffusé, rouge sinon);
	\item \textbf{archivage}: le document a été archivé ou non (case verte si le document a été
	archivé, rouge sinon);
	\item \textbf{référence}: nom complet du document;
	\item \textbf{localisation}: où a été diffusé le document;
	\item \textbf{chemin}: chemin complet d'accès au document sur le \git.
\end{itemize}


\section{État de Configuration}
\label{EC}

Un État de configuration décrit l'état des documents demandés par leur destinataire à une date donnée. Un état de configuration doit être généré entre autres à chaque livraison. Ils suivent le modèle présenté en Annexe \ref{modèle EC}. Il doit être composé de tous les documents nécessaire à la livraison : 
\begin{itemize}
\item \textbf{Pour une livraison client:}
	\begin{itemize}
	\item les livrables (code source, procédure d'installation, documentation technique);
	\item les documents de spécification.
	\end{itemize}
\item \textbf{Pour une inspection technique:}
	\begin{itemize}
	\item le code source;
	\item les documents de spécification;
	\item les documents qualité;
	\item les documents de gestion de projet (exemple: comptes-rendus de réunion) nécessaires à la compréhension des autres documents.\\
	\end{itemize}
\end{itemize}


Chaque fiche comprend un tableau indiquant les informations suivantes pour chaque document :
\begin{itemize}
	\item \textbf{Type}: le type du document. \\
	Exemple : \verb+PQ+;
	\item \textbf{Destinataire}: le destinataire du document;
	\item \textbf{Référence du document}: la référence complète du document. \\
	Exemple : \verb+PGC_Q_Unipik_v1.00+;
	\item \textbf{Date de création}: date de création du document.\\
	Format : \verb+JJ/MM/AAAA+ ;
	\item \textbf{Raisons}: le but de l'envoi de ce document.\\
	Exemple : \verb+Livraison lot 1+.
\end{itemize}


\bigskip
Par sa signature, le vérificateur de l'État de Configuration atteste avoir vérifié la présence
de tous les éléments nécessaires à la livraison ainsi que leur cohérence, tant au niveau de
la forme que du fond.

\paragraph{Précision du suffixe :}
Le suffixe de l'état de configuration est de type date et commentaire. Le suffixe \verb+date+ correspond à la date à laquelle les documents ont été envoyés par l’émetteur. Le suffixe \verb+commentaire+ doit définir la raison de la création de l'état de configuration. Les valeurs possibles sont :
\begin{itemize}
\item \verb+Inspection+;
\item \verb+Livraison+;
\item \verb+<NomDoc>+; \\
Exemple : \verb+cPQ+
\item \verb+Autre+.
\end{itemize}
Exemple : \verb+EC_Q_Unipik_d16-03-05_cLivraison+

\section{Fiche de Réception de Documents}
\label{FRD}

Une fiche de réception de documents décrit les documents remis par des personnes extérieures au PIC. Elles suivent le modèle présenté en Annexe \ref{modèle FRD}. Chaque fiche comprend un tableau indiquant les informations suivantes pour chaque document :
\begin{itemize}
\item \textbf{Date de réception} : la date de réception du document (format : \verb+JJ/MM/AAAA+);
\item \textbf{Emetteur} : l'émetteur du document;
\item \textbf{Raisons} : le but de l'envoi de ce document;
\item \textbf{Description du document} : la description du contenu du document;
\item \textbf{Référence du document} : le nom complet du document;
\item \textbf{Chemin du document} : le chemin complet du document (où il est stocké).
\end{itemize}

\paragraph{Précision du suffixe :}
Le suffixe de la fiche de réception de documents est de type date. Le suffixe \verb+date+ correspond à la date à laquelle les documents ont été envoyés par l’émetteur.