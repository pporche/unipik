% version 1.00	Auteur Kafui Atanley date 30/11/2016

\documentclass [a4paper] {article}
\usepackage[utf8]{inputenc}
\usepackage[francais]{babel}
\usepackage[top=2cm, bottom=4cm, left=2cm, right=2cm]{geometry} 
\usepackage{fancyhdr}
\usepackage{graphicx}
\usepackage{color, colortbl}
\usepackage{longtable}
\usepackage{vocabulaireUnipik}
\usepackage{hyperref}
\DeclareUnicodeCharacter{00A0}{ }
\pagestyle{fancy}
\definecolor{Gray}{gray}{0.8}



%--- En-t�te et pied de page ---%

\renewcommand{\footrulewidth}{0,01cm}
\rhead{}
\chead{\huge{Compte-rendu de réunion de Tutorat Pédagogique}}					%titre
\begin{document}

30/11/2016			 				%Date 
\hfill   
\hfill 	 14:03 - 14:33				%Heure de d�but, heure de fin.


\lfoot{Version : 1.00} 			% version
%--- Fin en-t�te et pied de page ---%
\section*{Historique des révisions}
\begin{center}
			\begin{tabular}{| p{2.5cm} | p{3cm} | p{3cm} | p{3cm} | p{3.5cm} |}
				\hline
				\rowcolor{Gray}
				Version & Date & Auteur(s) & Modification(s) & Partie(s) modifiée(s)		 \\
				\hline
				1.00 & 30/11/2016 & \Kafui & Création & Toutes \\
				\hline			
			\end{tabular}
		\end{center}

\section*{Signatures}

		\begin{center}
			\begin{tabular}{| p{2.5cm} | p{4cm} | p{3cm} | p{3cm} | p{2.5cm} |}
				\hline
				\rowcolor{Gray}
				Rôle & Fonction & Nom & Date & Visa		 \\
				\hline
				Vérificateur & \RGC & \Melissa & -- & -- \\[30pt]
				\hline
				Validateur & \CP & \Pierre &  -- & -- \\[30pt]	
				\hline
			\end{tabular}
		\end{center}

%--- Réunion --%
\section{Résumé de la semaine passée}
La semaine précédente s'est orientée autour de 4 axes :  
\begin{itemize}
	\item Finir le formulaire anonyme (100 \% fait);
	\item Les ventes ont été avancées (40 \% fait);
	\item Avancer les tests unitaires sur les entités (70 \% fait);
	\item La réunion chez le client ce Lundi.
\end{itemize} 

\section{Réunion chez le client}
\Pierre{} en tant que \CP{} et \Julie{} en tant que \RD{} ainsi que quatre bénévoles étaient présents au cours de cette réunion. La réunion avait pour but d'initier un bénévole lambda à l'application.  Nous avons eu de nombreux retours notamment au niveau de la navigabilité. La date de livraison du lot 3 a été précisé pour Mercredi prochain à 14h30. Elle se fera en présence de Cathy Guest, responsable des actions éducatives (anciennement dénommé plaidoyer), et au moins un autre bénévole. La livraison du lot 4 se fera en présence de Cathy Guest et de Véronic barbier (présidente de UNICEF 76).
\nomTuteurPedago{} nous conseille d'inviter les représentants au niveau national de l'Unicef pour la revue finale.\\


\section{Problème avec le serveur de production}
Nous avons rencontré un problème que nous ne parvenons à résoudre avec le serveur de production. Notre apprlication fonctionne correctement en pré-production. Les environnements sont identiques mais l'application ne fonctionne pas sur le serveur de production. \nomTuteurPedago{} nous conseille de vérifier le codage de l'application. D'expérience, ceux-ci est du à une variation de la taille du swap.
\\

\section{Géolocalisation}
\nomTuteurPedago{} nous fait remarquer qu'il serait intéressant d'intégrer la recherche d'itinéraire pour notre localisation. Ceux-ci seront intégrés dans le module si le temps se présente. La recherche de point géographique se fait à vol d'oiseau. Il serait judicieux de proposer plusieurs modes de recherche. Les filtres sur la géolocalisation sont au point sur toutes les listes excepté celle des bénévoles.\\

\section{Avancement dans les tests}
Nous avons également effectué des tests en boîte blanche sur le Front. Ceux-ci passent et le taux de couverture avoisinent la totalité du code.\\

\section{Hébergement}
Nous n'avons pas eu de nouvelles de Quantic Télécom. Le \CP{} va se charger de les relancer cette semaine.
%--- fin de réunion ---%
\newpage



\end{document}





















