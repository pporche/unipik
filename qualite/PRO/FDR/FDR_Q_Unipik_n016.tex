% version 1.00	date 28/03/2016  	Auteur Leriche Florian

\section*{Informations générales}

\begin{table}[H]
\centering
	\begin{tabularx}{16.8cm}{|X|X|}
	\hline
	\rowcolor{gray!40} Numéro du risque & Type du risque \\
	\hline
	016 & Attaque extérieure \\
	\hline
	\end{tabularx}
\end{table}

\begin{table}[H]
\centering
	\begin{tabularx}{16.8cm}{|X|X|X|}
	\hline
	\rowcolor{gray!40} Date & Visa du \RQ & Visa du \CP \\
	\hline
	 28/03/16 & pgpic & pgpic \\
	\hline
	\end{tabularx}
\end{table}

\begin{table}[H]
\centering
	\begin{tabularx}{16.8cm}{|X|X|X|X|}
	\hline
	\rowcolor{gray!40} Pilote & Activité WBS & Compte WBS & Phase d'apparition \\
	\hline
	 \Florian & Suivre les Risques et Opportunités & 1.2.3.2 & Tout le long du \PICCourt. \\
	\hline
	\end{tabularx}
\end{table}

\section*{Description du risque}

\subsection*{Résumé}
	Une attaque extérieur pourrait entrainer la perte de nombreuses données à caractères personnels fournies par le client. 
	
\subsection*{Analyse des causes}
	voir figure \ref{Attaque exterieure}.

\subsection*{Criticité}

\begin{table}[H]
\centering
	\begin{tabularx}{16.8cm}{|>{\columncolor{gray!40}}X|X|}
	\hline
	Gravité & 4\\
	\hline
	Probabilité & 1\\
	\hline
	Criticité & Critique\\
	\hline
	\end{tabularx}
\end{table}
\newpage

\section*{Actions}
\subsection*{Actions préventives}

%\begin{table}[H]
\centering
	\begin{longtable}{|p{7cm}|p{7cm}|}
	\hline
	\rowcolor{gray!40} Numéro de cause & Actions préventives \\
	\hline
	1 & \begin{itemize}
	 	\item Mettre en place un système de sécurité.
	 \end{itemize} \\
	\hline
	2 & \begin{itemize}
		\item Ne pas stocker les mots de passe utilisateur en clair dans la base de données.
	\end{itemize}	 \\
	\hline
	3 & \begin{itemize}
		\item Utiliser des outils qui permettent de développer de manière sécurisée.
	\end{itemize} \\
	\hline
	\end{longtable}
%\end{table}

\flushleft
\subsection*{Plan de contournement}

\begin{enumerate}
	\item Prévenir le client ainsi que le tuteur pédagogique.
	\item Informer la CNIL.
        \item Informer les personnes concernées.
\end{enumerate}

\section*{Décision de clôture}
Par le \CP{} et le pilote du risque.
\begin{table}[H]
\centering
	\begin{tabularx}{16.8cm}{|X|X|}
	\hline
	\rowcolor{gray!40} Date de clôture & Raison de la clôture \\
	\hline
	  & \\
	\hline
	\end{tabularx}
\end{table}

\section*{Historique des modifications}
\begin{table}[H]
\centering
	\begin{tabularx}{16.8cm}{|X|X|}
	\hline
	\rowcolor{gray!40} Date & Modification \\
	\hline
	  & \\
	\hline
	\end{tabularx}
\end{table}
\newpage


\begin{figure}
	\centering
	\includegraphics[scale=0.8]{images/nPourquoiFDR016}
	\caption{\label{Attaque exterieure}risque Attaque extérieure - méthode des n pourquoi}
\end{figure}
