% version 1.00, date 20/02/16, auteur Michel Cressant
\section{Objectif du document}
	Ce document présente la stratégie, l'organisation, l'environement et la planification des activités de test de validation du logiciel produit par l'équipe \nomEquipe{}.
	
\section{Responsabilité}
	Les responsables de ce document sont le \CP, le \RQ{} et le \RD.
	
\section{Présentation de l'application}
	L'application permettra la gestion des interventions externes de \nomClient.
	
\section{Contexte de réalisation}
	Cette application est réalisée par l'équipe \nomEquipe{} dans le cadre des \PIC.

\section{Déroulement d'une recette}


Le déroulement d'une recette se déroule de la manière suivante : \\
Un premier cahier de recettes vierge est envoyé pour une approbation des tests de validation.
\begin{itemize}
	\item Soit le cahier de recettes est approuvé;
	\item Soit le cahier de recette n'est pas approuvé, il est alors corrigé puis de nouveau envoyé pour approbation.
\end{itemize}
Ensuite se déroule la recette provisoire où sont envoyés le cahier de recette et le lot.
\begin{itemize}
	\item Soit aucunes remarques ne sont remontées sur le lot, il est donc approuvé par le client. Le cahier de recettes provisoire devient définitif et il y a diffusion finale du lot. 
	\item Soit des remarques sont remontées lors de la recette, le lot n'est pas approuvé, le lot passe donc en période probatoire.
\end{itemize}
Durant la période probatoire, le lot est corrigé en fonctions des remarques exprimées par le client dans le cahier de recette provisoire. \\
Enfin se déroule la recette définitive où sont envoyés le chahier de recettes défnitif et le lot corrigé. De même que précédemment plusieurs scénarii sont possibles :
\begin{itemize}
	\item Aucunes remarques sur le lot, le lot est approuvé, il y a donc diffusion finale du lot;
	\item Des remarques mineures sont soulevées sur le lot, le lot est donc corrigé en conséquence, vient ensuite une approbation définitive du lot avant la diffusion finale du lot;
	\item Le lot est accepté avec report de fonctionnalités;
	\item Des remarques majeures sont soulevées sur le lot, il n'est donc pas approuvé. Il y a alors retour en période probatoire.
\end{itemize}


