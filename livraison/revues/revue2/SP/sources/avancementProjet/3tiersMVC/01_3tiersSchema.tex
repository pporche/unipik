% version 1.01, date 07/03/16, auteur Matthieu Martins-Baltar
% =================================================
% Set up a few colours
\colorlet{lcfree}{Gray}
\colorlet{lcnorm}{Gray}
\colorlet{lccong}{Gray}
\colorlet{lccyan}{Cyan}
% -------------------------------------------------
% Set up a new layer for the debugging marks, and make sure it is on
% top
\pgfdeclarelayer{marx}
\pgfsetlayers{background,main,marx}
% A macro for marking coordinates (specific to the coordinate naming
% scheme used here). Swap the following 2 definitions to deactivate
% marks.
%\providecommand{\cmark}[2][]{%
%   \begin{pgfonlayer}{marx}
%   \node [nmark] at (c#2#1) {#2};
%  \end{pgfonlayer}{marx}
%  } 
\providecommand{\cmark}[2][]{\relax} 
% -------------------------------------------------

\begin{tikzpicture}[%
	auto,
    >=triangle 60,              % Nice arrows; your taste may be different
    start chain=going below,    % General flow is top-to-bottom
    node distance=6mm and 60mm, % Global setup of box spacing
    every join/.style={double},   % Default linetype for connecting boxes
    ]
% ------------------------------------------------- 
% A few box styles 
% <on chain> *and* <on grid> reduce the need for manual relative
% positioning of nodes
\tikzset{
  base/.style={draw, on chain, on grid, align=center, minimum height=4ex},
  rule/.style={base, rectangle, text width=8em, font={\scriptsize}, fill=lccyan!25},
  % coord node style is used for placing corners of connecting lines
  coord/.style={coordinate, on chain, on grid, node distance=6mm and 25mm},
  % nmark node style is used for coordinate debugging marks
  %nmark/.style={draw, cyan, circle, font={\sffamily\bfseries}},
  % -------------------------------------------------
  % Connector line styles for different parts of the diagram
  double/.style={<->, draw, lccyan}
}
% -------------------------------------------------
% Start by placing the nodes
\node [rule] 		(P)			{Présentation\\ (Navigateur Web)};
% Use join to connect a node to the previous one 
\node [rule, join]	(S)	{Métier\\ (Serveur Web)};
\node [rule, join]	(DB)		{Données\\ (Serveur SQL)};


\end{tikzpicture}