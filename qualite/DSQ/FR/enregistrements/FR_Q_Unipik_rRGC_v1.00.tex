% version 1.00	Date 01/03/2016	Auteur Pierre Porche

\documentclass[11pt]{article}
\usepackage{draftcopy}
\usepackage[francais]{babel}
\usepackage[utf8]{inputenc}
\usepackage{tabularx}
\usepackage{graphicx}
\usepackage[table]{xcolor}
\usepackage{fancyhdr}
\usepackage{vocabulaireUnipik}

\begin{document}
\chead{\huge Fiche de rôle \RGC}
\section*{Introduction}

Au début du projet, le \RGC{} doit s'occuper de la rédaction du \PGCCourt{} qui sert à contrôler l’activité de gestion des configurations pendant toute la durée du \PICCourt. L’élaboration de ce document permet donc de fixer toutes les règles de la gestion des configurations. Ce dernier sera amené à évoluer tout au long du projet afin que cette gestion soit toujours adaptée au \PICCourt.

\section*{Tâches liées à sa fonction}

Le \RGC{} devra remplir les missions suivantes :
\begin{itemize}
	\item Mettre en place le gestionnaire de sources du \PICCourt;
	\item Rédiger le \PGC;
	\item Fixer les règles de la gestion des configurations;
        \item Clôturer les ordres de corrections après avoir vérifié que la procédure de correction/vérification avait bien été respectée;
        \item Se charger de l'effacement du dépôt des sources;
        \item S'assurer que les membres de l'équipe \PICCourt{} respectent le \PGCCourt.
\end{itemize}


\end{document}