% version 1.00, date 12/11/16, auteur Kafui Atanley
Ce chapitre présente les maquettes pour chaque fonctionnalité pour chaque fonctionnalité additionnelle du lot 3 par rapport au lot 2.
\section{Lot 2}
\subsection{Fonctionnalité 1}
Ce paragraphe décrit les maquettes concernant la fonctionnalité 1. \\

La figure suivante (figure \ref{maquette1-1}) montre la maquette pour la création d'un profil.
\begin{figure}[H]
	\centering
	\includegraphics[scale=0.40]{images/maquettes/fonctionnalite1CreationDUnProfil.png}
	\caption{Maquette~: Création d'un profil }
	\label{maquette1-1}
\end{figure}

La figure suivante (figure \ref{maquette1-2}) montre la maquette pour la modification d'un profil.
\begin{figure}[H]
	\centering
	\includegraphics[scale=0.40]{images/maquettes/fonctionnalite1ModificationDUnProfil.png}
	\caption{Maquette~: Modification d'un profil}
	\label{maquette1-2}
\end{figure}

La figure suivante (figure \ref{maquette1-3}) montre la maquette pour la modification d'un profil par un administrateur.
\begin{figure}[H]
	\centering
	\includegraphics[scale=0.5]{images/maquettes/fonctionnalite1ModificationDUnProfilAdmin.png}
	\caption{Maquette~: Modification profil par un administrateur}
	\label{maquette1-3}
\end{figure}

\subsection{Fonctionnalité 3}
Ce paragraphe décrit la maquette concernant la fonctionnalité 3. \\

La figure suivante (figure \ref{maquette3-1}) montre la maquette pour le formulaire de demande d'intervention.
\begin{figure}[H]
	\centering
	\includegraphics[scale=0.3]{images/maquettes/fonctionnalite3FormulaireDeDemandeDInterventions.png}
	\caption{Maquette~: Formulaire de demande d'intervention}
	\label{maquette3-1}
\end{figure}

La figure suivante (figure \ref{maquette3-2}) montre l'email type de demande d'intervention aux établissements.
\begin{figure}[H]
	\centering
	\includegraphics[scale=0.3]{images/maquettes/fonctionnalite3MailType.png}
	\caption{Maquette~: Email type de demande d'intervention}
	\label{maquette3-2}
\end{figure}

\subsection{Fonctionnalité 4}
Ce paragraphe décrit la maquette concernant la fonctionnalité 4. \\

La figure suivante (figure \ref{maquette4-1}) montre l'email type de confirmation de réception de demande.
\begin{figure}[H]
	\centering
	\includegraphics[scale=0.4]{images/maquettes/fonctionnalite4MailConfirmationReceptionDemande.png}
	\caption{Maquette~: Email type de confirmation réception de demande}
	\label{maquette4-1}
\end{figure}

La figure suivante (figure \ref{maquette4-2}) montre l'email type pour l'administrateur de confirmation de suppression d'une demande.
\begin{figure}[H]
	\centering
	\includegraphics[scale=0.4]{images/maquettes/fonctionnalite4MailConfirmationSuppressionDemandePourAdmin.png}
	\caption{Maquette~: Email type administrateur de confirmation suppression de demande}
	\label{maquette4-2}
\end{figure}

La figure suivante (figure \ref{maquette4-3}) montre l'email type pour l'établissement de confirmation de suppression d'une demande.
\begin{figure}[H]
	\centering
	\includegraphics[scale=0.4]{images/maquettes/fonctionnalite4MailConfirmationSuppressionDemandePourEtablissement.png}
	\caption{Maquette~: Email type établissement de confirmation suppression de demande}
	\label{maquette4-3}
\end{figure}

\subsection{Fonctionnalité 5}
Ce paragraphe décrit la maquette concernant la fonctionnalité 5. \\

La figure suivante (figure \ref{maquette5-1}) montre la maquette de la géolocalisation des interventions ainsi que l'affectation d'un plaideur à celles ci. \\
\begin{figure}[H]
	\centering
	\includegraphics[scale=0.35]{images/maquettes/fonctionnalite5CarteDesInterventions.png}
	\caption{Maquette~: Géolocalisation des interventions et affectation à une intervention}
	\label{maquette5-1}
\end{figure}

La figure suivante (figure \ref{maquette5-2}) montre l'email type de prise en charge. \\
\begin{figure}[H]
	\centering
	\includegraphics[scale=0.35]{images/maquettes/fonctionnalite5MailDePriseEnCharge.png}
	\caption{Maquette~: Email type de prise en charge}
	\label{maquette5-2}
\end{figure}


\subsection{Fonctionnalité 6}
Ce paragraphe décrit la maquette concernant la fonctionnalité 6. \\

La figure suivante (figure \ref{maquette6}) montre l'email type récapitulatif des informations d'une intervention pour l'établissement.
\begin{figure}[H]
	\centering
	\includegraphics[scale=0.4]{images/maquettes/fonctionnalite6MailDInformationPourLEtablissement.png}
	\caption{Maquette~: Email type récapitulatif des informations d'une intervention pour l'établissement}
	\label{maquette6}
\end{figure}

\subsection{Fonctionnalité 7}
Ce paragraphe décrit la maquette concernant la fonctionnalité 7.\\

La figure suivante (figure \ref{maquette7}) montre l'email type de rappel pour le plaideur.
\begin{figure}[H]
	\centering
	\includegraphics[scale=0.4]{images/maquettes/fonctionnalite7MailDeRappelPourLePlaideur.png}
	\caption{Maquette~: Email type de rappel pour le plaideur}
	\label{maquette7}
\end{figure}

La figure suivante (figure \ref{maquette7}) montre l'email type de rappel pour l'établissement.
\begin{figure}[H]
	\centering
	\includegraphics[scale=0.4]{images/maquettes/fonctionnalite7MailDeRappelPourLEtablissement.png}
	\caption{Maquette~: Email type de rappel pour l'établissement}
	\label{maquette7}
\end{figure}
\section{Lot 3}
\subsection{Fonctionnalité 8}
Ce paragraphe décrit les maquettes concernant la fonctionnalité 8 soit la géolocalisation. \\

La figure suivante \ref{maquette8-1} montre la maquette d'une fiche descriptive d'une entité possédant une adresse. Un espace devra être réservé pour afficher la location de l'entité comme visible sur la fmaquette.
\begin{figure}[H]
	\centering
	\includegraphics[scale=0.40]{images/maquettes/fonctionnalite9Geolocalisation.png}
	\caption{Maquette~: Fiche descriptive d'une entité possédant une adresse}
	\label{maquette8-1}
\end{figure}
La figure suivante \ref{maquette8-2} présente la maquette des filtres de tri attendus sur la géolocalisation dans les pages permettant de lister les entités.
\begin{figure}[H]
	\centering
	\includegraphics[scale=0.40]{images/maquettes/fonctionnalite9Filtre.png}
	\caption{Maquette~: Modèle de filtre attendues dans les liste de tri}
	\label{maquette8-2}
\end{figure}

\subsection{Fonctionnalité 9}
Ce paragraphe décrit les maquettes concernant la fonctionnalité 9 soit l'attribution de frimousse. \\

La figure suivante \ref{maquette9-1} montre l'email type de prise en charge d'une intervention de type frimousse.
\begin{figure}[H]
	\centering
	\includegraphics[scale=0.35]{images/maquettes/fonctionnalite5MailDePriseEnCharge.png}
	\caption{Maquette~: Email type de prise en charge}
	\label{maquette9-1}
\end{figure}
 L'utilisateur pourra s'attribuer une intervention de type frimousse via la liste d'intervention ou via la fiche descriptive de cette intervention.


\subsection{Fonctionnalité 10}
Ce paragraphe décrit les maquettes concernant la fonctionnalité 10 soit la gestion des ventes. \\

La figure suivante montre la maquette \ref{maquette10-5} de la page permettant d'ajouter une vente .
\begin{figure}[H]
	\centering
	\includegraphics[scale=0.40]{images/maquettes/fonctionnnalite10AjouterVente.png}
	\caption{Maquette~: Ajouter une vente}
	\label{maquette10-5}
\end{figure}

La figure suivante \ref{maquette10-3} montre la maquette de la page de la fiche descriptive d'une vente.
\begin{figure}[H]
	\centering
	\includegraphics[scale=0.40]{images/maquettes/fonctionnnalite10ConsulterVente.png}
	\caption{Maquette~: Consulter la fiche descriptive d'une vente}
	\label{maquette10-3}
\end{figure}

La figure suivante \ref{maquette10-4} Listing des ventes montre la maquette de la page permettant d'effectuer le listing des ventes.
\begin{figure}[H]
	\centering
	\includegraphics[scale=0.40]{images/maquettes/fonctionnalite10Ventes.png}
	\caption{Maquette~: Listing des ventes}
	\label{maquette10-4}
\end{figure}

La figure suivante \ref{maquette10-4} Listing des ventes montre la maquette de la page permettant d'effectuer le listing des ventes.

Il est possible de supprimer une vente à partir de la fiche descriptive d'une vente 


