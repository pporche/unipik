\documentclass [a4paper] {article}
\usepackage[utf8]{inputenc}
\usepackage[francais]{babel}
\usepackage[top=2cm, bottom=4cm, left=2cm, right=2cm]{geometry} 
\usepackage{fancyhdr}
\usepackage{graphicx}
\usepackage{color, colortbl}
\usepackage{longtable}
\usepackage{vocabulaireUnipik}
\pagestyle{fancy}
\definecolor{Gray}{gray}{0.8}



%--- En-t�te et pied de page ---%

\renewcommand{\footrulewidth}{0,01cm}
\rhead{}
\chead{\huge{Compte-rendu de réunion de Tutorat Pédagogique}}					%titre
\begin{document}

05/02/2016			 				%Date 
\hfill   
\hfill 	 9:02 - 10:00 	14:01 - 16:30			%Heure de d�but, heure de fin.


\lfoot{Version : 1.00} 			% version
%--- Fin en-t�te et pied de page ---%
\section*{Historique des révisions}
\begin{center}
			\begin{tabular}{| c | c | c | c | p{4cm} |}
				\hline
				\rowcolor{Gray}
				Version & Date & Auteur(s) & Modification(s) & Partie(s) modifiée(s)		 \\
				\hline
				1.00 & 22/02/2016 & \Pierre, \Kafui & Création & Toutes \\
		\hline		
			\end{tabular}
		\end{center}

\section*{Signatures}

		\begin{center}
			\begin{tabular}{| c | c | c | c | p{4cm} |}
				\hline
				\rowcolor{Gray}
				Rôle & Fonction & Nom & Date & Visa		 \\
				\hline
				Vérificateur & \RGC & \Mathieu & 23/02/2016 & pgpic \\[30pt]
				\hline
				Validateur & \CP & \Sergi & 23/02/2016 & pgpic \\[30pt]	
				\hline
			\end{tabular}
		\end{center}

%--- Réunion --%

\section{Résumé semaine passée et questions}
Cette semaine, quatre personnes ont travaillé sur le \DSE{}, dont \Michel{} qui s'est plus tard réorienté sur le \PTV{}. Le \DSECourt{} est très proche d'être terminé et nous avons des questions concernant le \PTVCourt{}.
\paragraph{}
Celui ci doit comprendre la manière employée pour valider les livrables. Il faut expliquer comment nous allons démontrer que chacune des affirmations (ex. : "l'ensemble des composant est interfaçable") que nous faisons. Le \PTVCourt{} dit uniquement ce que l'on va tester et comment (ex. : "La validation se fera au travers d'un programme qui montrera la connexion". Les détails sur les tests sont dans le \CDR{}.
\\
Un exemple sur les tests de l'interface : "nous utiliserons tel logiciel" est une affirmation qui appartient au \PTVCourt{}. "nous cliqueront à tel et tel endroit" est une affirmation qui appartient au \CDR{}. \nomTuteurPedago{} nous demande de lui envoyer le \PTV{} en même temps qu'au client.
\\
Il faudra tester indépendamment le modèle, la vue et le contrôleur. Pour la vue, il existe des "logiciels de clic" et pour la Base de Données, il faudra l'interroger.
\paragraph{}
La période probatoire dure une semaine durant laquelle des tests ont lieu avec le client qui nous tient informé de leur déroulement. au bout d'une semaine, sans nouvelles de la part du client, la recette est acceptée.
\paragraph{}
Une réunion s'est tenue dans les locaux du client le 03/02/2016. Ceux-ci sont de petite taille et ne sont que peu équipés donc peu pratiques. La réunion a été trop longue pour plusieurs raisons, parmi lesquelles le fait que nous soyons trop nombreux. Il n'y avait pas d'heure de fin prévue et la conduite de réunion était trop "souple". Afin de remédier à ces problèmes, il faut réduire le nombre de participants, fixer une heure de fin, établir un Ordre du Jour, prévoir un timing et servir des cafés et des biscuits.
\\
Nous avons ensuite rapporté à \nomTuteurPedago{} les modifications sur le Cahier des Charges suite à la réunion client (Cf. CRC\_Q\_Unipik\_d16-02-03).
\paragraph{}
Concernant les formations, il existe le site coursenligne.insa-rouen.fr qui pourra nous aider à trouver du contenu adapté. Il faut également ajouter des formations à l'UML dans le plan de formation.
\paragraph{}
Au sujet du serveur mis à disposition, M. Bonnegent nous a fait comprendre que cela ne serait pas possible. \nomTuteurPedago{} nous explique que nous aurions du nous adresser plus haut dans la hiérarchie, mettre en avant l'aspect mécénat et arriver avec une solution "clef en main". L'idée serait d'aller parler de cela avec des personnes comme Anne Caldin et Maxime Reynet puis de signer une convention avec un professeur référent en passant par des personnes clef : Gilles Gasso, Jean Macquet, UNICEF France, etc. .


\section{Revue du modèle E/A}



\section{Revue du GIT}
\nomTuteurPedago{} a regardé le git et certains problèmes persistent : il faut mettre des commentaires dans tous les fichiers pour spécifier leur rédacteur, leur version et leur date de rédaction. Les conventions de nommage n'ont pas été définies ou respectées. Les makefile se répètent ou font référence à des dossiers inexistants.
\\
Le modèle E/A doit se trouver dans le dossier spécifications.




%--- fin de réunion ---%
\newpage



\end{document}





















