Ce document présente le \DSE{} (\DSECourt) du \PIC{} \nomPIC. Ce logiciel permettra la gestion des interventions externes de l'organisation \nomClient. \\
	
	Les spécifications externes entrent dans le cadre de l'organisation et de la gestion du développement logiciel. Elles ont pour objectif de décrire exactement ce que sera le système du point de vue de l'utilisateur.
	
	
\section*{Portée du document}
	Ce document est destiné :
	\begin{itemize}
		\item au client \nomClient{} dont la représentante est \representantClient,
		\item à l'équipe \PICCourt{} \nomEquipe.
	\end{itemize}
	
\section*{Document de référence}
	Le présent document fait référence au Cahier des Charges du projet fourni par le client et est rédigé en fonctions des clauses qualités définies dans le \PQ.
	
\section*{Contenu du document}
	Ce document aborde dans un premier point une description générale sur la réalisation du produit, comme le respect des exigences, ou encore, les facteurs pouvant influer sur ce dernier. Ensuite les besoins détaillés seront explicités, c'est à dire les fonctionnalités détaillées  explicitement par le client, les spécifications d'interfaces et spécifications opérationnelles. La description des fournitures sera abordée en dernier point.