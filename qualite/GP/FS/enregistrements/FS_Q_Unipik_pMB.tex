%version 1.00,	date 12/02/2016	auteur(s) Pierre Porche
\documentclass [a4paper] {article}
\usepackage[utf8]{inputenc}
\usepackage[francais]{babel}
\usepackage[top=2cm, bottom=4cm, left=2cm, right=2cm]{geometry} 
\usepackage{fancyhdr}
\usepackage{graphicx}
\usepackage{color, colortbl}
\usepackage{longtable}
\usepackage{vocabulaireUnipik}
\usepackage{tabularx}
\usepackage{xcolor}
\definecolor{Gray}{gray}{0.8}

\pagestyle{fancy}

%--- En-t�te et pied de page ---%

\renewcommand{\footrulewidth}{0,01cm}
\rhead{}
\chead{\huge{Fiche de Suivi}}

\begin{document}

\section*{\Melissa}

\centering
	\begin{longtable}{|>{\columncolor{gray!40}}p{2cm}|p{12cm}|}
	\hline
	Semaine 5 & \begin{itemize}
		\item Formation argoUML
		\item Évaluation argoUML
		\item Diagramme de package pour le \DCPCourt
		\item Formation Symfony
		\item Correction du \DSICourt (maquettes et cas d'utilisation)
		\item Préparation de la revue
		\end{itemize} \\
	\hline
	
	Semaine 6 & \begin{itemize}
		\item Squelette du \DCPCourt{} et modification du makefile
		\item Correction du \DSECourt
		\item Correction du \DSICourt
		\item Préparation de la revue
		\end{itemize} \\
	\hline
	
	Semaine 7 & \begin{itemize}
		\item Formation Symfony
		\end{itemize} \\
	\hline
	
	Semaine 8 & \begin{itemize}
		\item Formation Symfony
		\end{itemize} \\
	\hline
	
	Semaine 9 & \begin{itemize}
		\item diagrammes d’interaction et squelette du \DCPCourt
		\item Formation Symfony
		\end{itemize} \\
	\hline
	
	Semaine 10 & \begin{itemize}
		\item Estimation du temps pour réaliser la formation PHPUnit
		\item Vérification diagramme de bundle et diagramme de navigation
		\item Vérification et correction de la base de donnée
		\item Recherche pour mettre en place la base de donnée 
		\item Création de types pour les valeurs multivalués de la base données
		\end{itemize} \\
	\hline
	
	Semaine 11 & \begin{itemize}
		\item Recherche et tests pour la mise en place de la BD en bottom-up
		\item Implémentation de méthodes dans les contrôleurs
		\end{itemize} \\
	\hline
	Semaine 12 & \begin{itemize}
	\item Préparation de la revue.
	\end{itemize} \\
	\hline
	
	Semaine 15 & \begin{itemize}
	\item Modifications du \PGCCourt .
	\item Scripts de remplissage de la base de données
	\item Vérification \CRTPCourt.
	\end{itemize} \\
	\hline
	
	
	
\end{longtable}

\end{document}