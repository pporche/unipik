% version 1.00	Auteur Sergi Colomies

\documentclass[asi]{picInsa}
\DeclareGraphicsRule{*}{pdf}{*}{}
\usepackage{pdfpages}


\usepackage{vocabulaireUnipik}

\setcounter{secnumdepth}{4}
\setcounter{tocdepth}{4}
\newcommand{\ligneMaj}[3] {
	\rowcolor[gray]{0.55} \textbf{\textit{#1}} & #2  &  #3\\
	\hline
}
\newcommand{\ligneSup}[3] {
	\rowcolor[gray]{0.65} |\textunderscore \textbf{\textit{#1}} & #2  &  #3\\
	\hline
}
\newcommand{\ligneMed}[3] {
	\rowcolor[gray]{0.75} \hspace{0.25cm} |\textunderscore #1  & #2 & #3 \\
	\hline
}
\newcommand{\ligneSub}[3] {
	\rowcolor[gray]{0.85}  \hspace{0.5cm} |\textunderscore #1 & #2 & #3\\
	\hline
}
\newcommand{\ligneSubSub}[3] {
	\rowcolor[gray]{0.95}  \hspace{0.75cm} |\textunderscore #1 & #2 & #3\\
	\hline
}
\newcommand{\ligneTache}[3] {
	\hspace{1.00cm} |\textunderscore #1 & #2 & #3\\
	\hline
}
\title{Débriefing \CP{}}
\author{\Sergi}


\titreGeneral{Débriefing \CP{}}
\sousTitreGeneral{\nomEquipe}
\titreAcronyme{\Huge{debriefing}}
\version{v1.00}
\titreDetaille{\large{debriefing\_L\_Unipik\_d16-05-23}}
\referenceVersion{debriefing\_L\_Unipik\_d16-05-23}
\auteurs{\Sergi{}}
\destinataires{Unité P3}
\resume{Le présent document contient la présentation du débriefing \CP{} \nomEquipe.}
\motsCles{débriefing, \CP, gestion de projet}
\natureDerniereModification{Création}
\modeDiffusionControle{}

\begin{document}

\couverture{}

\informationsGenerales{}


\tableofcontents

\setcounter{chapter}{0}

\chapter{Introduction}
\label{Introduction}
Le présent document constitue le rapport de débriefing du \CP{} de l'équipe \nomEquipe{} pour le premier semestre du Projet INSA Certifié dans le département Architecture des Systèmes d'Information. Dans ce document sera décrit la fonction de \CP{} et les expériences acquises à ce poste.\vspace{0.5cm}\\
Le \CP{} a pour fonction principale la gestion d'équipe et la relation client. Il est également présent dans la démarche qualité et sert d'intermédiaire entre l'équipe et l'extérieur : INSA ou client.\vspace{0.5cm}\\
Dans un premier temps une description du PIC dans sa généralité sera effectuée avec une présentation du client et du sujet du projet. Dans un deuxième temps sera décrit l'avancement du projet et les points non traités ce semestre. Puis dans un troisième temps seront décrites les méthodes mises en place pour la gestion d'équipe et la gestion de projet. Finalement, ce document sera conclu par mon ressenti personnel sur ce semestre passé en tant que \CP.




\chapter{Présentation du PIC}
\label{presentation_PIC}
\section{Le client}
L'\nomClient{} Seine Maritime est un comité de l'\nomClient{},  agence de l'Organisation des Nations unies consacrée à l'amélioration et à la promotion de la condition des enfants.  L'\nomClient{} a activement participé à la rédaction, la conception et la promotion de la Convention relative aux droits de l'enfant (CIDE), adoptée lors du sommet de New York le 20 novembre 1989. L'\nomClient{} a reçu le prix Nobel de la paix le 12 janvier 1965. L'\nomClient{} s'est donné les objectifs prioritaires suivants :
\begin{itemize}
	\item l'éducation des filles
	\item la vaccination et la lutte contre le SIDA et le VIH
	\item la protection de l'enfance
	\item la santé des nouveau-nés
	\item l'égalité hommes-femmes\vspace{0.5cm}
\end{itemize}

Pour notre projet nous travaillons avec l'\nomClient{} Seine-Maritime. Les fonctions de ce comité sont : l'organisation de plaidoyers dans les écoles sur des thèmes centraux de l'\nomClient, l'organisation d'opérations frimousses consistant à la fabrication de poupées "frimousses" par des élèves afin d'être revendues par la suite, et des actions ponctuelles telles que des ventes lors d’événements particuliers.



\section{Ses besoins}
Afin de pouvoir gérer ses différentes actions (plaidoyer, frimousse, actions ponctuelles,...), l'\nomClient{} Seine-Maritime a besoin d'un outil de gestion des interventions externes de ses bénévoles. Cet outil devra être capable de gérer l'ensemble des bénévoles du comité et de les relier aux différentes interventions externes de manière intelligente.\\
Le parc informatique étant assez hétérogène, tant au niveau matériel que logique, notre service devra fonctionner sous différents environnements (Windows, Linux, MAC). De plus les connaissances informatiques du client étant limitées dans certains cas, nous devrons produire un service intuitif et facile à prendre en main.



\section{Les livrables}
Le projet à réaliser a été découpé en quatre livrables sur la période de deux semestres.\vspace{0.5cm}\\
Le premier livrable constate de l'architecture matérielle et logicielle de l'outil final. Nous avons décidé de réaliser un service web développé en PHP avec une base de données en PostgreSQL. Le framework Symfony permettra de faire le lien entre les deux. Nous utiliserons une architecture trois tiers avec le patron de conception Modèle-Vue-Contrôleur (MVC).\vspace{0.5cm}\\
Le deuxième livrable couvre toute la partie de gestion des plaidoyers. Le livrable doit fonctionner à la fin du semestre pour pouvoir commencer à être utilisé par l'UNICEF dès la rentrée scolaire.\vspace{0.5cm}\\
Le troisième livrable couvrira la partie gestion des frimousses.\vspace{0.5cm}\\
Enfin le dernier livrable couvrira la partie gestion des actions ponctuelles et devra être rendu pour la fin du second semestre de PIC.



\section{L'équipe}
L'équipe PIC se décompose suivant l'organigramme fonctionnel ci-dessous (figure \ref{organigramme}) :
\begin{figure}[!h]
	\begin{center}
	\includegraphics[scale=0.35]{images/organigrammeFonctionnel.pdf}
	\label{organigramme}
	\caption{Organigramme fonctionnel}
	\end{center}
\end{figure} 





\chapter{L'avancement du PIC}
\label{avancement}
\section{Prévisions pour le premier semestre}
En collaboration avec notre client, l'\nomClient{} Seine-Maritime, nous avons divisé le projet en quatre grands livrables : deux livrables par semestre. Notre client nous a bien fait comprendre qu'il souhaitait pouvoir commencer à utiliser le service pour les plaidoyers dès la fin du premier semestre. Il fallait donc que toute la partie concernant les plaidoyers soit traitée au premier semestre. Le client a aussi laissé sous-entendre que les livrables attendus au second semestre n'étaient pas les plus importants à leurs yeux et qu'ils seraient amenés à changer.\vspace{0.5cm}\\
Dans les parties suivantes vont être décrits les lots du premier semestre tels qu'ils ont été déterminés initialement.

\subsection{Lot 1}
Le premier lot est décrit de la manière suivante dans le cahier des charges : <<Le lot 1 couvre l'architecture matérielle et logicielle à mettre en place pour le fonctionnement de l'application dans sa globalité>>. Pour ce livrable il est donc attendu une réflexion de l'équipe sur l'architecture du système à mettre en place, non pas au niveau des langages de développement mais au niveau structurel et physique. Cette architecture a également été mise en place sur un exemple basique que l'on a développé. Ce livrable doit être rendu à $T_{0}+5 semaines$. 

\subsection{Lot 2}
Le deuxième lot va couvrir les fondamentaux et la gestion des plaidoyers. Il implémente sept fonctionnalités et doit être achevé pour la fin du premier semestre. Les fonctionnalités demandées sont les suivantes :
\begin{itemize}
	\item gestion des bénévoles avec la création, la suppression et la modification de ceux-ci dans la base de données
	\item gestion des établissements avec la création, la suppression et la modification de ceux-ci dans la base de données
	\item envoi de mails automatique aux établissements pour la démarche de création d'intervention
	\item création d'un formulaire intelligent permettant aux établissements de s'inscrire dans la liste des interventions facilement et rapidement
	\item affectation d'interventions aux plaideurs en fonction de leur géolocalisation et par envoi de mails automatiques
	\item modification des interventions par un bénévole
	\item envoi de mails de rappels une semaine avant la date prévue de l'intervention
\end{itemize}



\section{Réalisations du premier semestre}
\subsection{Lot 1 : l'architecture}
Le premier lot correspond à l'architecture logicielle et matérielle du service demandé. Nous avons décidé de réaliser une architecture trois tiers composée d'un client, un serveur et une base de données. Le client représente les ordinateurs des bénévoles qui auront accès à l'interface du service et pourront l'utiliser sans se soucier de la logique métier et de la base données. Le serveur hébergera notre service et répondra aux requêtes des clients. Enfin la base de données sera séparée du serveur d'application afin de bien délimiter les différents cadres de développement. Nous avons également décidé de mettre en place le patron de conception Modèle-Vue-Contrôleur (MVC).\vspace{0.5cm}\\
En ce qui concerne la base de données, le client nous a imposé de la développer en PostgreSQL. Elle doit également pouvoir utiliser des objets spatiaux via la technologie PostGIS. Nous avons aussi pris en considération la faible volumétrie de la base : au niveau de la Seine-Maritime ne sont dénombrés qu'une dizaine de plaideurs pour 1500 établissements et un taux de connexion qui ne dépassera pas les dix connexions par seconde.\vspace{0.5cm}\\
En ce qui concerne le service web, nous avons décidé de le développer en PHP. Pour faire ce choix nous avons réalisé un état de l'art puis une comparaison des différents langages possibles. Suite à cette comparaison (figure \ref{comparaison_PHP}), il est apparu que le PHP est le langage de développement qui correspond le mieux à notre projet.
\begin{figure}[!h]
\begin{center}
    \begin{tabular}[h]{|p{0.45\textwidth}|p{0.45\textwidth}|}
    
	\hline
	\cellcolor{blue!15}PHP & \cellcolor{blue!15}Java J2EE \\\hline
        Serveur de type web & Serveur d'application ou conteneur de servlet \\\hline
        Moins lourd, consomme moins de ressources sur des applications peu complexes & Efficace sur des applications complexes \\\hline
        Evolutions fréquentes sans rétrocompatibilité & Moins d'évolutions et rétrocompatibilité avec Java \\\hline
        Langage adapté à l'univers web & Formation et utilisation du code une fois pour toute \\\hline
        Plus difficile de gérer la sécurité & Contrôle et validation des données inclue au coeur du langage \\\hline
        Intégration et apprentissage rapide des développeurs & Nécessité d'un bon niveau d'abstraction \\\hline
    \end{tabular}
    \caption{Comparaison des langages de développement web}
    \label{comparaison_PHP}
\end{center}
\end{figure}

Enfin afin de faire le lien entre notre base de données et notre application web, nous avons pris la décision d'utiliser le framework Symfony. En effet celui-ci possède de nombreux avantages dont celui de générer du code PHP à partir d'une base de données. Ses autres principaux avantages sont : être un logiciel open source, inclure un développement MVC avec une bonne séparation des trois couches et avoir des tests unitaires et fonctionnels intégrés.

\subsection{Lot 2 : les plaidoyers}
Le second livrable couvre toute la partie gestion des plaidoyers. Pour rappel, les plaidoyers sont des interventions menées par des bénévoles de l'\nomClient{} 76 dans des établissements scolaires sur des thèmes précis. Les grandes fonctionnalités à développer sont : la création de bénévoles dans la base de données, la création d'établissements dans la base de données, la création d'interventions dans la base de données et l'association de ces trois grandes entités pour former une intervention par un bénévole dans un établissement. Sont aussi demandées l'envoi de mails automatiques et la géolocalisation des interventions.\vspace{0.5cm}\\
Durant ce semestre nous avons réalisé une base de données propre afin de pouvoir développer par la suite notre service à partir de bonnes bases et sans devoir revenir sur le modèle. Nous avons également développé des vues ainsi que des contrôleurs. Il nous est possible d'envoyer des mails automatiquement et nous pouvons gérer la connexion et la déconnexion d'un utilisateur.
Cependant, nous n'avons pas pu développer toute la logique métier dans les temps.

\subsection{Éléments non traités}
Ayant du retard sur le deuxième livrable, nous n'avons pas pu réaliser une partie de la logique métier concernant les plaidoyers. Nous n'avons pas non plus géré les objets spatiaux via la technologie PostGIS.


\section{Prévisions du second semestre}
\subsection{Lot 2 : les plaidoyers}
Le deuxième livrable est à achever avant la fin septembre, délai que nous avons négocié avec le client. Il s'agit là de terminer la logique métier et de la tester. Il faudra être capable de gérer la gestion des établissements et des interventions.


\subsection{Lot 3 : les frimousses}
Le troisième lot qui s'étend sur la première moitié du second semestre devra couvrir la totalité de la gestion des frimousses. Pour rappel, les opérations frimousses sont des opérations menées en établissements scolaires et consistant en la réalisation de poupées par les élèves afin que celles-ci puissent être vendues par la suite.\\
Les deux grandes fonctionnalités attendues pour ce livrable sont les suivantes :
\begin{itemize}
	\item envoie de mails automatiques
	\item association de bénévoles aux interventions de type frimousse\vspace{0.5cm}
\end{itemize}

Ce livrable est attendu pour la date $T_{1}+6$ semaines avec $T_{1}$ la date de début du second semestre.

\subsection{Lot 4 : les actions ponctuelles}
La dernière fonctionnalité qui s'étend sur la deuxième moitié du second semestre couvrira quant à elle la gestion des actions ponctuelles. Pour rappel les actions ponctuelles sont des ventes sur stands organisées par l'\nomClient{} ou bien des projets étudiants encadrés par l'\nomClient{}.\\
Les grandes fonctionnalités attendues pour ce lotissement sont les suivantes : 
\begin{itemize}
	\item géolocalisation des activités ponctuelles
	\item association entre un bénévole et une action ponctuelle dans la base de données
	\item gestion des statistiques de l'\nomClient{}\vspace{0.5cm}
\end{itemize}
Ce dernier livrable devra être rendu à la fin du temps imparti pour le PIC.





\chapter{La gestion du projet et de l'équipe}
\label{gestion_equipe}
Dans cette partie vont être décrites les différentes méthodes mises en place pour la gestion de projet et la gestion de l'équipe. Dans un premier temps sera abordée la méthodologie de travail mise en place. Dans un deuxième temps sera abordée la gestion de l'équipe dans ses détails. Dans un troisième temps sera abordée la gestion des tâches et du temps tout au long de ce premier semestre de PIC. Finalement, dans un dernier temps sera abordé le point de la communication dans le PIC.

\section{Méthodologie de travail}
\subsection{Modèle en spirale}
\begin{figure}[!h]
	\begin{center}
	\includegraphics[scale=0.4]{images/modeleSpirale.pdf}
	\label{diagrammeSpirale}
	\caption{Méthodologie en spirale}
	\end{center}
\end{figure}

Initialement nous souhaitions mettre en place une méthodologie de type Agile. Cependant nous nous sommes vite rendus compte qu'une telle méthodologie n'était pas adaptée à notre projet et au respect de la Qualité décrite dans la DGQ2. Nous avons donc mis en place une méthodologie qui correspond mieux à notre projet : la méthodologie en spirale. Cette méthodologie peut être assimilée à une succession de petits cycles en V. Elle se décompose en 6 étapes différentes : spécifications du problème, analyse des solutions possibles et existantes, conception de la solution choisie, implémentation de cette solution, tests sur le code développé et pour finir validation de notre solution par le client. Cette succession d'étapes est réalisée quatre fois, une fois par lot.

\subsection{Front-end / Back-end}
Au fur et à mesure que le projet a avancé, il est apparu la nécessité de diviser l'équipe de développeurs en deux. Une équipe s'occuperait du front-end, c'est-à-dire des vues et de l'interface de notre service, tandis que l'autre équipe s'occuperait du back-end, c'est-à-dire des contrôleurs et de la logique métier permettant de faire remonter les résultats vers les vues. Dans chaque équipe a été nommé un responsable afin de structurer le travail et d'avancer de manière efficace.



\section{Gestion de l'équipe}
\subsection{Rôles}
Dès le début du PIC nous avons décidé d'attribuer des rôles au maximum de personnes afin que chaque membre du PIC ait des responsabilités et afin de bien séparer les différentes tâches. Cela nous a permis de gagner du temps et d’établir une certaine hiérarchie à respecter. Certains nouveaux rôles sont apparus au fur et à mesure que le PIC avançait. Les principaux rôles sont les suivants : 
\begin{itemize}
	\item \CP{}
	\item \CPA{}
	\item \RQ{}
	\item \RQA{}
	\item \RGC{}
	\item \RD{}
	\item \RRS{}
	\item \RS{}
\end{itemize}

\subsection{Temps de travail}
À cause des différentes spécialités et des différents cours de chaque membre du PIC, nous évitons au maximum d'imposer des créneaux horaires obligatoires. Cependant nous avons mis en place avec le \RQ{} des indicateurs de temps de travail en salle PIC. Ainsi, pour l'ensemble de l'équipe, excepté le \CP{} et le \RQ{}, le seuil minimum de temps de présence a été fixé à 17h et la valeur cible à 22h. Pour le \CP{} et le \RQ{}, le seuil minimum est de 22h et la valeur cible est de 27h.

\subsection{Team Building}
L'environnement de travail est quelque chose de réellement important si l'on souhaite faire avancer le projet rapidement. Une bonne cohésion d'équipe se traduit par un travail rapide et efficace. C'est pourquoi régulièrement nous effectuons du team building, tels que des sorties au restaurant, afin de souder le groupe et d'éviter les conflits.



\section{Gestion des tâches et du temps}
\subsection{Diagramme de Gantt}
La base de toute gestion de projet est la décomposition de celui-ci en un ensemble de sous-tâches minimales auxquelles sont associées des durées de réalisation et des ressources humaines et matérielles. Toutes ces sous-tâches peuvent ensuite être mises bout à bout afin d'estimer la durée du projet et de réaliser un planning d'avancement.\\
Dans le cadre du PIC, nous avons décidé de mettre en place un diagramme de Gantt grâce au logiciel open source GanttProject. Ce diagramme, mis à jour de manière hebdomadaire, nous permet de suivre l'avancement du PIC en temps réel et de planifier l'ensemble des tâches à venir dans le temps imparti.

\subsection{Diagramme de Pert}
Le logiciel GanttProject nous permet également de réaliser à partir du diagramme de Gantt un diagramme de Pert. Ce diagramme nous permet de visualiser le chemin critique de manière plus aisée et donc la fin estimée du projet. Tout comme le diagramme de Gantt, ce diagramme est mis à jour régulièrement afin de suivre l'avancement en temps réel.

\subsection{Kanban}
En interne, même si nous n'avons pas mis en place une méthodologie de travail de type Agile, nous avons conservé quelques uns de ses éléments tels que le kanban. Le kanban est un tableau de quatre colonnes permettant de regrouper les tâches à accomplir. Les quatre colonnes sont : à faire, en cours, à vérifier et fait. Ainsi lorsqu'une tâche est rajoutée au programme, elle va se situer dans la colonne à faire. Si un membre de l'équipe décide de réaliser cette tâche, il la fait passer dans la colonne en cours. Une fois qu'il finit la tâche, celle ci passe dans la colonne à vérifier. Un autre membre de l'équipe décidera alors de vérifier le travail réalisé pour finalement placer la tâche dans la colonne fait.\\
Le kanban est physique et non logiciel. Il s'agit là d'un choix personnel : un tableau physique est bien plus interactif pour l'ensemble de l'équipe et les membres peuvent ainsi se réunir autour du tableau afin de discuter et se répartir les tâches.\\
Un système de couleurs à également été mis en place afin de différencier les tâches liées à la qualité de celles liées au développement.



\section{Communication}
La communication est un point essentiel pour la réussite d'un projet. Que ce soit entre membre de la maîtrise d’œuvre ou bien avec la maîtrise d'ouvrage. Dans cette partie seront décrites les procédures de communication que nous avons mises en place.

\subsection{Communication interne}
Sur le plan de la communication interne, entre membres de l'équipe, nous effectuons des réunions hebdomadaires d'une heure pour revenir sur les points de la semaine passée, définir les points à réaliser la semaine prochaine et faire remonter les éventuelles remarques. Cette réunion permet à chaque membre de l'équipe de s'exprimer et de donner son avis. En plus de cette réunion, des réunions informelles sont effectuées lorsque le besoin s'en fait ressentir.\vspace{0.5cm}\\
Afin de communiquer hors des réunions, plusieurs dispositifs ont été mis en places. Tout d'abord, la salle PIC étant organisée en open-space, il est assez aisé de communiquer avec un tiers directement. Nous utilisons également des conversations par mails pour les communications officielles qui nécessitent un suivi. Pour les autres communications nous utilisons le logiciel libre Discord. L'avantage de ce logiciel est qu'il comporte une messagerie instantanée avec possibilité de conversations audio.\vspace{0.5cm}\\
Nous avons également mis en place une boîte à idée. Cette boîte à idée permet de faire remonter de nouvelles idées dans l'optique d'améliorer l'environnement du PIC. Elle permet également d'effectuer des remarques anonymes. Les idées déposées dans la boîte sont lues une fois par semaine.

\subsection{Communication tuteurs}
Dès les toutes premières semaines du PIC, il a été défini des réunions hebdomadaires avec nos trois tuteurs : communication, qualité et pédagogique. Ces réunions permettent de faire le point chaque semaine et de s'assurer qu'on aille dans la bonne direction pour la réalisation du projet. Ces réunions sont également un apport important pour la réalisation du PIC puisque nos tuteurs partagent avec nous leurs expériences.\\
En plus des réunions, une communication régulière par mails ou par téléphone permet de faire parvenir aux tuteurs les remarques et questions plus importantes rapidement.

\subsection{Communication client}
La communication avec le client est un point clé de l'aboutissement d'un projet. En effet sans une bonne communication, la compréhension du projet peut ne pas être la même entre les deux parties et nous pourrions développer un service qui ne corresponde pas aux attentes. Afin de contrecarrer cette problématique, nous avons effectué des réunions régulières lors de nos phases de spécifications et de conception. Ainsi avant de se lancer dans le développement, nous nous assurons de bien avoir compris et cerné les attentes du client. Bien évidemment nous conversons avec le client par mail et par téléphone pour les autres remarques qui ne nécessitent pas d'effectuer des réunions.





\chapter{Conclusion}
Le rôle de \CP{} est un des rôles les plus formateurs du PIC. L'apport en expérience est réellement important sur différents fronts. J'ai ainsi pu acquérir des compétences en gestion de projet, en communication, en travail de groupe et en relation client. J'ai été affronté à certains problèmes mais le PIC m'a appris à les résoudre.\vspace{0.5cm}\\
La gestion de projet n'a pas été chose aisée pour cause du manque d’expérience à l'initiation du PIC. Cependant, au fur et à mesure que le projet avançait et grâce à la collaboration de mon \CPA{} \Pierre, nous sommes montés en puissance en gestion de projet.\vspace{0.5cm}\\
Une composante majeure du rôle de \CP{} est la relation client. Une nouvelle fois, l'expérience initiale étant nulle, le PIC a été l'occasion d'acquérir des compétences dans ce domaine. La grande difficulté qui est apparue est le manque de connaissances techniques du client. Il a fallu apprendre à communiquer sans utiliser de termes techniques afin de pouvoir être compris par le client. De plus les clients étant des bénévoles, leurs différentes activités ont limité en temps leurs présences et leur rapidité de réponse.\vspace{0.5cm}\\
La passation avec mon \CPA{}, \Pierre, se fera de manière assez naturelle étant donnée que nous avons œuvré en collaboration étroite tout au long du PIC. Nous partagions également le même bureau : il pouvait ainsi garder un œil sur le travail que j'effectuais afin de se préparer pour son futur rôle.\vspace{0.5cm}\\
Je me suis porté volontaire au début du PIC pour assumer la fonction de \CP{} car je souhaitais obtenir de l'expérience dans ce domaine et je ressors pleinement satisfait de ce semestre. Mon seul regret aura été de m'être trop détaché du développement et par conséquent d'avoir la sensation de ne pas pouvoir aider mon équipe lors des phases de développement.


\pageQuatriemeCouverture

\end{document}
