% version 1.01	Auteur Pierre Porche

\begin{pagesService}
	\begin{historique}
		% nouvelles versions à rajouter AU-DESSUS en recopiant les lignes suivantes et en les modifiant :
		\unHistorique{3.01}{03/10/2016}{\Melissa}{Modifications relatives à l'utilisation de courriel comme visa pour la vérification de document}{Chapitre 7}
		\unHistorique{3.00}{28/09/2016}{\Kafui \newline \Pierre}{Modifications relatives au changement d'équipe, 
									\newline Modifications relatives à l'utilision de Redmine,
									\newline Modifications relatives à l'ajout d'un nouveau membre dans l'équipe,
									\newline Modifications relatives à l'utilisation de GanttProject}{Chapitres 1, 2, 3, 4, 6, 9}
		\unHistorique{2.00}{29/03/2016}{\Pierre}{Modifications (FFT n°004, FOC n°004)}{Chapitres 4, 5, 6, annexe B}
		\unHistorique{1.00}{22/02/2016}{\Sergi \newline \Pierre \newline \Michel \newline \Kafui \newline \Matthieu \newline \Mathieu \newline \Julie \newline \Melissa \newline \Florian }{Création}{Toutes}

	\end{historique}

        \begin{suiviDiffusions}

            % On place ici les diffusions
		\unSuivi{3.00}{28/09/2016}{\nomTuteurQualite{},  \nomApprobateur{}, \nomEquipe{}}
        	\unSuivi{2.00}{29/04/2016}{\nomTuteurQualite{},  \nomApprobateur{}, \nomEquipe{}}
        	\unSuivi{1.00}{14/03/2016}{\nomTuteurQualite{},  \nomApprobateur{}, \nomEquipe{}}
          
          
        \end{suiviDiffusions}

%%Signataires
        \begin{signatures}
	   \uneSignature{Vérificateur}{\RGC{}}{\Melissa{}}{21/09/2016}{PGPic}
       \uneSignature{Validateur}{\CP{}}{\Pierre}{21/09/2016}{PGPic}
	   \uneSignature{Approbateur}{Direction Qualité Unité P3}{\nomApprobateur}{28/09/2016}{courriel}
        \end{signatures}
	
	
	\begin{documentsReference}
		\begin{listeDeReferences}
			\uneReference{NF EN ISO 9001}{Octobre 2015}
			\uneReference{\MQ{}}{ASI-MQ-MQASI}
			\uneReference{\DGQ{} du Processus "\DGQUN{}"}{ASI-DGQ-DGQ1}
			\uneReference{\DGQ{} du Processus "\DGQDEUX{}"}{ASI-DGQ-DGQ2}
			\uneReference{\DGQ{} du Processus "\DGQTROIS{}"}{ASI-DGQ-DGQ3}
			\uneReference{\PGC }{\PGCCourt\_ Q\_\nomEquipe }
			\uneReference{NF EN ISO 9000}{Octobre 2015}
		\end{listeDeReferences}
	\end{documentsReference}
	
	\begin{terminologie}
		La terminologie (définitions et abréviations) utilisée dans le présent document est centralisée dans le \MQ{} \ASICourt{} (\emph{cf. ASI-MQ-MQASI}) de l'Unité P3.

		\begin{listeDAbreviations}
			\uneAbreviation{\asiCourt}{\asi}
			\uneAbreviation{CR}{Compte-Rendu}
			\uneAbreviation{\CRICourt}{\CRI}
			\uneAbreviation{CDC}{Cahier Des Charges}
			\uneAbreviation{\CRCCourt}{\CRC}
			\uneAbreviation{\CRTPCourt}{\CRTP}
			\uneAbreviation{\CRTQCourt}{\CRTQ}
			\uneAbreviation{\CTFTCourt}{\CTFT}
			\uneAbreviation{\DGQCourt}{\DGQ}
			\uneAbreviation{\DSECourt}{\DSE}
			\uneAbreviation{\DSICourt}{\DSI}
			\uneAbreviation{\DTUCourt}{\DTU}
			\uneAbreviation{\DTVCourt}{\DTV}
			\uneAbreviation{\DTICourt}{\DTI}
			\uneAbreviation{\FFTCourt}{\FFT}
			\uneAbreviation{\FTCourt}{\FT}
			\uneAbreviation{\FRCourt}{\FR}
			\uneAbreviation{\INSACourt}{\INSA}
			\uneAbreviation{ISO}{International Standard Organisation}
			\uneAbreviation{\MGPICourt}{\MGPI}
			\uneAbreviation{\MQCourt}{\MQ}
			\uneAbreviation{\OCCourt}{\OC}
			\uneAbreviation{\PGCCourt}{\PGC}
			\uneAbreviation{\PICCourt}{\PIC}
			\uneAbreviation{\PQCourt}{\PQ}
			\uneAbreviation{\PRCourt}{\PR}
			\uneAbreviation{\POCourt}{\PO}
			\uneAbreviation{\PTICourt}{\PTI}
			\uneAbreviation{\PTUCourt}{\PTU}
			\uneAbreviation{\PTVCourt}{\PTV}
			\uneAbreviation{\PVCourt}{\PV}
			\uneAbreviation{\PVVVCourt}{\PVVV}
			\uneAbreviation{\RFDCourt}{\RFD}
			\uneAbreviation{RTP}{Réunion Tuteur Pédagogique}
			\uneAbreviation{RTQ}{Réunion Tuteur Qualité}
			\uneAbreviation{\SMQCourt}{\SMQ}
			\uneAbreviation{Unité P3}{Unité Pédagogique Par Projet}
			\uneAbreviation{\WBSCourt}{\WBS}
			\uneAbreviation{\FIICourt}{\FII}
		\end{listeDAbreviations}
	\end{terminologie}
       
	\begin{listeDeDefinitions} 
		\uneTerminologie{\DGQCourt 1}{Processus Rechercher, choisir et contractualiser les PIC.}
		\uneTerminologie{\DGQCourt 2}{Processus Réaliser les PIC.}
		\uneTerminologie{\DGQCourt 3}{Processus Manager la Qualité.}
		\uneTerminologie{Incrément}{Logiciel fonctionnel produit d'un cycle  et implémentant une partie du carnet de produit.}
		\uneTerminologie{\Lintranet}{Plate-forme conçue par geotopic pour gérer entre autres la gestion des heures de travail, l'état des configurations et le suivi des réunions.}
		\uneTerminologie{Jours de travail \PICCourt{}}{L'équipe \PICCourt{} ne travaille pas à temps plein. Ce qui implique que certains jours ne comptent pas dans le temps de travail \PICCourt. Les jours de travail \PICCourt{} sont donc les jours où l'équipe travaille sur le \PICCourt{}.}
		\uneTerminologie{Semaine \PICCourt{}}{Terme désignant la semaine comme si elle ne comportait que des jours \PICCourt{}.}
	\end{listeDeDefinitions}
\end{pagesService}
