% Version 1.00	Date 01/03/2016	Auteur Pierre Porche

\subsection*{Introduction}


Le \CP{} doit garantir le bon déroulement du \PICCourt. Il possède des missions d’organisation et de validation du travail effectué par les membres de l’équipe. Il est également l’interlocuteur privilégié du tuteur pédagogique, du tuteur qualité et du client.

\subsection*{Tâches liées à sa fonction}

Une passation devra être mise en place entre les \CPs{} du premier et second semestre. Cette passation devra si possible être formalisée sous la forme d’une formation.

\subsection*{Tâches effectuées au démarrage du \PICCourt}

Le \CP{} devra se conformer aux exigences de la période de démarrage du \PICCourt :
\begin{itemize}
	\item Organiser l’équipe \PICCourt au premier semestre et de la fin du premier au second semestre en tenant compte des éventuels départs à l’étranger, redoublements, réorientations ou retours de mobilité académique. Ainsi qu’organiser la formation en début du second semestre du \PICCourt

	\item Créer le dépôt \git{} sur https://monprojet.insa-rouen.fr et des espaces publics et privés des membres de l’équipe \PICCourt ;
	\item Rédiger ou faire rédiger, puis valider les \FC{} de son équipe et éventuellement attester certaines compétences d’ordre personnel;
	\item Établir l’organigramme des fonctions de son \PICCourt et les descriptifs de ces fonctions;
	\item Associer les ressources aux fonctions de son \PICCourt.
\end{itemize}

\subsection*{Tâches effectuées au cours du \PICCourt}

Le \CP{} devra remplir au cours du \PICCourt les missions suivantes :

\begin{itemize}
	\item Mettre à jour les \FC{} des membres du \PICCourt;
	\item Archiver de manière hebdomadaire l’ensemble des espaces privés des membres et les conserver sur clé USB jusqu’à la semaine suivante;
	\item Garantir les journaux de tests;
	\item Participer à la \CTFT (CTFT);
	\item Gérer le budget de fonctionnement du \PICCourt;
	\item Représenter les ressources globales du \PICCourt et le solde des ressources consommées par un \WBSCourt minimal, un OBS, un \RBSCourt{} ou un \FBSCourt;

	\item Collecter les risques de tâches mis en lumière par les membres de l’équipe;
	\item Participer à la réunion de la conduite de projet ou suivi prévisionnel tenue au minimum une fois par semestre;
	\item Effectuer le débriefing à la revue de \PICCourt;
	\item Accorder les dérogations de possibilité de diffusion des non-conformités en accord avec le client.
\end{itemize}

Le \CP{} devra également remplir ou déléguer ces missions au \CPA :
\begin{itemize}
	\item Animer la réunion d’avancement hebdomadaire;
	\item Établir les diagrammes de Gantt pour le suivi des tâches passées, en cours et futures;
	\item Reprendre le planning de la semaine passée et le mettre à jour en fonction des retards estimés, des réunions exceptionnelles et des corrections possibles;
	\item Vérifier les fiches de suivi hebdomadaire des membres de l’équipe.
\end{itemize}

\subsection*{Tâches effectuées en fin de période du \PICCourt}

Le \CP{} devra en fin de période remplir les missions suivantes :
\begin{itemize}
	\item Livrer au secrétariat de la Direction du Département \ASICourt{} l’archivage de l’espace public des membres de l’équipe en fin de semestre;
	\item Garantir avec la Direction Qualité dans un \PVCourt{} la fin de la phase d’intégration.
\end{itemize}