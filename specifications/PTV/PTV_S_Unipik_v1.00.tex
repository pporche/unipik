\documentclass[asi]{picINSA}
\DeclareGraphicsRule{*}{pdf}{*}{}
\usepackage{pdfpages}

%\usepackage{colortbl}
\usepackage{fancyhdr}
\usepackage{listings} 
\usepackage{mathrsfs}
\usepackage{url}
\usepackage{lmodern}
\usepackage{color}
\usepackage{xcolor}
\usepackage{wrapfig}
\usepackage{graphicx}
\usepackage{pdflscape}
\usepackage{longtable}
\usepackage{sectsty}
\usepackage{lastpage}
\usepackage{multirow}
\usepackage{float}
\usepackage{eso-pic}
\usepackage[french]{minitoc}
\usepackage[babel=true]{csquotes}
\usepackage{tikz}
\addto\captionsfrench{\def\tablename{Tableau}}
\usepackage{../../ressources/Unipik/vocabulaire/vocabulaireUnipik}
%\usepackage{../../ressources/Unipik/vocabulaire/vocabulaireEastpic}

\setcounter{secnumdepth}{4}
\setcounter{tocdepth}{4}
\newcommand{\ligneMaj}[3] {
	\rowcolor[gray]{0.55} \textbf{\textit{#1}} & #2  &  #3\\
	\hline
}
\newcommand{\ligneSup}[3] {
	\rowcolor[gray]{0.65} |\textunderscore \textbf{\textit{#1}} & #2  &  #3\\
	\hline
}
\newcommand{\ligneMed}[3] {
	\rowcolor[gray]{0.75} \hspace{0.25cm} |\textunderscore #1  & #2 & #3 \\
	\hline
}
\newcommand{\ligneSub}[3] {
	\rowcolor[gray]{0.85}  \hspace{0.5cm} |\textunderscore #1 & #2 & #3\\
	\hline
}
\newcommand{\ligneSubSub}[3] {
	\rowcolor[gray]{0.95}  \hspace{0.75cm} |\textunderscore #1 & #2 & #3\\
	\hline
}
\newcommand{\ligneTache}[3] {
	\hspace{1.00cm} |\textunderscore #1 & #2 & #3\\
	\hline
}
\title{\PTV{}}
\author{\Pierre} %à changer


\titreGeneral{\PTV}
\sousTitreGeneral{\nomEquipe}
\titreAcronyme{\PTVCourt}
\version{V1.00}
\titreDetaille{\PTVCourt\_S\_\nomEquipe\_\versionPrive}
\referenceVersion{\PTVCourt\_S\_\nomEquipe\_\versionPrive}
\auteurs{\Matthieu{} \& \Kafui{} \& \Melissa{} \& \Sergi{} \& \Michel{} \& \Pierre{}}
\destinataires{}%\nomApprobateur{}, \nomTuteurQualite, \nomEquipe, \nomPIC{}}
\resume{Le présent document contient la présentation du \PTV{} \nomEquipe.}
\motsCles{\PTVCourt{}}
\natureDerniereModification{ }
\modeDiffusionControle{}

\begin{document}

\couverture{}

 \informationsGenerales{}
\begin{pagesService}
	\begin{historique}
		% nouvelles versions à rajouter AU-DESSUS en recopiant les lignes suivantes et en les modifiant :
		\unHistorique{1.00}{02/02/2016}{\Sergi \newline \Pierre \newline \Michel}{Création}{Toutes}

	\end{historique}

        \begin{suiviDiffusions}

            % On place ici les diffusions
        	\unSuivi{1.00}{}{\nomEquipe{}, \nomPIC{}}
          
          
        \end{suiviDiffusions}

%%Signataires
        \begin{signatures}
	   \uneSignature{Vérificateur}{\RQ{},\newline \CPA}{\Pierre{}}{26/01/2016}{courriel}
       \uneSignature{Validateur}{\CP{}}{\Sergi}{26/01/2016}{courriel}
	   \uneSignature{Approbateur}{Le client}{\nomClient}{}{}
        \end{signatures}
	
	

	
	
\end{pagesService}


\tableofcontents

\setcounter{chapter}{0}


\chapter{Introduction}
\label{introduction}
Ce document présente le \DSI{} (\DSICourt) du \PIC{} \nomPIC. Ce logiciel permettra la gestion des interventions externes de l'organisation \nomClient. \\

	Les spécifications internes entrent dans le cadre de l'organisation et de la gestion du développement logiciel. Elles ont pour objectif de décrire quelles sont les solutions retenus pour répondre aux demandes du client et au exigences du \DSECourt{}. 
	
\section*{Portée du document}
	Ce document est destiné :
	\begin{itemize}
		\item à l'équipe \PICCourt{} \nomEquipe.
	\end{itemize}
	
\section*{Document de référence}
	Le présent document fait référence au \DSE{} et est rédigé en fonctions des clauses qualités définies dans le \PQ.
	
\section*{Contenu du document}
	Ce document aborde dans un premier point, l'architecture globale du projet ainsi que les dépendances. Ensuite la présentation de la solution qui précisera les points qui ne figurent pas déjà dans le \DSECourt{}. La description détaillée des lots sera abordée en dernier point.

\chapter{Lot 1}
\label{lot1}
	Ce chapitre a pour but de valider les points évoquer dans le \DSE.
	
\section{Description du lot}
	Ce paragraphe rappelle les exigences demandées pour le lot 1 :
	\begin{itemize}
		\item La mise en place d'une architecture matérielle et logicielle pour le fonctionnement de l'application dans sa globalité,
		\item La base de données doit donc être opérationnelle et peut petre manipulée via des scripts,
		\item Cette structure sera locale.
		\item La livraison du lot se fera à T0 + 5 semaines (T0 étant la date du lancement du projet).
	\end{itemize}
	
\section{Conception du lot}
	Ce paragraphe explique les méthodes suivies pour la conception de ce lot :
	\begin{itemize}
		\item La base de données sera subdivisée en plusieurs sous fichiers,
		\item Le code devra respecter les bonnes conventions de codage (indentation, nom de variable explicite,... ),
		\item Le code devra être documenté.
	\end{itemize}
	
\section{Déroulement de la recette}
	Ce paragraphe explique comment doit se dérouler la recette : 
	\begin{itemize}
		\item La recette sera effectué à \INSA,
		\item La recette sera effectué sur une machine de l'équipe \nomEquipe,
		\item La recette sera effectué sur le lot 1.
	\end{itemize}

\section{Description générale des tests}
	Ce paragraphe présente la descriptions générales des tests effectués sur le lot 1 :	
	\begin{itemize}
		\item Les tests seront des scripts permettant la création, la visualisation et la modification du contenu de la base de données.
	\end{itemize}

\section{Validation}	
	\begin{itemize}
		\item Le lot 1 ne pourra partir en validation que si aucun tests unitaires ne produit d'erreur,
		\item La validation sera effectué via des scripts permettant la manipulation du contenu de la base de données.
	\end{itemize}
	




\begin{appendix}
\part*{Annexes}
\addcontentsline{toc}{part}{Annexes}

\listoffigures
\addcontentsline{toc}{chapter}{Table des figures}
	 
\listoftables
\addcontentsline{toc}{chapter}{Liste des tableaux}
\end{appendix}
\pageQuatriemeCouverture

\end{document}
