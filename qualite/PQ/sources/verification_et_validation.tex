Chaque document produit par le PIC doit faire l'objet d'une vérification, d'une validation
et d'une approbation avant diffusion. Ce cycle est présent et décrit dans le \PGC (\PGCCourt). Il est cependant rappelé ici de façon à faciliter la recherche d'informations concernant ce cycle.

\section{Vérification}

\subsubsection*{A faire avant de commencer la vérification}
Avant toute chose, il est nécessaire de vérifier sur la plate-forme installée sur le serveur
(\lintranet) que le rédacteur a bien signalé que la rédaction du document est terminée.

\subsubsection*{A faire pour la vérification}
Le vérificateur doit vérifier les points suivants :
\begin{itemize}
\item vérification orthographique du document ;
\item vérification de la mise en page ;
\item vérification du suivi du document.
\end{itemize}
Des commandes LATEX peuvent être mises en place par exemple dans le cas où le rédacteur
n'est pas sûr sur un certain point. Ces commandes peuvent insérer des caractères dans le texte.
Le vérificateur devra vérifier qu'il n'y a plus d'insertion de ce type dans le document.

\subsubsection*{A faire après avoir effectué la vérification}

Après avoir vérifié un document, la personne en charge de cette tâche devra :
\begin{itemize}
\item le signaler sur \lintranet ;
\item signer le document papier si nécessaire ;
\item modifier le document numérique pour indiquer la date et indiquer qu'il a appliqué son
visa.
\end{itemize}

Le visa dans le document numérique peut être matérialisé de trois façons différentes :
\begin{itemize}
\item \lintranet : le visa est matérialisé seulement sur \lintranet par une case cochée.
\item Courriel : le visa a été matérialisé par courriel.
\item Signé : le visa a été matérialisé sur le document papier par une signature.
\end{itemize}

\subsection{Validation}

\subsubsection*{A faire avant de commencer la validation}
Le validateur doit vérifier que le vérificateur a bien indiqué sur \lintranet et sur le document que le document en question est vérifié.

\subsubsection*{A faire pour la validation}
Le validateur doit valider les points suivants :
\begin{itemize}
\item la pertinence du document ;
\item la complétude du contenu par rapport aux objectifs fixés.
\end{itemize}

\subsubsection*{A faire après avoir effectué la validation}
Après avoir validé un document, la personne en charge de cette tâche devra :
\begin{itemize}
\item le signaler sur \lintranet ;
\item signer le document papier si nécessaire ;
\item modifier le document numérique pour indiquer la date et indiquer qu'il a appliqué son
visa.
\end{itemize}

Le visa dans le document numérique peut être matérialisé de trois façons différentes :
\begin{itemize}
\item \lintranet : le visa est matérialisé seulement sur \lintranet par une case cochée.
\item Courriel : le visa a été matérialisé par courriel.
\item Signé : le visa a été matérialisé sur le document papier par une signature.
\end{itemize}


\subsection{Approbation}

Pour les documents qui ont une portée extérieure au PIC, une approbation sera nécessaire par la personne concernée extérieure au PIC. Le document une fois approuvé devient un enregistrement.

\subsubsection*{Approbation par les tuteurs}
Après chaque réunion avec un des tuteurs, un compte rendu ayant parcouru le cycle de
vérification et validation doit être approuvé.


\subsubsection*{Approbation par les tuteurs}
En ce qui concerne les documents approuvables par le client, si aucune remarque n'est effectuée
par le client sur le relevé de conclusions de réunions sous sept jours après l'envoi de ce dernier,
le compte-rendu est considéré comme approuvé.

\subsection{Diffusion}
Une fois approuvé (si le document nécessite une approbation), le document peut être diffusé.
Pour les documents sans approbation, c'est le rédacteur qui le diffuse. Pour le reste, c'est au
Chef PIC de s'en charger.

\section{Vérification et validation de lots}
Pour certaines livraisons, il se peut que le lot soit un ensemble de documents. Dans ce cas,
le vérificateur doit vérifier que tous les documents sont présents.
\\
Ce vérificateur devra rédiger un \PVVV (\PVVVCourt).