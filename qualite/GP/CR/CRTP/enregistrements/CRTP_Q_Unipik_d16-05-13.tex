% version 1.00	Auteur Pierre Porche date 13/05/2016

\documentclass [a4paper] {article}
\usepackage[utf8]{inputenc}
\usepackage[francais]{babel}
\usepackage[top=2cm, bottom=4cm, left=2cm, right=2cm]{geometry} 
\usepackage{fancyhdr}
\usepackage{graphicx}
\usepackage{color, colortbl}
\usepackage{longtable}
\usepackage{vocabulaireUnipik}
\usepackage{hyperref}
\pagestyle{fancy}
\definecolor{Gray}{gray}{0.8}



%--- En-t�te et pied de page ---%

\renewcommand{\footrulewidth}{0,01cm}
\rhead{}
\chead{\huge{Compte-rendu de réunion de Tutorat Pédagogique}}					%titre
\begin{document}

13/05/2016			 				%Date 
\hfill   
\hfill 	 9:02 - 10:14				%Heure de d�but, heure de fin.


\lfoot{Version : 1.00} 			% version
%--- Fin en-t�te et pied de page ---%
\section*{Historique des révisions}
\begin{center}
			\begin{tabular}{| p{2.5cm} | p{3cm} | p{3cm} | p{3cm} | p{3.5cm} |}
				\hline
				\rowcolor{Gray}
				Version & Date & Auteur(s) & Modification(s) & Partie(s) modifiée(s)		 \\
				\hline
				1.00 & 13/05/2016 & \Pierre & Création & Toutes \\
		\hline		
			\end{tabular}
		\end{center}

\section*{Signatures}

		\begin{center}
			\begin{tabular}{| p{2.5cm} | p{4cm} | p{3cm} | p{3cm} | p{2.5cm} |}
				\hline
				\rowcolor{Gray}
				Rôle & Fonction & Nom & Date & Visa		 \\
				\hline
				Vérificateur & \RQA & \Kafui &  & pgpic \\[30pt]
				\hline
				Validateur & \CP & \Sergi &  & pgpic \\[30pt]	
				\hline
			\end{tabular}
		\end{center}

%--- Réunion --%

\section{Résumé de la semaine passée}
Cette semaine, malgré le temps réduit passé en PIC dû aux autres projets se terminant à ce moment, s'est orientée principalement sur quatre axes : 
\begin{itemize}
\item L'avancement dans la conception de la BD ;
\item La préparation de la revue ;
\item L'inspection technique ;
\item Rencontre avec la DSI et le service juridique.
\end{itemize} 

\section{Base de données}
La conception de la base de données avance bien, nous sommes en phase de vérification de ce qui a été généré par l'ORM, notamment les DELETE ON CASCADE. Il est à noter que nous ne compilons qu'une fois afin de ne pas mapper plusieurs fois. \nomTuteurPedago{} nous explique qu'il faut réaliser un maximum de contrôles hauts.

\section{Revue}
Le plan a été établi, les tâches ont été partagées et la partie principale sera l'avancement du projet. Il est à noter que Véronique Barbier sera absente et que nous ne savons pas pour le moment qui sera présent. \nomTuteurPedago{} nous suggère d'adapter notre présentation à notre interlocuteur.

\section{Inspection technique}
Nous avons reçu le retour de l'inspection technique. \nomTuteurPedago{} nous rappelle que ce n'est pas un audit et que si il y a des non-conformités, il faut générer des \FT{} afin de les corriger, suivre leur réapparition et de les clore.
Concernant la tracabilité, sujet de notre non-conformité, \nomTuteurPedago{} nous conseille d'être plus rigoureux. Il serait recommander de créer des scripts afin de nous aider dans cette démarche.


\section{Rencontre DSI et service juridique}
\Sergi{} est allé voir M. Blondel-Angot dans son bureau pour lui présenter notre projet de partenariat. Durant cette réunion, il a également pu rencontrer des membres de la DSI. La réponse semble plutôt négative car nous ne leur avons pas apporté de solution concernant l'administrateur de notre service. Il n'aurait pas fallu y aller seul, que ce soit pour prendre un compte-rendu ou même pour être soutenu dans ses propos. Afin de mener à bien une négociation, \nomTuteurPedago{} nous informe qu'il serait souhaitable d'arriver avec des arguments et un plan B. La prochaine étape serait de rattrapper le coup en conviant la DSI à notre livraison afin de lui montrer que notre projet  est solide et de le faire rencontrer le client. Pour cela, il serait souhaitable de rencontrer M. Vasseur de manière informelle.


\section{Divers}
\nomTuteurPedago{} nous demande de lui envoyer : 
\begin{itemize}
\item Le diagramme de navigation ;
\item Le diagramme de classe et dictionnaire de données ;
\item Le schéma de BD ;
\item Le diagramme de packages.
\end{itemize}



%--- fin de réunion ---%
\newpage



\end{document}





















