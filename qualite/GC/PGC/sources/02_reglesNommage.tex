% version 1.01 Date 31/03/2016	Auteur Mathieu Medici

Dans le but d'identifier clairement et de manière unique les éléments de configuration du \picCourt{}, il est nécessaire de spécifier des règles de nommage. Chaque élément inclus dans un référentiel doit respecter la règle de nommage qui lui convient.\\

L'ensemble des fichiers suit le modèle suivant :
\begin{center}
  \verb+<Type>_<Referentiel>_Unipik_<Suffixe>+
\end{center}
On utilisera des abréviations pour chaque élément (Type, Référentiel et Suffixe).\\
Exemple de nommage: \verb+PGC_Q_Unipik_v1.00+\\
Exemple de stockage : \verb+pic_unicef/qualite/GC/PGC/+ 


\section{Nom du \picCourt{}}
Ce champ indique l'appartenance du document au \PICCourt{} peu importe sa provenance
 (PIC, client, département \ASI{}, etc). Il a pour valeur \textbf{\nomEquipe}.

\section{Référentiels}

Tous les éléments réalisés au cours du PIC appartiennent à l'un des 5 référentiels suivants : Développement, Livraison, Qualité, Ressources et Spécifications.
\begin{table}[H]
\centering
	\begin{tabularx}{11cm}{|X|c|X|}
	\hline
	\rowcolor[gray]{0.85} Référentiels & Abréviations & Noms des dossiers sous \git{} \\
	\hline
	Développement & D & developpement\\
	\hline
	Livraison & L & livraison\\
	\hline
	Qualité & Q & qualite\\
	\hline	
	Ressources & R & ressources\\
	\hline 
	Spécifications & S & specifications\\ 
	\hline
	\end{tabularx}
\caption{Abréviations associées à chaque type}
\label{Référentiel}
\end{table}

\section{Types}
Chaque document est défini par un type et sera stocké dans le dossier de même nom. Chaque type est associé à zéro, un ou deux suffixes. Voici l'ensemble des documents associé pour chacun à son type et ses suffixes:

\begin{longtable}{|p{12cm}|c|c|}
    \hline
    \rowcolor[gray]{0.85} Documents et Enregistrements & Type & Suffixe 			\endfirsthead
    \hline	
    \rowcolor[gray]{0.85} Documents et Enregistrements & Type & Suffixe \endhead
    \hline
    \multicolumn{3}{|c|}{\textbf{\bsc{suite ...}}}\endfoot
    \hline
    \endlastfoot
    \hline
    \multicolumn{3}{|c|}{\textbf{\bsc{Référentiel Qualité}}}\\
    \hline
    Dossier de Suivi de la Qualité & DSQ & -\\
    \hline
    	\hspace{1cm} Faits Techniques & FT & -\\
    	\hline   
    	\hspace{2cm} Commission de Traitement des Faits Techniques & CTFT & n,d\\
    	\hline
    	\hspace{2cm} Fiche de Fait Technique & FFT & c,n,d\\
    	\hline
    	\hspace{2cm} Fiche d'Ordre de Correction & FOC & n,d\\
    \hline
    \hspace{1cm} Fiche de Compétences & FC & p,v\\
    \hline
    \hspace{1cm} Fiche de Formation & FF & c\\
    \hline
    \hspace{2cm} Fiche Récapitulative de Cours & FRC & c\\    
    \hline
    \hspace{1cm} Fiche de Rôle & FR & r,v\\
    \hline
    \hspace{1cm} Organigramme Fonctionnel & OF & v\\
    \hline
    \hspace{1cm} Plan de Formation & PF & v\\
    \hline
    \hspace{1cm} Questionnaire de Satisfaction & QS & -\\
    \hline
    \hspace{2cm} Questionnaire de Satisfaction Client & QSC & d\\
    \hline
    \hspace{2cm} Questionnaire de Satisfaction Élève & QSE & d\\
    \hline
    \hspace{2cm} Questionnaire de Satisfaction Service Support & QSSS & d\\
    \hline
    \hspace{2cm} Questionnaire de Satisfaction Tuteur & QST & d\\    
    \hline
    \hspace{1cm} Rapport d'Audit Interne & RAI & d\\
    \hline
    \hspace{1cm} Tableau de Bord & TB & s\\
    \hline
    Gestion de Projet & GP & -\\
    \hline
    \hspace{1cm} Fiche de Suivi & FS & p\\
    \hline
    \hspace{1cm} Planification Projet & PP & s\\    
    \hline
    \hspace{1cm} Compte-Rendu & CR & -\\
    \hline
    \hspace{2cm} Compte-Rendu de réunion Client & CRC & d\\
    \hline
    \hspace{2cm} Compte-Rendu de réunion Exceptionnelle & CRE & d\\
    \hline
    \hspace{2cm} Compte-Rendu de réunion Interne & CRI & d\\
    \hline
    \hspace{2cm} Compte-Rendu de réunion Inter-PIC & CRIP & d\\
    \hline
    \hspace{2cm} Compte-Rendu de réunion de Fait Technique & CRFT & d\\
    \hline
    \hspace{2cm} Compte-Rendu de réunion Tuteur Pédagogique & CRTP & d\\
    \hline
    \hspace{2cm} Compte-Rendu de réunion Tuteur Qualité & CRTQ & d\\
    \hline
    \hspace{1cm} Emails & mails & -\\
    \hline
    \hspace{2cm} Mail du Client & MC & d,c\\
    \hline
    \hspace{2cm} Mail du Directeur Qualité & MDQ & d,c\\
    \hline
    \hspace{2cm} Mail de Livraison & ML & d,c\\
    \hline
    \hspace{2cm} Mail de l'Unité P3 & MP3 & d,c\\
    \hline
    \hspace{2cm} Mail du Tuteur Communication & MTC & d,c\\
    \hline
    \hspace{2cm} Mail du Tuteur Pédagogique & MTP & d,c\\
    \hline
    \hspace{2cm} Mail du Tuteur Qualité & MTQ & d,c\\
    \hline
    \hspace{1cm} Procès-Verbal & PV & -\\
    \hline
    \hspace{2cm} Revue Formelle de Démarrage & RFD & -\\
    \hline
    \hspace{3cm} Procès-Verbal de Démarrage & PVD & d\\
    \hline
    \hspace{2cm} Revue Formelle de Fin de Phase de Conception Préliminaire & RFFPCP & -\\
    \hline
    \hspace{3cm} Procès-Verbal de Fin de Phase de Conception Préliminaire & PVFPCP & d\\
    \hline
    \hspace{2cm} Revue Formelle de Fin de Phase d'Intégration & RFFPI & -\\
    \hline
    \hspace{3cm} Procès-Verbal de Fin de Phase d'Intégration & PVFPI & d\\
    \hline
    \hspace{2cm} Revue Formelle de Fin de Phase de Spécifications & RFFPS & -\\
    \hline
    \hspace{3cm} Procès-Verbal de Document de Spécifications Externes & PVDSE & d\\
    \hline
    \hspace{3cm} Procès-Verbal de Document de Spécifications Internes & PVDSI & d\\
    \hline
    \hspace{3cm} Procès-Verbal de Fin de Phase de Spécifications & PVFPS & d\\
    \hline
    \hspace{3cm} Procès-Verbal de Plan de Tests de Validation & PVPTV & d\\
    \hline
    \hspace{2cm} Revue Formelle de Recettes & RFR & -\\
    \hline
    \hspace{3cm} Procès-Verbal de Recettes & PVR & d\\
    \hline
    Gestion des Configurations & GC & -\\
    \hline
    \hspace{1cm} Etat de Configuration & EC & d,c\\
     \hline
    \hspace{1cm} Fiche d'Etat des Données Client & FEDC & aucun\\   
    \hline
    \hspace{1cm} Plan de Gestion des Configurations & PGC & v\\
    \hline
    Portefeuille des Risques et Opportunités & PRO & -\\
    \hline
    \hspace{1cm} Fiche De Risque & FDR & n\\
    \hline
    \hspace{1cm} Fiche D'Opportunité & FDO & n\\
    \hline
    Plan Qualité & PQ & v\\
    \hline
 \multicolumn{3}{|c|}{\textbf{\bsc{Référentiel Spécifications}}}\\
    \hline
    Document de Spécifications Externes & DSE & v\\
    \hline
    Document De Spécifications Internes & DSI & v\\
    \hline
    Plan de Tests de Validation & PTV & v\\
    \hline
 \multicolumn{3}{|c|}{\textbf{\bsc{Référentiel Développement}}}\\
    \hline
    Dossier d'Audit de Code & DAC & d\\
    \hline
    Implémentation & implementation & -\\
    \hline
    LotX & lotX & -\\
    \hline
    \hspace{1cm} \CDR & CDR & l, v\\
    \hline
    \hspace{1cm} Dossier de Conception Détaillée & DCD & l\\
    \hline
    \hspace{1cm} Dossier de Conception Préliminaire & DCP & l\\
    \hline    
    \hspace{1cm} Dossier de Tests & DT & -\\
    \hline
    \hspace{2cm} Dossier de Tests d'Intégration & DTI & l \\
    \hline
    \hspace{3cm} Journal de Tests d'Intégration & JTI & l \\ 
    \hline
    \hspace{3cm} Plan de Tests d'Intégration & PTI & l \\
    \hline
    \hspace{2cm} Dossier de Tests Unitaires & DTU & l \\
    \hline
    \hspace{3cm} Journal de Tests Unitaires & JTU & l \\
    \hline
    \hspace{3cm} Plan de Tests Unitaires & PTU & l \\
    \hline
    \hspace{1cm} Guide Utilisateur & GU & l\\
    \hline
    \hspace{1cm} Procédure d'Installation & PI & l\\
    \hline
 \multicolumn{3}{|c|}{\textbf{\bsc{Référentiel Livraison}}}\\
    \hline   
    Lots & lots & -\\
    \hline
    \hspace{1cm} Lot numéro X & lotX & d\\
    \hline
    Réunion Bilan & RB & -\\
    \hline
    \hspace{1cm} Réunion Bilan \CP & RBCP & -\\
    \hline
    \hspace{1cm} Réunion Bilan \RQ & RBRQ & -\\
    \hline
    Revues & revues & -\\
    \hline
    \hspace{1cm} Revue numéro X & revueX & -\\
    \hline
    \hspace{2cm} Support Présentation & SP & n\\
    \hline
  \caption{Formalisme des différents Types}
  \label{Formalisme Types}  
\end{longtable}

\section{Suffixes}

Lors du nommage nous utilisons des suffixes adaptés en fonction du type de document. Un document peut avoir plusieurs suffixes, dans ce cas ils sont séparés par un "$\_$". La présence de chaque suffixe demandé est obligatoire et dois respecter l'ordre indiqué. Les suffixes possibles sont les suivants : 

	\begin{table}[H]
		\centering
		\begin{tabularx}{10cm}{|X|c|c|}
		\hline
		\rowcolor[gray]{0.85} Suffixe & Abréviation & Format\\
		\hline
		Date & d & dAA-MM-JJ\\
		\hline
		Version & v & vX.YY\\
		\hline
		Lot & l & lX\\		
		\hline
		Numéro & n & nXXX\\
		\hline
		Semaine & s & sXX\\
		\hline
		Rôle & r & rRole\\
		\hline
		Personne & p & pInitiales\\
		\hline
		Commentaire & c & cCommentaire\\
		\hline
		\end{tabularx}
	\caption{Abréviations associées à chaque suffixe}
	\label{Suffixes}
	\end{table}
	


\subsection{Suffixe Date}
\label{suffixe_date}

Le suffixe \verb+Date+ suit le format \textbf{AA-MM-JJ}, AA pour l'année, MM pour le mois et JJ pour le jour. Ce format permet de classer les documents par ordre chronologique.\\

Exemple : \verb+d16-01-27+

\subsection{Suffixe Version}
\label{suffixe_version}
Le suffixe \verb+Version+ suit le format \textbf{vX.YY}. La gestion des versions et des révisions n'étant pas la même selon si le document doit être approuvé ou non, nous allons séparer la convention du suffixe \verb+Version+ en deux parties. \\

\paragraph{Cas des documents avec approbation\\}

Dans le cas des documents à approuver (par exemple PQ ou PGC), X est le numéro d'approbation et commence à 0, YY est le numéro de correction et commence à 00. Lorsqu'un document est approuvé, le numéro d'approbation est incrémenté et le numéro de correction revient à 00. Lorsqu'un document est corrigé suite à une demande de correction, le numéro de correction est incrémenté.

Pour plus de clarté, voici le mode de fonctionnement étape par étape : 
\begin{itemize}
\item[$\rightarrow$] Création d'un document en v0.00;
\item[$\rightarrow$] Envoi du document en v0.00 pour approbation;
\item[$\rightarrow$] Demande de correction sur le document en v0.00;
\item[$\rightarrow$] Correction du document en v0.01;
\item[$\rightarrow$] Envoi du document en v0.01 pour approbation;
\item[$\rightarrow$] ...;
\item[$\rightarrow$] Approbation du document en v1.00 et diffusion obligatoire du document en version v1.00;
\item[$\rightarrow$] Envoi du document en v1.01 pour approbation;
\item[$\rightarrow$] Demande de correction sur le document en v1.01;
\item[$\rightarrow$] Correction du document en v1.02;
\item[$\rightarrow$] Envoi du document en v1.02 pour approbation;
\item[$\rightarrow$] ...;
\item[$\rightarrow$] Approbation du document en v2.00 et diffusion obligatoire du document en version v2.00;
\end{itemize}

\paragraph{Cas des documents sans approbation\\}

Dans le cas des documents n'ayant pas besoin d'être approuvés (par exemple DSI), X est le numéro de version classique et commence à 1, YY n'est pas utilisé.

Pour plus de clarté, voici le mode de fonctionnement étape par étape :
\begin{itemize}
\item[$\rightarrow$] Création du document en v1.00;
\item[$\rightarrow$] Modification du document en v2.00;
\item[$\rightarrow$] Modification du document en v3.00;
\item[$\rightarrow$] ...
\end{itemize}


\subsection{Suffixe Lot}
\label{suffixe_lot}


Le suffixe \verb+Lot+ suit le format \textbf{lX}. X correspond au numéro du lot.\\

Exemple : \verb+l1+\\


\subsection{Suffixe Numéro}
\label{suffixe_numero}

Le suffixe \verb+Numéro+ suit le format \textbf{nXXX}. XXX est un nombre entier compris entre 001 et 999. Il commence à la valeur 001 et est incrémenté à chaque nouveau document.\\

Exemple : \verb+n042+\\

Remarque : Petit cas particulier pour les numéros des documents situés dans un dossier \verb+revuesX+. Le suffixe numéro correspond dans ce cas précis au numéro de la revue et suit le format \textbf{vX} avec X le numéro de la revue.\\


Exemple : \verb+n1+\\

\subsection{Suffixe Semaine}
\label{suffixe_semaine}

Le suffixe \verb+Semaine+ suit le format \textbf{sXX}. XX correspond au numéro de la semaine en cours, la première semaine du PIC étant 01. Seules les semaines comportant au moins un jour ouvré sont numérotées.\\

Exemple : \verb+s09+

\subsection{Suffixe Rôle}
\label{suffixe_role}

Le suffixe \verb+Rôle+ suit le format \textbf{rRole} où Role correspond au rôle concerné par le document. Les valeurs possibles pour Role sont : 
\begin{itemize}
\item CP :  \CP;
\item CPA : \CPA;
\item RQ : \RQ;
\item RQA : \RQA;
\item RGC : \RGC;
\item RRS : \RRS;
\item RS : \RS;
\item RD : \RD;
\item D : \D;
\item PR : \PDR.\\
\end{itemize}

Exemple : \verb+rCP+

\subsection{Suffixe Personne}
\label{suffixe_personne}

Le suffixe \verb+Personne+ suit le format \textbf{pInitiales} où Initiales correspond aux initiales de la personne concernée par le document. Dans le cas où plusieurs personnes sont concernées le suffixe aura le format \textbf{pInitiales1-Initiales2}. Les initiaux possibles sont les suivants :
\begin{table}[H]
	\centering
	\begin{tabularx}{8cm}{|X|c|}
	\hline
	\rowcolor[gray]{0.85} Personnes & Initiales\\
	\hline
	\Florian & FL \\
	\hline
	\Julie & JP \\
	\hline
	\Kafui & KA \\
	\hline
	\Melissa & MB \\
	\hline
	\Michel & MC \\
	\hline
	\Mathieu & MM \\
	\hline
	\Matthieu & MMB \\
	\hline
	\Pierre & PP \\
	\hline
	\Sergi & SC \\
	\hline
	\end{tabularx}
	\caption{Initiales associées à chaque personne}
	\label{Initiales}
\end{table}

Exemple : \verb+pFL+

\subsection{Suffixe Commentaire}
\label{suffixe_commentaire}

Le suffixe \verb+Commentaire+ suit le format \textbf{cCommentaire} où Commentaire a une valeur bien précise selon le type de document. Pour \nomEquipe, il y a cinq documents différents ayant un suffixe \verb+commentaire+, voici leurs valeurs associées :
\begin{itemize}
\item Fiche de Formation (FF) : sujet de la formation;\\
 Exemple : \verb+cLaTeX+
 \item Fiche Récapitulative de Cours (FRC) : sujet du cours;\\
 Exemple : \verb+cLaTeX+
\item emails (MC, MDQ, ML, MP3, MTC, MTP, MTQ) : sujet de l'email;\\
 Exemple : \verb+cdatesRevues+
\item Fiche de Fait Technique (FFT) : peut prendre deux valeurs, \verb+RC+ si c'est un fait technique suite à une réclamation client, \verb+autre+ sinon;\\
 Exemple : \verb+cRC+
\item Etat de Configuration (EC) : voir la partie \ref{EC}.
\end{itemize}
