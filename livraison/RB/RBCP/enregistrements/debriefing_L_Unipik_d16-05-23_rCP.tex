% version 1.00	Auteur Sergi Colomies

\documentclass[asi]{picInsa}
\DeclareGraphicsRule{*}{pdf}{*}{}
\usepackage{pdfpages}


\usepackage{vocabulaireUnipik}

\setcounter{secnumdepth}{4}
\setcounter{tocdepth}{4}
\newcommand{\ligneMaj}[3] {
	\rowcolor[gray]{0.55} \textbf{\textit{#1}} & #2  &  #3\\
	\hline
}
\newcommand{\ligneSup}[3] {
	\rowcolor[gray]{0.65} |\textunderscore \textbf{\textit{#1}} & #2  &  #3\\
	\hline
}
\newcommand{\ligneMed}[3] {
	\rowcolor[gray]{0.75} \hspace{0.25cm} |\textunderscore #1  & #2 & #3 \\
	\hline
}
\newcommand{\ligneSub}[3] {
	\rowcolor[gray]{0.85}  \hspace{0.5cm} |\textunderscore #1 & #2 & #3\\
	\hline
}
\newcommand{\ligneSubSub}[3] {
	\rowcolor[gray]{0.95}  \hspace{0.75cm} |\textunderscore #1 & #2 & #3\\
	\hline
}
\newcommand{\ligneTache}[3] {
	\hspace{1.00cm} |\textunderscore #1 & #2 & #3\\
	\hline
}
\title{Débriefing \CP{}}
\author{\Sergi}


\titreGeneral{Débriefing \CP{}}
\sousTitreGeneral{\nomEquipe}
\titreAcronyme{\Huge{debriefing}}
\version{v1.00}
\titreDetaille{\large{debriefing\_L\_Unipik\_d16-05-23}}
\referenceVersion{debriefing\_L\_Unipik\_d16-05-23}
\auteurs{\Sergi{}}
\destinataires{Unité P3}
\resume{Le présent document contient la présentation du débriefing \CP{} \nomEquipe.}
\motsCles{débriefing, \CP, gestion de projet}
\natureDerniereModification{Création}
\modeDiffusionControle{}

\begin{document}

\couverture{}

\informationsGenerales{}
%% version 1.00, date 29/02/16, auteur Michel Cressant
\begin{pagesService}
	\begin{historique}
		% nouvelles versions à rajouter AU-DESSUS en recopiant les lignes suivantes et en les modifiant :
		\unHistorique{1.00}{02/02/2016}{\Michel}{Création}{Toutes}

	\end{historique}

%        \begin{suiviDiffusions}
%
%            % On place ici les diffusions
%        	\unSuivi{1.00}{}{\nomEquipe{}}
%          
%          
%        \end{suiviDiffusions}

%%Signataires
        \begin{signatures}
	   \uneSignature{Vérificateur}{\RRS}{\Matthieu{}}{17/03/2016}{PGPic}
       \uneSignature{Validateur}{\CP{}}{\Sergi}{}{PGPic}
        \end{signatures}
	
	

	
	
\end{pagesService}

\begin{pagesService}
	\begin{historique}
		% nouvelles versions à rajouter AU-DESSUS en recopiant les lignes suivantes et en les modifiant :
		\unHistorique{1.00}{25/04/2016}{\Sergi{}}{Création}{Toutes}

	\end{historique}

        \begin{suiviDiffusions}

            % On place ici les diffusions
        	\unSuivi{1.00}{}{}
          
       
        \end{suiviDiffusions}

%%Signataires
        \begin{signatures}
	   \uneSignature{Vérificateur}{}{}{}{pgpic}
       \uneSignature{Validateur}{\CPA{}, \newline \RQ{}}{\Pierre}{}{pgpic}
	   \uneSignature{Approbateur}{}{}{}{}
        \end{signatures}
\end{pagesService}

\tableofcontents

\setcounter{chapter}{0}

\chapter{Introduction}
\label{Introduction}





\chapter{Présentation du PIC}
\label{presentation_PIC}
\section{Le client}
L'\nomClient{} Seine Maritime est un comité de l'\nomClient{},  agence de l'Organisation des Nations unies consacrée à l'amélioration et à la promotion de la condition des enfants.  L'\nomClient{} a activement participé à la rédaction, la conception et la promotion de la Convention relative aux droits de l'enfant (CIDE), adoptée lors du sommet de New York le 20 novembre 1989. L'\nomClient{} a reçu le prix Nobel de la paix le 12 janvier 1965. L'\nomClient{} s'est donné les objectifs prioritaires suivants :
\begin{itemize}
	\item l'éducation des filles,
	\item la vaccination et la lutte contre le SIDA et le VIH,
	\item la protection de l'enfance,
	\item la santé des nouveau-nés,
	\item l'égalité hommes-femmes\vspace{0.5cm}.
\end{itemize}

Pour notre projet nous travaillons avec l'\nomClient{} Seine-Maritime. Les fonctions de ce comité sont : l'organisation de plaidoyers dans les écoles sur des thèmes centraux de l'\nomClient, l'organisation d'opérations frimousses consistant à la fabrication de poupées "frimousses" par des élèves afin d'être revendues par la suite, et des actions ponctuelles telles que des ventes lors d’événements particuliers.

\section{Ses besoins}
Afin de pouvoir gérer ses différentes actions (plaidoyer, frimousse, actions ponctuelles,...), l'\nomClient{} Seine-Maritime a besoin d'un outil de gestion des interventions externes de ses bénévoles. Cette outil devra être capable de gérer l'ensemble des bénévoles du comité et de les relier aux différentes interventions externes de manière intelligente.\\
Le parc informatique étant assez hétérogène, tant au niveau matériel que logique, notre service devra fonctionner dans différents environnements (Windows, Linux, MAC). De plus les connaissances informatiques du client étant limitées dans certains cas, nous devrons produire un service intuitif et facile à prendre en main.

\section{Les livrables}
Le projet à réaliser a été découpé en quatre livrables sur la période des deux semestres.\vspace{0.5cm}\\
Le premier livrable constate de l'architecture matérielle et logicielle de l'outil final. Nous avons décidé de réaliser un service web développé en PHP avec une base de données en PostgreSQL. Le framework Symfony permettra de faire le lien entre les deux. Nous utiliserons une architecture trois tiers avec le patron de conception Modèle-Vue-Contrôleur (MVC).\vspace{0.5cm}\\
Le deuxième livrable couvre toute la partie de gestion des plaidoyers. Le livrable doit fonctionner à la fin du semestre pour pouvoir commencer à être utilisé par l'UNICEF dès la rentrée scolaire.\vspace{0.5cm}\\
Le troisième livrable couvrira la partie gestion des frimousses.\vspace{0.5cm}\\
Enfin le dernier livrable couvrira la partie gestion des actions ponctuelles et devra être rendu pour la fin du second semestre de PIC.

\section{L'équipe}
L'équipe PIC se décompose suivant l'organigramme fonctionnel ci-dessous (figure \ref{organigramme} :
\begin{figure}[!h]
	\begin{center}
	\includegraphics[scale=0.35]{images/organigrammeFonctionnel.pdf}
	\label{organigramme}
	\caption{Organigramme fonctionnel}
	\end{center}
\end{figure} 





\chapter{L'avancement du PIC}
\label{avancement}

\section{Prévisions pour le premier semestre}
En collaboration avec notre client, l'\nomClient{} Seine-Maritime, nous avons divisé le projet en quatre grands livrables : deux livrables par semestre. Notre client nous a bien fait comprendre qu'il souhaitait pouvoir commencer à utiliser le service pour les plaidoyers dès la fin du premier semestre. Il fallait donc que toute la partie concernant les plaidoyers soit traitée au premier semestre. Le client a aussi laissé sous-entendre que les livrables attendus au second semestre n'étaient pas les plus importants à leurs yeux et qu'ils seraient amenés à changer.\vspace{0.5cm}\\
Dans les parties suivantes vont être décrits les lots du premier semestre tels qu'ils ont été déterminés initialement.

\subsection{Lot 1}
Le premier lotissement est décrit de la manière suivante dans le cahier des charges : <<Le lot 1 couvre l'architecture matérielle et logicielle à mettre en place pour le fonctionnement de l'application dans sa globalité>>. Pour ce livrable il est donc attendu une réflexion de l'équipe sur l'architecture du système que l'on mettra en place, non pas au niveau des langages de développement mais au niveau structurel et physique. Cette architecture sera également mise en place sur un exemple de base que l'on développera.

\subsection{Lot 2}


\section{Réalisations du premier semestre}
\subsection{Lot 1 : l'architecture}


\subsection{Lot 2 : les plaidoyers}


\subsection{Éléments non traités}


\section{Prévisions du second semestre}

\subsection{Lot 2 : les plaidoyers}


\subsection{Lot 3 : les frimousses}


\subsection{Lot 4 : les actions ponctuelles}






\chapter{La gestion d'équipe}
\label{gestion_equipe}

 
\begin{appendix}
\listoffigures
\addcontentsline{toc}{chapter}{Table des figures}
	 
\listoftables
\addcontentsline{toc}{chapter}{Liste des tableaux}
\end{appendix}
\pageQuatriemeCouverture

\end{document}
