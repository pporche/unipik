Dans ce chapitre, le mot \og{}~archivage~\fg{} désignera à la fois le stockage des documents en vigueur et le stockage des documents périmés (archivage à proprement parler).\\

Cette partie explicite la façon dont les documents produits par le \PICCourt{}
 \nomEquipe{} seront archivés tout au long du projet. 
\nomEquipe{} prévoit un archivage informatique comprenant un archivage
de l'espace public sur le serveur \git{}, mis à disposition par le département \ASI{}, pour les documents relatifs à la démarche qualité et un archivage de l'espace privé sur un serveur interne. 
Un archivage papier est aussi réalisé pour certains documents, comme ceux signés à la main.
Le produit du \PICCourt{} est archivé sur la plate-forme \url{monprojet.insa-rouen.fr} tout au long du projet et y sera conservé après la fin du \PICCourt. \\

%\subsection{Archivage de l'espace privé sur un serveur interne}

%La gestion des configurations prévoit un archivage de l'espace privé de chacun des membres du \picCourt{}. 
 %Cet archivage est défini par les sauvegardes du répertoire \verb+\home\user+ de chaque membre du \picCourt{}. 
 %Ces sauvegardes sont effectuées automatiquement de façon quotidienne au minimum. 
 %Les espaces privés sont enregistrés sur le disque dur du serveur de la salle \picCourt{}, 
 %dans le répertoire \verb+/home/user+.

