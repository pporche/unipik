% version 1.00	Date 01/03/2016	Auteur Pierre Porche

\documentclass[11pt]{article}
\usepackage{draftcopy}
\usepackage[francais]{babel}
\usepackage[utf8]{inputenc}
\usepackage{tabularx}
\usepackage{graphicx}
\usepackage[table]{xcolor}
\usepackage{fancyhdr}
\usepackage{vocabulaireUnipik}

\begin{document}
\chead{\huge Fiche de rôle \CPA}

\section*{Introduction}

Durant le premier semestre, le \RQA{} doit seconder le \RQ{} et effectuer les tâches qu’il lui aura déléguées. À la fin de ce semestre, dans la majorité des cas, il devra se préparer à assurer le rôle de \RQ .

\section*{Tâches liées à sa fonction}

Le \RQA{} devra remplir les missions suivantes :
\begin{itemize}
	\item Tenir à jour les différents indicateurs mis en place pour le \PICCourt;
	\item Remplacer le \RQ{} si celui-ci est indisponible ;
	\item Seconder le \RQ{} dans la réalisation du \PQ, du \PGC{} ainsi que des autres documents liés au projet;
        \item Participer au développement des fonctionnalités nécessaires au projet. 
\end{itemize}


\end{document}