\documentclass [a4paper] {article}
\usepackage[utf8]{inputenc}
\usepackage[francais]{babel}
\usepackage[top=2cm, bottom=4cm, left=2cm, right=2cm]{geometry} 
\usepackage{fancyhdr}
\usepackage{graphicx}
\usepackage{color, colortbl}
\usepackage{longtable}
\usepackage{vocabulaireUnipik}
\pagestyle{fancy}
\definecolor{Gray}{gray}{0.8}



%--- En-t�te et pied de page ---%

\renewcommand{\footrulewidth}{0,01cm}
\rhead{}
\chead{\huge{Compte-rendu de réunion de Tutorat Pédagogique}}					%titre
\begin{document}

05/02/2016			 				%Date 
\hfill   
\hfill 	 9:01 - 10:03 	14:01 - 16:30			%Heure de d�but, heure de fin.


\lfoot{Version : 1.00} 			% version
%--- Fin en-t�te et pied de page ---%
\section*{Historique des révisions}
\begin{center}
			\begin{tabular}{| c | c | c | c | p{4cm} |}
				\hline
				\rowcolor{Gray}
				Version & Date & Auteur(s) & Modification(s) & Partie(s) modifiée(s)		 \\
				\hline
				1.00 & 22/02/2016 & \Pierre, \Kafui & Création & Toutes \\
		\hline		
			\end{tabular}
		\end{center}

\section*{Signatures}

		\begin{center}
			\begin{tabular}{| c | c | c | c | p{4cm} |}
				\hline
				\rowcolor{Gray}
				Rôle & Fonction & Nom & Date & Visa		 \\
				\hline
				Vérificateur & \RGC & \Mathieu & 23/02/2016 & pgpic \\[30pt]
				\hline
				Validateur & \CP & \Sergi & 23/02/2016 & pgpic \\[30pt]	
				\hline
			\end{tabular}
		\end{center}

%--- Réunion --%

\section{Présentation du modèle Entité/Association}

\Julie{} prend la parole pour commencer à expliquer le modèle entité/association.
\nomTuteurPedago{} intervient pour nous signifier que nous aurions dû considérer toutes les alternatives possibles avant de  faire le choix d'utiliser un modèle Entité/Association. Le tuteur pédagogique suggère à l'équipe d'utiliser la méthode MERISE car le modèle Entité/association est très peu utilisé en entreprise. \\
\nomTuteurPedago{} conseille de changer d'outil pour la modélisation car dia n'est pas adapté au formalisme requis. \nomTuteurPedago{} nous conseille de bien faire la différence entre une clef structurelle (indépendante de l'application) et une clef sémantique (dépendante de l'application) dans notre modèle entité/association car celle-ci n'est actuellement pas faite de manière correcte.
\nomTuteurPedago{} nous demande de corriger la contrainte d'administrateur globale en en assignant plusieurs car si il est unique son absence spontanée peut poser problème. \\
\nomTuteurPedago{} nous signale qu'aucun mot de passe n'a été considéré pour les utilisateurs dans le modèle entité/association. Ceci aurait pu être résolu en pensant à regarder les design pattern. 
Le modèle entité/association présenté au client doit seulement prendre en compte le modèle conceptuel. Le modèle logique ne servira qu'à l'équipe de développement. 
Les structures "adresse de l'établissement" et "adresse de la personne" doivent être ramenées au même niveau d'abstraction afin de faciliter la recherche de proximité.
\nomTuteurPedago{} rappelle au \RGC que les règles de nommage sont à respecter partout en commençant notamment par le modèle entité/association actuel où les attributs ne les respectent pas.
\\ ~
Il manque des entités faibles qui seraient pertinentes sémantiquement (projet et vente notamment), "plaidoyer" est une entité mais aussi un attribut de "réalise", il pourrait donc y avoir association ternaire. Dans l'entité "Établissement", trois attributs contiennent le mot "contact", il pourrait donc y avoir une structure ou une association ternaire. \nomTuteurPedago{} nous rappelle que lorsque deux concepts sont reliés logiquement, ils doivent être reliés structurellement. Concernant les dates souhaitées par le demandeur, afin de ne pas avoir de produit cartésien à faire sur les tables, il est souhaitable de créer deux attributs : "momentSouhaite" et "momentAEviter".
\\
Concernant les cardinalités, "contact" devrait avoir une cardinalité (0,N) avec "établissement".
\\
Nous faisons de la généralisation (regrouper des entités proches en sens), il faut donc modifier le schéma en conséquence.
\\ ~
Il ne faut pas utiliser de "etc." dans le texte. Il faut dire "on réalise un schéma selon le modèle E/A" et non "on réalise un modèle E/A".



\section{Résumé semaine passée et questions}
Cette semaine, quatre personnes ont travaillé sur le \DSE{}, dont \Michel{} qui s'est plus tard réorienté sur le \PTV{}. Le \DSECourt{} est très proche d'être terminé et nous avons des questions concernant le \PTVCourt{}.
\paragraph{}
Celui ci doit comprendre la manière employée pour valider les livrables. Il faut expliquer comment nous allons démontrer que chacune des affirmations (ex. : "l'ensemble des composant est interfaçable") que nous faisons. Le \PTVCourt{} dit uniquement ce que l'on va tester et comment (ex. : "La validation se fera au travers d'un programme qui montrera la connexion". Les détails sur les tests sont dans le \CDR{}.
\\
Un exemple sur les tests de l'interface : "nous utiliserons tel logiciel" est une affirmation qui appartient au \PTVCourt{}. "nous cliqueront à tel et tel endroit" est une affirmation qui appartient au \CDR{}. \nomTuteurPedago{} nous demande de lui envoyer le \PTV{} en même temps qu'au client.
\\
Il faudra tester indépendamment le modèle, la vue et le contrôleur. Pour la vue, il existe des "logiciels de clic" et pour la Base de Données, il faudra l'interroger. Il faudra ensuite tester les liaisons entre les composants du MVC.
\paragraph{}
La période probatoire dure une semaine durant laquelle des tests ont lieu avec le client qui nous tient informé de leur déroulement. au bout d'une semaine, sans nouvelles de la part du client, la recette est acceptée.
\paragraph{}
Une réunion s'est tenue dans les locaux du client le 03/02/2016. Ceux-ci sont de petite taille et ne sont que peu équipés donc peu pratiques. La réunion a été trop longue pour plusieurs raisons, parmi lesquelles le fait que nous soyons trop nombreux. Il n'y avait pas d'heure de fin prévue et la conduite de réunion était trop "souple". Afin de remédier à ces problèmes, il faut réduire le nombre de participants, fixer une heure de fin, établir un Ordre du Jour, prévoir un timing et servir des cafés et des biscuits.
\\
Nous avons ensuite rapporté à \nomTuteurPedago{} les modifications sur le Cahier des Charges suite à la réunion client (Cf. CRC\_Q\_Unipik\_d16-02-03).
\paragraph{}
Concernant les formations, il existe le site coursenligne.insa-rouen.fr qui pourra nous aider à trouver du contenu adapté. Il faut également ajouter des formations à l'UML dans le plan de formation.
\paragraph{}
Au sujet du serveur mis à disposition, M. Bonnegent nous a fait comprendre que cela ne serait pas possible. \nomTuteurPedago{} nous explique que nous aurions du nous adresser plus haut dans la hiérarchie, mettre en avant l'aspect mécénat et arriver avec une solution "clef en main". L'idée serait d'aller parler de cela avec des personnes comme Anne Caldin et Maxime Reynet puis de signer une convention avec un professeur référent en passant par des personnes clef : Gilles Gasso, Jean Macquet, UNICEF France, etc. .


\section{Revue du modèle E/A}



\section{Revue du GIT}
\nomTuteurPedago{} a regardé le git et certains problèmes persistent : il faut mettre des commentaires dans tous les fichiers pour spécifier leur rédacteur, leur version et leur date de rédaction. Les conventions de nommage n'ont pas été définies ou respectées. Les makefile se répètent ou font référence à des dossiers inexistants.
\\
Le modèle E/A doit se trouver dans le dossier spécifications.




%--- fin de réunion ---%
\newpage



\end{document}





















