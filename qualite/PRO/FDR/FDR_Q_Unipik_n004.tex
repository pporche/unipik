% version 1.02	date 20/04/2016  	Auteur Pierre Porche
% version 1.01	date 29/03/2016  	Auteur Pierre Porche
% version 1.00	date 29/01/2016  	Auteur Julie Pain

\section*{Informations générales}
 
\begin{table}[H]
\centering
	\begin{tabularx}{16.8cm}{|X|X|}
	\hline
	\rowcolor{gray!40} Numéro du risque & Type du risque \\
	\hline
	004 & Mauvaise communication avec le client \\
	\hline
	\end{tabularx}
\end{table}

\begin{table}[H]
\centering
	\begin{tabularx}{16.8cm}{|X|X|X|}
	\hline
	\rowcolor{gray!40} Date & Visa du \RQ & Visa du \CP \\
	\hline
	 29/01/2016 & pgpic & pgpic \\
	\hline
	\end{tabularx}
\end{table}

\begin{table}[H]
\centering
	\begin{tabularx}{16.8cm}{|X|X|X|X|}
	\hline
	\rowcolor{gray!40} Pilote & Activité WBS & Compte WBS & Phase d'apparition \\
	\hline
	 \Julie & Suivre les Risques et Opportunités & 1.2.3.2 & À partir du début du projet\\
	\hline
	\end{tabularx}
\end{table}

\section*{Description du risque}

\subsection*{Résumé}
	Une mauvaise communication ou une absence de communication avec le client peut entraîner une mauvaise compréhension des besoins du client. Cela peut avoir des conséquences sur la réalisation du produit, notamment une insatisfaction du client ou un rendu en retard du produit. \\
	La mauvaise communication peut être dûe à la lenteur ou à l'absence de réponse du client aux mails du \CP{} ou à un défaut ou retard d'envoi des mails au client.
\subsection*{Analyse des causes}
	voir figure \ref{risque mauvaise communication client}.

\subsection*{Criticité}

\begin{table}[H]
\centering
	\begin{tabularx}{16.8cm}{|>{\columncolor{gray!40}}X|X|}
	\hline
	Gravité & 3\\
	\hline
	Probabilité & 3\\
	\hline
	Criticité & Critique \\
	\hline
	\end{tabularx}
\end{table}
\newpage

\section*{Actions}
\subsection*{Actions préventives}

\centering
	\begin{longtable}{|p{7cm}|p{7cm}|}
	\hline
	\rowcolor{gray!40} Numéro de cause & Actions préventives \\
	\hline
	1 & \begin{itemize}
		\item Mettre en place un planning.
		\end{itemize} \\
	\hline
	2 & \begin{itemize}
		\item Faire valider les \CRC{} par le client.
		\end{itemize} \\
	\hline
	3 & \begin{itemize}
		\item Répondre aux mails du client le plus tôt possible.
		\end{itemize} \\
	\hline
	4 & \begin{itemize}
		\item Définir des règles de communication.
	\end{itemize} \\
	\hline
	5 & \begin{itemize}
		\item Réaliser des actions de Team Building.
	\end{itemize} \\
	\hline
	6 & \begin{itemize}
		\item Réaliser des tutorats communication.
	\end{itemize} \\
	\hline
	\end{longtable}

\flushleft
\subsection*{Plan de contournement}

\begin{enumerate}
	\item Récupérer la version du document la plus récente possible.
	\item Ré-imprimer le document et le refaire valider si besoin.
	\item Refaire le travail perdu le plus vite possible.
\end{enumerate}

\section*{Décision de clôture}
Par le \CP{} et le pilote du risque.
\begin{table}[H]
\centering
	\begin{tabularx}{16.8cm}{|X|X|}
	\hline
	\rowcolor{gray!40} Date de clôture & Raison de la clôture \\
	\hline
	  & \\
	\hline
	\end{tabularx}
\end{table}

\section*{Historique des modifications}
\begin{table}[H]
\centering
	\begin{tabularx}{16.8cm}{|X|X|}
	\hline	
        \rowcolor{gray!40} Date & Modification \\
        
        \hline
	29/03/2016 & modification de la probabilité de 2 à 3 pour cause de vacances du client. \\
	\hline
	23/03/2016 & modification de la probabilité de 1 à 2 pour cause de vacances du client. \\
	\hline
	14/03/2016 & modification de la probabilité\\
	\hline
	\end{tabularx}
\end{table}
\newpage

\begin{figure}
	\centering
	\includegraphics[scale=0.8]{images/nPourquoiFDR004}
	\caption{\label{risque mauvaise communication client}risque mauvaise communication client - méthode des n pourquoi}
\end{figure}
