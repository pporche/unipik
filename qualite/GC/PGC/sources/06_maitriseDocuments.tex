% version 1.01 Date 14/03/2016	Auteur Mathieu Medici

\section{Historique des évolutions}

Les modifications apportées à un document apparaissent dans sa page de service. À chaque modification il faut indiquer les informations suivantes :
\begin{itemize}
 \item \textbf{version}: indique le numéro de version du document suite à la modification apportée;
\item \textbf{date} : indique la date de la dernière modification du document dans le format \verb+JJ/MM/AAAA+;
\item \textbf{auteur} : indique les noms complets des personnes ayant effectué les modifications;
\item \textbf{modification} : décrit brièvement le type de modification apportée. Si le document a
été diffusé, cette section comprend obligatoirement :
	\subitem \textbullet{ }le numéro de la Fiche d'\OC{} (FOC) qui a donné lieu à la modification;
	\subitem \textbullet{ }le numéro de la Fiche de Fait Technique (\FFTCourt{}) qui lui est associée.
\item \textbf{parties modifiées} : indique les numéros de chapitres et de sections modifiés.
\end{itemize}

\section{Suivi des diffusions}

Le suivi des diffusions est indiqué dans la page de service du document. Il comprend trois informations pour chaque diffusion :
\begin{itemize}
\item \textbf{version} : indique la version du document concerné par la diffusion ;
\item  \textbf{date} : indique la date de diffusion de document ;
\item \textbf{destinataire} : indique les destinataires du document. 
\end{itemize}
 Le suivi des diffusions n’est rempli qu’une fois le document approuvé.

\section{Signatures}

Le contrôle de la vérification, de la validation et de l’approbation d’un document est visible dans la page de service. Il comporte trois lignes correspondant aux trois personnes en charge de ces étapes. Les trois personnes doivent être différentes. L'auteur du document ne peut pas en faire partie.\\

Le \textbf{vérificateur} est responsable de la relecture du document afin d’en vérifier l’orthographe et la cohérence du contenu. Il ne peut en aucun cas être l'auteur du document vérifié..\\

Le \textbf{validateur} évalue la pertinence du document après qu’il ait été vérifié, il engage aussi sa responsabilité en ce qui concerne la correction du document. Il est en charge de la diffusion du document vers les destinaires identifiés et de la mise à jour des dossiers \verb+approbations+.\\

L’\textbf{approbateur} est une personne externe au \pic{}, en charge de l’acceptation du document.

Pour chacune des trois étapes doivent apparaître les informations suivantes dans la page de service :
\begin{itemize}
\item \textbf{fonction} : la fonction de la personne ayant effectué l’étape ;
\item  \textbf{nom} : le nom de la personne ayant effectué l’étape ;
\item \textbf{date} : la date où l’étape a été effectuée ;
\item \textbf{visa} : la mention "signé" est ajoutée aux documents électroniques, une fois le document signé numériquement. La signature numérique doit être archivée, c'est le seul document permettant de certifier de la signature du document.

\end{itemize}

