% version 1.00, date 05/11/16, auteur Kafui Atanley
%La fonctionnalité 9 permettra une gestion des ventes engrangées par une intervention
%de type frimousse ou effectuées dans un établissement. Il devra être possible de tracer le
%montant collecté lors de la vente afin de permettre de faire un bilan en fin d’année. Les
%informations associées à la vente seront :
% la date de celle-ci,
% le chiffre d’affaire,
% l’établissement dans laquelle la vente a eu lieu,
% l’intervention étant à l’origine de celle-ci si tant donné qu’il y en ait une,
% les remarques éventuelles.
% L’équipe Projet INSA Certifié devra être force de proposition compte au rendu de celle-ci.
\subsection{Fonctionnalité 10}

La fonctionnalité 10 consistera à gérer les ventes au sein de l'application.
Les ventes peuvent être enclenchées par des interventions de type frimousse. 
Les ventes ont lieu au sein d'un établissement.
Les ventes sont caractérisées par : 
\begin{itemize}
\item la date de celle-ci,
\item le chiffre d’affaire,
\item l’établissement dans laquelle la vente a eu lieu,
\item l’intervention étant à l’origine de celle-ci si tant donné qu’il y en ait une,
\item les remarques éventuelles.
\end{itemize}

L'équipe \PIC{} mettra en forme des filtres sur les critères de caractérisation de temporalité, du chiffre d'affaire ainsi que de lieu sur les ventes.
La maquette \ref{fonctionnalite9Ventes} présente comment sera figurée la liste des ventes.
 \\

\begin{figure}[H]
	\centering
	\includegraphics[scale=0.4]{images/maquettes/fonctionnalite9Ventes.png}
	 \caption{Maquette~: Visualiser les demandes et s'attribuer des interventions}
	 \label{fonctionnalite9Ventes}
\end{figure}
