Ici sont décris les différents tests unitaires qui seront effectués sur le lot 1 :

\section{Test Unitaire TU001}
	-> Quel est le but de ce test \\
	
	-> Qu'est ce qu'il fait \\
	
	-> ce que je dois faire (ecrire phpunit + le package jusqua la classe a tester dans un terminal...) \\
	
	-> Le resultat attendu (test OK avec toutes les assertions OK)
	
	
\section{Test Unitaire TU002}

Ce test vérifie la bonne fonctionnalité de la classe "truc"
  		\begin{center}
    	 		\begin{tabular}[h]{|p{0.45\textwidth}|p{0.45\textwidth}|}
			\hline
				Eléments testés & Classe "truc" \\\hline
    				A faire & Ecrire phpunit +package/jusqua/la/classeTest \\\hline
    				Résultat attendu & Test OK et assertion(s) OK \\\hline
     		\end{tabular}
  		\end{center}	
  		
  		
  		
\section{Test Unitaire TU003}

Ce test vérifie la bonne fonctionnalité de la classe "truc"
  		\begin{center}
    	 		\begin{tabular}[h]{|p{0.45\textwidth}|p{0.45\textwidth}|}
			\hline
				Eléments testés & Classe "truc" \\\hline
    				A faire & Ecrire phpunit +package/jusqua/la/classeTest \\\hline
    				Résultat attendu & Test OK et assertion(s) OK \\\hline
     		\end{tabular}
  		\end{center}	
  	
  	
  		
\section{Test Unitaire TU004}

Ce test vérifie la bonne fonctionnalité de la classe "truc"
  		\begin{center}
    	 		\begin{tabular}[h]{|p{0.45\textwidth}|p{0.45\textwidth}|}
			\hline
				Eléments testés & Classe "truc" \\\hline
    				A faire & Ecrire phpunit +package/jusqua/la/classeTest \\\hline
    				Résultat attendu & Test OK et assertion(s) OK \\\hline
     		\end{tabular}
  		\end{center}	



\section{Test Unitaire TU005}

Ce test vérifie la bonne fonctionnalité de la classe "truc"
  		\begin{center}
    	 		\begin{tabular}[h]{|p{0.45\textwidth}|p{0.45\textwidth}|}
			\hline
				Eléments testés & Classe "truc" \\\hline
    				A faire & Ecrire phpunit +package/jusqua/la/classeTest \\\hline
    				Résultat attendu & Test OK et assertion(s) OK \\\hline
     		\end{tabular}
  		\end{center}	