\documentclass [a4paper] {article}
\usepackage[utf8]{inputenc}
\usepackage[francais]{babel}
\usepackage[top=2cm, bottom=4cm, left=2cm, right=2cm]{geometry} 
\usepackage{fancyhdr}
\usepackage{graphicx}
\usepackage{color, colortbl}
\usepackage{longtable}
\usepackage{vocabulaireUnipik}
\usepackage{tabularx}
\usepackage{xcolor}
\definecolor{Gray}{gray}{0.8}

\pagestyle{fancy}

%--- En-t�te et pied de page ---%

\renewcommand{\footrulewidth}{0,01cm}
\rhead{}
\chead{\huge{Fiche de Suivi}}

\begin{document}

\section*{\Kafui}

\begin{tabularx}{16.8cm}{|>{\columncolor{gray!40}}l|X|}
	\hline
	Semaine 5 & \begin{itemize}
	\item Évaluation ArgoUML
	\item Corrections \DSICourt
	\item Vérifications compte-rendus
	\item Nettoyage données client
	\end{itemize}\\
	\hline
        Semaine 6 & \begin{itemize}
	\item Préparation de la revue.
	\item Début de réalisation des controleurs pour le lot 1.
    \item Corrections des différents comptes-rendu.  
	\end{itemize} \\
	\hline
        Semaine 7 & \begin{itemize}
	\item Réalisation des controlleurs pour le lot 1.
	\item Corrections des différents comptes-rendu.
	\end{itemize} \\
	Semaine 8 & \begin{itemize}
        \item Formation Symfony.
        \item Livraison client et démonstration du lot 1.  
	\end{itemize} \\
	\hline
	Semaine 9 & \begin{itemize}
        \item  Vérification de compte-rendu
        \item  Approche top down : Développement de la base de données
	\end{itemize} \\
	\hline
	Semaine 10 & \begin{itemize}
        \item Vérification de compte-rendu
        \item Approche top down : Développement de la base de données
	\end{itemize} \\
	\hline
\end{tabularx}

\end{document}
