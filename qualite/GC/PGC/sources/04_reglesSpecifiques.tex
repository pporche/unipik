% version 1.00	Auteur Mathieu Médici

\section{Référentiels spécifiques}

Le référentiel \textbf{Ressources} ne sera pas soumis aux conventions de nommage mais certaines règles doivent être respectées. Chaque dossier et chaque fichier devra être nommé en lowerCamelCase et de manière évoquateur quant au contenu. 

\section{Dossiers spécifiques}

\subsection{Dossiers images}

\`{A} la racine du répertoire des documents nécessitant l'inclusion d'images, de diagrammes ou
d'autres figures, se trouve un répertoire \textbf{images}. Il est ajouté au \git{} par le
rédacteur du document si besoin est. Dans ce répertoire sont stockés les fichiers du projet
(\verb+.dia+, \verb+.xcf+, \verb+.jpg+, etc\dots). Les \verb+.pdf+ de ces fichiers sont générés
à la compilation par le \verb+makefile+.
Les fichiers du répertoire \verb+images+ ne sont pas soumis aux conventions de nommage mais certaines règles doivent être respectées. Chaque fichier devra être nommé en lowerCamelCase et de manière évoquateur quant au contenu. 

\subsection{Dossiers sources}

\`{A} la racine du répertoire des documents nécessitant l'inclusion de sous-fichiers \verb+.tex+
(correspondant à des chapitres, des sections, etc\dots) se trouve un répertoire \textbf{sources} pour
les documents longs nécessitant d'être subdivisés. 
Ces sous-fichiers \verb+.tex+ seront placés dans ce répertoire et ils seront inclus via
le \verb+.tex+ général. Ces sous-fichiers ne sont pas soumis aux conventions de nommage mais certaines règles doivent être respectées.\\
Chaque fichier devra être nommé de la manière suivante :
\begin{center}
\verb+<Numéro>_<Nom>+
\end{center}
\begin{itemize} 
\item Numéro : numéro du document variant de \verb+01+ à \verb+99+ et prenant la valeur particulière \verb+00+ pour les pages de services;
\item Nom : nom du document en lowerCamelCase et évocateur quant au contenu.\\
\end{itemize}
Exemples : \\
\verb+00_pageService.tex+\\
\verb+01_introduction.tex+\\
\verb+02_partie1.tex+\\

Remarque : si un fichier source appelle d'autres sources, il faut créer un dossier du même nom que le nom du document et dont les fichiers respectent aussi les règles ci-dessus.\\
Exemple : \\
\verb+\partie1+\\
\hspace*{1cm} \verb+01_sousPartie1.tex+\\
\hspace*{1cm} \verb+02_sousPartie2.tex+\\
\hspace*{1cm} \verb+03_sousPartie3.tex+\\

\subsection{Dossiers annexes}

\`{A} la racine du répertoire des documents nécessitant l'inclusion d'annexes se trouve un répertoire \textbf{annexes}.
Ces documents seront placés dans ce répertoire et ils seront inclus via
le \verb+.tex+ général. Ces sous-fichiers ne sont pas soumis aux conventions de nommage mais certaines règles doivent être respectées. Chaque fichier devra être nommé en lowerCamelCase et de manière évoquateur quant au contenu. 

\subsection{Dossiers pdf}

\`{A} la racine du répertoire de chaque document se trouve un répertoire \textbf{pdf}. C'est
le répertoire dans lequel est placé le \verb+pdf+ généré par la compilation \LaTeX{}. Le \verb+pdf+ n'est pas stocké sur le \git{}.

\subsection{Dossiers approbations}

Lorsqu'un document est approuvé par une signature manuscrite, la copie papier de l'enregistrement est archivée dans la salle PIC de  \nomEquipe{} et une numérisation de la page de signatures est effectuée et stockée dans le dossier \textbf{approbations}. Les numérisations seront nommées de la manière suivante:
\begin{center}
\verb+ApprobSignature_<Nom du document>+
\end{center}

Exemple : \verb+ApprobSignature_PGC_Q_Unipik_v1.00+

\subsection{Dossiers enregistrements}

Les dossiers \textbf{enregistrements} permettent de stocker les sources des documents présents dans le répertoire associé à ces documents.\\
Exemple : Le répertoire \verb+pic_unicef/qualite/GP/CR/CRC+ regroupe l'ensemble des Comptes Rendus Client, on y trouve alors un dossier \verb+enregistrements+ qui contient l'ensemble des sources, un dossier \verb+pdf+ pour le stockage sur SA MACHINE des pdf et le makefile qui compile toutes les sources en même temps.


\section{Précisions sur les répertoires de l'arborescence}

\subsection{Répertoire I}

Localisation : \verb+pic_unicef/developpement/I+\\

Le répertoire I (Implementation) est constitué de dossiers représentatifs des différentes applications à produire par le PIC. Le nom de ces dossiers et leur organisation sont gérés par le responsable développement. Chaque nom de dossier et de fichier doit respecter le nommage lowerCamelCase.

\subsection{Répertoire lotX}

Localisations : \\
\verb+pic_unicef/developpement/lotX+\\
\verb+pic_unicef/livraison/lots/lotX+\\

Les répertoires lotX contiendront, en plus des livraisons officielles prévues, le Cahier De Recette du lot concerné annoté par le client.

\subsection{Répertoire client}

Localisation : \verb+pic_unicef/ressources/client+\\

Le répertoire client contient l’ensemble des documents fournis par le client à \nomEquipe.

\section{Règles de rédaction des dates}

Les dates figurant dans les documents sont toutes au format \textbf{JJ/MM/AAAA}. Attention à ne pas confondre avec le format de date des  suffixes dans la convention de nommage (\verb+dAA-MM-JJ+).

