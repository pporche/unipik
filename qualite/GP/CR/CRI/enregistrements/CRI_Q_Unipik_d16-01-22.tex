% version 1.00	Auteur Pierre Porche

\documentclass [a4paper] {article}
\usepackage[utf8]{inputenc}
\usepackage[francais]{babel}
\usepackage[top=2cm, bottom=4cm, left=2cm, right=2cm]{geometry} 
\usepackage{fancyhdr}
\usepackage{graphicx}
\usepackage{color, colortbl}
\usepackage{longtable}
\usepackage{vocabulaireUnipik}
\pagestyle{fancy}
\definecolor{Gray}{gray}{0.8}

\pagestyle{fancy}


%--- En-t�te et pied de page ---%
\renewcommand{\footrulewidth}{0,01cm}
\rhead{}
\chead{\huge{Compte-rendu de réunion interne}}					%titre
\begin{document}

22/01/2016			 				%Date 
\hfill   
\hfill 	 16:03-17:12 				%Heure de d�but, heure de fin.


\lfoot{Version 1.00} 			% Rédacteur

%--- Fin en-t�te et pied de page ---%
\section*{Historique des révisions}
\begin{center}
			\begin{tabular}{| c | c | c | c | p{4cm} |}
				\hline
				\rowcolor{Gray}
				Version & Date & Auteur(s) & Modification(s) & Partie(s) modifiée(s)		 \\
				\hline
				1.00 & 22/01/2016 & \Pierre & Création & Toutes \\
		\hline		
			\end{tabular}
		\end{center}

\section*{Signatures}

		\begin{center}
			\begin{tabular}{| c | c | c | c | p{4cm} |}
				\hline
				\rowcolor{Gray}
				Rôle & Fonction & Nom & Date & Visa		 \\
				\hline
				Vérificateur & \RQA & \Kafui & 22/01/2016 & pgpic \\[30pt]
				\hline
				Validateur & \CP & \Sergi & 22/01/2016 & pgpic \\[30pt]	
				\hline
			\end{tabular}
		\end{center}
		
\newpage		

%--- Réunion --%

\section{Lecture du cahier des charges}
L'ensemble de l'équipe a lu le cahier des charges en notant les remarques de chacun au fur et à mesure. Dans l'ordre de lecture :
\begin{itemize}
\item 1.1 : L'hébergeur n'imposera pas de technologie ;
\item 1.2 : Terme "léger" à clarifier ;
\item 1.2 : Il existe d'autres navigateurs que ceux cités ;
\item 1.2 : Le mécénat risque être plus que complexe à mettre en place connaissant les hébergeurs ;
\item 1.2 : Des langages de BD sont imposés sans que nous comprenions pourquoi ;
\item 1.3 : Il faudrait clarifier avec un schéma car nous n'avons pas saisi ;
\item 1.4 : Que sont les informations externes à l'application? ;
\item 1.5 : Les interlocuteurs sont ils au courant du projet? ;
\item 2.1 : Lors de la suppression, attention à la CNIL, quels champs faut il garder? ;
\item 2.1 : Que sont les VAE? ;
\item 2.1 : Si le bénévole est investi dans deux activités, quel responsable prévenir? ;
\item 2.2 : Il faudrait clarifier le champ "ville de rattachement". De plus, un même code postal peut référencer deux villages différents.
\item 2.3.1 : Le terme "distance d'une ville" est peu clair : distance d'une ville à quoi? ;
\item 2.3.2 : Champs pré-remplis et/ou suggestions pendant le remplissage ;
\item 2.3.4 : L'implémentation de la fonction SMS ne sera pas possible ;
\item 2.4 : F9, ne faudrait il pas rajouter un lieu de vente également? .
\end{itemize}
Ces remarques seront présentées au client lors de la réunion du 26/01/2016.

\section{Analyse des risques et opportunités}
Après avoir passé en revu chaque risque et opportunité, aucune modification n'est à prévoir sur le nombre de ceux ci ou leur criticité.

%--- fin de réunion ---%
\end{document}





