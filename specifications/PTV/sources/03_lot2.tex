% version 1.00 date 08/12/2016 auteur(s) Pierre Porche

	Ce chapitre a pour but de valider les points évoqués dans le \DSE{} et d'expliquer notre démarche pour s'assurer du bon fonctionnement de notre lot.
	
\section{Description du lot}
	Ce paragraphe rappelle les exigences demandées pour le lot 2 :
	\begin{itemize}
		\item Une gestion individuelle des opérations conventionnelles comme la création d'un individu, la modification des caractéristiques le concernant ou encore la suppression d'un bénévole;
		\item Une gestion individuelle des opérations conventionnelles comme la création d'un établissement, la modification des caractéristiques le concernant ou encore la suppression d'un établissement;
		\item Un envoi d'email d'invitation aux établissements à remplir le formulaire de demande d'intervention;
		\item Une aide au remplissage du formulaire grâce à une complétion automatique;
		\item Un envoi d'email de confirmation de demande d'intervention permettant aussi l'annulation de cette demande;
		\item Un affichage d'une carte montrant les lieux où des demandes d'interventions on été faites pour un créneau que l'utilisateur aura entré;
		\item Un envoi d'email d'information, à l'établissement, de prise en charge de l'intervention demandée;
		\item Un envoi d'email de rappel quelques jours avant l'intervention au plaideur et à l'établissement concernés.
	\end{itemize}
	
\section{Conception du lot}
	Ce paragraphe explique les méthodes suivies pour la conception de ce lot :
	\begin{itemize}
		\item Le code devra respecter les bonnes conventions de codage (indentation, nom de variable explicite,... ),
		\item Le code devra être documenté.
	\end{itemize}
	
\section{Déroulement de la recette}
	Ce paragraphe explique comment doit se dérouler la recette : 
	\begin{itemize}
		\item La recette sera effectuée au sein de l'INSA ;
		\item La recette sera effectuée sur une machine du groupe UNIPIK ;
		\item La recette sera effectuée sur le lot 2.
	\end{itemize}

\section{Description générale des tests}
	Ce paragraphe présente la description générale des tests effectués sur le lot 2 :	\\
	
	Les tests seront effectués sur l'application web demandée, développée en Modèle-Vue-Controleur en architecture trois tiers. Ils permettront de démontrer la fonctionnalité des exigences sitées ci-dessus.

\section{Validation}	
	\begin{itemize}
		\item Le lot 2 ne pourra partir en validation que si aucun test unitaire ne produit d'erreur ;
		\item Le lot 2 ne pourra partir en validation que si aucun test d’intégration ne produit d'erreur.
	\end{itemize}
	
