\section{Suivi des configurations}

Afin de permettre le suivi de ces documents ainsi que leur état en temps réel, \nomEquipe{} utilise l'intranet \lintranet. 
Il permet de savoir quelle est la dernière version applicable d'un document et récapitule l'ensemble du contenu des référentiels. 
Cette page est actualisée par la personne concernée lors de la création, la vérification, la validation, l'approbation, la diffusion et l'archivage d'un 
document. 
Sur cette page figurent les informations suivantes:
\begin{itemize}
	\item \textbf{référentiel}: nom du référentiel auquel le document appartient;
	\item \textbf{document}: nom du document;
	\item \textbf{rédacteur}: nom du rédacteur qui a initié le document;
	\item \textbf{vérificateur}: nom du vérificateur;
	\item \textbf{validateur}: nom du validateur;
	\item \textbf{approbation}: le document a été approuvé ou non (case verte si le document a été
	approuvé, rouge sinon);
	\item \textbf{diffusion}: le document a été diffusé ou non (case verte si le document a été
	diffusé, rouge sinon);
	\item \textbf{archivage}: le document a été archivé ou non (case verte si le document a été
	archivé, rouge sinon);
	\item \textbf{référence}: nom complet du document;
	\item \textbf{localisation}: où a été diffusé le document;
	\item \textbf{chemin}: chemin complet d'accès au document sur le \git.
\end{itemize}


\section{L'État de Configuration}

Un État de Configuration est composé d'un ou plusieurs documents.
Il sera effectué avant chaque livraison au \client. 
Seront présents dans l'EC seulement les articles dont a besoin le destinataire 
(cf. modèle présenté en Annexe \ref{modèle EC}).
 

Les informations suivantes seront spécifiées dans l'EC:
\begin{itemize}
	\item \textbf{les raisons} qui ont motivé la création de ce document ;
	\item \textbf{le destinataire} des documents livrés ;
	\item \textbf{date de création} de l'EC ;
	\item \textbf{liste des documents} composant l'EC demandé.
\end{itemize}


\bigskip
Par sa signature, le vérificateur de l'État de Configuration atteste avoir vérifié la présence
de tous les éléments nécessaires à la livraison ainsi que leur cohérence, tant au niveau de
la forme que du fond.

