% version 1.00	Auteur Julie Pain

\documentclass[11pt]{article}
\usepackage{draftcopy}
\usepackage[francais]{babel}
\usepackage[utf8]{inputenc}
\usepackage{tabularx}
\usepackage{graphicx}
\usepackage[table]{xcolor}
\usepackage{fancyhdr}
\usepackage[margin=0.5in]{geometry}
\usepackage{vocabulaireUnipik}
\usepackage{longtable}


\begin{document}

\chead{\huge Fiche de compétence}

\begin{center}
\begin{table}[!hp]

	\begin{tabularx}{\linewidth}{|X|X|}
	\hline
	\rowcolor{gray!40} Élève ingénieur & Date et signature \\
	\hline
	\end{tabularx}
	\begin{tabularx}{\linewidth}{|X|X|}
	Nom : Pain &  \\ 
	Prénom : Julie & \\
	Semestre : 8 & \\
	\hline
	\end{tabularx}
\end{table}
\end{center}

\section*{\large\FR}

\begin{table}[!hp]
\centering
	\begin{tabularx}{\linewidth}{|X|X|}
	\hline
	\rowcolor{gray!40} Référence \WBSCourt & Description du rôle \\
	\hline
	 1.3 & Réaliser les produits \\
	 \hline
	\end{tabularx}
\end{table}


\section*{\large\FC}


\begin{table}[!hp]
\centering
	\begin{tabularx}{\linewidth}{|X|X|}
	\hline
	\rowcolor{gray!40} Niveau inscrit & Notes correspondantes \\
	\hline
	 Faible & 10 et moins \\
	 \hline
	 Moyen & 10 - 13 \\
	 \hline
	 Bon & 13 - 16 \\
	 \hline
	 Très bon & 16 - 20 \\
	 \hline
	\end{tabularx}
\end{table}

\begin{table}[!hp]
\centering
	\begin{tabularx}{\linewidth}{|X|X|X|X|}
	\hline
	\rowcolor{gray!40} Référence & Compétence & Justificatif & Niveau \\
	\hline
	 BD & Base de donées & Examen & Bon \\
	\hline
	 SE & Système d'exploitation & Examen & Moyen \\
	\hline
	 Prograv & JAVA & Examen & Bon \\
	\hline
	 RI & Fondamentaux des réseaux informatiques & Examen & Bon \\
	\hline
	 TW & Technologie Web & Examen & Moyen \\
	\hline
	TI & Théorie de l'information  & Examen & Moyen \\
	\hline
	\end{tabularx}
\end{table}

\begin{table}[!hp]
\centering
	\begin{tabularx}{\linewidth}{|X|}
	\hline
	\rowcolor{gray!40} Compétences à acquérir par formation complémentaire \\
	\hline
	 Symfony, Bootstrap \\
	\hline
	\end{tabularx}
\end{table}

\section*{\large Suivi de formation}




%\begin{table}[!hp]
\centering
	\begin{longtable}{|p{0.11\textwidth}|p{0.11\textwidth}|p{0.12\textwidth}|p{0.09\textwidth}|p{0.11\textwidth}|p{0.06\textwidth}|p{0.12\textwidth}|p{0.12\textwidth}|}
	\hline
	\rowcolor{gray!40} \tiny Date début & \tiny Date fin & \tiny Intitulé Formation & \tiny Nature Formation & \tiny Evaluateur & \tiny Avis & \tiny Signature & \tiny Évaluation à froid \\
	\hline
	07/03/2016 &07/03/2016 &ArgoUML &auto formation & N/A & terminé & &  \\
	\hline
	29/03/2016 & 17/04/2016 & Bootstrap &formation en ligne & \Matthieu & reçu & &  \\
	\hline
	21/03/2016 & 24/03/2016 & Symfony &formation en ligne & \Florian & reçu & &  \\
	\hline
	\end{longtable}
%\end{table}

\end{document}