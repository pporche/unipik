% version 1.00 date 08/12/2016 auteur(s) Pierre Porche

	Ce chapitre a pour but de valider les points évoqués dans le \DSE{} et d'expliquer notre démarche pour s'assurer du bon fonctionnement de notre lot.
	
\section{Description du lot}
	Ce paragraphe rappelle les exigences demandées pour le lot 3 :
	\begin{itemize}
		\item La fonctionnalité 8 permet l’attribution d’intervention de type frimousse ;
		\item L’outil devra faciliter la géolocalisation des établissements, bénévoles et interventions ;
		\item La fonctionnalité 10 permettra une gestion des ventes engrangées par une intervention de type frimousse ou effectuées dans un établissement.
	\end{itemize}
	
\section{Conception du lot}
	Ce paragraphe explique les méthodes suivies pour la conception de ce lot :
	\begin{itemize}
		\item Le code devra respecter les bonnes conventions de codage (indentation, nom de variable explicite,... ),
		\item Le code devra être documenté.
	\end{itemize}
	
\section{Déroulement de la recette}
	Ce paragraphe explique comment doit se dérouler la recette : 
	\begin{itemize}
		\item La recette sera effectuée au sein de l'INSA ;
		\item La recette sera effectuée sur une machine du groupe UNIPIK ;
		\item La recette sera effectuée sur le lot 3.
	\end{itemize}

\section{Description générale des tests}
	Ce paragraphe présente la description générale des tests effectués sur le lot 3 :	\\
	
	Les tests seront effectués sur l'application web demandée, développée en Modèle-Vue-Controleur en architecture trois tiers. Ils permettront de démontrer la fonctionnalité des exigences sitées ci-dessus.

\section{Validation}	
	\begin{itemize}
		\item Le lot 3 ne pourra partir en validation que si aucun test unitaire ne produit d'erreur ;
		\item Le lot 3 ne pourra partir en validation que si aucun test d’intégration ne produit d'erreur.
	\end{itemize}
	
