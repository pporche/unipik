% version 1.00	Auteur Kafui Atanley date 13/09/2016

\documentclass [a4paper] {article}
\usepackage[utf8]{inputenc}
\usepackage[francais]{babel}
\usepackage[top=2cm, bottom=4cm, left=2cm, right=2cm]{geometry} 
\usepackage{fancyhdr}
\usepackage{graphicx}
\usepackage{color, colortbl}
\usepackage{longtable}
\usepackage{vocabulaireUnipik}
\usepackage{hyperref}
\DeclareUnicodeCharacter{00A0}{ }
\pagestyle{fancy}
\definecolor{Gray}{gray}{0.8}



%--- En-t�te et pied de page ---%

\renewcommand{\footrulewidth}{0,01cm}
\rhead{}
\chead{\huge{Compte-rendu de réunion de Tutorat Pédagogique}}					%titre
\begin{document}

12/09/2016			 				%Date 
\hfill   
\hfill 	 13:35 - 14:32				%Heure de d�but, heure de fin.


\lfoot{Version : 1.00} 			% version
%--- Fin en-t�te et pied de page ---%
\section*{Historique des révisions}
\begin{center}
			\begin{tabular}{| p{2.5cm} | p{3cm} | p{3cm} | p{3cm} | p{3.5cm} |}
				\hline
				\rowcolor{Gray}
				Version & Date & Auteur(s) & Modification(s) & Partie(s) modifiée(s)		 \\
				\hline
				1.00 & 13/09/2016 & \Kafui & Création & Toutes \\
		\hline		
			\end{tabular}
		\end{center}

\section*{Signatures}

		\begin{center}
			\begin{tabular}{| p{2.5cm} | p{4cm} | p{3cm} | p{3cm} | p{2.5cm} |}
				\hline
				\rowcolor{Gray}
				Rôle & Fonction & Nom & Date & Visa		 \\
				\hline
				Vérificateur & \RGC & \Melissa &  14/09/2016 & pgpic \\[30pt]
				\hline
				Validateur & \CP & \Pierre &  & pgpic \\[30pt]	
				\hline
			\end{tabular}
		\end{center}

%--- Réunion --%
\section{Changement d'équipe}
Le début de ce semestre entraîne le départ de trois membres du \PICCourt à l'étranger en les personnes de \Mathieu, \Michel{} et \Sergi{} et l'intégration de nouveaux membres que sont \Francois{} et \Juliana. Il y a donc eu une redistribution de rôle :
\begin{itemize}
\item \Pierre{} devient \CP;
\item \Francois{} devient \CPA;
\item \Kafui{} devient \RQ;
\item \Julie{} devient \RD;
\item \Melissa{} devient \RGC.
\end{itemize}

\section{Résumé de la semaine passée}
La semaine précédente s'est orientée autour de 5 axes :  
\begin{itemize}
\item Installation du personnel ;
\item Passation entre les différents membres de l'équipe ;
\item Modification des documents relatifs la qualité ;
\item Reprise en main du projet;
\item Numérisation de la documentation relative à la mise en place de l'environnement de développement.
\end{itemize} 

\section{Base de données}
La base de données à été testée par des requêtes. Les tests sur la liaison entre les couches hautes et basses ont été concluants. 

Selon le processus CRUD (créer - lire - retrouver - supprimer) nous avons actuellement implémenté et testé 80 \% des liaisons entre les couches hautes et basses pour la consultation. La mise à jour des données n'a pas été testée. \nomTuteurPedago{} nous fait remarquer que notre base de données actuelle ne respecte pas les bonnes pratiques de transformation d'UML vers la base de données. 
À l'issue de la réunion nous devrons exporter une copie de la base de données depuis phpPgAdmin que nous enverrons à \nomTuteurPedago{}.

\section{Semaine suivante}
\Francois{} et \Juliana{} affineront leur formation au framework Symfony3. 
\Pierre{} veut prendre rendez-vous pour la livraison du deuxième lot la première semaine d'octobre. Il y a désaccord entre le \RD{} et \Pierre{} sur le fait de savoir si le lot sera fini à cette date. \Pierre{} discutera avec \Julie{} en réunion interne pour prendre la décision adéquate au niveau de la date de livraison. En ce qui concerne l'hébergement, une solution gratuite semble actuellement difficile à envisager. Le \CP{} va s'occuper de trouver la solution la moins chère pour en faire part au client.

\section{Gestion de Projet}
L'utilisation de la méthode SWOT sera mise en avant pour la prise de décision. Les fonctionnalités seront décomposées en tâches unitaires.
\nomTuteurPedago{} nous fait remarquer que nous devons absolument suivre notre chemin critique et y affecter les ressources nécessaires.
\nomTuteurPedago{} conseille que le \RQ{} décentralise son travail et agisse en tant que vérificateur du retour de travail effectué afin de pouvoir mobiliser son temps sur la partie développement du PIC.
\nomTuteurPedago{} nous informe que le processus de qualité au second semestre occupe un temps moindre comparativement au premier semestre et ne devrait mobiliser le \RQ{} seulement deux semaines avant l'audit. Une ressource est actuellement mobilisée pour l'écriture de script pour remplir la base de données ce qui est trop selon \nomTuteurPedago. Nous devons mettre en place une structure de l'équipe Backend/Frontend afin d'éviter un manque de communication.


%--- fin de réunion ---%
\newpage



\end{document}





















