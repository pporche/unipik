Le processus de \textit{gestion de projet} se décompose en trois sous processus : 
\begin{itemize}
\item Planifier le projet
\item Réaliser le suivi
\item Gérer les risques et opportunités
\end{itemize}

\section{Planification du projet}

Le processus \textit{Planifier le projet} peut se découper en trois sous-processus.

\begin{itemize}
\item Analyser le projet : consiste à établir la liste de l'ensemble des activités du projet.
\item Modéliser le projet : consiste en la création des tâches.
\item Ordonnancer les activités : a pour objectif d'établir un planning du projet, sous la forme d'un diagramme GANTT.
\end{itemize}

Nous étudierons tout d'abord ces trois sous-processus, puis nous présenterons le planning de principe et enfin nous verrons les outils  disposition pour la gestion de projet. 

\subsection{Analyse du projet}
Le but de l'analyse est de préparer et de faciliter la modélisation. Nous utiliserons pour cela la structure la structure WBS qui permet de décomposer chaque activité en allant du plus général au particulier. Chaque feuille de la WBS représente une activité. Les WBS proviennent du besoin du client exprimé dans le Cahier des charges et au début de chaque sprint. \\

Dans le cadre de la plannification, nous pourrons également utiliser d’autres structures hiérarchiques comme la \FBSCourt (\FBS), l’ OBS (Object \BS) ou encore la \RBSCourt (\RBS). L’analyse du projet va nous permettre de bien préparer l’étape suivante, c’est-à-dire la modélisation.


\subsection{Modélisation du projet}

Au cours de cette modélisation, nous allons reprendre le WBS réalisé lors de l’analyse du projet et nous allons décomposer chaque activité en une ou plusieurs tâches de manière à établir une liste des différentes tâches. Ensuite nous allons définir chaque tâche en précisant sa durée, sa charge,les effectifs en ressources nécessaires et les prédécesseurs de cette tâche. Cette étape va nous aider à préparer l’ordonnancement des tâches.\\

Chaque attribution de tâche se fera via PGPic et/ou sur le kanbanboard utilisé par les membres de l'équipe, notamment pour les tâches de réalisation d’un Sprint. Les tâches seront attribuées aux différents membres de l’équipe PIC en prenant en compte des compétences et des disponibilités de chacun.

\subsection{Ordonnancement}

L'objectif de cette phase est d'ordonnancer les activités afin d'établir un planning du projet. Ce planning sera sous la forme d'un GANTT. Ce type de diagramme permet d’avoir une représentation visuelle dans le temps.\\

Pour répertorier les compétences de chacun, des fiches de compétences seront créées au début du projet et mises à jour au fur et à mesure des formations. L’ensemble des formations ainsi que les analyses effectuées sont renseignées dans la Fiche de Suivi des Formations.\\ 

Ce planning sera mis à jour par le \CP ou par son adjoint chaque semaine lors de la réunion d’avancement hebdomadaire. Il sera par la suite disponible sur PGPic.


\subsection{Planning de principe}

Comme dit ci-dessus, le planning du \PICCourt est fait par le \CP et le \CPA. Le responsable développement pourra aussi être sollicité selon le type de tâche.

\subsection*{Semestre 1 : Janvier 2015 - Mai 2015}

Durant le premier semestre, nous disposons de 12 semaines de \PICCourt. Le planning global sera donc le suivant : 
\begin{itemize}
\item Installation, initialisation des procédures de qualité, prise en main des technologies internes (1 semaine);
\item Premier sprint: réalisation de 1er lot (4 semaines);
\item Deuxième sprint (4 semaines);
\item Troisième sprint: finalisation du 2ème lot (3 semaines);
\end{itemize}

\subsection*{Semestre 2 : Septembre 2015 - Janvier 2016}

Durant le second semestre, nous disposons de 14 semaines de \PICCourt. Le planning global sera donc le suivant : 
\begin{itemize}
\item Quatrième sprint: réalisation de 3ème lot (6 semaines);
\item Cinquième sprint (4 semaines);
\item Sixième sprint: finalisation du 4ème lot (4 semaines);\\
\end{itemize}

Ce planning est théorique et pourra être modifié à l'avenir. 

\subsection{Outils pour la gestion du projet}

Nous allons présenter dans cette partie le logiciel et les méthodes utilisés pour la gestion du \PICCourt. \\

\subsubsection*{PGPic}

PGPic est un logiciel développé initialement par l'équipe Geotopic, une équipe \PICCourt de l'\INSACourt. Il est utilisé pour une grande partie de la gestion de projet. \\

Il est disponible à l’adresse \url{https://pgpic.insa-rouen.fr}. Il possède de nombreuses fonctionnalités : 
\begin{itemize}
\item \textbf{Créer et attribuer les tâches}. Le \CP définit les tâches pour chaque membre du \PICCourt, et ceux-ci sélectionnent la tâche sur laquelle ils sont en train de travailler. Le \CP peut donc voir le temps passé par les membres de son équipe sur chaque tâche, étant donné que PGPic a également un outil de comptage du temps de travail installé sur les machines de la salle \PICCourt
\item \textbf{Organiser les tâches}. Le \CP peut agencer les tâches en visualisant un diagramme GANTT. Il est possible de retrouver facilement le diagramme GANTT d'une journée partculière car celui-ci est sauvegardé tous les jours. 
\item \textbf{Gérer le suivi des tâches}. Chaque membre peut consulter la fiche de tâche qui le concerne. Ces fiches évoluent chaque jour et sont enregistrées
\item \textbf{Voir le temps de travail} sur la semaine de chaque membre du \PICCourt. Tous les membres doivent se connecter en arrivant en salle \PICCourt à PGPic, et rester connectés tout au long de leur travail. En fin de semaine, le \CP peut donc vérifier si chaque membre du PIC a bien réalisé le nombre d’heures requis
\item \textbf{Voir le calendrier}. Le Chef PIC peut y ajouter les dates des réunions, audits, fins de sprint, livraison afin que chaque membre du PIC ait une vision globale des dates limites du semestre \\ 
\end{itemize}

En revanche, les réclamations du client doivent être prises en compte et PGPic ne permet pas ce suivi. Un autre outils de gestion devra donc être utilisé en parallèle. 


\subsubsection*{Kanbanboards} 

Les Kanbanboards sont des tableaux de suivi des tâches. Celles-ci sont triées dans quatre catégories :
\begin{itemize}
\item \`A faire
\item En cours
\item \`A vérifier
\item Terminé \\ 
\end{itemize}

Nous utiliserons cet outil pour superviser l'avancement technique du projet.

\section{Suivi du projet} 

Dans cette partie, nous allons aborder le processus de \textit{Suivi du projet} en étudiant tout d'abord le suivi d'avancement du projet et enfin les communications internes et externes.  

\subsection{Suivi d'avancement}

Nous mettons en place un suivi hebdomadaire ayant pour objectif de mesurer l'avancement du \PICCourt. Nous organiserons entre autre des réunions hebdomadaire qui permettront d'établir et de discuter du planning à venir. Lors de ces réunion,  et de déclarer et analyser les risques potentiels. Avant chacunes de ces réunions, chaque membre du PIC doit décrire l'avancement des tâches qui lui sont attribuées.\\

À la suite de ces réunions, le \CP{} distribue des éventuelles fiches de tâches modifiées et établit un diagramme de GANTT des tâches futures pour l'ensemble de l'équipe. Un compte rendu sera rédigé à chaque réunion et mis à disposition de l'équipe. \\

De plus, des \textit{daily Scrum} sont mis en place les jours où nous sommes tous présents afin de coordonner les tâches en cours. Le suivi des tâches globales sera fait à l’aide de PGPic.

\subsection{Communication}

La communication est l'un des aspects les plus importants de la conduite de projet. Nous verrons tout d'abord comment nous communiquons avec les Tuteurs (\tuteurPedagogique \tuteurQualite), ensuite comment s'organise la communication avec le client et enfin les communications internes et inter\PICCourt.\\ 

Toutes les réunions ne se déroulent pas de la même façon. Cependant, l’archivage d’un compte-rendu se fait toujours en version numérique et parfois en version papier (lorsqu’il est signé).

\subsubsection*{La communication avec les Tuteurs}
La communication se base sur des réunions hebdomadaires par tuteur dont le jour est définis en début de \PICCourt et en cas de besoin, l’envoi de courriels. Des réunions en dehors de ce créneau pourront être organisées exceptionnellement. Ces dernières se déroulent de la façon suivante :
\begin{enumerate}
\item Le \CP{} définit avant chaque réunion un ordre du jour qui recense toutes les questions et points à aborder pendant la réunion. Chaque membre du \PICCourt{} a la possibilité de consulter et de rajouter des éléments à cet ordre du jour. Cela peut se faire par oral ou bien via le bloc-notes de PGPic. 
\item Un secrétaire est désigné à chaque début de réunion pour prendre des notes sur les points abordés et rédiger le compte-rendu au plus tard dans les 7 jours qui suivent.
\item Une fois rédigé, le compte-rendu est soumis à vérification puis validation. Le compte-rendu doit être validé avant la prochaine réunion au plus tard.
\item Une fois validé, le compte-rendu est envoyé au tuteur. Si le compte-rendu n’est pas approuvé, les modifications nécessaires sont apportées, puis il est renvoyé pour approbation.
\item Si le compte-rendu n’est pas approuvé, il est réédité, vérifié, validé et soumis à approbation à la prochaine réunion.
\item Une fois approuvé, le compte-rendu est archivé. 
\end{enumerate}

\subsubsection*{La communication avec le client}

La communication avec le client repose sur : 
\begin{itemize}
\item Des courriels
\item Des appels
\item Des réunions entre l'équipe et le client \\
\end{itemize}

Le \CP est l’interlocuteur privilégié du client. En cas d’absence et de nécessité, le \CPA peut devenir cet interlocuteur. La fréquence des échanges sera au minimum hebdomadaire. En cas de nécessité, une prise de contact téléphonique peut être initiée par
le client ou le \CP. L’échange d’e-mails est mené depuis l’adresse mise en place dans le cadre des \PICCourt, associée au \CP. Ces e-mails sont archivés et classés selon les libellés définis dans le \PGC. Les réunions avec le client sont prédéfinies à l’avance entre l’équipe \PICCourt et ce dernier. La réunion avec le client se déroule de la façon suivante :
\begin{enumerate}
\item Le \CP et le \CPA définissent avant chaque réunion un ordre du jour qui recense toutes les questions et points à aborder pendant la réunion. Chaque membre du \PICCourt a la possibilité de consulter et de rajouter des éléments à cet ordre du jour. Cela peut se faire par oral ou bien via le bloc-notes de PGPic.
\item Un secrétaire est désigné à chaque début de réunion pour prendre des notes sur les points abordés et rédiger le compte-rendu au plus tard dans les 7 jours qui suivent.
\item Une fois rédigé, le compte-rendu est envoyé pour vérification puis validation. La validation est obligatoirement effectuée par le \CP.
\item Une fois validé, le compte-rendu est envoyé par courriel au client pour approbation. Si le client ne fait aucune remarque sur le compte-rendu dans les 7 jours suivants l’envoi du mail, le compte-rendu est considéré comme approuvé.
\item Une fois approuvé, le compte-rendu est archivé.
\end{enumerate}


\subsubsection*{Les communications internes}

La communication interne comprend :
\begin{itemize}
\item Des réunions hebdomadaires
\item Des \textit{Daily Scrum}
\item Courriels
\item Groupe Facebook privé \\
\end{itemize}

Les réunions hebdomadaires auront lieu avec l’équipe au complet. Cette réunion aura pour but de traiter de l’avancement du projet ainsi que de la gestion des risques. Lors de cette réunion seront aussi inspectés les indicateurs de Qualité si ceux-ci sont disponibles au moment de la réunion. De plus, c’est lors de cette réunion que se fera l’attribution des tâches pour la semaine suivante. Avant chaque réunion, un secrétaire est désigné afin de prendre des notes et de rédigé par la suite un compte-rendu qui sera archivé sous forme numérique.\\

Les \textit{Daily Scrum} sont des petites réunions quotidienne, d’une durée de 15 minutes maximum, s’effectuant debout, le matin. Lors de ces réunions, chaque membre de l’équipe prendra la parole à tour de rôle pour répondre aux questions suivantes :
\begin{itemize}
\item Qu’as-tu fait hier ?
\item Quelles difficultés as-tu rencontré ? Et inversement, de quoi es-tu fier ?
\item Que vas-tu faire aujourd’hui ?
\end{itemize}
Cette réunion permet d’avoir un suivi précis de l’avancement du projet, de soulever d'éventuels problèmes, et d’aider chaque membres à prendre du recul sur les tâches qu’il effectue.\\

Les courriels et le groupe Facebook sont utilisés pour une communication moins formelle. 

\subsubsection*{Les communications inter\picCourt{}}  

Il est important que les équipes des différents \PICCourt puissent communiquer entre elles afin d’échanger des informations concernant le déroulement des \PICCourt et de partager leurs expériences. C'est pourquoi il est important d'organiser des réunions.\\

Les réunions inter\PICCourt{} vont être effectuées soit entre les différents chefs \PICCourt{}, soit entre les différents Responsables Qualité. Le but de ces réunions est de définir des lignes directrices communes aux projets, répondre aux différentes questions qui peuvent survenir lors de la mise en place et du
suivi de la Qualité dans les \PICCourt et prendre des décisions. Elles permettent également d’assurer la Qualité de manière uniforme dans tous les \PICCourt, et de constater les éven-
tuelles différences de Management de la Qualité. Ces réunions donneront lieu à des comptes rendus ou des relevé de décision écrit.

\section{Gestion des risques et opportunités} 

Le processus \textit{Gérer les risques et opportunités} est découpé en trois sous-parties : 
\begin{itemize}
\item identifier les risques et opportunités
\item suivre les risques et opportunités
\item réduire les risques et améliorer les oportunités \\
\end{itemize}

L'objectif de la gestion des risques et opportunités est de souligner les points du projet sur lesquels il faut porter une attention particulière. 

\subsection{Identification des risques et opportunités}

Tous les membres de l'équipe participent à l'identification des risques ou opportunités. Les risques sont iddentifiés en début de semestre et aboutissent à un portefeuille de risque. Un portefeuille d'opportunités est également associé aux opportunités identifiés. Cette identification se poursuit tout au long du semestre. \\ 

L'identification des risques a un double objectif : 
\begin{itemize}
\item prévenir l'apparition d'écart par rapport à un objectif 
\item prévenir l'apparition d'une non-conformité. \\
\end{itemize}
Si un risque est découvert et validé par l'équipe \PICCourt{}, une Fiche de Risques sera rédigée et ajoutée au \PR.\\

L'identification d'une opportunité a plusieurs objectifs : 
\begin{itemize}
\item signaler l'apparition d'une possibilité de meilleur avancement du projet 
\item signaler  une amélioration possible d'un point du projet\\
\end{itemize}
Si une opportunité est découverte et validée par l'équipe \PICCourt, une Fiche d'Opportunités sera rédigée et ajoutée au \PO. \\

\subsection{Suivi des risques et opportunités}

\subsection{Réduction des risques et améliorer les opportunités}




%\subsection{Identification des risques} \label{identificationRisques}

%	 L'identification des risques a un double objectif : 
%	\begin{itemize}
%		\item prévenir l'apparition d'écart par rapport à un objectif ;
%		\item prévenir l'apparition d'une non-conformité.
%	\end{itemize}
%	 Si un risque est découvert et validé par l'équipe \PICCourt{}, une Fiche de Risques sera rédigée et ajoutée au Portefeuille de Risques.\\
%	\newline
%	Tous les risques n'ont pas le même impact. Celui-ci peut-être :
%	\begin{itemize}
%		\item mineur : le risque peut entraîner une difficulté à achever une tâche mais n'entraîne pas de retard ;
%		\item moyen : le risque peut entraîner un retard sur le projet ;
%		\item important : le risque entraîne forcement un retard sur le projet ;
%		\item maximum : le risque bloque le projet.\\
%	\end{itemize}
%
%Chaque risque a également une priorité. Cette priorité est définie en fonction de l'impact du risque mais aussi de sa probabilité d'apparition.
%
%Nous devons également définir sa probabilité :
%\begin{itemize}
% \item peu probable : (P < 20\%) ;
% \item possible : (20\% < P < 40\%) ;
% \item probable : (40\% < P < 60\%) ;
% \item très probable (P > 60\%).\\
%
%\end{itemize}
%
%La priorité du risque est exprimée en fonction de l'impact et de la probabilité d'apparition.
%
%\begin{table}[h]
%\centering
%\begin{tabular}{c|
%>{\columncolor[HTML]{009901}}c |c|c|
%>{\columncolor[HTML]{FE0000}}c |}
%\cline{2-5}
% & \cellcolor[HTML]{BBDAFF}Peu probable & \cellcolor[HTML]{BBDAFF}Possible & \cellcolor[HTML]{BBDAFF}Probable & \cellcolor[HTML]{BBDAFF}Très probable \\ \hline
%\multicolumn{1}{|c|}{\cellcolor[HTML]{BBDAFF}Mineur} & {\color[HTML]{000000} \textbf{Acceptable}} & \cellcolor[HTML]{009901}{\color[HTML]{000000} \textbf{Acceptable}} & \cellcolor[HTML]{FFC702}{\color[HTML]{000000} \textbf{À Surveiller}} & {\color[HTML]{000000} \textbf{Critique}} \\ \hline
%\multicolumn{1}{|c|}{\cellcolor[HTML]{BBDAFF}Moyen} & {\color[HTML]{000000} \textbf{Acceptable}} & \cellcolor[HTML]{FFC702}{\color[HTML]{000000} \textbf{À Surveiller}} & \cellcolor[HTML]{FFC702}{\color[HTML]{000000} \textbf{À Surveiller}} & {\color[HTML]{000000} \textbf{Critique}} \\ \hline
%\multicolumn{1}{|c|}{\cellcolor[HTML]{BBDAFF}Important} & {\color[HTML]{000000} \textbf{Acceptable}} & \cellcolor[HTML]{FFC702}{\color[HTML]{000000} \textbf{À Surveiller}} & \cellcolor[HTML]{FE0000}{\color[HTML]{000000} \textbf{Critique}} & {\color[HTML]{000000} \textbf{Critique}} \\ \hline
%\multicolumn{1}{|c|}{\cellcolor[HTML]{BBDAFF}Maximum} & \cellcolor[HTML]{FFC702}{\color[HTML]{000000} \textbf{À Surveiller}} & \cellcolor[HTML]{FE0000}{\color[HTML]{000000} \textbf{Critique}} & \cellcolor[HTML]{FE0000}{\color[HTML]{000000} \textbf{Critique}} & {\color[HTML]{000000} \textbf{Critique}} \\ \hline
%\end{tabular}
%\caption{Les priorités des risques} \label{table:prioriteRisque}
%\end{table}
%
%Les risques peuvent être de differente nature :
%\begin{itemize}
% \item juridique (lié au contrat) ;
% \item technique ;
% \item projet (équipe, organisationnel) ;
% \item externe (conjoncturel, par exemple : grève, épidémie, catastrophe naturelle, fermeture
%des locaux) ;
% \item infrastructure (matériel et locaux à disposition de l’équipe PIC) ;
% \item humain (maladies, problemes de differentes types) .
%\end{itemize}
%
%La phase d'apparition du risque est également indiquée sur la fiche de risque. Elle correspond à la phase du projet dans laquelle le risque peut survenir.
%
%\subsection{Suivi des risques}  \label{suivi_des_risques}
%
%Cette phase permettra de garder à jour le \PR{} . Les évolutions de gestion des risques seront soumises, validées et ré-actualisées collectivement lors des réunions hebdomadaires. Les Fiches de Risques pourront donc être mises à jour au fur et à mesure de l'évolution des risques. Chaque membre de l'équipe \PICCourt{} peut proposer l'ajout d'un nouveau risque au \PR{}.\\
%
%Si on estime que le risque ne présente plus aucun problème pour le projet, ce dernier est clôturé. \\
%
%Toute modification opérée sur le risque va entrainer un changement sur sa Fiche de Risque. Le suivi des risques permet de maîtriser et minimiser un maximum de risques.
%
%\subsection{Réduction des risques}  \label{reductionRisques}
%
%L'équipe PIC doit essayer de réduire au maximum les risques qui sont identifiés. \\
%
%L'émission d'une nouvelle Fiche de Risque donnera lieu à une analyse de risque afin de définir des actions préventives et de proposer un plan de contournement. Le résultat de cette analyse sera présent dans la Fiche de Risque.
